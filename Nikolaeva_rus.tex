\begin{englishtitle}
\title{The method of fictitious domains for an equilibrium problem of a Kirchhoff-Love plate with a flat rigid inclusion}
% First author
\author{N.~A.~Nikolaeva}
\institute{North Eastern Federal University, Yakutsk, Russia\\
  \email{niknataf@mail.ru}
 }
% etc

\maketitle

\begin{abstract}
An equilibrium problem for a plate with a flat rigid inclusion is considered. It is assumed that a flat rigid inclusion reaches the boundary at the zero angle and partially contacts a rigid body. The existence and uniqueness of a solution to the problem are proved.
We present differential and variational formulations of the problem. 

\keywords{plate, flat rigid inclusion, fictitious domain}
\end{abstract}
\end{englishtitle}



\iffalse
\documentclass[12pt]{llncs}
\usepackage[T2A]{fontenc}
\usepackage[utf8]{inputenc}
\usepackage[english,russian]{babel}
\usepackage[russian]{nla}

%\usepackage[english,russian]{nla}

% \graphicspath{{pics/}} %Set the subfolder with figures (png, pdf).

%\usepackage{showframe}
\begin{document}
%\selectlanguage{russian}
\fi

\title{Метод фиктивных областей в задаче о равновесии пластины Кирхгофа-Лява с плоским жестким включением\thanks{Работа выполнена при поддержке гранта РНФ, проект \textnumero~24-21-00081.}}
% Первый автор
\author{Н.~А.~Николаева
} % обязательное поле
\institute{Северо-Восточный федеральный университет им. М. К. Аммосова,
г. Якутск, Россия\\
  \email{niknataf@mail.ru}
  }
% Другие авторы...

\maketitle

\begin{abstract}
Исследуется задача о равновесии пластины, которая содержит плоское жесткое включение. Предполагается, что плоское жесткое включение выходит на границу под нулевым углом и частично контактирует с недеформируемым телом. Доказана теорема существования и единственности решения задачи. Приведены дифференциальная и вариационная формулировки задачи. 

\keywords{пластина, плоское жесткое включение, фиктивная
область}
\end{abstract}

%\section{Основные результаты} % не обязательное поле

%Текст доклада на русском языке.



% Список литературы.
\begin{thebibliography}{99}
\bibitem{1}
Khludnev A.\,M., Kovtunenko V.\,A., Analysis of cracks in solids.  Southampton New York: WIT-Press, 2000.
\bibitem{2}
% Format for Journal Reference:
Nikolaeva N. A., Kirchhoff--Love plate with a flat rigid inclusion. Chelyab. Fiz.-Mat. Zh. 2023. vol~8, No~1. Pp.~29-46.
\bibitem{3}
Nikolaeva N. A., Method of fictitious areas in a task about balance of a plate of Kirchhoff–Lyava. J. Math. Sci. 2017. vol.~221, No~6.Pp. 872–882.

\end{thebibliography}


% Обязательная информация на английском языке:




%\end{document}


