\begin{englishtitle}
\title{On singularities of caustics in spaces of low dimension}
% First author
\author{Vyacheslav D. Sedykh}
\institute{Gubkin University, Moscow, Russia\\
  \email{sedykh@mccme.ru}
}
\maketitle

\begin{abstract}
Caustic is the set of critical values of a Lagrangian map. A germ of a Lagrangian map is a germ of a sweep of a gradient mapping. By Arnold's theorem on Lagrangian singularities, simple stable Lagrangian germs are defined by versal deformations of germs of smooth functions at critical points of types $A,D,E$. Multisingularity of a Lagrangian map at a point of the target space is the unordered set of singularities of the mapping at the preimages of this point. We will talk about the adjacencies of multisingularities of a generic Lagrangian map into a space of dimension $n\leq5$. 

\keywords
{Lagrangian map, caustic, $ADE$ singularities, multisingularities, adjacency index}
\end{abstract}
\end{englishtitle}

\iffalse
\documentclass[12pt]{llncs}
\usepackage[T2A]{fontenc}
\usepackage[utf8]{inputenc}
\usepackage[english,russian]{babel}
\usepackage[russian]{nla}

%\usepackage[english,russian]{nla}

% \graphicspath{{pics/}} %Set the subfolder with figures (png, pdf).

%\usepackage{showframe}
\begin{document}
%\selectlanguage{russian}
\fi

\title{Об особенностях каустик в пространствах небольшой размерности}
% Первый автор
\author{В.~Д.~Седых 
} % обязательное поле
\institute{Губкинский университет, Москва, Россия\\
\email{sedykh@mccme.ru}
}

\maketitle

\begin{abstract}
Каустикой называется множество критических значений лагранжева отображения. Росток лагранжева отображения является ростком развертки градиентного отображения. По теореме Арнольда о лагранжевых особенностях простые устойчивые лагранжевы ростки определяются версальными деформациями ростков гладких функций в критических точках типов $A,D,E$. Мультиособенностью лагранжева отображения в точке пространства-образа называется неупорядоченный набор особенностей отображения в прообразах этой точки. Мы расскажем о примыканиях мультиособенностей лагранжева отображения общего положения в пространство размерности $n\leq5$. 

\keywords{
лагранжево отображение, каустика, особенности $ADE$, мультиособенности, индекс примыкания}
\end{abstract}


% Обязательная информация на английском языке:




%\end{document}


