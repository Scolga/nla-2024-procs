\begin{englishtitle} 
\title{On stability of a linear difference equation with complex coefficients}

\author{Ilya Aksenenko}
\institute{Perm National Research Polytechnic University, Perm, Russia \\
  \email{ilya156@list.ru}}


\maketitle

\begin{abstract}
The stability of a linear autonomous difference equation with two (generally speaking, complex) coefficients and various delays is investigated. To construct an exponential stability domain in the parameter space, the D-partition method is used for the above-mentioned difference equation: geometric stability criteria are found and exponential stability domains in the four-dimensional space of coefficients, as well as their three-dimensional sections, are described.

\keywords{difference equations, stability, D-decomposition} 
\end{abstract}
\end{englishtitle}

\iffalse

\documentclass[12pt]{llncs}  
\usepackage{iftex}
\ifPDFTeX
\usepackage[T2A]{fontenc}
\usepackage[utf8]{inputenc} 
\usepackage[english,russian]{babel}
\fi
\usepackage{todonotes} 
\usepackage[russian]{nla}

\renewcommand{\Re}{\mathop{\rm Re}\nolimits}
\renewcommand{\Im}{\mathop{\rm Im}\nolimits}


\begin{document}
\fi

\title{Об устойчивости линейного разностного уравнения с~комплексными коэффициентами\thanks{Работа выполнена при поддержке Министерства науки и высшего образования Российской Федерации, проект \textnumero~FSNM-2023-0005.}}
% Первый автор
\author{И.~А.~Аксененко}

% Аффилиации пишутся в следующей форме, соединяя каждый институт при помощи \and.
\institute{Пермский национальный исследовательский политехнический университет (ПНИПУ), Пермь, Россия \\
  \email{ilya156@list.ru}
}

\maketitle

\begin{abstract}
Исследуется устойчивость линейного автономного разностного уравнения с двумя комплексными коэффициентами.
С помощью метода D-разбиений описаны области экспоненциальной и равномерной устойчивости в четырехмерном действительном пространстве коэффициентов и их трехмерные сечения. 

\keywords{линейное разностное уравнение, устойчивость, D-разбиение, области устойчивости}
\end{abstract}

Целью работы является исследование устойчивости разностного уравнения вида
$$
x(n) + ax(n - 1) + bx(n - k) = 0,\quad n \in \mathbb{N},\eqno(1)
$$
для произвольных $k\in\mathbb{N}$ и $a,b \in \mathbb{C}$.

Очевидно, что в случае $k=1$ уравнение (1) экспоненциально устойчиво, если и только если $|a+b|<1$, и равномерно устойчиво, если и только если $|a+b| \leq 1$.
Поэтому ниже считаем, что $k>1$.

Сначала рассмотрим случай $a \ge 0$, $b \in \mathbb{C}$.
Положим $b=\alpha + i\beta$.
Поверхность D-разбиения~[1] пространства параметров уравнения (1) задается параметрическими уравнениями
$$
	\begin{cases}
		\alpha  = a\cos ((k-1)\varphi) - \cos (k\varphi),& \\
		\beta  = a\sin((k-1)\varphi) - \sin (k\varphi).& 
	\end{cases}
$$
где $\varphi \in [-\pi, \pi]$.
Эта поверхность разбивает пространство параметров на конечное число областей, среди которых следует выбрать область устойчивости.

Обозначим через ${C}_{k}$ открытую область, границы которой определяются изменением параметров $a$ и $\varphi$ в пределах $a \in [0,\frac{k}{k-1}]$, $\varphi \in [-\arccos {t_{0}},\arccos {t_{0}}]$, где ${t_0}$~--- первый положительный корень уравнения 
$$
a{U_{k - 2}}(t) = {U_{k - 1}}(t),
$$
a $U_k$ -- многочлены Чебышёва 2-го рода.

Через $[{C}_{k}]$ обозначим замыкание области ${C}_{k}$.

%Теперь сформулируем критерии экспоненциальной устойчивости и устойчивости по Ляпунову.

\textbf{Теорема 1.} \textit{Пусть $a \ge 0$ и $b \in \mathbb{C}$.
Уравнение~{\rm (1)} экспоненциально устойчиво, если и только если точка с координатами $\{\mathop{\rm Re}\nolimits b,\mathop{\rm Im}\nolimits b,a\}$ принадлежит области~${C}_{k}$.}

\textbf{Теорема 2.} \textit{Пусть $a \ge 0$ и $b \in \mathbb{C}$.
Уравнение~{\rm (1)} равномерно устойчиво, если и только если точка с координатами $\{\mathop{\rm Re}\nolimits b,\mathop{\rm Im}\nolimits b,a\}$ принадлежит множеству $$[C_{k}] \setminus \{ \frac{1}{k-1},0,\frac{k}{k-1}\}.$$}

Теперь рассмотрим общий случай $a,b\in\mathbb{C}$.
Обозначим $a =|a|{e^{i\omega }}$. % и сделаем замену $x(n) = {e^{i\omega n}}y(n)$.
Поверхность D-разбиения задается параметрическими уравнениями
$$
\begin{cases}
	\alpha  = |a|\cos ((k-1)\varphi + k\omega) - \cos (k\varphi + k\omega),& \\
	\beta  = |a|\sin((k-1)\varphi+ k\omega) - \sin (k\varphi + k\omega).& 
\end{cases}
$$
При фиксированном параметре $\omega$ эта поверхность разбивает пространство параметров на конечное число областей.
 Назовем $C_{k\omega}$ открытую область, ограниченную поверхностью D-разбиения при тех же изменениях параметров $a$ и $\varphi$, что и в случае $a\ge 0$. Легко видеть, что область $C_{k\omega}$ получается из $C_{k}$ поворотом на угол $k\omega$.

\textbf{Теорема 3.} \textit{Пусть $a,b \in \mathbb{C}$.
Уравнение~{\rm (1)} экспоненциально устойчиво, если и только если точка с координатами $\{\mathop{\rm Re}\nolimits b,\mathop{\rm Im}\nolimits b,|a|\}$ принадлежит области~${C}_{k\omega}$.}

\textbf{Теорема 4.} \textit{Пусть $a,b \in \mathbb{C}$.
Уравнение~{\rm (1)} равномерно устойчиво, если и только если точка с координатами $\{\mathop{\rm Re}\nolimits b,\mathop{\rm Im}\nolimits b,|a|\}$ принадлежит множеству $$[C_{k\omega}] \setminus \{ \frac{\cos k\omega}{k-1},\frac{\sin k\omega}{k-1},\frac{k}{k-1}\}.$$}

\begin{thebibliography}{9} 

\bibitem{Neymark} Неймарк Ю.И. Устойчивость линеаризованных систем (дискретных и распределенных). Л.: ЛКВВИА, 1949.% – 140 с. 

\end{thebibliography}
 


%\end{document}

