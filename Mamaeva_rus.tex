\begin{englishtitle} % Настраивает LaTeX на использование английского языка
% Этот титульный лист верстается аналогично.
\title{Analysis of natural vibrations of a liquid column in a vertical well}
% First author
\author{Z.Z. Mamaeva
}
\institute{Mavlyutov Institute of Mechanics UFRC RAS, Ufa, Russia\\
  \email{zilia16@mail.ru}
}
% etc

\maketitle

\begin{abstract}
The problem of natural oscillations of a liquid column in a tubing string placed in a vertical well, arising after a sudden opening or closing of the well, is considered. A mathematical model has been constructed and studied to describe the dynamics of fluid pressure in a vertical well and the bottom-hole zone in order to determine the natural frequencies of oscillations. The influence of various reservoir characteristics of the formation and hydraulic fracture on the main wave parameters of the natural oscillations of the liquid column is analyzed.

\keywords{oil well, formation, liquid column, natural vibrations, hydraulic fracture} % в конце списка точка не ставится
\end{abstract}
\end{englishtitle}

\iffalse
\documentclass[12pt]{llncs}

% При использовании pdfLaTeX добавляется стандартный набор русификации babel.

\usepackage{iftex}

\ifPDFTeX
\usepackage[T2A]{fontenc}
\usepackage[utf8]{inputenc} % Кодировка utf-8, cp1251 и т.д.
\usepackage[english,russian]{babel}
\fi

\usepackage{todonotes}

\usepackage[russian]{nla}


\begin{document}
\fi

\title{Анализ собственных колебаний столба жидкости в вертикальной скважине}
% Первый автор
\author{З.~З.~Мамаева}  % \inst ставит циферку над автором.
 

% Аффилиации пишутся в следующей форме, соединяя каждый институт при помощи \and.
\institute{ИМех УФИЦ РАН, Уфа, Россия\\
  \email{zilia16@mail.ru}.
}

\maketitle

\begin{abstract}
 Рассмотрена задача о собственных колебаниях столба жидкости в насосно-компрессорной колонне, помещенной в вертикальную скважину, возникающих после внезапного открытия или закрытия скважины. Построена и исследована математическая модель, описывающая динамику давления жидкости в вертикальной скважине и призабойной зоне с целью определения собственных частот колебаний. Проанализировано влияния различных коллекторских характеристик пласта и трещины ГРП на основные волновые параметры собственных колебаний столба жидкости.  
\keywords{нефтяная скважина, пласт, столб жидкости, собственные колебания, трещина гидравлического разрыва} % в конце списка точка не ставится
\end{abstract}

\section{Основные результаты} % не обязательное поле

В настоящее время добыча нефти является сложным и наукоемким процессом, который непрерывно модернизируется и развивается. Одной из актуальных проблем в нефтяной отрасли является снижение дебита большинства добывающих скважин и, как следствие, увеличение добычи трудноизвлекаемых запасов и необходимость проведения работ по обработки призабойной зоны с целью улучшения ее коллекторских характеристик, например, создание ГРП трещин. Данные процессы требуют исследования состояния пластов и получения информации о геометрии полученных трещин. 

В данной работе представлен один из возможных методов исследования пластов и трещин ГРП, основанный на возбуждении собственных колебаний столба жидкости в скважине и анализе волновых характеристик колебаний. Построена математическая модель собственных колебаний столба жидкости в скважине, возникших вследствие гидроудара  \cite{1,2}. Изучено влияние коллекторских характеристик пласта, наличия трещины на основные волновые характеристики колебаний. Показано, что в случае низкопроницаемых пластов (порядка миллидарси и меньше), после проведения ГРП может происходить двукратное снижение собственных частот колебаний. Проанализирована динамика колебаний давления на устье, в середине и на забое скважины.  


% Рисунки и таблицы оформляются по стандарту класса article. Например,


% кавычки << >>.

% В конце текста можно выразить благодарности, если этого не было
% сделано в ссылке с заголовка статьи, например,

%

\begin{thebibliography}{9} % или {99}, если ссылок больше десяти.
\bibitem{1} Шагапов В.Ш., Башмаков Р.А., Рафикова Г.Р., Мамаева З.З. Затухающие собственные колебания жидкости в скважине, сообщающейся с пластом // ПМТФ. 2020. Т.61, №4(362). С.5-14. 

\bibitem{2}	Ляпидевский В. Ю., Неверов В. В., Кривцов А. М. Математическая модель гидроудара в вертикальной скважине // Сиб. электрон.мат. изв. 2018. № 15. C. 1687–1696..


\end{thebibliography}

% После библиографического списка в русскоязычных статьях необходимо оформить
% англоязычный заголовок и аннотацию.




%\end{document}