\begin{englishtitle}
\title{Mixed problem for a system of first order differential equations}
% First author
\author{E.Yu. Grazhdantseva}
\institute{Irkutsk State  University, Institute of Mathematics and Information Technologies, Irkutsk, 664003 Russia;\\
Melentiev Energy Systems Institute Siberian Branch of the Russian Academy of Sciences, Irkutsk, 664033 Russia\\
  \email{grelyur@mail.ru}}
 % etc

\maketitle

\begin{abstract}
An exact solution to a mixed problem for a system of first-order linear differential equations is constructed.

\keywords{partial derivative, trigonometric series}
\end{abstract}

\end{englishtitle}

\iffalse
\documentclass[12pt]{llncs}
\usepackage[T2A]{fontenc}
\usepackage[utf8]{inputenc}
\usepackage[english,russian]{babel}
\usepackage[russian]{nla}

%\usepackage[english,russian]{nla}

% \graphicspath{{pics/}} %Set the subfolder with figures (png, pdf).

%\usepackage{showframe}
\begin{document}
%\selectlanguage{russian}
\fi

\title{Смешанная задача для системы дифференциальных уравнений первого порядка\thanks{Работа выполнена при поддержке гранта FWEU-2021-0006 программы фундаментальных исследований РФ на 2021-2030 гг.} 
 }
% Первый автор
\author{Е.~Ю.~Гражданцева }

\institute{ИГУ, Институт математики и информационных технологий, Иркутск, Россия;\\
Институт систем энергетики им. Л.\,А.\,Мелентьева СО РАН, Иркутск, Россия\\
  \email{grelyur@mail.ru}}

% Другие авторы...

\maketitle

\begin{abstract}
Построено точное решение смешанной задачи для системы линейных дифференциальных уравнений первого порядка. 

\keywords {частная производная, тригонометрический ряд}
\end{abstract}

\section{Основные результаты} % не обязательное поле

Рассматривается система дифференциальных уравнений  вида
\begin{equation} \label{1} 
\left\lbrace \begin{array}{cc}
\displaystyle\frac{\partial u}{\partial x}+ a \frac{\partial v}{\partial t}=0, \\
\\
\displaystyle\frac{\partial v}{\partial x}+ c \frac{\partial u}{\partial t}=0,
\end{array} \right.
\end{equation}
где 
$u=u(x,y),$
$v=v(x,y)$ ,   -- неизвестные функции независимых переменных 
$x$ 
 и 
 $t,$
 таких, что $x \in (0,l),$
 $l$
 и
 $t$
 -- действительные положительные числа,
 $a$
 и
 $c$
 -- действительные числа, причем
 $ac>0.$ 

Пусть функции
$u=u(x,t)$
и
$v=v(x,t)$
удовлетворяют условиям
\begin{equation}
u|_{x=0}=0 ,\,\,\,\,\, v|_{x=l} =0 
\end{equation}
\begin{equation}
u|_{t=0}=\varphi (x), \,\,\,\, v|_{t=0} =\psi (x).
\end{equation}
Данная система (1) является системой гиперболического типа, условия (2) -- краевыми условиями для неизвестной функции 
$u=u(x,t)$
данной системы, условия (3) -- начальными (Коши) условиями для обеих неизвестных функций. Таким образом, задачу (1)--(3) можно рассматривать как начально-краевую (смешанную) задачу.

Используя принцип построения решения дифференциального уравнения методом разделения переменных (методом Фурье) [1-4] для рассматриваемой задачи построено точное решение в виде тригонометрического ряда, а именно, решением задачи (1)-(3) можно считать функции 
$$u=u(x,y)=\sum\limits_{n=1}^{\infty} \left( \varphi_n cos\frac{\pi nt}{l\sqrt{ac}} + \sqrt{\frac{a}{c}} \psi_n sin \frac{\pi nt}{l\sqrt{ac}} \right) sin \frac{\pi nx}{l}, $$
$$v=v(x,t)= \sum\limits_{n=1}^{\infty} \left( \psi_n cos\frac{\pi nt}{l\sqrt{ac}} - \sqrt{\frac{c}{a}} \varphi_n sin \frac{\pi nt}{l\sqrt{ac}} \right) cos \frac{\pi nx}{l} $$
где
$\varphi_n,$ $n=1,2,...,$
являются коэффициентами Фурье разложения функции
$\varphi (x)$
в ряд Фурье по синусам, а 
$\psi_n$
 -- коэффициентами Фурье разложения функции   
$\psi(x)$ 
 в ряд Фурье по косинусам при
$x \in (0,l),$
т.е.
$$\varphi_n=\frac{2}{l}\int\limits_{0}^{l} \varphi (x) sin \frac{\pi nx}{l} dx, \,\,\,\,\, \psi_n=\frac{2}{l}\int\limits_{0}^{l} \psi (x) cos \frac{\pi nx}{l} dx. $$					

% Список литературы.
\begin{thebibliography}{99}
\bibitem{1}
% Format for books
Владимиров~В.С. Уравнения математической физики. М.:~Наука,~1988.

\bibitem{2}
Курант~Р. Уравнения с частными производными.М.:~Мир,~1964.

\bibitem{3}
Михлин~С.Г. Курс математической физики.М.:~Наука,~1968.

\bibitem{4}
Тихонов~А.Н., Самарский~А.А. Уравнения математической физики.М:~Наука,~1999.

\end{thebibliography}





%\end{document}

%%% Local Variables:
%%% mode: latex
%%% TeX-master: t
%%% End:
