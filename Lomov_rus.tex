\begin{englishtitle} % Настраивает LaTeX на использование английского языка
% Этот титульный лист верстается аналогично.
\title{On the stability of solutions to the variational \\ Prony problem\thanks{The study was carried out within the framework of the state contract of the Sobolev Institute of Mathematics (project no. FWNF-2022-0008)}}
% First author
\author{Andrei Lomov
}
\institute{Sobolev Institute of Mathematics, Novosibirsk State University, \\ Novosibirsk, Russia, \email{lomov@math.nsc.ru}
}
% etc

\maketitle

\vspace{-1ex}
\begin{abstract}
Using the Mean value theorem, new
stability constants of solutions to the variational Prony problem are obtained.

\keywords{the variational Prony problem, local stability, guaranteed estimates} % в конце списка точка не ставится
\end{abstract}
\end{englishtitle}

\iffalse
%%%%%%%%%%%%%%%%%%%%%%%%%%%%%%%%%%%%%%%%%%%%%%%%%%%%%%%%%%%%%%%%%%%%%%%%
%
%  This is the template file for the 6th International conference
%  NONLINEAR ANALYSIS AND EXTREMAL PROBLEMS
%  June 25-30, 2018
%  Irkutsk, Russia
%
%%%%%%%%%%%%%%%%%%%%%%%%%%%%%%%%%%%%%%%%%%%%%%%%%%%%%%%%%%%%%%%%%%%%%%%%


\documentclass[12pt]{llncs}  

% При использовании pdfLaTeX добавляется стандартный набор русификации babel.

\usepackage{iftex}

\ifPDFTeX
\usepackage[T2A]{fontenc}
\usepackage[utf8]{inputenc} % Кодировка utf-8, cp1251 и т.д.
\usepackage[english,russian]{babel}
\fi

\usepackage{todonotes} 

\usepackage[russian]{nla}

\begin{document}
\fi
\title{Об устойчивости решений вариационной задачи Прони\thanks{Работа выполнена в рамках государственного задания Института математики им. С.Л. Соболева СО РАН, проект \textnumero~FWNF-2022-0008}}
% Первый автор
\author{А.~А.~Ломов
}

% Аффилиации пишутся в следующей форме, соединяя каждый институт при помощи \and.
\institute{ИМ СО РАН им. С.Л. Соболева, Новосибирский государственный университет, \\ Новосибирск, Россия,
  \email{lomov@math.nsc.ru}
}

\maketitle

\begin{abstract}
По теореме об оценке конечных приращений получены новые константы
устойчивости решений вариационной задачи Прони.

\keywords{вариационная задача Прони, локальная устойчивость, гарантированные оценки} % в конце списка точка не ставится
\end{abstract}

\newcommand{\T}{\top}

\section{Основные результаты} % не обязательное поле

Вариационной задачей Прони называем обратную задачу идентификации
вектора параметров $\theta\in\mathbb{R}^{v}$ разностного уравнения
\begin{equation}
\alpha_{n}x_{k+n}+\alpha_{n-1}x_{k+n-1}+\ldots+\alpha_{0}x_{k}=0,\quad k=\overline{1,N-n},\label{SL/eq:RU}
\end{equation}
с матричными коэффициентами $\alpha_{i}=\alpha_{i}(\theta)\in\mathbb{R}^{r\times\left(r+m\right)}$,
$i=\overline{0,n}$ по возмущенным наблюдениям
$y=x+\delta x\in\mathbb{R}^{N\left(r+m\right)}$
процесса $x\doteq\left[x_{1};\ldots;x_{N}\right]$. Предполагается,
что $\alpha=D\theta+d$, где $\alpha\doteq\mathrm{vect}\left[\alpha_{0};\ldots;\alpha_{n}\right]$,
столбцы $\left[D,d\right]$ линейно независимы, и задача вычисления $\theta$ по $x$ однозначно разрешима.

Устойчивость решений
$\hat{\theta}(x+\delta x)=\theta+\delta\theta$
к возмущениям зависит от целевой функции; при аддитивных возмущениях предпочтительна \cite{Lomov--Rusinova-2022} вариационная ЦФ:
\[
\hat{\theta}=\arg\min_{\theta}J(\theta,y),\quad J(\theta,y)\doteq\left\Vert y-\hat{x}(\theta)\right\Vert ^{2},\quad\hat{x}(\theta)\doteq\left(I-GCG^{\T}\right)y,\quad C\doteq\left(G^{\T}G\right)^{-1},
\]
где $G^{\T}\doteq\left\backslash \alpha(\theta)\right\backslash $ есть клеточно-теплицевая
матрица системы уравнений \eqref{SL/eq:RU}.

Нас интересуют гарантированные границы для нормы отклонений
$\left\Vert \delta\theta\right\Vert $,
в отличие от анализа с точностью до малых более высоких порядков \cite{Abatzoglu-et-al-1991}.

\smallskip
\textbf{Теорема (об оценке конечных приращений)}
\textsl{
Пусть отображение $F:\Omega\to\mathbb{R}^{v}$ непрерывно
дифференцируемо в выпуклой компактной области пространства
$\mathbb{R}^{N\left(r+m\right)}$.
Тогда $$\left\Vert F(y)-F(x)\right\Vert \leq\sup_{\xi\in\Omega}\left\Vert F^{\prime}(\xi)\right\Vert \cdot\left\Vert y-x\right\Vert .$$
}\smallskip


Для неявной функции $\hat{\theta}(y)\doteq F(y)$, вычисляемой из
условия $J_{\theta}^{\prime}(\hat{\theta},y)=0$, имеем производную Фреше
$F^{\prime}=-\left(J_{\theta\theta}^{\prime\prime}\right)^{-1}J_{\theta y}^{\prime\prime}\doteq S^{\T}$.
Пусть $\|\cdot\|$ — операторная норма, тогда
\[
\|F^{\prime}(y)\|=\|S(\hat{\theta}(y),y)\|=\lambda_{\text{max}}^{1/2}\left(S^{\T}S\right).
\]

\smallskip
\textbf{Лемма.}\textsl{
При возмущении $R\to R+\Delta R\in\mathbb{R}^{n}$ симметричной положительно определенной матрицы $R$ верна гарантированная оценка для наименьшего
собственного числа $\lambda_{i}\doteq\lambda_{i}\left(R\right)$,
$\lambda_{1}<\lambda_{2}\leq\ldots\leq\lambda_{n}$
($\left\Vert \cdot \right\Vert_{F} $ --- фробениусовская норма):
\[
\lambda_{1}+\Delta\lambda_{1}\in\left[\lambda_{1}\pm\left(\left\Vert \Delta R\right\Vert _{F}+\frac{2n}{\left|\lambda_{2}-\lambda_{1}\right|}\left\Vert \Delta R\right\Vert _{F}^{2}\right)\right].
\]
}\smallskip

\textbf{Предложение.}\textsl{
В вариационной задаче Прони
$S^{\T}S=\left(D^{\T}\hat{V}^{\T}C\hat{V}D\right)^{-1}\doteq R^{-1}$,
где $\hat{V}$ есть клеточно-ганкелевая матрица из отсчетов процесса
$\hat{x}$: $G^{\T}\hat{x}\equiv\hat{V}\left(D\theta+d\right).$ Тогда
$$\lambda_{\mathrm{max}}^{1/2}\left(S^{\T}S\right)=\lambda_{\mathrm{min}}^{-1/2}\left(R\right).$$
}\smallskip

Как следствие теоремы об оценке конечных приращений имеет место неравенство
\(
\|\delta\theta\| \leq\frac{\varepsilon}{\inf_{\|\delta x\|\leq\varepsilon,\;\left\Vert \delta\theta\right\Vert \leq\mu\varepsilon\lambda_{\mathrm{min}}^{-1/2}\left(R\right)}\,\lambda_{\mathrm{min}}^{1/2}\left(R\right)},
\)
где $\mu\geq1$ — априорный коэффициент.

\smallskip
\textbf{Теорема.}\textsl{
В вариационной задаче Прони верна следующая гарантированная оценка для ошибки вектора параметров: $\left\Vert \delta\theta\right\Vert \leq\mu\frac{\varepsilon}{\sqrt{\lambda_{1}}},\quad\mu\geq1,$
$\lambda_{1}\doteq\lambda_{\mathrm{min}}\left(R\right)$,
при техническом условии
$\varepsilon\leq\frac{\left(\sqrt{2}-1\right)}{5n\left\Vert D\right\Vert ^{2}\left\Vert V\right\Vert ^{2}\left\Vert C\right\Vert ^{3/2}}\cdot\frac{\left(1-\mu^{-2}\right)\lambda_{1}^{3/2}}{\left(\mu+\frac{3\sqrt{n+1}}{5n\left\Vert x\right\Vert \left\Vert C\right\Vert ^{1/2}}\sqrt{\lambda_{1}}\right)}.$
}\smallskip

Проведены расчеты для границ нормы ошибки
$\left\Vert \delta\theta\right\Vert $
в задаче Ланцоша восстановления экспоненциальных слагаемых из возмущенных наблюдений суммы трех экспонент \cite{Lanczos-1956} и на ряде более обусловленных примеров. Показано, что ранее полученные границы нормы
ошибки $\left\Vert \delta\theta\right\Vert $ (ссылки в \cite{Lomov--Rusinova-2022,Lomov-2024-MTrudy}),
основанные на оценках остаточного члена второго порядка в разложении в ряд Тейлора неявной функции $\hat{\theta}(y)$ (по сути на оценках конечных приращений производной Фреше), на несколько порядков менее эффективны, как и оценки на основе неравенств типа Уилкинсона.

% В конце текста можно выразить благодарности, если этого не было
% сделано в ссылке с заголовка статьи, например,
% Работа выполнена при поддержке РФФИ (РНФ, другие фонды), проект \textnumero~00-00-00000.
%



\begin{thebibliography}{9} % или {99}, если ссылок больше десяти.
\bibitem{Lomov--Rusinova-2022}\emph{Ломов\,А.\,А., Русинова\,Е.\,А.}
Сравнение целевых функций в задаче Прони для аппроксимации данных
измерений // Вестник ЮУрГУ. Серия «Вычислительная математика и информатика».
2022. Т. 11, \textnumero 2. С. 18--27. DOI: 10.14529/cmse220202

\bibitem{Abatzoglu-et-al-1991}\emph{Abatzoglu\,T.J., Mendel\,J.M.,
Harada\,G.A.} The Constrained Total Least Squares Technique and its
Applications to Harmonic Superresolution. IEEE Trans. Signal Processing.
1991. Vol.\,SP-39. Pp.\emph{\,}1070--1087.

\bibitem{Lanczos-1956}\emph{Lanczos\,C}. Applied analysis. Prentice
Hall, Englewood Cliffs, N.J., 1956. 539\,p. Пер.: \emph{Ланцош\,К}.
Практические методы прикладного анализа. М.: ФМЛ, 1961. 524\,с.

\bibitem{Lomov-2024-MTrudy}\emph{Ломов\,А.А}. О локальной устойчивости
в полной задаче Прони // Математические труды. \textnumero 1. 2024. Принято к
публикации.

\bibitem{VanHuffel--Vandewalle-1991}\emph{Van\,Huffel\,S., Vandewalle\,J}.
The total least squares problem. SIAM, Philadelphia, 1991.

\end{thebibliography}

% После библиографического списка в русскоязычных статьях необходимо оформить
% англоязычный заголовок.






%\end{document}

