
\iffalse
%%%%%%%%%%%%%%%%%%%%%%%%%%%%%%%%%%%%%%%%%%%%%%%%%%%%%%%%%%%%%%%%%%%%%%%%
%
% This is the template file for the 6th International conference
% NONLINEAR ANALYSIS AND EXTREMAL PROBLEMS
% June 25-30, 2018
% Irkutsk, Russia
%
%%%%%%%%%%%%%%%%%%%%%%%%%%%%%%%%%%%%%%%%%%%%%%%%%%%%%%%%%%%%%%%%%%%%%%%%
% The preparation of the article is based on the standard llncs class
% (Lecture Notes in Computer Sciences), which is adjusted with style
% file of the conference.
%
% There are two ways of compilation of the file into PDF
% 1. Use pdfLaTeX (pdflatex), (LaTeX+DVIPS will not work);
% 2. Use LuaLaTeX (XeLaTeX will work too).
% When using LuaLaTeX You will need TTF or OTF CMU fonts
% (Computer Modern Unicode). The fonts are installed with 'cm-unicode' package in
% a distribution of LaTeX % (https://www.ctan.org/tex-archive/fonts/cm-unicode),
% either by downloading and installing these fonts system wide, the address of their page is
% http://canopus.iacp.dvo.ru/%7Epanov/cm-unicode/
% The second option won't work in XeLaTeX.
%
% For MiKTeX (LaTeX distribution for Windows),
%  1. Package 'cm-unicode' is installed manually with the MiKTeX administration Console.
%  2. For the compilation of this example, namely, the stub figure, one will also need to
% download package 'pgf' manually. This package uses in the popular
% package tikz.
%  3. Tests showed that the rest of the required packages MiKTeX loads automatically (if
%     it is allowed). The 'auto download' option is
%     configured in 'Settings' section in MiKTeX Console.
%
%
% The easiest way to compile an article is to use pdfLaTeX, but
% the final layout of the book will be compiled with LuaLaTeX,
% as a result will be of better quality thanks to the package 'microtype' and
% use vector OTF instead of standard raster fonts of pdfLaTeX.
%
% In the case of questions and problems with the article compilation,
% write letters to e-mail: eugeneai@irnok.net, Cherkashin Evgeny.
%
% New version of the correcting style file will be available at the website:
%     https://github.com/eugeneai/nla-style
%     file - nla.sty
%
% Further instructions are in the text body of the template. The template itself
% is an article example.
%
% The LaTeX2e format is used!

% 12 points font size is used.
\documentclass[12pt]{llncs}

% The correcting style file is added.
\usepackage{todonotes}

\usepackage{nla} % This package is needed for compiling
                 % this template, it should be removed
                 % from your article.

% Many popular packages (amsXXX, graphicx, etc.) are already imported in the style file.
% If there is a conflict with your packages, try disabling them and compile
% the text.
%
% It would be convenient in the layout of the proceedings if the file names
% of the figures of different authors do not clash.
% To minimize the clash, the drawings can be placed in a separate subfolder
% named after the author or the title of the paper.
%
% \graphicspath{{ivanov-petrov-pics/}} % specifies the folder with images in png, pdf formats.
% or
% \graphicspath{{great-problem-solving-paper-pics/}}.

\begin{document}
\fi
% Text should be formatted in accordance with the 'article' class, using extensions like
% AMS.
%
\title{Abstract convexity and subdifferentiability of functions with respect to Lipschitz concave functions\thanks{The work was supported by the National Program for Scientific Research of the Republic of Belarus for 2021-2025  Convergence--2025, project No.~1.3.01.}}
% First author
\author{Valentin V. Gorokhovik 
}
\institute{Institute of Mathematics, National Academy of Sciences of Belarus, Minsk, Belarus\\
  \email{gorokh@im.bas-net.by}}

\maketitle

\begin{abstract}
For extended real-valued functions defined on a set $X$ Rubinov A.M. (sse \cite{Rub2000}) have introduced the notion of abstract convexity with respect to the set $H$ of functions from $R^X$. In the talk we realize this notion for the case when $X$ is a real normed space and $H$ is the set ${\mathcal{L}\widehat{C}}:={\mathcal{L}\widehat{C}}(X,{\mathbb{R}})$ of Lipschitz continuous classically concave functions.

\keywords{abstract convex function, subgradient, subdifferential}
\end{abstract}

% at the end of the list, there should be no final dot

\section{Abstract ${{\mathcal L}\widehat{C}}$-convexity and ${{\mathcal L}\widehat{C}}$-subdifferentiability.} 

Let $X$ be a real normed space and let ${\mathcal{L}\widehat{C}}:={\mathcal{L}\widehat{C}}(X,{\mathbb{R}})$ be the set of real-valued Lipschitz continuous classically concave functions defined on $X$.

A function $f: X \mapsto \overline{\mathbb R}$ is called \cite{GorTyk1,GorTyk2} \textit{abstract ${{\mathcal L}\widehat{C}}$-convex} if $f(x) =\sup\{h(x) \mid h \in S^-({{\mathcal L}\widehat{C}},f)\}$, where
$S^-({{\mathcal L}\widehat{C}},f):=\{h \in {{\mathcal L}\widehat{C}} \mid h(x) \le f(x)\,\,\forall\,\,x \in {\rm dom}f\}  \neq \emptyset$ is \textit{the lower $\mathcal{{{\mathcal L}\widehat{C}}}$-support set} of $f$, while a functions $h$ from $S^-({{\mathcal L}\widehat{C}},f)$ are called \textit{${{\mathcal L}\widehat{C}}$-minorants} of $f.$


\begin{theorem}
An arbitrary function $f: X \mapsto \overline{\mathbb{R}}$ is abstract ${{\mathcal L}\widehat{C}}$-convex if and only if it is lower semicontinuous and bounded from below by a Lipschitz continuous function.
\end{theorem}

An ${{\mathcal L}\widehat{C}}$-minorant $h \in S^-({{\mathcal L}\widehat{C}},f)$ is said \cite{GorTyk1} to be \textit{support from below to the function $f$ at a point $x \in {\rm dom}f$} if $f(x) = h(x)$. The set of all minorants $h \in S^-({{\mathcal L}\widehat{C}},f)$ that are support from below to the function $f$ at a point $x \in {\rm dom}f$ we denote by $S^-({{\mathcal L}\widehat{C}},f,x)$.

We say \cite{GorTyk2} that a function $f: X \mapsto \overline{\mathbb{R}}$ is ${{\mathcal L}\widehat{C}}$-\textit{subdifferentiable at a point} $x \in {\rm dom}f$ if $S^-({{\mathcal L}\widehat{C}},f,x) \neq \emptyset$.

Any function $f$ which is Lipschitz continuous on $X$ is ${{\mathcal L}\widehat{C}}$-subdifferentiable at any point $x \in X$.

For abstract ${{\mathcal L}\widehat{C}}$-convex functions the following counterpart of the classical Br{\o}ndsted-Rockafellar theorem holds.

\begin{theorem}
Let $X$ be a complete normed vector space.Then for any ${{\mathcal L}\widehat{C}}$-convex function $f:X \mapsto \overline{\mathbb{R}}$ the set of points at which $f$ is ${\mathcal{L}\widehat{C}}$-subdifferentiable  is dense in ${\rm dom}\,f.$
\end{theorem}

\section{${{\mathcal L}\widehat{C}}_\theta$-subgradients and ${{\mathcal L}\widehat{C}}_\theta$-subdifferentials}
 
 By ${{\mathcal L}\widehat{C}}_\theta:={{\mathcal L}\widehat{C}}_\theta(X,{\mathbb{R}})$ we denote the subset of ${{\mathcal L}\widehat{C}}$ consisting of such functions $l \in  {{\mathcal L}\widehat{C}}$  for which the equality  $l(\theta) = 0$ holds.

A function $l \in {{\mathcal L}\widehat{C}}_\theta$ is called \cite{Gor} \textit{an ${{\mathcal L}\widehat{C}}_\theta$-subgradient of a function $f:X \mapsto \overline{\mathbb{R}}$ at a point $\bar{x} \in {\rm dom}\,f$} if
\begin{equation*}
f(x) - f(\bar{x}) \ge l(x - \bar{x})\,\,\text{for all}\,\,x \in X.
\end{equation*}
The set of all ${{\mathcal L}\widehat{C}}_\theta$-subgradients of a function $f:X \mapsto \overline{\mathbb{R}}$ at a point $\bar{x} \in {\rm dom}\,f$,  denoted by $D_{{{\mathcal L}\widehat{C}}_\theta}f(\bar{x})$, is called \textit{the full ${{{\mathcal L}\widehat{C}}_\theta}$-subdifferential of $f$ at $\bar{x}$}.

\vspace{3pt}

An ${{\mathcal L}\widehat{C}}_\theta$-subgradient $l \in D_{{{\mathcal L}\widehat{C}}_\theta}f(\bar{x})$ that is a maximal (with respect to the pointwise ordering) element of $D_{{{\mathcal L}\widehat{C}}_\theta}f(\bar{x})$ is called \textit{a maximal ${{\mathcal L}\widehat{C}}_\theta$-subgradient of $f$ at ${\bar{x}}$}.
The set of all maximal ${{\mathcal L}\widehat{C}}_\theta$-subgradients of a function $f:X \mapsto \overline{\mathbb{R}}$ at a point $\bar{x} \in {\rm dom}\,f$,  denoted by $\partial_{{{\mathcal L}\widehat{C}}_\theta}f(\bar{x})$, is called \textit{the thin ${{{\mathcal L}\widehat{C}}_\theta}$-subdifferential of $f$ at $\bar{x}.$}

\begin{theorem}
A function $l \in {{\mathcal L}\widehat{C}}_\theta(X,{\mathbb{R}})$ is an ${{\mathcal L}\widehat{C}}_\theta$-subgradient of a function $f:X \mapsto \overline{\mathbb{R}}$ at a point $\bar{x} \in {\rm dom}\,f$ if and only if the function $h: x \mapsto l(x-\bar{x}) + f(\bar{x})$ belongs to $S^-({\mathcal{L}\widehat{C}},\,f,\,\bar{x})$, while  a function $l \in {{\mathcal L}\widehat{C}}_\theta(X,{\mathbb{R}})$ is  \textit{a maximal ${{\mathcal L}\widehat{C}}_\theta$-subgradient of a function $f:X \mapsto \overline{\mathbb{R}}$ at a point $\bar{x} \in {\rm dom}\,f$} if and only if  the function $h: x \mapsto l(x-\bar{x}) + f(\bar{x})$ is maximal (with respect to the pointwise ordering) in $S^-({\mathcal{L}\widehat{C}},\,f,\,\bar{x})$.
\end{theorem}

We show that for conventional convex functions the notion of the thin subdifferential coincides with the classical notion of the Fenchel--Rockafellar subdifferential.

The relationship with other notions of subdifferentials also is discussed in the talk.

As an application we derive the next criterium of global minimum for ${\mathcal{L}}\widehat{C}$-subdifferential functions.

\begin{theorem}[\rm{global extremum criteria}]
 If a function $f:X \mapsto \overline{\mathbb{R}}$ is ${{{\mathcal L}{\widehat{C}}}}$-subdifferentiable at a point $\bar{x} \in {\rm dom}\,f$,
then  $f$ achieves its global minimum over $X$ at the point $\bar{x} \in {\rm dom}\,f$ if and only if
$$
 0_{X^*} \in \partial_{{{\mathcal L}{\widehat{C}}_\theta}}f(\bar{x}).
$$
Here $X^*$ is the vector space of continuous linear functions defined on $X$, $\partial_{{{\mathcal L}{\widehat{C}}_\theta}}f(\bar{x})$ stands for the thin ${{{\mathcal L}{\widehat{C}}_\theta}}$-subdifferential of $f$ at $\bar{x}$, and $0_{X^*}$ is the null linear function defined on $X$ $($the null element of $X^*$).
\end{theorem}

Symmetric notions of abstract ${\mathcal{L}\check{C}}$-concavity and ${\mathcal{L}\check{C}}$-superdifferentiability of functions with respect to the set ${\mathcal{L}\check{C}}:={\mathcal{L}\check{C}}(X,{\mathbb{R}})$ of Lipschitz continuous and convex in classical sense functions are considered as well.

Some calculus rules for full and thin ${{{\mathcal L}\widehat{C}}_\theta}$-subdifferentials are established.

%

\begin{thebibliography}{9} % or {99}, if there is more than ten references.

\bibitem{Rub2000}
Rubinov~A.M. Abstract convexity and global optimization. Kluwer Academic Publishers, Dordrecht, 2000.

\bibitem{GorTyk1}
Gorokhovik~V.V., Tykoun~A.S. Support points of lower semicontinuous functions
with respect to the set of Lipschitz concave functions. Doklady of the National Academy of Sciences of Belarus. 2019. Vol.~63. No.~6. Pp.~647-653.

\bibitem{GorTyk2}
Gorokhovik~V.V., Tykoun~A.S. Abstract convexity of functions with respect to the set of Lipschitz (concave) functions. Proceedings of the Steklov Institute of Mathematics.~2020. Vol.~309. no.~Suppl.~1. Pp.~S36-S46.

\bibitem{Gor} Gorokhovik V.V. Regularly abstract convex functions with respect to the set of Lipschitz continuous concave functions. Optimization.~2023. Vol.~72, no.1. Pp.~241--261.

\end{thebibliography}
%\end{document}
