
\iffalse
\documentclass[12pt]{llncs}

\usepackage{todonotes}

\usepackage{nla} 

\begin{document}
\fi

\title{Boundary value problems for a class of pseudohyperbolic equations\thanks{The work is supported by the Mathematical Center in Akademgorodok under
agreement No. 075-15-2022-282 with the Ministry of Science and Higher
Education of the Russian Federation.}}

\author{V.V. Shemetova
}
\institute{Novosibirsk State University,  Novosibirsk, Russia\\
  \email{valentina501@mail.ru}
 }

\maketitle

\begin{abstract}
In this paper is considered one class of equations that are unresolved with respect to the highest order derivative. Initial boundary value problems in the quarter plane are researched. The study of unique solvability is carried out in an anisotropic Sobolev space with exponential weight.

\keywords{pseudohyperbolic equation, anisotropic Sobolev space, initial boundary value problem}
\end{abstract}

This work is dedicated to the investigation of the initial boundary value problem
$$
\begin{array}{l} 
(I - D_x^2) D_t^2 u + D_x^4 u - a^2D_x^2 u = f(t, x), \ \ x>0, \ \ t>0,  \\
(b_{11}u+b_{12}D_xu+b_{13}D^2_xu+b_{14}D^3_xu)\bigl|_{x=0} =0, \\
(b_{21}u+b_{22}D_xu+b_{23}D^2_xu+b_{24}D^3_xu)\bigl|_{x=0} =0,\\
u\bigl|_{t=0} = 0, \ \ 
D_t u\bigl|_{t=0}= 0,
\end{array}
\eqno (1)
$$
where $I$ is the identity operator, $a\in\mathbb{R}$ and $b_{i,j} \in \mathbb{R}$, $i,j = 1,2$. The differential equation in (1) is unresolved with respect to the highest order derivative. Such equations are usually called Sobolev-type equations after Sobolev's pioneering works [1]. In the monograph [2], a classification of such equations is introduced. This equation belongs to the class of pseudohyperbolic equations. Equation in (1) arises in modeling torsional [3] or longitudinal [4] vibrations of elastic rods. 

Let us introduce the notation
$$
\begin{array}{ll}
A_1=b_{11}b_{22}-b_{12}b_{21}, \ \ \ &
A_2=b_{11}b_{23}-b_{13}b_{21},\\
A_3=b_{12}b_{23}-b_{13}b_{22},  \ \ \ &
A_4=b_{11}b_{24}-b_{14}b_{21},\\
A_5=b_{12}b_{24}-b_{14}b_{22},  \ \ \  &
A_6=b_{13}b_{24}-b_{14}b_{23}.\\
\end{array}
\eqno(2)
$$

We present a result on the unique solvability of the given problem in Sobolev spaces with an exponential weight. It should be noted that a class of initial and boundary value problems has been identified, for which the conditions for the right-hand side are the same as those of the Cauchy problem [5]. 

{\bf Theorem 1.} 
Let the Lopatinskii condition hold for boundary value problem (1) and 
$$
A_4-A_5+A_6\ne 0,
$$
then there exists a positive constant $\gamma_0>0$ such that, for every
$$
f(t,x)\in W_{2,\gamma}^{1,0}(\mathbb{R}^2_{++}), \ \ f(t,x)\bigr|_{t=0}=0, \ \ \gamma>\gamma_0,
$$
boundary value problem (1)  admits a unique solution in the space $W_{2,\gamma}^{2,4}(\mathbb{R}^2_{++})$, $\gamma>\gamma_0$, such that $D_t^2 D_x^2 u(t,x)\in L_{2,\gamma}(\mathbb{R}^2_{++})$ and the estimate
$$
\|D_t^2 D_x^2 u(t, x), L_{2,\gamma}(\mathbb{R}^2_{++})\|+\|u(t, x), W_{2,\gamma}^{2,4}(\mathbb{R}^2_{++})\|\le c(\gamma_0) \|f(t,x),  W_{2,\gamma}^{1,0}(\mathbb{R}^2_{++})\|,
$$
is valid where the constant $c(\gamma_0)$  does not depend on the function $f(t,x)$.

Moreover, it was possible to denote a class of initial boundary value problems that require stricter requirements for the existence of derivatives of a function $f(t,x)$ than the Cauchy problem.

{\bf Theorem 2.} 
Let the Lopatinskii condition hold for boundary value problem (1) and 
$$
A_4-A_5+A_6= 0, \ \  A_2-A_3+A_6\ne 0, \ \ \mbox{ or } \ \
A_4=A_5\ne 0, \ \ A_6=0,
$$
then there exists a positive constant $\gamma_0>0$ such that, for every
$$
f(t, x)\in W_{2,\gamma}^{2,0}(\mathbb{R}^2_{++}), \ \ \gamma>\gamma_0, \ \
f(t, x)\bigr|_{t=0}=D_t f(t, x)\bigr|_{t=0}=0,
$$
boundary value problem (1)  admits a unique solution in the space $W_{2,\gamma}^{2,4}(\mathbb{R}^2_{++})$, $\gamma>\gamma_0$, such that $D_t^2 D_x^2 u(t,x)\in L_{2,\gamma}(\mathbb{R}^2_{++})$ and the estimate
$$
\|D_t^2 D_x^2 u(t, x), L_{2,\gamma}(\mathbb{R}^2_{++})\|+\|u(t, x), W_{2,\gamma}^{2,4}(\mathbb{R}^2_{++})\|\le c(\gamma_0) \|f(t,x),  W_{2,\gamma}^{2,0}(\mathbb{R}^2_{++})\|,
$$
is valid where the constant $c(\gamma_0)$  does not depend on the function $f(t,x)$.

{\bf Theorem 3.} 
Let the Lopatinskii condition hold for boundary value problem (1) and 
$$
A_4-A_5+A_6= 0, \ \  A_2-A_3+A_6= 0, \ \ A_6\ne 0,
$$
then there exists a positive constant $\gamma_0>0$ such that, for every
$$
f(t, x)\in W_{2,\gamma}^{3,0}(\mathbb{R}^2_{++}), \ \ \gamma>\gamma_0, \ \
f(t, x)\bigr|_{t=0}=D_t f(t, x)\bigr|_{t=0}=D^2_t f(t, x)\bigr|_{t=0}=0,
$$
boundary value problem (1)  admits a unique solution in the space $W_{2,\gamma}^{2,4}(\mathbb{R}^2_{++})$, $\gamma>\gamma_0$, such that $D_t^2 D_x^2 u(t,x)\in L_{2,\gamma}(\mathbb{R}^2_{++})$ and the estimate
$$
\|D_t^2 D_x^2 u(t, x), L_{2,\gamma}(\mathbb{R}^2_{++})\|+\|u(t, x), W_{2,\gamma}^{2,4}(\mathbb{R}^2_{++})\|\le c(\gamma_0) \|f(t,x),  W_{2,\gamma}^{3,0}(\mathbb{R}^2_{++})\|,
$$
is valid where the constant $c(\gamma_0)$  does not depend on the function $f(t,x)$.

%Conditions for unambiguous solvability of the specified problem are formulated and conditions are obtained for the right hand side of the equation, under which boundary value problems in Sobolev spaces have a unique solution with exponential weights. The classes of initial-boundary value problems are highlighted for which requirements on the right-hand side coincide with those for the Cauchy problem and the requirements are stricter than for the Cauchy case.



\begin{thebibliography}{9} % or {99}, if there is more than ten references.

\bibitem{SLSobolev2003_ShV} Sobolev S.L. 
 Selected Works of S. L. Sobolev. Vol. I: Mathematical Physics, Computational Mathematics, and Cubature Formulas, Sobolev Inst. Math., Novosibirsk,  2003. [in Russian]

\bibitem{GVDemidenko1998_ShV}
Demidenko G.V., Uspenskii S.V. Partial Differential Equations and Systems Not Solvable with Respect to the Highest-Order Derivative. Nauchnaya Kniga, Novosibirsk, 1998. [in Russian]

\bibitem{VZVlasov1940_Sh} Vlasov V.Z. Thin-Walled Elastic Beams. Stroyizdat, Moscow-Leningrad,  1940. [in Russian]

\bibitem{REBishop1952_Sh} Bishop R.E.\,D. Longitudinal waves in beams. 
Aeronautical Quarterly. 1952. Vol.~3,  No.~4, Pp.~280--293 

\bibitem{GVDemidenko2001_ShV} Demidenko G.V. On solvability of the Cauchy problem for pseudohyperbolic equations.
Sib. Adv. Math. 2001. Vol.~11, No~4. Pp.~25--40.





\end{thebibliography}
%\end{document}
