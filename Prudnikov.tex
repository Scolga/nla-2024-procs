
\iffalse
\documentclass[12pt]{llncs}

\usepackage{todonotes}

\usepackage{nla}

\begin{document}
\fi

% Text should be formatted in accordance with the 'article' class, using extensions like
% AMS.
%
\title{The star sets and solution of DC problem}
% First author
\author{Igor Prudnikov 
}
\institute{Scientific Center of Smolensk State Medical University, Smolensk, Russia\\
  \email{prudnik09@yandex.ru}}


\maketitle

\begin{abstract}
%Insert your english abstract here. Include 3-6 keywords below.

The author considers star sets on the plane and studies their
connection with the formulated problem of representing a Lipschitz
function of two variables as a difference of two convex functions.
A geometric interpretation is also given.

\keywords{Lipschitz functions, convex functions, variation of a
function, curvature of a curve, turn of a curve.}
\end{abstract}

% at the end of the list, there should be no final dot
\section{The main results}

The solution to this problem is very interesting for the fields of
optimization ( for quasidifferential calculus ) and geometry. In
\cite{aleksandrov1} A.D. Aleksandrov began to study surfaces,
which are graphs of functions representing a difference of two
convex functions (so-called DC functions), to construct internal
geometry of such surfaces. This problem has been studied by many
authors (\cite{aleksandrov2}-\cite{zalgaller1}).

The necessary and sufficient conditions for such a representation
of a function of one variable are well known. These conditions can
be written in the following way.

Let $x \rightarrow f(x): [a,b] \rightarrow \mathbb{R}$ be a
Lipschitz function. Define a set, where the function $f(\cdot)$ is
differentiable, as $N_f$. Since the function $ f(\cdot)$  is
Lipschitz, $N_f$ is the set  of full measure on [a,b].


In order for the function $f(\cdot): [a,b] \rightarrow \mathbb{R}$
to be representable as a difference of two convex functions, it is
necessary and sufficient that the following condition be satisfied
$$
sup_{ \{x_i\} \subset N_f} \,\, \sum_i \mid f'(x_i) - f'(x_{i-1})
\mid \equiv  \vee (f'; a,b) < \infty
$$
where the derivatives are taken where they exist. The sign  $\vee$
means the variation of the function $ f'$ on the segment  [a,b].


In \cite{aleksandrov1} A.D. Aleksandrov formulated the problem of
representing a function as a difference of two convex functions if
it is  DC on any line in the domain of definition. The answer to
this question is negative \cite{vesely}.


According to the Aleksandrov's terminology we define a many-sided
function as a function whose graph consists of finite number of
parts of planes.

In the article \cite{proudconvex1} the necessary and sufficient
conditions for a representation of any one degree homogeneous
Lipschitz function of three  variables as a difference of two
convex functions are given. This result can be extended to the
$m^{th}$ degree homogeneous functions. Now I ignore the
homogeneity condition and will consider any Lipschitz function
$f(\cdot): (x,y) \rightarrow D $ with Lipschitz constant $L$,
where $D$ is an open bounded convex set in $\mathbb{R} ^2$ such
that its closure $\bar{D}$ is compact. I will describe a
representation with an algorithm giving a sequence of pairs of
convex functions. These convex functions converge uniformly on $D$
to a pair of convex functions if the conditions of the theorems
formulated below are satisfied.

Let $\wp(D)$ be a class of curves on the plane  $XOY$  such that
bound star regions  in the region D. Any curve $r \in \wp(D)$ will
be parameterized in natural way i.e. the parameter $\tau$  of a
point M on the curve  $r(\cdot)$ is equal to the length of the
curve $r(\cdot)$ between an initial point and the point M. I will
denote such a curve by $r(t), t \in [0,T_r]$.

Note that the class of curves on the $XOY$ plane that bound closed
convex sets in the domain $D$ belongs to the class $\wp(D)$. Let
us denote such a class of curves by $\varrho(D)$.

With the help of the curves $r \in \wp (D)$  the necessary and
sufficient conditions for a representation of the function
$f(\cdot)$  as a difference of two convex functions can be written
in the following way.

{\bf Theorem 1.} {\em In order that a Lipschitz function $z
\rightarrow f(z):D \rightarrow \mathbb{R}$ was represented as a
difference of two convex functions it is necessary and sufficient
that for any curve $r(\cdot) \in \wp(D)$ and any its subregions
the inequality
$$
 (\exists c_1(D,f), c_2(D,f)>0) (\forall r \in \wp(D)) ,
$$
\begin{equation}
 \vee (\Phi'; 0,T_r)<  c_1(D,f) + c_2(D,f) \vee (r';
 0,T_r), \label{difconv3}
\end{equation} was correct where $\Phi(t)=f(r(t)) \;\;\;\; \forall t \in
[0,T_r]$. \label{difconvexthm1}}


{\bf Remark 1.} {\em The condition of Theorem \ref{difconvexthm1}
means the following. Some constants  $ c_1(D,f)$, $c_2(D,f)>0 $ exist
that for any $r \in \wp(D)$ and any subregions $[T_i, T_{i+1} ]
\subset [0, T_r] $, $i \in 1:m,$
$$
 \sum_1^m \vee (\Phi'; T_i,T_{i+1})<  c_1(D,f) + c_2(D,f) \sum_1^m \vee (r';
 T_i,T_{i+1}).
$$
We included the mention on subregions in Theorem
\ref{difconvexthm1} for which  \ref{difconv3} is correct,  since
the variation $\vee (r'; 0,T_r)$ can be unlimited.}

{\bf Remark 2.} {\em The similar conditions for the variation of
$\Phi (\cdot)$ on curves $r(\cdot)$ were written the first time in
\cite{proudconvex1}}.

\section{Geometric interpretation of Theorem 1}


I will give another geometrical interpretation of  Theorem 1. Let
me introduce a turn of the curve $ r(\cdot)$  on the graph
$\Gamma_f = \{(x,y,z) \in \mathbb{R}^3 \mid z = f(x , y)\} .$

Consider on $\Gamma_f$  the curve $ R(t)=(r(t),f(r(t)))$  where
$r(\cdot) \in \wp(D).$  As the function $f(\cdot,\cdot)$ is
Lipschitz, the derivative $R'(\cdot)$  exists almost everywhere on
$[0,T_r]$,  which I will denote by $\tau(\cdot)=R'(\cdot).$

{\bf Definition.} {\em The turn of the curve $R(\cdot)$  on the
manifold $\Gamma_f$ is called the value
$$
sup_{ \{t_i\} \subset N_r} \,\, \sum_i \Vert
\frac{\tau(t_i)}{\Vert \tau (t_i) \Vert }  -
\frac{\tau(t_{i-1})}{\Vert \tau(t_{i-1}) \Vert } \Vert = O_r
$$ }

{\bf Theorem 2.} {\em For any Lipschitz function $z \rightarrow
f(z) :D \rightarrow R$ to be DC function on the compact set $ D
\in \mathbb{R}^2$ it is necessary and sufficient that some
constants $c_2(D,f), c_3(D,f)>0$ existed for all $r(\cdot) \in
\wp(D)$ such that the turn of the curve $R(\cdot)$ and any subsets
of $R(\cdot)$ on the $\Gamma_f$ were bounded from above
\begin{equation} O_r \leq c_2(D,f) + c_3(D,f) \vee(r';0,T_r)
\;\; \forall r \in \wp(D). \label{difconv6} \end{equation}
\label{difconvexthm2} }

{\bf Corollary 1.} {\em In order that a Lipschitz function $z
\rightarrow f(z):D \rightarrow \mathbb{R}$ was represented as a
difference of two convex functions it is necessary  that for any
curve $r(\cdot) \in \varrho(D)$  the inequality
$$
 (\exists c_5(D,f)>0) (\forall r \in \varrho(D)) ,
 \vee (\Phi'; 0,T_r)<  c_5(D,f)
$$
was correct where $\Phi(t)=f(r(t)) \;\;\;\; \forall t \in
[0,T_r]$.}



{\bf Corollary 2}. {\em For any Lipschitz function $z \rightarrow
f(z) :D \rightarrow R$ to be DC function on the compact set $ D
\in \mathbb{R}^2$ it is necessary that some constant $c_6(D,f)>0$
existed for all $r(\cdot) \in \varrho(D)$ such that the turn of
the curve $R(\cdot)$  on the $\Gamma_f$ was bounded from above
$$
O_r \leq c_6(D,f)  \;\; \forall r \in \varrho(D).
$$ }

The article \cite{veselyzajicek} provides an example confirming
the incorrectness of the sufficiency of the statements of
Consequences. In fact, the authors of the article gave an example
proving that the curves of the set $\varrho(D)$ are not enough to
check the representability of a function as a difference of convex
functions. The class of curves $\wp(D)$ is much wider than the
class of curves $\varrho(D)$.



\begin{thebibliography}{13}

\bibitem{aleksandrov1} Aleksandrov A.D. About the surfaces represented
in a view of a difference of two convex functions, Izvestiya
Kazakh. Akad. Sci.  Math. 1949. no~3.  Pp.~3-20. ~[In Russian].
\bibitem{aleksandrov2} Aleksandrov A.D. Surfaces represented as a difference
of two convex   functions,  Russian Acad. Sci. Dokl. Math. 1950,
no~1. ~[In Russian].
\bibitem{vesely} L. Vesely, L. Zajicek. Delta-convex mappings between Banach
spaces and applications, Dissertationes Math. (Rozprawy Mat.).
1989. Vol.~289.
\bibitem{Hartman} Hartman P. On functions representable as a difference of convex
functions. Pacific J.Math. 1959. no~9. Pp.~707-713.
\bibitem{proudconvex2} Prudnikov I.M. On the question of the representability
of a function of two variables as a difference of convex
functions. Sibirsk. Mat. Zh.  2014. Vol.~55, no~6. Pp.~1368-1380.
~[In Russian].
\bibitem{proudconvex1} Proudnikov I.M. The necessary and sufficient
conditions of a representation of a homogeneous function in a view
of a difference of two convex functions.  Izvestiya Russian Acad.
Sci. Math. 1992. Vol.~59, no~5.  Pp.~1116-1128. ~[In Russian].
\bibitem{veselyzajicek} L.Vesely, L.Zajicek. A non-DC function which is DC
along all convex curves. J. Math. Anal. Appl. 2018. Vol.~463.
Pp.~167-175.
\bibitem{zalgaller1} Zalgaller V.A. About representation of a function
of two variables in a view of a difference of two convex
functions. Vestnik of Leningrad University. 1963. Vol.~1.
Pp.~44-45. ~[In Russian].
\end{thebibliography}

%\end{document}
