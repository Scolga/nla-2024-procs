

\iffalse
\documentclass[12pt]{llncs}

\usepackage{todonotes}

\usepackage{nla} 

\begin{document}
\fi

\title{Conditions for the existence of periodic solutions to \\ one class of systems of \\ nonlinear differential equations\thanks{The work is supported 
by the Mathematical Center in Akademgorodok under agreement 075-15-2022-282 
with the Ministry of Science and Higher Education of the Russian Federation.}}

\author{Gennadii V. Demidenko
}
\institute{Novosibirsk State University, Novosibirsk, Russia\\
\email{g.demidenko@g.nsu.ru}
}

\maketitle

\begin{abstract}
A class of systems of nonlinear differential equations is considered.
It is assumed that the linear parts of the systems have periodic coefficients
and are exponentially dichotomous.
Conditions for the existence of periodic solutions are established
and their stability is proved under small perturbations of coefficients
of the linear parts and nonlinear terms.
\keywords{periodic solutions, exponential dichotomy, Lyapunov equation}
\end{abstract}

We consider systems of nonlinear differential equations
\begin{equation}\label{1}
\frac{dy}{dt} = A(t)y + f(t,y), \quad -\infty < t< \infty,
\end{equation}
where
$A(t)$
is an $(n\times n)$-matrix with $T$-periodc entries, 
the continuous vector-function
$f(t,y)$
satisfies the Lipschitz condition locally with respect to
$y$ 
and the following conditions
\begin{equation}\label{2}
f(t + T, y) \equiv f(t,y), \quad \|f(t,y)\| \le q(1 + \|y\|)^{\omega}, 
\end{equation}
where
$q > 0$
and
$\omega \ge 0$
are constants. 
We assume that the linear systems are exponentially dichotomous (see, for example, [1]). 
According to the spectral criterion, this is equivalent to the fact that the spectrum 
of the monodromy matrix does not intersect with the unite circle. In [2]  
we established a criterion of the exponential dichotomy for systems of differential equations 
with periodic coefficients. The criterion is formulated in terms of the solvability of a special 
boundary value problem for the Lyapunov differential equation 
$$
\frac{d}{dt}H + HA(t) + A^*(t)H = C(t), \quad 0 < t < T,
$$ 
where 
$C(t) = C^*(t)$ 
is a special matrix. 
Using this criterion, one can obtain estimates for all dichotomy parameters,  
establish the existence of 
$T$-periodic solutions to (\ref{1})  
and prove their stability under small perturbations of the right-hand sides.

The obtained results can be used for more detailed study of stability of 
periodic oscillations of an inverted pendulum whose suspension point oscillates
along a straight line having a small angle with the vertical (see, for example, [3]): 
$$
\varphi'' + \varepsilon \varphi' - \frac{g - a \omega^2 \sin (\omega t) \cos \alpha}{l} \sin \varphi 
+ \frac{a \omega^2 \sin (\omega t) \sin \alpha}{l} \cos \varphi = 0.
$$		
	
\begin{thebibliography}{9} % or {99}, if there is more than ten references.

\bibitem{1}
Daleckii Ju.L., Krein, M.G.:
Stability of Solutions of Differential Equations in Banach Space.
Am. Math. Soc., Providence, 1974.

\bibitem{2}
Demidenko G.V. On conditions for exponential dichotomy 
of systems of linear differential equations with periodic coefficients.  
Int. J. Dynamical Systems and Differential Equations. 2016. Vol.~6, no.~1. Pp.~63--74.  

\bibitem{3}
Demidenko G.V., Dulepova A.V. On periodic solutions of one second-order 
differential equation. Journal of Mathematical Sciences (United States). 2024. Vol.~278, 
no.~2. Pp.~314-327.

\end{thebibliography}

%\end{document}