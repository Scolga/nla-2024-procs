\begin{englishtitle} % Настраивает LaTeX на использование английского языка
% Этот титульный лист верстается аналогично.
\title{A dual method for solving a game problem with arbitrary situations}
% First author
\author{Akmal Mamatov
%  \and
%  Name FamilyName2\inst{2}
%  \and
%  Name FamilyName3\inst{1}
}
\institute{Samarkand State University named after Sh. Rashidov, Samarkand, Uzbekistan\\
  \email{akmm1964@rambler.ru}}
%  \and
%Affiliation, City, Country\\
%\email{email@example.com}}
% etc

\maketitle

\begin{abstract}
A dual method for solving a game problem with arbitrary situations is proposed.
\keywords{game problem, players, dual method, support} % в конце списка точка не ставится
\end{abstract}
\end{englishtitle}

\iffalse
%%%%%%%%%%%%%%%%%%%%%%%%%%%%%%%%%%%%%%%%%%%%%%%%%%%%%%%%%%%%%%%%%%%%%%%%
%
%  This is the template file for the 6th International conference
%  NONLINEAR ANALYSIS AND EXTREMAL PROBLEMS
%  June 25-30, 2018
%  Irkutsk, Russia
%
%%%%%%%%%%%%%%%%%%%%%%%%%%%%%%%%%%%%%%%%%%%%%%%%%%%%%%%%%%%%%%%%%%%%%%%%


%  В случае возникновения вопросов и проблем с версткой статьи,
%  пишите письма на электронную почту: eugeneai@irnok.net, Черкашин Евгений.



\documentclass[12pt]{llncs}

% При использовании pdfLaTeX добавляется стандартный набор русификации babel.
% Если верстка производится в LuaLaTeX, то следующие три строки надо
% закомментировать, русификация будет произведена в корректирующем стиле автоматом.
\usepackage{iftex}

\ifPDFTeX
\usepackage[T2A]{fontenc}
\usepackage[utf8]{inputenc} % Кодировка utf-8, cp1251 и т.д.
\usepackage[english,russian]{babel}
\fi

% Для верстки в LuaLaTeX текст готовится строго в utf-8!

% В операционной системе Windows для редактирования в кодировке utf-8
% можно использовать программы notepad++ https://notepad-plus-plus.org/,
% techniccenter http://www.texniccenter.org/,
% SciTE (самая маленькая по объему программа) http://www.scintilla.org/SciTEDownload.html
% Подойдет также и встроенный в свежий дистрибутив MiKTeX редактор
% TeXworks.

% Добавляется корректирующий стилевой файл строго после babel, если он был включен.
% В параметре необходимо указать russian, что установит не совсем стандартные названия
% разделов текста, настроит переносы для русского языка как основного и т.п.

\usepackage{todonotes} % Этот пакет нужен для верстки данного шаблона, его
                       % надо убрать из вашей статьи.

\usepackage[russian]{nla}

% Многие популярные пакеты (amsXXX, graphicx и т.д.) уже импортированы в корректирующий стиль.
% Если возникнут конфликты с вашими пакетами, попробуйте их отключить и сверстать
% текст как есть.
%
%


% Было б удобно при верстке сборника, чтобы названия рисунков разных авторов не пересекались.
% Чтоб минимизировать такое пересечение, рисунки нужно поместить в отдельную подпапку с
% названием - фамилией автора или названием статьи.
%
% \graphicspath{{ivanov-petrov-pics/}} % Указание папки с изображениями в форматах png, pdf.
% или
% \graphicspath{{great-problem-solving-paper-pics/}}.


\begin{document}
\fi
% Текст оформляется в соответствии с классом article, используя дополнения
% AMS.
%

\title{Двойственный метод решения решения игровой задачи с произвольными ситуациями}
% Первый автор
\author{А.~Р.~Маматов%\inst{1}  % \inst ставит циферку над автором.
%  \and  % разделяет авторов, в тексте выглядит как запятая.
% Второй автор
%  И.~О.~Фамилия\inst{2}
%  \and
% Третий автор
%  И.~О.~Фамилия\inst{1}
} % обязательное поле

% Аффилиации пишутся в следующей форме, соединяя каждый институт при помощи \and.
\institute{Самаркандский государственный университет имени Ш. Рашидова, Самарканд, Узбекистан\\
  \email{akmm1964@rambler.ru}
%  \and   % Разделяет институты и присваивает им номера по порядку.
%Институт (название в краткой форме), Город, Страна\\
%  \email{email@example.com}
% \and Другие авторы...
}

\maketitle

\begin{abstract}
Предлагается двойственный метод решения игровой задачи с произвольными ситуациями.

\keywords{игровая задача, игроки, двойственный метод, опора} % в конце списка точка не ставится
\end{abstract}

\section{Основные результаты} % не обязательное поле

Пусть имеется два игрока, которые выбирают стратегии $x$ и $y$ соответственно из множеств
$$\quad X=\{x\mid f_*\leq x \leq f^*\}, Y(x)=\{y\mid g_* \leq y \leq g^*,\quad Ax+By = b\},$$
поочередно, сначала первый игрок выбирает стратегию $x$, затем, зная $x$, второй игрок выбирает стратегию $y$.

Здесь \begin{math} x,f_*,f^* \in \mathbb{R}^n, y,g_{*}, g^{*} \in \mathbb{R}^l, b\in \mathbb{R}^m, A \in \mathbb{R}^{mxn}, B\in \mathbb{R}^{mxl},  rankB<l.\end{math}

Цель первого игрока --- найти $\hat{x}$, доставляющий максимальное значение  функции $$\varphi(x)=\min_{y\in Y(x)}\Psi(x,y),x\in X, \mbox{т.е. } \varphi(\hat{x})=\max_{x\in X}\varphi(x),$$ цель второго игрока --- найти $\hat{y}$, минимизирующий функцию $$\Psi(\hat{x},y),y\in Y(\hat{x}),\mbox{ т.е. } \Psi(\hat{x},\hat{y})=\min_{y\in Y(\hat{x})}\Psi(\hat{x},y) .$$
$$\mbox{ Здесь }\Psi(x,y)=\begin{cases}
c'x+d'y, \text{ если } x\in X,y\in Y(x),c \in \mathbb{R}^n, d \in \mathbb{R}^l;\\
+ \infty, \text{ если } x\in X, Y(x)=\emptyset.
\end{cases}$$

Другими словами, имеем  линейную игровую задачу с произвольными ситуациями \cite{Ivanilov1972},\cite{Mamatov2006}:
$$
\varphi(x)=\min_{y\in Y(x)}\Psi(x,y)\rightarrow \max_{x\in X}. \eqno (1)
$$
Задача максимизации функции $\psi(\mu,s,t)$  по  $(\mu,s,t ) \in M$,  т.е.
$$
\psi(\mu,s,t)= \min_{(\lambda,\nu )\in \Lambda(\mu,s,t )}(b'\mu + g_{*}'s -g^{*'}t + f^{*'}\lambda -f_{*}'\nu)\rightarrow \max_{(\mu,s,t)\in M}, \eqno (2)
$$
$$M=\{(\mu,s,t) \mid B'\mu - t + s =d; s \geq 0, t \geq 0\},$$
$$\Lambda(\mu,s,t )=\{ (\lambda,\nu )\mid A'\mu -\nu +
\lambda = c; \nu  \geq 0,\lambda\geq 0\},$$
называется двойственной к задаче (1).

Используя задачу (2), а также результаты работы  \cite{Mamatov2000} предложен метод решения задачи (1).

При этом основным инструментом исследование является понятия опоры \cite{Gabasov1984}.

Приводятся иллюстративный пример задачи (1) при следующих значениях параметров:
$$c'=(1;-1;0),d'=(1;0;-1;1;5),$$
$$f_*'=(-5;-30;0), f^{*'}=(3;25;50),$$ $$g_*'=(-109;-6;-101;-10;-3),g^{*'}=(44;6;298;10;15),$$
$$A=
\left(
\begin {array}{ccc}
1& 0& -1\\
0& 1& 1
\end{array}
\right), B= \left(
\begin {array}{ccccc}
6& 3& 2& 3& 4\\
4& 2& 1& 2& 3
\end{array}
\right),
b'=(5;4).$$

% Рисунки и таблицы оформляются по стандарту класса article. Например,

%\begin{figure}[htb]
%  \centering
  % Поддерживаются два формата:
  %\includegraphics[width=0.7\linewidth]{figure.pdf} % Растровый формат
  %\includegraphics[width=0.7\linewidth]{figure.png} % Векторный и растровый формат
  %
  % Векторные рисунки можно рисовать в редакторе Inkscape
  % https://inkscape.org/ru/download/
  % Основной формат этого редактора - SVG, поэтому рисунки необходимо экспортировать в
  % PDF или PNG (с разрешением - минимум 150 dpi, максимум - 300dpi).
%  \begin{center}
%    \missingfigure[figwidth=0.7\linewidth]{Уберите меня из статьи!}
%  \end{center}
%  \caption{Заголовок рисунка}\label{fig:example}
%\end{figure}


% В конце текста можно выразить благодарности, если этого не было
% сделано в ссылке с заголовка статьи, например,
%Работа выполнена при поддержке РФФИ (РНФ, другие фонды), проект \textnumero~00-00-00000.
%

% Список литературы оформляется подобно ГОСТ-2008.
% Примеры оформления находятся по этому адресу -
%     https://narfu.ru/agtu/www.agtu.ru/fad08f5ab5ca9486942a52596ba6582elit.html
%

\begin{thebibliography}{9} % или {99}, если ссылок больше десяти.

\bibitem{Ivanilov1972} Иванилов Ю.П. Двойственные полуигры//Известия АН СССР. Серия
техническая кибернетика.1972. \textnumero~4. С.~3--9.

\bibitem{Mamatov2006} Маматов А.Р.Алгоритм решения одной игры двух лиц
с передачей информации ~// ЖВМиМФ. 2006. Т.46, \textnumero~10, С. 1784--1789.

\bibitem{Mamatov2000} Маматов А.Р. Двойственный алгоритм вычисления локального оптимума
одной максиминной задачи со связанными переменными // Узбекский
журн. Пробл. информатики и энергетики. 2000. \textnumero~1. С.7--12.

\bibitem{Gabasov1984} Габасов Р., Кириллова Ф.М., Тятюшкин А. И. Конструктивные
методы оптимизации. Ч.1. Линейные задачи. Минск:Университетское. 1984.



\end{thebibliography}

% После библиографического списка в русскоязычных статьях необходимо оформить
% англоязычный заголовок.




%\end{document}

%%% Local Variables:
%%% mode: latex
%%% TeX-master: t
%%% End:
