\begin{englishtitle} % Настраивает LaTeX на использование английского языка
% Этот титульный лист верстается аналогично.
\title{On the question of an alternative in a differential game for systems with the properties of generalized uniqueness and uniform boundedness}
% First author
\author{Alexander Chentsov   \and Dmitrii Serkov 
}
\institute{Krasovskii institute of mathematics and mechanics, Yekaterinburg, RF\\
  \email{chentsov@imm.uran.ru} ,\email{serkov@imm.uran.ru}}
% etc

\maketitle

\begin{abstract}
We consider constructions related to the alternative solvability of a nonlinear differential game (meaning the Krasovsky-Subbotin alternative) under the conditions of generalized uniqueness and uniform boundedness of the generalized trajectories used by A.\,V.\,Kryazhimskiy.
The constructions use procedures based on the method of program iterations.
\keywords{differential games, theorem on the alternative, condition of generalized uniqueness, method of program iterations} % в конце списка точка не ставится
\end{abstract}
\end{englishtitle}

\iffalse
%%%%%%%%%%%%%%%%%%%%%%%%%%%%%%%%%%%%%%%%%%%%%%%%%%%%%%%%%%%%%%%%%%%%%%%%
%
%  This is the template file for the 6th International conference
%  NONLINEAR ANALYSIS AND EXTREMAL PROBLEMS
%  June 25-30, 2018
%  Irkutsk, Russia
%
%%%%%%%%%%%%%%%%%%%%%%%%%%%%%%%%%%%%%%%%%%%%%%%%%%%%%%%%%%%%%%%%%%%%%%%%

%  Верстка статьи осуществляется на основе стандартного класса llncs
%  (Lecture Notes in Computer Sciences), который корректируется стилевым
%  файлом конференции.
%
%  Скомпилировать файл в PDF можно двумя способами:
%  1. Использовать pdfLaTeX (pdflatex), (LaTeX+DVIPS не работает);
%  2. Использовать LuaLaTeX (XeLaTeX будет работать тоже).
%  При использовании LuaLaTeX потребуются TTF- или OTF-шрифты CMU
%  (Computer Modern Unicode). Шрифты устанавливаются либо пакетом
%  дистрибутива LaTeX cm-unicode
%              (https://www.ctan.org/tex-archive/fonts/cm-unicode),
%  либо загрузкой и установкой в операционной системе, адрес страницы:
%              http://canopus.iacp.dvo.ru/%7Epanov/cm-unicode/
%  Второй вариант не будет работать в XeLaTeX.
%
%  В MiKTeX (дистрибутив LaTeX для ОС Windows):
%  1. Пакет cm-unicode устанавливается вручную в программе MiKTeX Console.
%  2. Для верстки данного примера, а именно, картинки-заглушки необходимо,
%     также вручную, загрузить пакет pgf. Этот пакет используется популярным
%     пакетом tikz.
%  3. Тест показал, что остальные пакеты MiKTeX грузит автоматически (если
%     ему разрешено автоматически грузить пакеты). Режим автозагрузки
%     настраивается в разделе Settings в MiKTeX Console.
%
%
%  Самый простой способ сверстать статью - использовать pdfLaTeX, но
%  окончательный вариант верстки сборника будет собран в LuaLaTeX,
%  так как результат получится лучшего качества, благодаря пакету microtype и
%  использованию векторных шрифтов OTF вместо растровых pdfLaTeX.
%
%  В случае возникновения вопросов и проблем с версткой статьи,
%  пишите письма на электронную почту: eugeneai@irnok.net, Черкашин Евгений.
%
%  Новые варианты корректирующего стиля будут доступны на сайте:
%        https://github.com/eugeneai/nla-style
%        файл - nla.sty
%
%  Дальнейшие инструкции - в тексте данного шаблона. Он одновременно
%  является примером статьи.
%
%  Формат LaTeX2e!

\documentclass[12pt]{llncs}  % Необходимо использовать шрифт 12 пунктов.

% При использовании pdfLaTeX добавляется стандартный набор русификации babel.
% Если верстка производится в LuaLaTeX, то следующие три строки надо
% закомментировать, русификация будет произведена в корректирующем стиле автоматом.
\usepackage{iftex}

\ifPDFTeX
\usepackage[T2A]{fontenc}
\usepackage[utf8]{inputenc} % Кодировка utf-8, cp1251 и т.д.
\usepackage[english,russian]{babel}
\fi

% Для верстки в LuaLaTeX текст готовится строго в utf-8!

% В операционной системе Windows для редактирования в кодировке utf-8
% можно использовать программы notepad++ https://notepad-plus-plus.org/,
% techniccenter http://www.texniccenter.org/,
% SciTE (самая маленькая по объему программа) http://www.scintilla.org/SciTEDownload.html
% Подойдет также и встроенный в свежий дистрибутив MiKTeX редактор
% TeXworks.

% Добавляется корректирующий стилевой файл строго после babel, если он был включен.
% В параметре необходимо указать russian, что установит не совсем стандартные названия
% разделов текста, настроит переносы для русского языка как основного и т.п.

\usepackage{todonotes} % Этот пакет нужен для верстки данного шаблона, его
                       % надо убрать из вашей статьи.

\usepackage[russian]{nla}

% Многие популярные пакеты (amsXXX, graphicx и т.д.) уже импортированы в корректирующий стиль.
% Если возникнут конфликты с вашими пакетами, попробуйте их отключить и сверстать
% текст как есть.
%
%


% Было б удобно при верстке сборника, чтобы названия рисунков разных авторов не пересекались.
% Чтоб минимизировать такое пересечение, рисунки можно поместить в отдельную подпапку с
% названием - фамилией автора или названием статьи.
%
% \graphicspath{{ivanov-petrov-pics/}} % Указание папки с изображениями в форматах png, pdf.
% или
% \graphicspath{{great-problem-solving-paper-pics/}}.


\begin{document}
\fi
% Текст оформляется в соответствии с классом article, используя дополнения
% AMS.
%

\title{К вопросу об альтернативе в дифференциальной игре для систем со свойствами обобщенной единственности и равномерной ограниченности}%\thanks{Работа выполнена при поддержке РФФИ (РНФ, другие фонды), проект \textnumero~00-00-00000.}
% Первый автор
\author{А.~Г.~Ченцов   \and  Д.~А.~Серков 
} % обязательное поле

% Аффилиации пишутся в следующей форме, соединяя каждый институт при помощи \and.
\institute{ИММ УрО РАН, Екатеринбург, РФ\\
   \email{chentsov@imm.uran.ru}, \email{serkov@imm.uran.ru}
}

\maketitle

\begin{abstract}
Рассматриваются конструкции, связанные с альтернативной разрешимостью нелинейной дифференциальной игры (имеется в виду альтернатива Красовского--Субботина) при условиях обобщенной единственности и равномерной продолжимости обобщенных траекторий, используемых А.\, В.\, Кряжимским.
Построения используют процедуры на основе метода программных итераций.

\keywords{дифференциальные игры, теорема об альтернативе, условие обобщенной единственности, метод программных итераций} % в конце списка точка не ставится
\end{abstract}

%\section{Основные результаты} % не обязательное поле
Для конфликтно управляемых динамических систем, удовлетворяющих условиям обобщенной единственности и равномерной ограниченности, изучаются вопросы разрешимости задачи на ми\-ни\-макс и иные свойства программных конструкций в классе обобщенных управлений (такие системы, не обладая, вообще говоря, липшицевостью по фазовой переменной, удовлетворяют теореме об альтернативе Н.\,Н.\,Кра\-сов\-ско\-го,\ А.\,И.\,Суб\-бо\-ти\-на (см. \cite{KraSub70PMM,KraSub74});
как показано \cite{Kry78}, именно условие обобщенной единственности А.\,В.\,Кряжимского позволяет распространить утверждение альтернативы на более широкий класс систем и, вместе с тем, условия обычной единственности для этого, вообще говоря, не достаточно \cite{KryDOK}).

В связи с построением альтернативного разбиения пространства позиций отметим пошаговые попятные процедуры построения стабильных мостов Н.\,Н.\,Красовского в \cite{Ushak1980} и метод программных итераций \cite{Chentsov76DAN226,Chentsov2021DU}.
В настоящем докладе сосредоточимся на варианте этого метода \cite[п.\,8]{Chentsov2021DU}, где указана итерационная процедура в пространстве множеств для случая нелинейной управляемой системы, удовлетворяющей условиям \cite{Kry78}.
В этой связи отметим процедуру \cite{ChentsovIMI2017}, имеющую смысл итераций на основе свойства, подобного стабильности в работах Н.\,Н.\,Красовского.
При этом в \cite{KraSub70PMM,KraSub74} рассматривался случай дифференциальной игры сближения--уклонения с замкнутыми (в пространстве позиций) целевыми множествами и множеством, определяющим фазовые ограничения.
В \cite{Chentsov2021DU,ChentsovIMI2017} рассматривалось при условиях \cite{Kry78} построение альтернативного разбиения в случае, когда целевое множество замкнуто, а множество, определяющее фазовые ограничения, имеет замкнутые временные сечения, но само может быть не замкнутым в пространстве позиций.
Важно отметить, что имеется обширный класс конфликтно управляемых систем, удовлетворяющих \cite{Kry78,KryDOK} и, вместе с тем, не удовлетворяющих локально условию Липшица по фазовой переменной, которое использовалось в \cite{KraSub70PMM,KraSub74} при доказательстве теоремы об альтернативе.


% Рисунки и таблицы оформляются по стандарту класса article. Например,

% Современные издательства требуют использовать кавычки-елочки << >>.

% В конце текста можно выразить благодарности, если этого не было
% сделано в ссылке с заголовка статьи, например,
%Работа выполнена при поддержке РФФИ (РНФ, другие фонды), проект \textnumero~00-00-00000.
%

% Список литературы оформляется подобно ГОСТ-2008.
% Примеры оформления находятся по этому адресу -
%     https://narfu.ru/agtu/www.agtu.ru/fad08f5ab5ca9486942a52596ba6582elit.html
%

\begin{thebibliography}{9} % или {99}, если ссылок больше десяти.


\bibitem{KraSub70PMM}{Красовский~Н.\,Н.,\ Субботин~A.\,И.} Альтернатива для игровой задачи сближения~// Прикладная математика и механика. 1970. Т.~34, {№}~6. С.~1005--1022.

\bibitem{KraSub74} Красовский Н. Н., Субботин A. И. Позиционные дифференциальные игры. М.: Наука. 1974.

\bibitem{Kry78}{ Кряжимский~А.\,В.} К теории позиционных дифференциальных игр сбли\-же\-ния-ук\-ло\-не\-ния~// Докл. АН СССР. 1978. Т.~239, {№}~4. С.~779--782.

\bibitem{KryDOK}Кряжимский~А.\,В.\, “Дифференциальные игры для нелипшицевых систем” Диссертация на соискание ученой степени доктора физико-математических наук, АН СССР УНЦ Институт математики и механики, Свердловск, 1980.

\bibitem{Ushak1980} Ушаков\,А.\, В. “К задаче построения стабильных мостов в дифференциальной игре сближения-уклонения,” Изв. АН СССР. Техн. кибернетика, no. 4, pp. 29–36, 1980.

\bibitem{Chentsov76DAN226} Ченцов А. Г. К игровой задаче наведения // Докл. АН СССР. 1976. Т. 226. С. 73–76.

\bibitem{Chentsov2021DU}Ченцов А. Г. Дифференциальная игра сближения-уклонения: альтернативная разрешимость и построение релаксаций // Дифференц. уравн. 2021. Т. 57. С. 1116–1141.

\bibitem{ChentsovIMI2017} Ченцов А. Г. Итерации стабильности и задача уклонения с ограничением на число переключений формируемого управления // Изв. ИМИ УдГУ. 2017. Т. 49. С. 17–54.


\end{thebibliography}

% После библиографического списка в русскоязычных статьях необходимо оформить
% англоязычный заголовок.




%\end{document}

%%% Local Variables:
%%% mode: latex
%%% TeX-master: t
%%% End:
