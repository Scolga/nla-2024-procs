
\iffalse
\documentclass[12pt]{llncs}

\usepackage{todonotes}

\usepackage{nla} 

\begin{document}
\fi

\title{The $14D$ Ricci-flat metric in the theory of fluid flows and the equations of rotation of a top\thanks{This work is supported by the Program SATGED 011303, Moldova State University.}}

\author{Valery Dryuma 
}
\institute{Institute of Mathematics and Computer Science ``V. Andrunachievici'',\\  Moldova State University, Chi\c sin\u au, Republic of Moldova\\
  \email{valdryum@gmail.com}
 }

\maketitle

\begin{abstract}
 Properties of Riemannian metric of14D space in local coordinates $(x,y,z,t,\eta,\rho,$ $m,u,v,w,p,\xi,\chi,n)$ which is the Ricci-flat on solutions of the Navier-Stokes system of equations are studied.Considered space consists of $6D$ space in local coordinates $(\eta,\rho,m,\xi,\chi,n)$ and of two $4D$ dual subspaces -in Euler coordinates $(x,y,z,t)$ and Lagrangian coordinates $(u,v,w,p)$. Structure of partition of $14D$ space depend significanltly on the properties of $6D$-flat space devided into two $3D$ subspaces. In this cotext properties of the Kovalevskaya top are studied in more detail.   

\keywords{equations, invariants, metrics,top}
\end{abstract}


\section{The Ricci-flat 14D metric and NS-equations}
\begin{theorem}      The metric of the form 
	\begin{equation}\label{metric1}
	\begin{gathered}
	ds^{2}=2\,{{\it dx}}^{2}+2\,{\it dx dy}+2\,{\it dx du}+2\,{{\it 
			dy}}^{2}+2\,{\it dy dz}+2\,{\it dy dv}+2\,{{\it dz}}^{2}+2\,{\it dz dw}+\\ +2
	\,{\it dt dp}+2\,{\it d\eta d\xi}+2\,{\it d\rho d\chi}+2\,{\it dm dn}+A{{\it 
			dt}}^{2}+B{{\it d\eta}}^{2}+C{{\it d\rho}}^{2}+E{{\it dm}}^{2},
	\end{gathered}
	\end{equation}
	where
	$$
	A=2-U \left( x,y,z,t \right) u-V \left( x,y,z,t \right) v-W \left( x,y
	,z,t \right) w,
	$$
	$$
	B= \left( -U  W +\mu\,{
		\frac {\partial }{\partial z}}U   \right) w+
	\left( -U  V  +\mu\,{
		\frac {\partial }{\partial y}}U \right) v+
	\left( \mu\,{\frac {\partial }{\partial x}}U  -
	\left( U  \right) ^{2}-P \right) u-U p,
	$$
	$$
	C= \left( -V  W +\mu\,{
		\frac {\partial }{\partial z}}V  \right) w+
	\left( \mu\,{\frac {\partial }{\partial y}}V -
	\left( V   \right) ^{2}-P \right) v+ \left( -U V  +\mu\,{\frac {\partial }{\partial x}}V 
	\right) u-V p,
	$$
	$$
	E= \left( -\mu\,{\frac {\partial }{\partial x}}U -\mu\,{\frac {\partial }{\partial y}}V  - 
	\left( W \right) ^{2}-P \right) w+ \left( -V W +\mu\,{\frac {\partial }{\partial y}}W \right) v+
	$$
	$$
	\left( -U  W +\mu\,{\frac {\partial }{\partial x}}W \right) u-W p
	$$
	is the Ricci-flat on solutions of Navier-Stokes system of equations (\ref{cub}) 
\end{theorem}
The Navier-Stokes system of equations
\begin{equation}\label{cub}
{\frac {\partial }{\partial t}}\vec U +\left(\vec U \cdot \vec\nabla\right)\vec U-\mu \Delta \vec U+\vec \nabla P=0, \qquad\vec\nabla \cdot \vec U=0,
\end{equation}
where  $\vec U=[U(\vec x,t), V(\vec x,t), W(\vec x,t)]$ is the fluid velocity , $P=P(\vec x,t)$, is the pressure, $\mu$ is the viscosity and 
$\vec x=(x,y,z)$  has numerous applications. 
The problem of their integration and classification  solutions is an important to the modern  mathematical and theoretical physics.
We will use  methods of the Riemannian geometry to studies properties of the system  (\ref{cub}).

 In the case of the Kovalevsky top $A=2C,B=2C,y0=0,z0=0$ the Euler-Poisson system of equations  
\begin{equation}\label{case6}
%\begin{gathered}
{\frac { d}{{ d}t}}x \left( t \right) =-{\frac { \left( C-B
		\right) yz}{A}}+{\frac {Mg \left( {\it y0}\,\omega2 \left( t \right) 
		-{\it z0}\,\omega1 \left( t \right)  \right) }{A}},$$$$
{\frac { d}{{ d}t}}y \left( t \right) =-{\frac { \left( A-C \right) x \left( t
		\right) z \left( t \right) }{B}}+{\frac {{\it z0}\,\omega \left( t
		\right) -{\it x0}\,\omega2 \left( t \right) }{B}}, $$$$
{\frac { d}{{d}t}}z \left( t \right) =-{\frac { \left( B-A
		\right) x \left( t \right) y \left( t \right) }{C}}+{\frac {{\it x0}
		\,\omega1 \left( t \right) -{\it y0}\,\omega \left( t \right) }{C}},$$$$
%\end{gathered}
%\end{equation}
%\begin{equation}\label{case7}
%\begin{gathered}
{\frac { d}{{ d}t}}\omega \left( t \right) =z \left( t \right) 
\omega1 \left( t \right) -y \left( t \right)
\omega2 \left( t \right),~~ 
{\frac {d}{{d}t}}\omega1 \left( t \right) =x \left( t \right)\omega2 \left( t \right) -z \left( t \right) \omega \left( t \right),~~$$$$ 
{\frac {d}{{d}t}}\omega2 \left( t \right) =y \left( t \right) 
\omega \left( t \right) -x \left( t \right) \omega1 \left( t \right),
%\end{gathered}
\end{equation}
 leads to the relations 
\begin{equation}\label{case8}  
 x \left( t \right) ={\frac {y \left( t \right) \omega \left( t
 		\right) \omega2 \left( t \right) +\omega \left( t \right) {\frac 
 			{ d}{{ d}t}}\omega \left( t \right) +\omega1 \left( t \right) {
 			\frac { d}{{d}t}}\omega1 \left( t \right) }{\omega2 \left( t
 		\right) \omega1 \left( t \right) }} 
\end{equation}
\begin{equation}\label{case9}z \left( t \right) ={\frac {y \left( t \right) \omega2 \left( t
		\right) +{\frac { d}{{ d}t}}\omega \left( t \right) }{\omega1
		\left( t \right) }}
\end{equation}
\begin{equation}
%\begin{gathered}\label{case10}z
y \left( t \right) =-2\,{\frac {C \left( {\frac {d}{{ d}t}}
		\omega1 \left( t \right)  \right) \omega \left( t \right) \omega1
		\left( t \right) }{\omega2 \left( t \right)  \left( 2\,C \left( -
		\left( \omega1 \left( t \right)  \right) ^{2}- \left( \omega2 \left( 
		t \right)  \right) ^{2}-1 \right) +2\,C \left( \omega1 \left( t
		\right)  \right) ^{2}+C \left( \omega2 \left( t \right)  \right) ^{2}
		\right) }}-$$$${\frac { \left( 2\,C \left( - \left( \omega1 \left( t
		\right)  \right) ^{2}- \left( \omega2 \left( t \right)  \right) ^{2}-
		1 \right) +C \left( \omega2 \left( t \right)  \right) ^{2} \right) {
			\frac {d}{{d}t}}\omega \left( t \right) }{\omega2 \left( t
		\right)  \left( 2\,C \left( - \left( \omega1 \left( t \right) 
		\right) ^{2}- \left( \omega2 \left( t \right)  \right) ^{2}-1
		\right) +2\,C \left( \omega1 \left( t \right)  \right) ^{2}+C \left( 
		\omega2 \left( t \right)  \right) ^{2} \right) }}+$$$$+{\frac {{\it C2}\,
		\omega1 \left( t \right) }{2\,C \left( - \left( \omega1 \left( t
		\right)  \right) ^{2}- \left( \omega2 \left( t \right)  \right) ^{2}-
		1 \right) +2\,C \left( \omega1 \left( t \right)  \right) ^{2}+C
		\left( \omega2 \left( t \right)  \right) ^{2}}}.
%\end{gathered}
\end{equation}

    Together with the known first integrals of system (10)
\begin{equation}
 \left( \omega \left( t \right)  \right) ^{2}+ \left( \omega1 \left( t
 \right)  \right) ^{2}+ \left( \omega2 \left( t \right)  \right) ^{2}- 1,
 Ax \left( t \right) \omega \left( t \right) +By \left( t \right) 
 \omega1 \left( t \right) +Cz \left( t \right) \omega2 \left( t
 \right) -{\it C2},$$$$
 A \left( x \left( t \right)  \right) ^{2}+B \left( y \left( t \right) 
 \right) ^{2}+C \left( z \left( t \right)  \right) ^{2}-2\,Mg \left( {
 	\it x0}\,\omega \left( t \right) +{\it y0}\,\omega1 \left( t \right) +
 {\it z0}\,\omega2 \left( t \right)  \right) +{\it C1},$$$$
 \left(  \left( x \left( t \right)  \right) ^{2}+ \left( y \left( t
 \right)  \right) ^{2}+c\omega \left( t \right)  \right) ^{2}+ \left( 
 2\,x \left( t \right) y \left( t \right) +c\omega1 \left( t \right) 
 \right) ^{2}={K}^{2}
\end{equation}    
the components of fluid-flow velocity and pressure $P(x,y,z,t)$ are determined using the Ricci-flat metric  on solutions of the system (\ref{cub}).
%\section{ Acknowledgements}
%The author was supported in part by the project $N 15.817.02.3F$.
%\bigskip
\begin{thebibliography}{9}
%\footnotesize
%\medskip
\bibitem{dryuma31:dryuma} {  Dryuma V. } {  On spaces related t the Navier-Stokes equations.}
Bul. Acad. \c Stiin\c te Repub. Moldova, Mat. 2010. Vol. 64. No.~3. P. 107--110.
%Mathematical Research, issue 122, Chisinau, "Stiinta",  1990, 93--103 (in Russian).
%\smallskip

 \bibitem{dryuma31:dryuma} {  Dryuma V. } {  The Ricci-flat spaces related t the Navier-Stokes equations.}
 Bul. Acad. \c Stiin\c te Repub. Moldova, Mat.  2012. Vol. 69. No.~2. P. 99--102.
 %Mathematical Research, issue 122, Chisinau, "Stiinta",  1990, 93--103 (in Russian).
 %\smallskip
 
  \bibitem{dryuma7} {  Golubev V. } {  Lectures on integrating the equations of motion of a rigid body around a fixed point} {  Gostexizdat, Moskva,1953}.
 
\end{thebibliography} 
%\end{document}


