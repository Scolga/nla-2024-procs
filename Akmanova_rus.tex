\begin{englishtitle} % Настраивает LaTeX на использование английского языка
% Этот титульный лист верстается аналогично.
\title{On the issue of optimal control of hybrid dynamic systems}
% First author
\author{S.~V.~Akmanova
}

\institute{Nosov Magnitogorsk State Technical University,  Magnitogosk, Russia\\
  \email{svet.akm\textunderscore74@mail.ru}
  }
  % etc
\maketitle

\begin{abstract}
The paper considers the issues of constructing optimal and acceptable software controls, respectively, for linear hybrid and nonlinear hybrid dynamic control systems. The construction of an acceptable control of nonlinear hybrid systems is associated with the solution of the corresponding linear optimal control problems.

\keywords{hybrid control system, optimal control, optimal movement, permissible control, stabilizing control} % в конце списка точка не ставится
\end{abstract}
\end{englishtitle}


\iffalse
\documentclass[12pt]{llncs}

% При использовании pdfLaTeX добавляется стандартный набор русификации babel.

\usepackage{iftex}

\ifPDFTeX
\usepackage[T2A]{fontenc}
\usepackage[utf8]{inputenc} % Кодировка utf-8, cp1251 и т.д.
\usepackage[english,russian]{babel}
\fi

\usepackage{todonotes}

\usepackage[russian]{nla}


\begin{document}
\fi

\title{К вопросу оптимального управления гибридными динамическими системами }
% Первый автор
\author{С.~В.~Акманова
 } % обязательное поле

% Аффилиации пишутся в следующей форме, соединяя каждый институт при помощи \and. 
\institute{ФГБОУ ВО <<МГТУ им. Г.И.Носова>>, Магнитогорск, Россия\\
  \email{svet.akm\textunderscore74@mail.ru}
    % Разделяет институты и присваивает им номера по порядку.
 }

\maketitle
 
\begin{abstract}
В работе рассматриваются вопросы построения оптимального и допустимого программных управлений соответственно для линейной гибридной и нелинейной  гибридной динамических систем управления. Построение допустимого управления нелинейными гибридными системами связано с решением соответствующих линейных задач оптимального управления.

\keywords{гибридная система управления, оптимальное управление, оптимальное движение, допустимое управление, стабилизирующее управление } 
\end{abstract}

\section{Основные результаты} % не обязательное поле

Гибридные (непрерывно-дискретные) динамические системы управления служат адекватными математическими моделями многих реальных систем управления \cite{Marchenko}. При изучении таких систем одним из основных является вопрос о построении оптимальных управлений.

В настоящей работе рассматривается  линейная гибридная динамическая система управления вида
$$
   \left\{ \begin{array}{c}
    x'(t)=A_1x(t)+B_1y(t_k),\ t_{k} \le t<t_{k+1},\\
    y(t_{k+1})=A_2x(t_{k+1})+B_2y(t_k)+Cu(t_k), k=0, 1, 2,...\\ 
      \end{array} 
   \right.  \eqno (1)
$$
   с начальными условиями 
 $$
  x(t_0)=x_0, y(t_0)=y_0,\eqno (2)
$$
где  $t_{k+1}-t_{k}=h>0$ -- постоянная величина ($k=0, 1, 2,...$), 
${x\in R^{n}}, {y\in R^{m}}$ -- векторы состояний, характеризующие поведение непрерывной и дискретной частей системы (1) соответственно, ${u\in R^{q}}$ -- вектор управления. 

При выбранном управлении $u=u(t_k)$, $k=0, 1, 2,...$ система (1) функционирует по стандартной схеме, при этом, учитывая (2), под решением системы (1), стартующим из точки $z_0=(x_0;y_0)$, будем понимать функцию
$$
  z(t)=(x(t),y(t)), \eqno (3)
$$
где $x(t)=\varphi_{k}(t)$, $y(t)=y(t_k)=y_k$
при $t_{k} \le t<t_{k+1}$, $k=0,1,2,...$,\\ причем
 $x(t)=\varphi_{k}(t)$ - решение задачи Коши $x'=A_1x+B_1y_k$, $x(t_k)=x_k$, $k=0,1,2,..$; 
 $y(t)$ -- кусочно-постоянная функция, меняющая свои значения в моменты времени $t=t_k$, $k=0,1,2,..$. .

Будем полагать, что система (1) управляема при $h=h_0>0$.  Ставится  задача построения оптимального программного управления $u^{0}=u^{0}(t_k)$, $k=0,1,2,..., l-1$ $(l \in N)$, переводящего систему (1), с учетом (3), из заданного начального состояния $z(t_0)=z^{(0)}$ в заданное конечное состояние
 $z(t_l)=z^{(1)}$ и минимизирующее функционал $I(u)$. Функционал $I(u)$  отвечает за качество процесса управления (количество расходуемой энергии) а имеет вид
$$
  I(u)=\sum_{k=0}^{l-1} u^{T}(t_k)u(t_k),\eqno (4)
$$
где $Т$ означает транспонирование.

Для решения поставленной задачи выполним дискретизацию системы (1) с шагом $h=h_0>0$ и перейдем к рассмотрению равносильной линейной дискретной системы вида
$$
    z_{k+1}=A_{0}z_{k}+Bu_{k}, \quad k=0, 1, 2, \ldots\,, \eqno (5)
$$	
где 
$$% \begin{equation}\label{mg2jgs5}
   z_{k}=\left[ \begin{array}{l}
		x_{k} \\
		y_{k}
	\end{array} \right]\,, \quad
     A_{0}=\left[\begin{array}{cc}
		e^{A_1h_0}  & A_1^{-1}(e^{A_1h_0}-I)B_1 \\
		A_2e^{A_1h_0}  & A_2A_1^{-1}(e^{A_1h_0}-I)B_1+B_2
	\end{array} \right]\,, \quad
    B=\left[ \begin{array}{l}
		0 \\
		C
	\end{array} \right]\,, \quad
  $$% \end{equation}	
$x_k=x(t_k)$, $y_k=y(t_k), u_k=u(t_k)$.

Равносильность систем (1) и (5) понимается в следующем смысле:

-- если $(x(t),y_k)$ -- решение системы (1), то $(x_k, y_k)$ -- решение системы (5), где $x_k=x(t_k)$;

-- если $(x_k, y_k)$ -- решение системы (5),  то $(x(t),y_k)$ -- решение системы (1), где $x(t)$ - решение задачи Коши   $x'=A_1x+B_1y_k$, $x(t_k)=x_k$.

Тогда поставленная для системы (1) задача сводится к задаче построения оптимального программного управления $u^{0}_k$, $k=0,1,2,...,l-1$, переводящего систему (5) из состояния $z_0=z^{(0)}$ в состояние $z_l=z^{(1)}$ при  условии минимизации функционала (4).

 Поскольку критерий оптимальности  является общим для систем (1) и (5), и данные системы равносильны, то, основываясь на работу \cite{Cazanova}, можно доказать, что искомое оптимальное управление для системы (1) имеет вид
$$
u^{0}(t_k)=S^{T}(k)F^{+}(0)d(0,z_0),\,\  k=0,1,2,...,l-1, \eqno (6)
$$
где  $S(k)=A^{l-k-1}_0 B,k=0,1,2,...,l-1$; $d(0,z_0)=z_l-A^{l}z_0$; $F^{+}(0)$ -- матрица, псевдообратная к матрице 
$$
F(0)=\sum_{k=0}^{l-1}S(k)S^{T}(k).
$$
Построенному оптимальноиу управлению (6)  соответствуют оптимальное движение \\
 ${z^{0}_k=[x^{0}_k \ \  y^{0}_k]^{T}}$ $(k=0,1,2,...,l)$ системы (5) и оптимальное движение системы (1) вида
 $$
  z^{0}(t)=(x^{0}(t),y^{0}(t)),
$$
где $x^{0}(t)=\varphi^{0}_k(t)$, $y^{0}(t)=y^{0}(t_k)=y^{0}_k$ при $t_{k} \le t<t_{k+1}$, $k=0,1,2,...,l$, \\
 причем
 $ \varphi^{0}_{k}(t_k)=x^{0}_k$, $k=0,1,2,...,l$.
 
 Далее в работе рассматривается нелинейная гибридная система управления вида
$$
   \left\{ \begin{array}{c}
    x'(t)=f(x(t), y(t_{k})),\ t_{k} \le t<t_{k+1},\\
    y(t_{k+1})=g(x(t_{k+1}),y(t_{k}),u(t_{k})), k=0, 1, 2,...,\\
    \end{array}
   \right. \eqno (7)
$$
с начальными условиями (2), при этом компоненты $x$, $y$, $u$  соответствуют описаниям  приведенным для системе (1); $f(x,y), g(x,y,u)$ -- непрерывно дифференцируемые по совокупности переменных функции. 

Будем полагать, что при $u=0$ система (7) имеет точку равновесия $x=0$, $y=0$, т.е. выполняются соотношения
$f(0,0)=g(0,0,0)=0$.

С учетом работ \cite{Akmanova,Cazanova,Albreht} строится допустимое управление  ${u^{*}=u^{*}(t_k)}$, ${k=0,1,  ...,l-1}$, переводящее систему (7) из начального состояния $z(t_0)=z^{(0)}$  в конечное состояние $z(t_l)=z^{(1)}$
и при этом функционал (4) удовлетворяет условию $I(u^{*})<+\infty$. 

Построение допустимого управления системы (7)  основано на  построении стабилизирующего управления для   этой системы и решении соответствующих линейных задач оптимального управления.


%

\begin{thebibliography}{9} % или {99}, если ссылок больше десяти.
\bibitem{Marchenko} Марченко В.М., Борковская И.М. Устойчивость и стабилизация линейных гибридных дискретно-непрерывных стационарных систем // Труды БГТУ. Физмат. науки и информатика. 2012. \textnumero~6.
С.~7--10. 

\bibitem{Cazanova} Сазанова Л.А. Об управлении дискретными  системми //   Автореф. дис. ... канд. физ.-мат. наук. Екат.еринбург,~2002.

 
\bibitem{Akmanova} Акманова С.В. Об управляемости и стабилизируемости нелинейных гибридных систем // Материалы междунар. науч. конф.
<<Уфимская осенняя математическая школа-2023>>. Уфа, 2023. Т.1.  С.~164--166.

\bibitem{Albreht}Альбрехт Э.Г., Сазанова Л.А. О построении допустимых управлений  в глобально управляемых  нелинейных дискретных системах // Известия УрГУ. 2003. \textnumero~26. С.~11--23. 

\end{thebibliography}



% После библиографического списка в русскоязычных статьях необходимо оформить
% англоязычный заголовок и аннотацию.




%\end{document}
