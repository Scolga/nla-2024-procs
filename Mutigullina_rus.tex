\begin{englishtitle} % Настраивает LaTeX на использование английского языка
% Этот титульный лист верстается аналогично.
\title{Decomposition of Webullah function into an orthogonal row}
% First author
\author{R.~R.~Mutigullina\inst{1}  \and  R.~D.~Murtazina\inst{2}
 }
\institute{UUST, Ufa, Russia\\
  \email{Regina.mutigullina@mail.ru}
  \and
  UUST, Ufa, Russia \\
\email{reginaufa@yandex.ru}
  }
% etc

\maketitle

\begin{abstract}
When expanding the Weibull distribution function into an orthogonal series of exponentials, the coefficients are uniformly convergent integrals with the parameter.

\keywords{Weibull function, orthogonal series, uniform convergence} % в конце списка точка не ставится
\end{abstract}
\end{englishtitle}


\iffalse
%%%%%%%%%%%%%%%%%%%%%%%%%%%%%%%%%%%%%%%%%%%%%%%%%%%%%%%%%%%%%%%%%%%%%%%%
%
%  This is the template file for the 6th International conference
%  NONLINEAR ANALYSIS AND EXTREMAL PROBLEMS
%  June 25-30, 2018
%  Irkutsk, Russia
%
%%%%%%%%%%%%%%%%%%%%%%%%%%%%%%%%%%%%%%%%%%%%%%%%%%%%%%%%%%%%%%%%%%%%%%%%

\documentclass[12pt]{llncs}  

% При использовании pdfLaTeX добавляется стандартный набор русификации babel.

\usepackage{iftex}

\ifPDFTeX
\usepackage[T2A]{fontenc}
\usepackage[utf8]{inputenc} % Кодировка utf-8, cp1251 и т.д.
\usepackage[english,russian]{babel}
\fi

\usepackage{todonotes} 

\usepackage[russian]{nla}

\begin{document}
\fi

\title{Разложение функции Вейбулла в ортогональный ряд}
% Первый автор
\author{Р.~Р.~Мутигуллина\inst{1}  \and Р.~Д.~Муртазина\inst{2}
 } % обязательное поле

% Аффилиации пишутся в следующей форме, соединяя каждый институт при помощи \and.
\institute{УУНиТ, Уфа, Россия \\
  \email{Regina.mutigullina@mail.ru}
  \and
  УУНиТ, Уфа, Россия \\
\email{reginaufa@yandex.ru}
  % \and Другие авторы...
}

\maketitle

\begin{abstract}
При разложении функции распределения Вейбулла в ортогональный ряд экспонент коэффициенты представляют собой равномерно сходящиеся интегралы с параметром.

\keywords{функция Вейбулла, ортогональный ряд, равномерная сходимость} % в конце списка точка не ставится
\end{abstract}

%\section{Основные результаты} % не обязательное поле

Рассмотрим функцию распределения Вейбулла
$$
f(t)=\frac{k}{r}\left(\frac{t}{r}\right)^{k-1}\cdot e^{-\left(\frac{t}{r}\right)^k},
$$
с параметрами $r=1$, $k=1,5$.


В работе получено разложение данной функции в ортогональный ряд
$$
f(t)=\sum_{l=1}^q a_l\phi_l(t),
$$ 
где ортогональные функции $\phi_l(t)$ выражаются в виде
$$
\phi_l(t)=\sum_{i=1}^l\lambda_{li}\cdot e^{-(i-1)\beta t},\quad l=1,2,\ldots.
$$ 

Будем считать, что $\beta=1$.
Коэффициенты $a_l$ ряда определяются так
$$
a_l=\int_0^{\infty}e^{-t}f(t)\phi_l(t)dt,
$$
а именно
$$
a_l=\frac{3}{2}\sum_{i=1}^l\lambda_{li}\int_0^{\infty}\sqrt{t}e^{-t\sqrt{t}-it}dt.
$$

Интеграл $\int_0^{\infty}\sqrt{t}e^{-t\sqrt{t}-it}dt$ равномерно сходится. Задавая границы интегрирования от $0$ до $5$ вычислим интегралы при $i=1,2,3,4,5,6$ методом прямоугольников, методом трапеций и методом Симпсона.
% В конце текста можно выразить благодарности, если этого не было
% сделано в ссылке с заголовка статьи, например,
%Работа выполнена при поддержке РФФИ (РНФ, другие фонды), проект \textnumero~00-00-00000.
%

\begin{thebibliography}{9} % или {99}, если ссылок больше десяти.
\bibitem{Ilin} Ильин~В.~А. Необходимые и достаточные условия базисности в $L^p$ и равносходимости с тригонометрическим рядом спектральных разложений по системе экспонент // Докл. АН СССР, 1983. Том~273. \textnumero~4. С.~789--793.

\bibitem{Xabib} Хабибуллин~Б.~Н. Полнота систем экспонент и множества единственности.  Уфа:~ РИЦ БашГУ, 2011.


\bibitem{KrTer}	Красичков-Терновский~И.Ф., Шилова~Г.Н. Абсолютная полнота систем экспонент на выпуклых компактах // Матем. сб., 2005. V.~196. \textnumero~12. С.~85--98.

\bibitem{Leo}	Леонтьев~А.Ф. О полноте системы экспонент на кривой // Сиб. матем. журнал, 1974. Т.~15. \textnumero~5. С.~1103--1114.

\bibitem{Leo2}	Леонтьев~А.Ф. Ряды экспонент. М.:~Наука, 1976.
\end{thebibliography}
% После библиографического списка в русскоязычных статьях необходимо оформить
% англоязычный заголовок.




%\end{document}
