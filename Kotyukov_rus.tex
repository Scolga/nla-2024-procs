\begin{englishtitle} % Настраивает LaTeX на использование английского языка
% Этот титульный лист верстается аналогично.
\title{Equilibrium as a coincidence point of two mappings}
% First author
\author{Alexander Kotyukov\inst{1}  \and  Natalya Pavlova\inst{2}
}
\institute{ICS RAS, Moscow, Russia\\
  \email{amkotyukov@mail.ru}
  \and
ICS RAS, Moscow, Russia\\
RUDN University, Moscow, Russia\\
MPTI, Dolgoprydniy, Russia
\email{natasharussia@mail.ru}}
% etc

\maketitle

\begin{abstract}
The report is dedicated to a research on equilibrium in complex systems using the results from the theory of covering mappings and coincidence points which is a extension of fixed point theory. Existence conditions for equilibrium in open market model are obtained. Equilibrium in this model is considered as a coincidence point for covering and Lipschitz-continuous mappings.

\keywords{coincidence point, covering mapping, equilibrium, supply, demand} % в конце списка точка не ставится
\end{abstract}
\end{englishtitle}


\iffalse
%%%%%%%%%%%%%%%%%%%%%%%%%%%%%%%%%%%%%%%%%%%%%%%%%%%%%%%%%%%%%%%%%%%%%%%%
%
%  This is the template file for the 6th International conference
%  NONLINEAR ANALYSIS AND EXTREMAL PROBLEMS
%  June 25-30, 2018
%  Irkutsk, Russia
%
%%%%%%%%%%%%%%%%%%%%%%%%%%%%%%%%%%%%%%%%%%%%%%%%%%%%%%%%%%%%%%%%%%%%%%%%


\documentclass[12pt]{llncs}

% При использовании pdfLaTeX добавляется стандартный набор русификации babel.

\usepackage{iftex}

\ifPDFTeX
\usepackage[T2A]{fontenc}
\usepackage[utf8]{inputenc} % Кодировка utf-8, cp1251 и т.д.
\usepackage[english,russian]{babel}
\fi

\usepackage{todonotes}

\usepackage[russian]{nla}

\begin{document}
\fi

\title{Положение равновесия как точка совпадения \\ двух отображений\thanks{Работа выполнена при поддержке РНФ, проект \textnumero~22-11-00042, https://rscf.ru/project/22-11-00042/, в ИПУ РАН.}}
% Первый автор
\author{А.М. Котюков\inst{1}    \and    Н.Г. Павлова\inst{2}
}

% Аффилиации пишутся в следующей форме, соединяя каждый институт при помощи \and.
\institute{ИПУ РАН, Москва, Россия \\
  \email{amkotyukov@mail.ru}
  \and   % Разделяет институты и присваивает им номера по порядку.
ИПУ РАН, Москва, Россия\\
РУДН, Москва, Россия\\
МФТИ, Долгопрудный, Россия
  \email{natasharussia@mail.ru}
% \and Другие авторы...
}

\maketitle

\begin{abstract}
Доклад посвящен исследованию равновесия в сложных системах с помощью результатов теории накрывающих отображений и точек совпадения, которые являются обобщением понятия неподвижной точки. Получены условия существования положения равновесия в модели открытого рынка. Положение равновесия рассмотрено как точка совпадения накрывающего и липшицевого отображений.

\keywords{точка совпадения, накрывающее отображение, равновесие, спрос, предложение} % в конце списка точка не ставится
\end{abstract}

\section{Основные результаты} % не обязательное поле

В исследовании различных моделей систем большую роль занимает вопрос об устойчивости системы. Под устойчивостью понимается способность системы возвращаться в исходное состояние после какого-либо изменения, вызванного либо самой системой, либо внешним воздействием. Устойчивость тесно связана с понятием равновесия системы. Равновесием называется способность системы поддерживать свое существование без внешнего воздействия. Часто при исследовании вопроса о существовании равновесия возникают системы неявных алгебраических или дифференциальных уравнений. Прежде всего интересно получить условия существования решений для этих систем, затем найти само решение. В докладе предложен подход к решению этих задач, основанный на использовании результатов теории накрывающих отображений и точек совпадения. Данный подход продемонстрирован для исследования вопроса о положении равновесия в математической модели рынка.

Пусть на рынке присутствует $n\in\mathbb{N}$ товаров, цены на которые описываются вектором $p\in\mathbb{R}^n, \ p=(p_1,...,p_n)$ таким, что $p_i\in[c_{1i},...,c_{2i}], i =\overline{1,n}$, где $c_1=(c_{11},...,c_{1n}), c_2 = (c_{21},...,c_{2n})\in\mathbb{R}: 0<c_{1i}<c_{2i}, i =\overline{1,n}$. Обозначим $P = [c_{11},...,c_{21}]\times...\times[c_{1n},c_{2n}]$. В рассматриваемой модели отображения спроса и предложения $D,S:P\to\mathbb{R}^n_+ = \left\{x=(x_1,...,x_n)\in\mathbb{R}^n \ | \ x_i > 0 \forall i = \overline{1,n}\right\}$ имеют специальный вид. Вектор $p\in P$ называется положением равновесия в модели открытого рынка, если
$$
S(p) = D(p).
$$



Работа выполнена при поддержке РНФ, проект \textnumero~22-11-00042,  https://rscf.ru/project/22-11-00042/, в ИПУ РАН.
%



\begin{thebibliography}{9} % или {99}, если ссылок больше десяти.
\bibitem{JMS23} Kotyukov A.M., Pavlova N.G. Equilibrium in Dynamic Market Models with Known Elasticity // Journal of Mathematical Sciences. 2023. 269. Pp. 847-852.

\bibitem{UBS24} Котюков А.М., Павлова Н.Г. Алгоритм поиска точек совпадения в сложных системах // Управление большими системами. 2024. Вып.107. С. 6-27.

\bibitem{MLSD23} Kotyukov A.M., Pavlova N.G. Equilibrium in Open Market Models with Nonconstant Elastisities / Proceedings of the 16th International Conference Management of Large-Scale System Development (MLSD). М.: IEEE, 2023. С. https://ieeexplore.ieee.org/document/10303986.


\end{thebibliography}

% После библиографического списка в русскоязычных статьях необходимо оформить
% англоязычный заголовок.




%\end{document}
