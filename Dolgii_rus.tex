\begin{englishtitle}
\title{Construction of approximations of optimal impulse stabilizing control for a delayed system}
% First author
\author{Y.F. Dolgii, A.N. Sesekin}
\institute{IMM UB RAS, Ekaterinburg, Russia\\
  \email{yurii.dolgii@imm.uran.ru}
  \and
UrFU, Ekaterinburg, Russia\\
\email{a.n.sesekin@urfu.ru}}
% etc

\maketitle

\begin{abstract}
The problem of optimal impulse stabilization for an autonomous linear system of differential equations with delay is considered. A method for constructing approximations for optimal impulse stabilizing control is proposed.

\keywords{optimal stabilization, impulse controls, retard}
\end{abstract}
\end{englishtitle}

\iffalse
\documentclass[12pt]{llncs}
\usepackage[T2A]{fontenc}
\usepackage[utf8]{inputenc}
\usepackage[english,russian]{babel}
\usepackage[russian]{nla}

%\usepackage[english,russian]{nla}

% \graphicspath{{pics/}} %Set the subfolder with figures (png, pdf).

%\usepackage{showframe}
\begin{document}
%\selectlanguage{russian}
\fi

\title{Построение приближений оптимального импульсного стабилизирующего управления для системы с запаздыванием}
% Первый автор
\author{Ю.~Ф.~Долгий   \and  А.~Н.~Сесекин } % обязательное поле
\institute{ИММ УрО РАН, Екатеринбург, Россия\\
  \email{yurii.dolgii@imm.uran.ru}
  \and
УрФУ, Екатеринбург, Россия\\
\email{a.n.sesekin@urfu.ru}}
% Другие авторы...

\maketitle

\begin{abstract}
Рассматривается задача оптимальной импульсной стабилизации для автономной линейной системы дифференциальных уравнений с запаздыванием. Предлагается метод построения  приближений для оптимального импульсного стабилизирующего управления.

\keywords{оптимальная стабилизация, импульсные управления,  запаздывание}
\end{abstract}

Объект управления описывается автономной
линейной системой дифференциальных уравнений с запаздыванием и импульсными управлениями
\begin{equation}\label{eq1}
\frac{dx(t)}{dt}=A_{0}x(t)+A_{\tau}x(t-\tau)+Bu.
\end{equation}
Здесь $t\in\mathbb{R}^{+}=(0,+\infty)$,  $x:[-\tau,+\infty)\rightarrow\mathbb{R}^{n}$, положительное число $\tau$ --- запаздывание; $A_{0}, A_{\tau}$ --- матрицы размерности $n\times n$, $B$ --- матрица размерности $n\times r$. Импульсные управления являются обобщенными функциями, определяемыми формулами  $u(t)=\frac{dv(t)}{dt},\,t\in \bar{\mathbb{R}}^{+},$ в которых импульсы управления $v: [0, +\infty)\rightarrow\mathbb{R}^{r}$ имеют ограниченные вариации на любом конечном отрезке положительной полуоси,  $v(0)=0$.

Требуется найти импульсное управление, формируемое по принципу  обратной связи, которое обеспечивает устойчивую работу системы
~\eqref{eq1}  и минимизирует заданный критерий качества переходных процессов
\begin{equation}\label{eq2}
 J=\int\limits_{0}^{+\infty}\left( x^{\top}(t)C_{x}x(t)+v^{\top}(t)C_{v}v(t)\right)\,dt,
\end{equation}
где  $C_{x}$, $C_{v}$  --- положительно определенные матрицы.

В работе \cite{1} получена система уравнений, определяющая непосредственно коэффициенты оптимального импульсного стабилизирующего управления. Найдены асимптотические решения регуляризованной задачи оптимальной стабилизации автономной линейной системы с запаздыванием для  импульсных управлений с вырожденным критерием качества \cite{2}. 

 В настоящей работе  изучаются конечномерные аппроксимации задачи оптимальной импульсной стабилизации автономной линейной системы с запаздыванием. Для управляемой системы с запаздыванием ~\eqref{eq1} рассматривается конечномерная аппроксимирующая система обыкновенных дифференциальных уравнений
\begin{equation}\label{eq8}
\frac{dx_{0}^{N}}{dt}=A_{0}x_{0}^{N}+A_{\tau}x_{N}^{N}+Bu(t),
\end{equation}
\begin{equation*}
\frac{dx_{k}^{N}}{dt}=\frac{N}{\tau}\left(x_{k-1}^{N}-x_{k}^{N}\right),\quad k=\overline{1,N},\,N\geq 3.
\end{equation*}
Для аппроксимирующей управляемой импульсной системы используем показатель качества переходных процессов
\begin{equation}\label{eq9}
 J^{N}=\int\limits_{0}^{+\infty}\left( x_{0}^{N\top}(t)C_{x}x_{0}^{N}(t)+v^{\top}(t)C_{v}v(t)\right)dt,\,N\geq 3.
\end{equation}
Применяя замены
\begin{equation*}
y_{0}^{N}(t)=x_{0}^{N}(t)-Bv(t),\,y_{k}^{N}(t)=x_{k}^{N}(t),\,t\in \mathbb{\bar{R}}^{+},\,k=\overline{1,N},\,N\geq 3,
\end{equation*}
переходим от конечномерной задачи импульсной стабилизации  ~\eqref{eq8}, ~\eqref{eq9}  к вспомогательной задаче  не импульсной стабилизации для автономной линейной системы обыкновенных дифференциальных уравнений.

Построены  приближения   оптимального импульсного стабилизирующего управления для  автономной линейной системы дифференциальных уравнений с запаздыванием.  Предложена система алгебраических уравнений для нахождения коэффициентов  приближенных импульсных управлений. Размерность этой системы существенно меньше размерности матричного уравнения Риккати.






% Список литературы.
\begin{thebibliography}{99}
% Format for Russian Journal Reference
\bibitem{1}
Долгий Ю.Ф. {\it Оптимальная импульсная стабилизация линейных автономных систем с запаздыванием}. Материалы международной конференции «Устойчивость, управление, дифференциальные игры (SCDG2019)», посвященной 95-летию со дня рождения академика Н.Н. Красовского. Екатеринбург: ИММ УрО РАН. 2019. С.  119—122.
\bibitem{2}
Долгий Ю.Ф., Сесекин А.Н.  {\it Исследование регуляризации вырожденной задачи импульсной стабилизации системы с запаздыванием}. Тр. Ин-та математики и механики УрО РАН. 2022. Т.~28, №~1. С.~74--95.
% etc
\end{thebibliography}






%\end{document}

%%% Local Variables:
%%% mode: latex
%%% TeX-master: t
%%% End:
