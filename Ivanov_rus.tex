\begin{englishtitle} % Настраивает LaTeX на использование английского языка
% Этот титульный лист верстается аналогично.
\title{On the stability of solutions of differential equations with a holomorphic right-hand side}
% First author
\author{Gennady Ivanov   \and  Gennady  Alferov  \and  Vladimir  Korolev 
}
\institute{St. Petersburg University, St.Petersburg, Russia\\
  \email{guennadi.ivanov@gmail.com}}
%  \and
% St. Petersburg University, St.Petersburg, Russia\\
%\email{ g.alferov@spbu.ru }
%\and
% St. Petersburg University, St.Petersburg, Russia\\
%\email{ v.korolev@spbu.ru}}
% etc

\maketitle

\begin{abstract}
The paper investigates the system of differential equations with holomorphic
the right part. 

%Keywords: differential equations, asymptotic sustainability

\keywords{differential equations, asymptotic sustainability} % в конце списка точка не ставится
\end{abstract}
\end{englishtitle}


\iffalse
\documentclass[12pt]{llncs}

% При использовании pdfLaTeX добавляется стандартный набор русификации babel.

\usepackage{iftex}

\ifPDFTeX
\usepackage[T2A]{fontenc}
\usepackage[utf8]{inputenc} % Кодировка utf-8, cp1251 и т.д.
\usepackage[english,russian]{babel}
\fi

\usepackage{todonotes}

\usepackage[russian]{nla}


\begin{document}
\fi

\title{Об  устойчивости решений дифференциальных уравнений с  голоморфной правой частью }
% Первый автор
\author{Г.~Г.~Иванов  \and    Г.~В.~Алферов  \and  В.~С.~Королев
} % обязательное поле

% Аффилиации пишутся в следующей форме, соединяя каждый институт при помощи \and.
\institute{Санкт-Петербургский государственный университет, Санкт-Петербург, Россия
}

\maketitle

\begin{abstract}
В работе исследуется система дифференциальных уравнений с голоморфной правой частью. Показано, что если нулевое решение системы автономных дифференциальных уравнений с голоморфной правой частью асимптотически устойчиво, то и нулевое решение системы уравнений с полиномиальной правой частью, где полином представляет собой сумму нескольких первых членов из голоморфного ряда исходной системы, будет асимптотически устойчивым.

\keywords{дифференциальные уравнения, асимптотическая устойчивость} % в конце списка точка не ставится
\end{abstract}

\section{Основные результаты} % не обязательное поле

Основной результат настоящего предложения состоит в следующем:
если нулевое решение системы автономных дифференциальных уравнений с голоморфной правой частью асимптотически устойчиво,
то и нулевое решение системы автономных дифференциальных уравнений с
полиномиальной правой частью, где полином представляет собой сумму
достаточно большого числа первых членов ряда, стоящего в правой части исходной системы,
 также будет асимптотически устойчивым.

Пусть правая часть уравнения
$$
\dot x=f(x),\quad x\in R^n,\eqno 1
$$
представляется в виде сходящегося в окрестности нуля ряда
$$
f_1(x)+f_2(x)+\dots +f_k(x)+\dots,
$$
в котором $f_k(x)$  ---  однородная  степени  $k$  форма  относительно
переменных $x$ с постоянными коэффициентами.

Теорема 1. Если решение   $x\equiv   0$   уравнения   (1)   асимптотически
устойчиво, то существует $l\geq 1$ такое, что
решение $x\equiv 0$ уравнения
$$
\dot x =g(x) =f_1(x)+f_2(x)+\dots +f_l(x) \eqno 2
$$
также будет асимптотически устойчивым.

Доказательство.
По предположению нулевое решение системы (1) асимптотически устойчиво.
Тогда существует окрестность нуля $A$, которая для системы (1)
будет областью  асимптотической устойчивости, и в силу
теоремы об области асимптотической устойчивости [1, \S 4] для
этой области будут существовать отрицательно определенная
функция $v(x)$, $-1 < v(x) < 0$, и положительно определенная функция $w(x)$, $w(x) > \alpha > 0$
при $\|x\| > \beta > 0$, связанные на решениях системы (1) соотношением
$$\frac{dv(x)}{dt} = w(1 +v).\eqno 3$$

Соотношение (3) запишем в виде
$$\frac{\partial v(x)}{\partial x}f(x) = w(1 +v(x). \eqno 4$$
Для некоторого $l$ определим функцию $h(x)$ равенством
$$h(x) =\sum_{k =l}^\infty f_k((x).$$
Тогда соотношение (4) можно записать так:
$$\frac{\partial v(x)}{\partial x}(g(x) +h(x)) =w(1 +v)$$
Отсюда ясно, что на решениях системы (2) будет выполняться соотношение
$$\frac{dv(x)}{dt} =w(1 +v) -\frac{\partial v(x)}{\partial x}h(x). \eqno 5$$

Из свойств функций $v(x)$ и $w(x)$ ясно, что, увеличивая значение
параметра $l$, можно добиться, чтобы второе слагаемое из правой
части соотношения (5) не меняло знак первого слагаемого. Но тогда
В силу теоремы об асимптотической устойчивости [2] нулевое решение
системы (2) будет асимптотически устойчивым, что и доказывает справедливость утверждения теоремы.

Замечания. Из доказательства теоремы следует, что:

1. увеличение значения параметра $l$ не будет влиять на устойчивость
системы (2). Поэтому для любого $m > l$ нулевое решение системы (2),
в которой $l$ заменено на $m$, будет также асимптотически устойчивым;

2. для проверки  асимптотической  устойчивости систем (1) и (2)
можно использовать одну и ту же функцию Ляпунова.

Из приведенных замечаний с очевидностью следует справедливость следующего
утверждения.

Теорема 2. Для того чтобы нулевое решение системы (1) было асимптотически устойчивым, необходимо и достаточно, чтобы существовало такое $l$, при
котором нулевое решение системы (2) будет асимптотически устойчивым.


% кавычки << >>.

%

\begin{thebibliography}{9} % или {99}, если ссылок больше десяти.
\bibitem{Kl} Ляпунов А.М. Общая задача об устойчивости движения. М.-Л., Гостехиздат, 1950.
\bibitem{A1} Иванов Г.Г., Алферов Г.В., Ефимова П.А. Условия устойчивости линейных однородных систем с переключениями  //  Вестник Пермского университета. Математика. Механика. Информатика. 2016. Вып.3(34). С. 37--48
\bibitem{A2} Иванов Г.Г., Алферов Г.В., Королев В.С.  Теорема об области асимптотической устойчивости и ее приложения   // Вестник Пермского университета. Математика. Механика. Информатика. 2022. Вып.1 (56). С. 5--13
\bibitem{A3} Alferov G., Ivanov G., Efimova P., Structural study of limited invariant sets of relay stabilized (Book Chapter). 2017. Mechanical systems: Research, Applications and Technology. Pp. 101--164.
\bibitem{K2} Kadry S., Alferov G., Ivanov G., Korolev V. Investigation of the stability of solutions of systems of ordinary differential  equations.     2020, AIP Conference  Proceedings  ICNAAM,  2293, 060004
\bibitem{K3} Kadry S., Alferov G.,  Korolev V. Shymanchuk D. Mathematical  Models of Control  Processes and Stability In Problems of Mechanics // (2022) AIP Conference Proceedings 2425, 080004.
\bibitem{A4} Alferov G.V., Korolev V.S., Polyakhova E.N., Kholshevnikov K.V. Modeling Problems of Dinamics and Development of Scientific Areas of Mechanics and Applied Mathematics // Vestnik St. Petersburg University: Mathematics. 2023. Vol. 56. no. 4. Pp. 470--477.
\end{thebibliography}

% После библиографического списка в русскоязычных статьях необходимо оформить
% англоязычный заголовок и аннотацию.




%\end{document} 
