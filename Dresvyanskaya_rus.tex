\begin{englishtitle} % Настраивает LaTeX на использование английского языка
% Этот титульный лист верстается аналогично.
\title{Application of piece-wise linear support functions in linear bilevel programming problems}
% First author
\author{N.~V.~Dresvyanskaya}

\institute{Melentiev Energy Systems Institute SB RAS, Irkutsk, Russia\\
  \email{nadyadresvyanskaya@gmail.com}
}
% etc

\maketitle

\begin{abstract}
The linear bilevel programming problem in the optimistic formulation is considered. 
It is reduced to a single-level linear programming problem with an additional inverse-convex constraint. 
A local search method based on the application of piece-wise linear convex support functions is proposed for solving the single-level problem. 
The convergence of the proposed method to the local minimum is established. Results of computational experiments are given.

\keywords{bilevel optimization, local search, support functions} % в конце списка точка не ставится
\end{abstract}
\end{englishtitle}


\iffalse
%%%%%%%%%%%%%%%%%%%%%%%%%%%%%%%%%%%%%%%%%%%%%%%%%%%%%%%%%%%%%%%%%%%%%%%%
%
%  This is the template file for the 6th International conference
%  NONLINEAR ANALYSIS AND EXTREMAL PROBLEMS
%  June 25-30, 2018
%  Irkutsk, Russia
%
%%%%%%%%%%%%%%%%%%%%%%%%%%%%%%%%%%%%%%%%%%%%%%%%%%%%%%%%%%%%%%%%%%%%%%%%


\documentclass[12pt]{llncs}  

% При использовании pdfLaTeX добавляется стандартный набор русификации babel.

\usepackage{iftex}

\ifPDFTeX
\usepackage[T2A]{fontenc}
\usepackage[utf8]{inputenc} % Кодировка utf-8, cp1251 и т.д.
\usepackage[english,russian]{babel}
\fi

\usepackage{todonotes} 
\usepackage{amssymb}

\usepackage[russian]{nla}

\begin{document}
\fi

\title{Применение кусочно-линейных опорных функций в задачах линейного двухуровневого программирования}
% Первый автор
\author{Н.~В.~Дресвянская}

% Аффилиации пишутся в следующей форме, соединяя каждый институт при помощи \and.
\institute{Институт систем энергетики им. Л.~А.~Мелентьева СО РАН (ИСЭМ СО РАН), Иркутск, Россия \\
  \email{nadyadresvyanskaya@gmail.com}
}

\maketitle

\begin{abstract}
В работе рассматривается задача линейного двухуровневого программирования в оптимистической постановке и 
осуществляется ее редукция к одноуровневой задаче линейного программирования с дополнительным обратно-выпуклым ограничением. 
Для решения одноуровневой задачи предлагается метод локального поиска, основанный на применении кусочно-линейных выпуклых опорных функций. 
Обосновывается сходимость метода к точке локального минимума. Приводятся результаты вычислительных экспериментов.

\keywords{двухуровневая оптимизация, локальный поиск, опорные функции} % в конце списка точка не ставится
\end{abstract}

% \section{Основные результаты} % не обязательное поле

Задача двухуровневого линейного программирования формулируется следующим образом:
\begin{gather}
c_1^Tx+d_1^Ty\rightarrow\min_{x,y}, \qquad\qquad\qquad\quad \label{blp1}\\
\qquad\qquad\qquad\: A_1x+B_1y\leqslant r_1,\;x\in X\subset\mathbb{R}^{n_1},\qquad\qquad \label{blp2}\\
\quad c_2^Tx+d_2^Ty\rightarrow\min_y,\qquad\qquad \label{blp3}\\
\qquad\qquad\qquad\qquad A_2x+B_2y\leqslant r_2,\;y\in Y\subset\mathbb{R}^{n_2},\qquad \label{blp4}
\end{gather}
где $A_i$ -- $m_i\times n_1$ матрицы, $B_i$ -- $m_i\times n_2$ матрицы, $r_i\in\mathbb{R}^{m_i}$, $i=1,2$, $X$ и $Y$ --- многогранные множества. 
Введем функцию оптимального значения второго уровня
\begin{eqnarray}
\psi(x)&=&\min_{y\in Y}\{c_2^Tx+d_2^Ty: A_2x+B_2y\leqslant r_2\}=\nonumber\\
&=&c_2^Tx+\min_{y\in Y}\{d_2^Ty: B_2y\leqslant r_2-A_2x\}=c_2^Tx+\varphi(x),\label{blp5}\\
\varphi(x)&=&\min_{y\in Y}\{d_2^Ty: B_2y\leqslant r_2-A_2x\}=\min_{y\in Y(x)}d^T_2y,\label{blp6}\\
Y(x)&=&\{y\in Y: B_2y\leqslant r_2-A_2x\}.\label{blp7}
\end{eqnarray}

Запишем исходную двухуровневую задачу \eqref{blp1}-\eqref{blp4} в виде следующей одноуровневой, используя функцию оптимального значения $\psi$
\begin{equation}
c_1^Tx+d_1^Ty\rightarrow\min_{x,y},
\label{blp8}
\end{equation}
\begin{equation}
A_1x+B_1y\leqslant r_1,\;x\in X\subset\mathbb{R}^{n_1},
\label{blp9}
\end{equation}
\begin{equation}
A_2x+B_2y\leqslant r_2,\;y\in Y\subset\mathbb{R}^{n_2},
\label{blp10}
\end{equation}
\begin{equation}
c_2^Tx+d_2^Ty\leqslant \psi(x).
\label{blp11}
\end{equation}

Везде далее будем предполагать, что ограничения \eqref{blp9}-\eqref{blp10} совместны. В силу линейности ограничений, проверка этого предположения не составляет труда. 
Если ограничения несовместны, то очевидно задача \eqref{blp1}-\eqref{blp4} не имеет решения.

В силу \eqref{blp5} неравенство \eqref{blp11} можно переписать следующим образом:
\begin{equation}
d_2^Ty\leqslant \varphi(x).
\label{blp12}
\end{equation}

Слагаемое $c_2^Tx$ в целевой функции \eqref{blp3} задачи второго уровня никак не влияет ни на решение задачи второго уровня, ни на общее решение всей двухуровневой задачи. 
Поэтому, в дальнейшем можно полагать $c_2=0$.

Определим функцию $\psi$ следующим образом:
\begin{eqnarray}
\psi(x)&=&\sup_{u\geqslant 0}\min_{y\in Y}\left\{d^T_2y+u^T(A_2x+B_2y-r_2)\right\}=\nonumber\\
&=&\sup_{u\geqslant 0}\left\{\min_{y\in Y}\left[(B_2^Tu+d_2)^Ty\right]+(A_2x-r_2)^Tu\right\}.\label{blp13}
\end{eqnarray}

Очевидно, $\psi$ --- функция оптимального значения задачи, двойственной задаче \eqref{blp6}. 
Пусть множество $Y$ есть неотрицательный ортант, $Y=\mathbb{R}^{n_2}_{+}=\{y\in\mathbb{R}^{n_2}: y\geqslant 0\}$. Тогда задача \eqref{blp13} эквивалентна следующей
\begin{equation}
(A_2x-r_2)^Tu\rightarrow\max_u,
\label{du1}
\end{equation}
\begin{equation}
u\in U=\{u\in\mathbb{R}^{m_2}:B_2^Tu+d_2\geqslant 0, \; u\geqslant 0\}.
\label{du2}
\end{equation}

В этом случае допустимое множество двойственной задачи не зависит от $x$. 

В дальнейшем будем предполагать, что допустимое множество двойственной задачи второго уровня \eqref{du1}-\eqref{du2} не пусто, $U\ne\emptyset$.

В условиях данного предположения $\varphi(x)=\psi(x)$, функцию оптимального значения задачи второго уровня будем далее обозначать $\varphi(x)$. 
Значение $\varphi(x)$ либо конечно, либо $\varphi(x)=+\infty$. Если $\varphi(x)=+\infty$, то ограничения прямой задачи \eqref{blp6} несовместны \cite{FPV_I_LP} и задача второго уровня --- несобственная задача 1-го рода. 
Если $\varphi(x)$ конечно, решение двойственной задачи \eqref{du1}-\eqref{du2} достигается в вершине $U$, прямая задача \eqref{blp6} также разрешима. Пусть $w_1,\ldots,w_P$ --- вершины множества $U$, тогда
\begin{equation}
\varphi(x)=\begin{cases}
+\infty, & Y(x)=\emptyset;\\
\max\limits_{1\leqslant i\leqslant P}w_i^T(A_2x-r_2), & Y(x)\ne\emptyset.
\end{cases}
\label{ObjDu}
\end{equation}

Идея локального поиска реализует общую схему из \cite{Khamisov} и состоит в следующем. Необходимо сначала построить функцию \eqref{ObjDu}, затем для каждого $i=1,\ldots,P$ решить задачу \eqref{blp8}-\eqref{blp10} 
с дополнительным ограничением $d^T_2y\leqslant w_i^T(A_2x-r_2)$. При таком подходе определяется локальное решение задачи двухуровневого программирования.

В докладе приводятся результаты численных экспериментов и обоснование сходимости метода к точке локального минимума.

\begin{thebibliography}{9} % или {99}, если ссылок больше десяти.
\bibitem{FPV_I_LP} Васильев~Ф.~П., Иваницкий~А.~Ю. Линейное программирование.  М.: Изд-во <<Факториал>>,~1998.  176 c.
\bibitem{Khamisov} Khamisov~O. Quadratic support functions in quadratic bilevel problems~// in: Operations Research Proceedings 2017. Springer International Publishing, 2018. P.~105--110.
\end{thebibliography}

% После библиографического списка в русскоязычных статьях необходимо оформить
% англоязычный заголовок.




%\end{document}

