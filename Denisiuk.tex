

\iffalse
\documentclass[12pt]{llncs}

\usepackage{todonotes}

\usepackage{nla}

\begin{document}
\fi

\title{Asymptotic properties of solutions to a system of nonlinear ordinary differential equations\thanks{The work is supported by the Mathematical Center in Akademgorodok under
agreement No. 075-15-2022-282 with the Ministry of Science and Higher
Education of the Russian Federation.}}

\author{Viktor A. Denisiuk
}
\institute{Novosibirsk State University, Novosibirsk, Russia\\
  \email{v.denisyuk@g.nsu.ru}}

\maketitle

\begin{abstract}
A system of nonlinear ordinary differential equations of large dimension is considered. We investigate asymptotic properties of solutions
to the system in dependence on the growth of the number of the equations. We prove that, for sufficiently large number of differential
equations, the last component of the solution to the Cauchy problem is an
approximate solution to an initial problem for one delay differential equation.

\keywords{system of ordinary differential equations of large dimension, asymptotic
properties of solutions, delay differential equation}
\end{abstract}

 %optional section
In this paper we consider the Cauchy problem for a system of nonlinear ordinary differential equations of the form
$$
%\left\{
%\begin{array}{l}
\displaystyle\frac{dz}{dt}=B_nz+f(t,z_n),\qquad 
       z|_{t=0}=z_0,
%\end{array}
%\right.
\eqno(1)
$$%
where $B_n$ is a constant matrix whose entries depend on $n$ and parameters. Here $f(t,z_n)\in C(\overline{\mathbb R}^2_+)$ is a bounded function and satisfies Lipschitz condition with respect to the second argument. Such systems appear in multistage synthesis models (see, for example, [1]), where $z_n(t)$ describes consentration of the final synthesis product. Since number of stages $n$ can be very large, then a problem of ``large dimension'' occurs  when finding $z_n(t)$.
\par In this paper we study properties of solutions to (1) for $n\gg 1$. Using methods proposed by G.V. Demidenko (see, for instance, [2,\ 3]), we prove closeness of $z_n(t)$ and a function $y(t)$  for $n\gg 1$, where $y(t)$ is a solution to an initial value problem for a delay differential equation
$$
\displaystyle\frac{dy(t)}{dt}=g(t-\tau, y(t-\tau)).
\eqno(2)
$$%
This paper is a continuation of our research of systems of nonlinear ordinary differential equations of large dimensions [4].

% At the end of the text, acknowledgments are expressed, if you haven't
% made a footnote from the title. For example, we can write

\begin{thebibliography}{3} % or {99}, if there is more than ten references.
\bibitem{DemLik2004} Demidenko G. V., Kolchanov N. A., Likhoshvai V. A., Matushkin Yu. G., Fadeev S. I. Mathematical modeling of regular contours of gene networks. Comput. Math. Math. Phys.~2004. Vol.~44, no.~12. Pp.~2276--2295.

\bibitem{LikFadDem2004}  Likhoshvai V. A., Fadeev S. I., Demidenko G. V., Matushkin Yu. G. Modeling nonbranching multistage synthesis by an equation with retarded argument. Sib. Zh. Ind. Mat.~2004. Vol.~7,  no.~1. Pp.~73--94.

\bibitem{Dem2012} Demidenko G. V. Systems of differential equations of higher dimension and delay equations. Siberian Math. J.~2012. Vol.~53, no.~6. Pp.~1021--1028.

\bibitem{DenMat2023} Denisiuk V. A., Matveeva I. I. Properties of solutions to one class of nonlinear systems of differential equations with a parameter. Chelyabinsk Physical and Mathematical Journal.~2023. Vol.~8, no.~4. Pp.~483--501.

\end{thebibliography}
%\end{document}
