

\iffalse
%%%%%%%%%%%%%%%%%%%%%%%%%%%%%%%%%%%%%%%%%%%%%%%%%%%%%%%%%%%%%%%%%%%%%%%%
%
% This is the template file for the 6th International conference
% NONLINEAR ANALYSIS AND EXTREMAL PROBLEMS
% June 25-30, 2018
% Irkutsk, Russia
%
%%%%%%%%%%%%%%%%%%%%%%%%%%%%%%%%%%%%%%%%%%%%%%%%%%%%%%%%%%%%%%%%%%%%%%%%
% The preparation of the article is based on the standard llncs class
% (Lecture Notes in Computer Sciences), which is adjusted with style
% file of the conference.
%
% There are two ways of compilation of the file into PDF
% 1. Use pdfLaTeX (pdflatex), (LaTeX+DVIPS will not work);
% 2. Use LuaLaTeX (XeLaTeX will work too).
% When using LuaLaTeX You will need TTF or OTF CMU fonts
% (Computer Modern Unicode). The fonts are installed with 'cm-unicode' package in
% a distribution of LaTeX % (https://www.ctan.org/tex-archive/fonts/cm-unicode),
% either by downloading and installing these fonts system wide, the address of their page is
% http://canopus.iacp.dvo.ru/%7Epanov/cm-unicode/
% The second option won't work in XeLaTeX.
%
% For MiKTeX (LaTeX distribution for Windows),
%  1. Package 'cm-unicode' is installed manually with the MiKTeX administration Console.
%  2. For the compilation of this example, namely, the stub figure, one will also need to
% download package 'pgf' manually. This package uses in the popular
% package tikz.
%  3. Tests showed that the rest of the required packages MiKTeX loads automatically (if
%     it is allowed). The 'auto download' option is
%     configured in 'Settings' section in MiKTeX Console.
%
%
% The easiest way to compile an article is to use pdfLaTeX, but
% the final layout of the book will be compiled with LuaLaTeX,
% as a result will be of better quality thanks to the package 'microtype' and
% use vector OTF instead of standard raster fonts of pdfLaTeX.
%
% In the case of questions and problems with the article compilation,
% write letters to e-mail: eugeneai@irnok.net, Cherkashin Evgeny.
%
% New version of the correcting style file will be available at the website:
%     https://github.com/eugeneai/nla-style
%     file - nla.sty
%
% Further instructions are in the text body of the template. The template itself
% is an article example.
%
% The LaTeX2e format is used!

% 12 points font size is used.
\documentclass[12pt]{llncs}

% The correcting style file is added.
\usepackage{todonotes}

\usepackage{nla} % This package is needed for compiling
                 % this template, it should be removed
                 % from your article.

% Many popular packages (amsXXX, graphicx, etc.) are already imported in the style file.
% If there is a conflict with your packages, try disabling them and compile
% the text.
%
% It would be convenient in the layout of the proceedings if the file names
% of the figures of different authors do not clash.
% To minimize the clash, the drawings can be placed in a separate subfolder
% named after the author or the title of the paper.
%
% \graphicspath{{ivanov-petrov-pics/}} % specifies the folder with images in png, pdf formats.
% or
% \graphicspath{{great-problem-solving-paper-pics/}}.

\begin{document}
\fi
% Text should be formatted in accordance with the 'article' class, using extensions like
% AMS.
%
\title{Method of equivalent control for discontinuous systems with delay\thanks{The work was carried out within the framework of the state task of the Ministry of Education and Science of Russia, project \textnumero~1210401300060-4.}}
\author{I.~A.~Finogenko }

\institute{ Matrosov Institute of System Dynamics and Control Theory SB~RAS, Irkutsk, Russia\\
	\email{fin@icc.ru}
}


\maketitle

\begin{abstract}
	General problems that arise when studying a system with a discontinuous right-hand side in the presence of delay are discussed. A detailed study describes the set of discontinuity points of the right-hand sides and the sliding mode method by the equivalent control method using invariantly differentiable functionals. Solutions of the given values are understood in the sense of A.F. Filippov, as solutions of functional-differential inclusions.
%Insert your english abstract here. Include 3-6 keywords below.

\keywords{functional-differential equation with discontinuous right-hand part, sliding mode, functional-differential inclusion, invariantly differentiable functional, equivalent control}
%\keywords{keyword, another keyword, \emph{etc.}}
\end{abstract}

% at the end of the list, there should be no final dot
\section{Introduction}
Functional differential equations are considered
\begin{equation}\label{Finogenko-eq1}
	\dot{x}=f(t,x_{t}(\cdot)), \,x_{t_{0}}(\cdot)=\phi_{0}(\cdot)
\end{equation}
where $f:R^{1}\times C_{\tau}\rightarrow R^{n}$ is a discontinuous function, $R^{n}$ is a $n$-dimensional vector space with norm $ \|\cdot\|$, $C_\tau$ --- the space of all continuous functions $\phi(\cdot)$ defined on the interval $[-\tau, 0]$ with
values in $R^n$ with the usual sup-norm
${\|\phi(\cdot)\|_C = \sup_{-\tau \leq \theta \leq 0} \| \phi(\theta) \|}$, $x_{t}(\theta)=x(t+\theta)$, $\tau \geq 0$ --- an arbitrary real number.

For systems without delay, it is convenient to represent the set of discontinuity points on the right-hand side of the equation as sets of zero Lebesgue measure. This allows the use of properties of Lebesgue measurable functions such as approximate continuity
\cite{Finogenko-second}.
In particular, $M$ can consist of
finite number of hypersurfaces. The movement along the intersection of the hypersurfaces of the discontinuity points is called the sliding mode. In our studies, the function $f$ is discontinuous on the set of points $M\subset R^{1}\times C_{\tau}$.

Movements of the system
(\ref{Finogenko-eq1})
over the set $M$, by analogy with discontinuous systems without delay, we will also call sliding modes. Although for sets $M$ in a function space this term may not always be justified. For example, motion through a set of functions $M=\{\phi(\cdot):\phi(-\tau)=0\}$ is completely determined by the initial function and does not depend on the right-hand side of the equation
(\ref{Finogenko-eq1}).

For discontinuous systems with delay, the structure of the set of discontinuity points on the right-hand sides of the equations can turn out to be much more complex. But we can single out one general property of  this set,  if we want to use  the theory of differential inclusions to describe solutions.  This property is boundary, i.e. the complement of $M$ must be everywhere
dense or, equivalently, the set $M$ must have an empty interior. 
Then for each point $(t,\phi(\cdot))\in M$ the set of limit values ??of the function $f(t',\phi'(\cdot))$ will be correctly defined for $(t', \phi'(\cdot))\rightarrow (t,\phi(\cdot))$, $(t',\phi'(\cdot))\not\in M$.
The convex, closed hull of all limit values ??of the function $f$ at the point
 $(t,\phi(\cdot))$ we denoteby $F(t,\phi(\cdot))$.
 
 The solution of equation
 (\ref{Finogenko-eq1})
 with initial function $x_{t_{0}}(\cdot)=\phi_{0}(\cdot)$
 is a continuous function $x(t)$ defined on 
 $[t_{0}-\tau,t_{1}]$, ${t_{1}>t_{0}}$, absolutely continuous on the segment $[t_{0},t_{1}]$
 and for almost all $t\in [t_{0},t_{1}]$ satisfying the differential inclusion
 \begin{equation}
 	\label{Finogenko-eq2}
 	\dot{x}(t)\in F(t,x_{t}(\,\cdot\,)).
 \end{equation}

\section{The main results}

Questions arise: how to describe the set of $M$ points of discontinuity of the right-hand side of functional differential equations so that it is possible to obtain sliding mode equations? In particular, what would an analogue of the equivalent control method look like?

In 
\cite{Finogenko-first}
the theory of i-smooth analysis was developed, which, within the framework of the direct Lyapunov method, was successfully applied in problems of stability of functional differential equations. The theory is based on invariant derivatives and invariantly differentiable functionals $W(t,x,\phi(\cdot))$.
We note: these functionals can be effectively used to describe a set of discontinuity points
functions $f(t,\phi(\cdot))$
and for deriving sliding mode equations for systems with delay. 

Consider a controlled system
\begin{equation}
	\label{Finogenko-eq4}
	\dot{x}=f(t,x_{t}(\cdot),u_{1},\ldots,u_{k})
\end{equation}
under the assumption that the function $f$ is continuous, each control $u_{i}=u_{i}(t,\phi(\cdot))$ (scalar function) is discontinuous on its manifold
${M_{i}=\{(\phi(\cdot)): W_{i}(t,\phi(0),\phi(\cdot))=0\}}$
and $u_{i}^{-}$, $u_{i}^{+}$ --- limit values of the function $u_{i}$ on both sides of the manifold $M_{i}$, $i=1, \dots,k$. {\em Equivalent controls} $u_{i}^{eq}=u_{i}^{eq}(t,\phi(\cdot))$ are determined from the equations
$$
\langle \nabla_x W_{i}[p],f(t,\phi(\cdot),u_{1}^{eq},\ldots,u_{k}^{eq})\rangle + \partial_\phi W_{i}[p]+\partial_{t}W_{i}[p] \bigg|_{p=(t,\phi(0),\phi(\cdot))} =0,
$$
where $\langle \cdot,\cdot\rangle$ --- scalar product, $\nabla_x W_{i}$ --- gradient with respect to $x$, $\partial_{t}W_{i}$ --- partial derivative with respect to $t$, $\partial_\phi W_{i}$ --- invariant derivative with respect to $\phi(\cdot)$ for the functional $W_{i}(t,x,\phi(\cdot)) $.

{\bf Theorem 1.} {\it If each value $u^{eq}_{i}$ is contained in a segment with endpoints $u_{i}^{-}$, $u_{i}^{+}$ , then the sliding mode equation will be
	\begin{equation}
		\label{Finogenko-eq5}
		\dot{x}=f(t,x_{t}(\cdot),u^{eq}_{1},\ldots,u^{eq}_{k}).
	\end{equation}
}


In Theorem 1, the equivalent control method known for systems without delay (see \cite{Finogenko-first}) is extended
to the equation
(\ref{Finogenko-eq4}).


\begin{thebibliography}{9}	
	
		\bibitem{Finogenko-first} Kim A.V. i-Smooth Analysis. A Wiley -- Interscience Publication. Chichester -- New York -- Brisban, 2014. 
	
	\bibitem{Finogenko-second} Filippov~A.F. Differential equations with discontinuous right-hand side. 	
	M.: Nauka, 1985. 
	
\end{thebibliography}

%\end{document}

%%% Local Variables:
%%% mode: latex
%%% TeX-master: t
%%% End:
