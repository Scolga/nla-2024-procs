\begin{englishtitle} % Настраивает LaTeX на использование английского языка
% Этот титульный лист верстается аналогично.
\title{An asymptotic behaviour for increments of sums of independent random variables and stochastic processes}
% First author
\author{Aleksei Bogarev}
\institute{Saint Petersburg University, Saint Petersburg, Russia\\
  \email{abogarevs1997@mail.ru}
}
% etc

\maketitle

\begin{abstract}
The asymptotic behaviour of increments of a stochastically continuous homogeneous process with independent increments from a domain of normal attraction of an asymmetric stable law has been studied.

\keywords{stochastically continuous homogeneous process with independent increments, domain of normal attraction of an asymmetric stable law, Borel--Cantelly lemma} % в конце списка точка не ставится
\end{abstract}
\end{englishtitle}

\iffalse
%%%%%%%%%%%%%%%%%%%%%%%%%%%%%%%%%%%%%%%%%%%%%%%%%%%%%%%%%%%%%%%%%%%%%%%%
%
%  This is the template file for the 8th International conference
%  NONLINEAR ANALYSIS AND EXTREMAL PROBLEMS
%  June 24-28, 2024
%  Irkutsk, Russia
%
%%%%%%%%%%%%%%%%%%%%%%%%%%%%%%%%%%%%%%%%%%%%%%%%%%%%%%%%%%%%%%%%%%%%%%%%


\documentclass[12pt]{llncs}  

% При использовании pdfLaTeX добавляется стандартный набор русификации babel.

\usepackage{iftex}

\ifPDFTeX
\usepackage[T2A]{fontenc}
\usepackage[utf8]{inputenc} % Кодировка utf-8, cp1251 и т.д.
\usepackage[english,russian]{babel}
\fi

\usepackage{todonotes} 

\usepackage[russian]{nla}

\usepackage{cmap} % Поиск по документу

\usepackage{amsmath,amssymb}

\renewcommand{\P}{\mathbb{P}}
\newcommand{\Exp}{\mathbb{E}}
\newcommand{\Z}{\mathbb{Z}}


\theoremstyle{remark}
\newtheorem{rem}{Замечание}


\begin{document}
\fi

\title{Асимптотическое поведение приращений сумм независимых случайных величин и случайных процессов\thanks{Работа выполнена при поддержке РНФ, грант \textnumero~23-21-00078.}}
% Первый автор
\author{А.\,С.~Богарев}

% Аффилиации пишутся в следующей форме, соединяя каждый институт при помощи \and.
\institute{СПбГУ, Санкт-Петербург, Россия \\
  \email{abogarevs1997@mail.ru}
}

\maketitle

\begin{abstract}
Исследовано асимптотичекское поведение приращений стохастически непрерывного однородного процесса с независимыми приращениями из области нормального притяжения асимметричного устойчивого закона.

\keywords{стохастически непрерывный однородный процесс с независимыми приращениями, область нормального притяжения асимметричного устойчивого закона, лемма Бореля--Кантелли} % в конце списка точка не ставится
\end{abstract}

\section{Основные результаты} % не обязательное поле

Пусть $X, X_1, X_2, \ldots X_n, \ldots$ -- последовательность независимых одинаково распределённых случайных величин и $\{a_n\}_{n=1}^{+\infty}$~-- неубывающая последовательность натуральных чисел, $1\leqslant a_n\leqslant n$, $S_n = X_1 + \ldots + X_n$, $S_0=0\text{ п.н.}$

Изучению асимптотического поведения приращений сумм $S_{n+ca_n}-S_n$, где $c>0$, посвящены работы множества авторов, среди которых У. Штадтмюллер~\cite{LanStadt}, Х. Ланцингер~\cite{Lan, LanStadt}, А.\,Н. Фролов~\cite{FrolovArt1}, М.\,Н. Тертеров~\cite{Terterov}. Для этого в зависимости от вида $a_n$ ищется нормирующая последовательность $\{b_n\}_{n=1}^{+\infty}$, для которой справедливо
\[
\limsup_{n\to +\infty}\frac{S_{n+ca_n}-S_n}{b_n} = 1\text{\quad п.н.}
\]

Например, при $a_n=n$, $c=1$, $\Exp X=0$, $\Exp X^2=1$ справедлив известный закон повторного логарифма Хартмана--Винтнера, где $b_n = \sqrt{2n\ln\ln n}$.

При $a_n = (\ln n)^p$, где $p>1$, предельное поведение приращений $S_{n+ca_n}-~S_n$ изучалось в работах Тетрерова~\cite{Terterov} и Фролова~\cite{FrolovArt1} для величин из областей притяжения нормального закона и асимметричных устойчивых законов, а также Ланцингера~\cite{Lan} и Ланцингера и Штадтмюллера~\cite{LanStadt} для величин с конечной дисперсией.

Пусть функция распределения $F(x)$ случайной величины $X$ принадежит области нормального притяжения асимметричного устойчивого закона с параметром $\alpha\in(1,2)$ и $a_n=(\ln n)^p$, где $p>1$. Положим $c_n=(\ln n)^{(p+\alpha-1)/\alpha}$, $t_0=\sup\{t\geqslant 0\mid\Exp e^{t(X^{+})^{\alpha/(p+\alpha-1)}}<\infty\}$,
\[
\varphi(c)=max\left\{\,x+y\;\middle|\;\frac{(\alpha-1)x^{\alpha/(\alpha-1)}}{\alpha c^{1/(\alpha-1)}}+t_0y^{\alpha/(p+\alpha-1)}\leqslant 1,\,x\geqslant 0,\,y\geqslant 0\,\right\}.
\]

В~\cite{Terterov} Тертеровым был получен следующий результат.

\begin{theorem}

Пусть $t_0\in(0,+\infty)$. Тогда
\[
\limsup_{n\to +\infty}\frac{S_{n+ca_n}-S_n}{c_n\varphi(c)} = 1\text{\quad п.н.}
\]

\end{theorem}

\begin{rem}

Под $S_h$ при $h\notin\Z$ подразумевается $S_{[h]}$, где $[h]$ --- целая часть числа $h$.

\end{rem}

Поведение однородных процессов с независимыми приращениями во многом похоже на поведение сумм независимых одинаково распределённых случайных величин (см.~\cite{FrolovArt2}). Это обстоятельство наводит нас на мысль о возможности перенесения результатов о поведении приращений сумм случайных величин на приращения случайных процессов.

Пусть всюду далее $\xi(t)$, $t\geqslant 0$ --- стохастически непрерывный однородный процесс с независимыми приращениями, с вероятностью $1$ непрерывный справа и имеющий пределы слева, $\xi(0)=0 \text{ п.н.}$ Пусть также функция распределения случайной величины $\xi(1)$ принадлежит области нормального притяжения асимметричного устойчивого закона с параметром $\alpha\in(1,2)$.

Положим $a_t=(\ln t)^p$, где $p>1$, и $c_t=(\ln t)^{(p+\alpha -1)/\alpha}$ при $t\geqslant 0$. Определим также $t_0=\sup\{t\geqslant 0\mid\mathbb{E}e^{t(\xi(1)^{+})^{\alpha/(p+\alpha-1)}}<\infty\}$,
\[
\varphi(c)=max\left\{\,x+y\;\middle|\;\frac{(\alpha-1)x^{\alpha/(\alpha-1)}}{\alpha c^{1/(\alpha-1)}}+t_0y^{\alpha/(p+\alpha-1)}\leqslant 1,\,x\geqslant 0,\,y\geqslant 0\,\right\},
\]
\[M_t=\sup\limits_{0\leqslant s\leqslant ca_t}\bigl(\xi(t+s)-\xi(t)\bigr).
\]
Желание перенести результат Тертерова~\cite{Terterov} на случай стохастически непрерывных однородных процессов с независимыми приращениями приводит нас к следующему утверждению.

\begin{theorem} \label{main}

Пусть $t_0\in (0,+\infty)$. Тогда
\[
\limsup_{t\to +\infty}\frac{\xi(t+ca_t)-\xi(t)}{\varphi(c)c_t}=\limsup_{t\to +\infty}\frac{M_t}{\varphi(c)c_t}=1\text{\quad п.н.}
\]

\end{theorem}

Доказательству теоремы \ref{main} была посвящена выпускная квалификационная работа автора.

\begin{thebibliography}{99}

\bibitem{Terterov} Тертеров М.\,Н.  О предельном поведении приращений сумм независимых случайных величин из областей притяжения асимметричных устойчивых распределений // Вестник
Санкт-Петербургского университета. Математика. Механика. Астрономия. Сер. 1.   2011.   в. 2.   С. 95--103.

\bibitem{FrolovBookEn} A.\,N. Frolov. Universal theory for strong limit theorems of probability. Singapore: World Scientific, 2019.

\bibitem{FrolovArt1} A.\,N. Frolov. One-sided strong laws for increments of sums of i.i.d. random variables // Studia Scientiarum Mathematicarum Hungarica. 2002. Vol. 39, \textnumero 3--4. Pp. 333--359.

\bibitem{FrolovArt2} Фролов А.\,Н. Универсальные предельные теоремы для приращений процессов с независимыми приращениями // Теория вероятностей и её применения.  2004.  Т. 49, в. 3.   С. 601--609.

\bibitem{Lan} H. Lanzinger. A law of the single logarithm for moving averages of random variables under exponential moment conditions. // Studia Scientiarum Mathematicarum Hungarica. 2002. Vol. 36. Pp. 65--91.

\bibitem{LanStadt} H. Lanzinger, U. Stadtmüller. Maxima of increments of partial sums for certain subexponential distributions. // Stochastic Processes and their Applications. 2000. Vol. 86. Pp. 307--322.

\bibitem{FrolovBookRu1} Фролов А.\,Н. Предельные теоремы теории вероятностей. Учебное пособие. Издательство СПбГУ, 2014.

\bibitem{FrolovBookRu2} Фролов А.\,Н. Устойчивые распределения и суммы независимых случайных величин. Учебное пособие. Издательство Лань, 2021.

\bibitem{FrolovBookRu3} Фролов А.\,Н. Безгранично делимые распределения и суммы независимых случайных величин и случайных процессов. Учебное пособие. Издательство Лань, 2023.

\end{thebibliography}

 

% После библиографического списка в русскоязычных статьях необходимо оформить
% англоязычный заголовок.




%\end{document}

