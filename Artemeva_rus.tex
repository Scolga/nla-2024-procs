
%\pagebreak


\begin{englishtitle} % Настраивает LaTeX на использование английского языка
% Этот титульный лист верстается аналогично.
\title{On the blow up the solution of the Cauchy problem for a~$(3+1)$--dimensional heat-electric model}
% First author
\author{M.V. Artemeva\inst{1}  \and  M.O. Korpusov\inst{1,} \inst{2}
%  \and
%  Name FamilyName3\inst{1}
}
\institute{Lomonosov Moscow State University, Moscow, Russia\\
  \email{artemeva.mv14@physics.msu.ru}
  \and
Nikolskiy Mathemathical institute PFUR, Moscow, Russia\\
\email{korpusov@gmail.com}}
% etc

\maketitle

\begin{abstract}
A heat–electric $(3+1)$--dimensional model of semiconductor heating in an electric field is considered. For the corresponding Cauchy problem, the existence of a nonextendable classical solution is proved and a global time a priori estimate is obtained.

\keywords{nonlinear Sobolev type equations, blow-up, local solvability, nonlinear capacity, estimates of the blow up time} % в конце списка точка не ставится
\end{abstract}
\end{englishtitle}

\iffalse
%%%%%%%%%%%%%%%%%%%%%%%%%%%%%%%%%%%%%%%%%%%%%%%%%%%%%%%%%%%%%%%%%%%%%%%%
%
%  This is the template file for the 6th International conference
%  NONLINEAR ANALYSIS AND EXTREMAL PROBLEMS
%  June 25-30, 2018
%  Irkutsk, Russia
%
%%%%%%%%%%%%%%%%%%%%%%%%%%%%%%%%%%%%%%%%%%%%%%%%%%%%%%%%%%%%%%%%%%%%%%%%


\documentclass[12pt]{llncs}  

% При использовании pdfLaTeX добавляется стандартный набор русификации babel.

\usepackage{iftex}

\ifPDFTeX
\usepackage[T2A]{fontenc}
\usepackage[utf8]{inputenc} % Кодировка utf-8, cp1251 и т.д.
\usepackage[english,russian]{babel}
\fi

\usepackage{todonotes} 

\usepackage[russian]{nla}

\begin{document}
\fi

\title{О разрушении решения задачи Коши одной $(3+1)$--мерной тепло--электрической модели\thanks{Работа выполнена при поддержке РНФ, проект \textnumero~23-11-00056.}}
% Первый автор
\author{М.~В.~Артемьева\inst{1}   \and  М.~О.~Корпусов\inst{1,} \inst{2}
%  \and
% Третий автор
%  И.~О.~Фамилия\inst{1}
}

% Аффилиации пишутся в следующей форме, соединяя каждый институт при помощи \and.
\institute{Московский государственный университет им. М.~В. Ломоносова, Москва, Россия \\
  \email{artemeva.mv14@physics.msu.ru}
  \and   % Разделяет институты и присваивает им номера по порядку.
Математический институт им. С.М. Никольского РУДН, Москва, Россия\\
  \email{korpusov@gmail.com}
% \and Другие авторы...
}

\maketitle

\begin{abstract}
В работе рассматривается одна тепло–электрическая $(3+1)$--мерная модель нагрева полупроводника в электрическом поле. Для соответствующей задачи Коши  доказано  существование  непродолжаемого  во  времени  классического решения и получена глобальная во времени априорная оценка.

\keywords{нелинейные уравнения соболевского типа, разрушение, blow-up, локальная разрешимость, нелинейная емкость, оценки времени раз	рушения} % в конце списка точка не ставится
\end{abstract}

В настоящей работе рассматривается следующая задача Коши для модельного $(3+1)$--мерного уравнения~\cite{KorpAtrTMF}:
\begin{equation}\label{1.4-0-1}
	\dfrac{\partial}{\partial t}\left(\Delta_x u(x,t)+|D_x u(x,t)|^q\right)+\Delta_xu(x,t)=0,\quad u(x,0)=u_0(x),
\end{equation}
$$
\Delta_x:=\sum\limits_{j=1}^3\dfrac{\partial^2}{\partial x_j^2}.
$$
Данное уравнение описывает тепло--электрический разогрев полупроводника. 

При $q>3/2$ доказывается существование непродолжаемого решения, что с физической точки зрения означает возникновение электрического <<пробоя>>. Для малых начальных данных доказано существование глобального во времени решения задачи Коши и получена оценка убывания по времени.

% В конце текста можно выразить благодарности, если этого не было
% сделано в ссылке с заголовка статьи, например,
Работа   выполнена   при   финансовой   поддержке   Российского   научного   фонда, проект   \textnumero23-11-00056.
%Работа выполнена при поддержке РФФИ (РНФ, другие фонды), проект ~00-00-00000.
%



\begin{thebibliography}{9} % или {99}, если ссылок больше десяти.
\bibitem{KorpAtrTMF} Артемьева М.В., Корпусов М.О. О разрушении решения одной (1+1)--мерной тепло–электрической модели //~ТМФ. 2024. Т.~219, \textnumero~2. С.~249--262.


\end{thebibliography}

% После библиографического списка в русскоязычных статьях необходимо оформить
% англоязычный заголовок.




%\end{document}

