\begin{englishtitle} % Настраивает LaTeX на использование английского языка
% Этот титульный лист верстается аналогично.
\title{Extremal properties of the reachable set boundary and the Lagrange principle}
% First author
\author{M.~I.~Gusev
}
\institute{ Krasovskii Institute of Mathematics and Mechanics, Ekaterinburg, Russia\\
  \email{gmi@imm.uran,ru}
}
% etc

\maketitle

\begin{abstract}
 The reachable set of a control system at a given time instant can be considered as an image of the set of admissible controls into  the state space under a nonlinear mapping. The article discusses some properties of  reachable sets, the main attention is paid  to the description  of the set boundary.

\keywords{nonlinear mapping, reachable set, boundary point, extremum conditions } % в конце списка точка не ставится
\end{abstract}
\end{englishtitle}

\iffalse
\documentclass[12pt]{llncs}

% При использовании pdfLaTeX добавляется стандартный набор русификации babel.

\usepackage{iftex}

\ifPDFTeX
\usepackage[T2A]{fontenc}
\usepackage[utf8]{inputenc} % Кодировка utf-8, cp1251 и т.д.
\usepackage[english,russian]{babel}
\fi

\usepackage{todonotes}

\usepackage[russian]{nla}

\begin{document}
\fi
\newtheorem{theo}{Теорема}

\title{Экстремальные свойства границы множества достижимости и принцип Лагранжа}
% Первый автор
\author{М.~И.~Гусев}


% Аффилиации пишутся в следующей форме, соединяя каждый институт при помощи \and.
\institute{Институт Математики и механики им. Н.Н.Красовского УрО РАН, Еrатеринбург, Россия \\
  \email{gmi@imm.uran.ru}  }

\maketitle

\begin{abstract}
Множество достижимости управляемой системы  в заданный момент времени можно рассматривать как образ множества допустимых управлений при нелинейном отображении в пространство состояний.  В докладе обсуждаются некоторые свойства множеств достижимости, основное внимание уделено описанию границы множеств.

\keywords{нелинейное отображение, множество достижимости, граничная точка, условия экстремума} % в конце списка точка не ставится
\end{abstract}




% Рисунки и таблицы оформляются по стандарту класса article. Например,

Множество достижимости (МД)  управляемой системы, описываемой дифференциальным  или  разностным уравнением,  можно  представить  в виде
 $G=\big\{y \in Y: y=F(u),\; u \in U \big\}$, где $U$ -- множество допустимых управлений, а $F$ -- непрерывное отображение, сопоставляющее  входу системы (управлению)  ее выход (вектор состояния в заданный момент времени).  Далее мы считаем, что  $U\subset X$, где $X$ -- действительное банахово пространство, а $Y=\mathbb R^n$.  В качестве $U$ будем рассматриваить множества вида $U=\{u:\varphi_i(u)\leq 1, i=1,...,m\}$, где $\varphi_i(u)$ -- непрерывные функционалы.  В таком виде можно записать многие геометрические и интегральные ограничения на управление, в том числе комбинацию таких ограничений.  В докладе рассматривается  задача характеризации граничных точек множества достижимости.

  Если управляемая система одномерна ($n=1$)  и множество $U= \{u:\varphi(u)\leq 1\}$  связно, то граничными точками МД могут быть  только величины максимума и минимума функции $F$ на $U$.  Условия экстремума для данных экстремальных задач  можно получить, используя функцию  Лагранжа $L(u, y, \lambda)=y F(u) +\lambda \varphi(u)$ ($y,\lambda$ - множители Лагранжа). Заметив, что $F(u)$ и $\varphi(u)$ входят в выражение для $L$ симметрично, можно попытаться поменять их роли, сделав $\varphi(u)$ целевой функцией, а $F(u)$ ограничением. Такая замена возможна и в общем случае, что утверждается следующей теоремой.

 \begin{theo}\label{MT} Пусть  $W$ -- окрестность множества  $U= \{u:\varphi(u)\leq 1\}$,  $\varphi (u)$ -- непрерывный функционал,  отображение   $F:W \to Y$ непрерывно дифференцируемо по Фреше в точке $\hat{u}\in U$ и выполнено условие $\mathrm{Im}\, F'(\hat{u})=Y.$ Для того, чтобы  $\hat{x}=F(\hat{u})\in \partial G$ необходимо, чтобы  $\hat{u}$ доставлял локальный минимум в задаче
 	\begin{equation}\label{pr1} \varphi(u)\to \min, \quad F(u)=\hat{x} \end{equation} и выполнялось равенство $\varphi (\hat{u})=1$.
 \end{theo}
Приравнивая нулю градиент функции Лагранжа  задачи (\ref{pr1}): $\varphi' (\hat{u})+F'^{*}(\hat{u})y^*=0$ ,  получим
 \begin{theo}\label{tm3} Пусть выполнены условия теоремы \ref{MT}, $\varphi(u)$ непрерывно дифференцируем по Фреше в точке $\hat{u}$ и $\varphi(\hat{u}) \neq 0$. Тогда найдется  $y^* \in \mathbb R^n$, $\|y^*\|=1$ такой, что  $\hat{u}$  удовлетворяет  необходимым условиям оптимальности для задачи  \begin{equation}\label{nc4} \langle y^*, F(u) \rangle\to \min, \quad \varphi (u) \leq 1. \end{equation}  \end{theo}
 Последнее утверждение нетрудно переформулировать для случая, когда   $\varphi(u) $ локально липшицева в точке $\hat{u}$. Заметим, что задача (\ref{nc4}) эквивалентна вычислению значения опорной функции МД $G$ в точке $-y^*$.

 Для управляемой системы вида \begin{equation}
 	\dot{x}(t)=f_1(t,x(t))+f_2(t,x(t))u(t),\quad t\in[t_0, t_1],  \quad x(t_0)=x_0, \quad u(\cdot)\in U.
 	\label{1s6}
 \end{equation}
 при не очень обременительных предположениях о правой части дифференциального уравнения отображение $F(u(\cdot))=x(t_1,u(\cdot))$ непрерывно дифференцируемо в $\mathbb L_k, r>1$,  и его производная Фреше вычисляется по формуле
 $ 	F^{\prime}(u(\cdot ))\delta u(\cdot )=\delta x(t_1) \label{(13)}$, $\delta x(t)$ -- решение линеаризованной системы $\dot{\delta x}(t)=A(t) \delta x(t)+B(t)\delta u(t), \quad\delta x(t_0)=0$. С учетом  равенства $F'^{*}(u(\cdot))y^*=B^\top(\cdot)\,p(\cdot)$, где  $\dot{p}(t)=-A^\top (t)p(t),\quad p(t_1)=y^*$ и приведенных выше утверждений по единой схеме  доказывается условие оптимальности в форме принципа максимума  Понтрягина для  управлений, приводящих на границу МД $G(t_1)$ \cite{Lee67, GZ18,PTF23} как для случая интегральных, так и выпуклых геометрических ограничений на управление.
Результаты   переносятся на случай нескольких ограничений типа неравенства и разнотипных ограничений на управление \cite{G23}.






\begin{thebibliography}{9} % или {99}, если ссылок больше десяти.
\bibitem{Lee67}
Lee~E.B., Marcus~L.Foundations of Optimal Control Theory.
New York:~J. ~Willey and Sons Inc.,~1967.\
\bibitem{GZ18}
Gusev~M.\,I., Zykov~I.\,V. On extremal properties of the boundary points of reachable sets for control systems with integral constraints.Proc. Steklov Inst. Math. 2018. Vol.~300. Suppl.~1. Pp.~114--125.
\bibitem{PTF23}
Пацко~В.С., Трубников~Г.И., Федотов~А.А. Множество достижимости машины Дубинса с интегральным ограничением на управление. МТИП. 2023. Vol.~5. No.~2.  Pp.~89–-104.
\bibitem{G23}
Gusev~ M.I. Computing the Reachable Set Boundary for an Abstract Control System: Revisited. Ural Math. J. 2023. Vol.~ 9. No~2. Pp.~99--108.

\end{thebibliography}

% После библиографического списка в русскоязычных статьях необходимо оформить
% англоязычный заголовок.




%\end{document}
