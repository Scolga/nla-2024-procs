\begin{englishtitle}
\title{Problems of control and optimal control by laser action on a two-layer material}
% First author
\author{Dariya Larionova}
\institute{Irkutsk State University, Irkutsk, Russia\\
\email{lardar24@yandex.ru}}
% etc

\maketitle

\begin{abstract}
This talk deals with a multilayer object subjected to laser irradiation. The object  consists of two biological layers which are heterogeneous in their thermophysical characteristics. Methods for constructing control functions and optimal control of the thermal effect of a laser action on the two-layer biomaterial are discussed.

\keywords{thermal conductivity, control, optimal control, two-layer biomaterial}
\end{abstract}
\end{englishtitle}

\iffalse
\documentclass[12pt]{llncs}
\usepackage[T2A]{fontenc}
\usepackage[utf8]{inputenc}
\usepackage[english,russian]{babel}
\usepackage[russian]{nla}

%\usepackage[english,russian]{nla}

% \graphicspath{{pics/}} %Set the subfolder with figures (png, pdf).

%\usepackage{showframe}
\begin{document}
%\selectlanguage{russian}
\fi

\title{Задачи управления и оптимального управления лазерным воздействием на двухслойный материал%\thanks{Работа выполнена при поддержке РФФИ (РНФ, другие фонды), проект \textnumero~00-00-00000.}
}
% Первый автор
\author{Д.~В.~Ларионова%\inst{1}
%  \and
% Второй автор
%  И.~О.~Фамилия\inst{2}
%  \and
% Третий автор
%  И.~О.~Фамилия\inst{2}
} 
\institute{Институт математики и информационных технологий\\ Иркутский государственный университет, Иркутск, Россия\\
  \email{lardar24@yandex.ru}
  }


\maketitle

\begin{abstract}
В докладе рассматривается многослойный объект, состоящий из
двух неоднородных по своим теплофизическим характеристикам
биологических слоев. Этот объект подвергается лазерному воздействию. Обсуждаются методы построения 
функций управления и оптимального управления тепловым воздействием
лазерного луча на двухслойный биоматериал. 

\keywords{уравнение теплопроводности, управление, оптимальное управление, двухслойный биоматериал}
\end{abstract}

%\section{Основные результаты} % не обязательное поле




Управление процессом лазерного воздействия на границе
неоднородного по своим физическим характеристикам биоматериала
осуществляется изменением интенсивности температуры лазерного
воздействия, влияя на тепловое состояние в биологической среде. В докладе предложен
конструктивный подход построения функции управления и оптимального
управления тепловым воздействием лазерного излучения на двухслойный биоматериал.

Предполагается,
что управление и оптимальное управление процессом теплового воздействия
лазерного луча осуществляется следующим образом: изменяя по времени на
верхней (левой) границе двухслойного биоматериала интенсивность
температуры лазерного луча, мы влияем на тепловое состояние в
биологическом субстрате. Для задач оптимального управления критерий
качества задан на всем промежутке времени. Цель управления состоит в изменении температуры из заданного начального состояния   в желаемое конечное состояние.

\iffalse
% Список литературы.
\begin{thebibliography}{99}
\bibitem{1}
% Format for Journal Reference:
Author1 N., Author2 N.  Article title. Journal. Year. Vol.~Volume, No~Number. Pp.~Page numbers.
% Format for books:
\bibitem{2}
Author N. Book title. Place: Publisher, year.
% Format for Russian Journal Reference:
\bibitem{3} Фамилия И.О. Название статьи~// Журнал. Год. Т.~том,  \textnumero~номер. С.~страницы.
% etc
\end{thebibliography}
\fi

% Обязательная информация на английском языке:




%\end{document}
