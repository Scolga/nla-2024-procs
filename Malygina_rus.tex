\begin{englishtitle} % Настраивает LaTeX на использование английского языка
% Этот титульный лист верстается аналогично.
\title{On sharp two-sided estimates for stable solutions to~differential equations with delay}
% First author
\author{Vera Malygina
%  \and
%  Name FamilyName2\inst{2}
%  \and
%  Name FamilyName3\inst{1}
}
\institute{Perm National Research Polytechnic University, Perm, Russia\\
  \email{tlsabatulina@list.ru}
%  \and
%Affiliation, City, Country\\
%\email{email}
}
% etc

\maketitle

\begin{abstract}
For linear autonomous differential equations with distributed delay that have a positive fundamental solution, we propose a method for obtaining new tests of exponential stability and two-sided estimates of the fundamental solution in the form of two exponential functions with exact efficiently calculated exponents and coefficients.

\keywords{autonomous functional differential equations, fundamental solution, exponential stability, sharp upper exponent} % в конце списка точка не ставится
\end{abstract}
\end{englishtitle}


\iffalse
\documentclass[12pt]{llncs}

% При использовании pdfLaTeX добавляется стандартный набор русификации babel.

\usepackage{iftex}

\ifPDFTeX
\usepackage[T2A]{fontenc}
\usepackage[utf8]{inputenc} % Кодировка utf-8, cp1251 и т.д.
\usepackage[english,russian]{babel}
\fi

\usepackage{todonotes}

\usepackage[russian]{nla}


\begin{document}
\fi
\title{О точных двусторонних оценках устойчивых решений дифференциальных уравнений с запаздыванием\thanks{Работа выполнена при поддержке Министерства науки и высшего образования Российской Федерации, проект \textnumero~FSNM-2023-0005.}}

\author{В.~В.~Малыгина  % \inst ставит циферку над автором.
} % обязательное поле

% Аффилиации пишутся в следующей форме, соединяя каждый институт при помощи \and.
\institute{Пермский национальный исследовательский политехнический университет (ПНИПУ), Пермь, Россия\\
  \email{mavera@list.ru}
}

\maketitle

\begin{abstract}
Для линейных автономных дифференциальных уравнений с распределенным запаздыванием, имеющих положительное фундаментальное решение, предлагается метод получения новых признаков экспоненциальной устойчивости и двусторонних оценок фундаментального решения в виде двух экспоненциальных функций с точными эффективно вычисляемыми показателями и коэффициентами.


\keywords{авнономные функционально-дифференциальные уравнения,  фундаментальное решение, экспоненциальная устойчивость, точный верхний показатель} % в конце списка точка не ставится
\end{abstract}


Определение экспоненциальной устойчивости линейного дифференциального уравнения с последействием обобщает классическое определение экспоненциальной устойчивости для обыкновенного дифференциального уравнения и предполагает существование констант $N,\gamma>0$ таких, что для каждого решения $x\colon[ t_0,\infty)\to\mathbb{R }$ справедлива оценка $|x(t)|\le Ne^{-\gamma(t-t_0)}\|\varphi\|$, где $\varphi $~--- начальная функция, определяющая решение.
Для уравнений с последействием задача оценки констант $N$ и $\gamma$ нетривиальна даже для случая скалярных уравнений; тем не менее, решать ее необходимо, поскольку без указания оценок для (или эффективного алгоритма их вычисления) проблему экспоненциальной устойчивости нельзя считать полностью решенной. 



Рассмотрим автономное функционально-дифференциальное уравнение
\begin{equation*}
\dot x(t)+ax(t)+\int\limits_{0}^{h}x(t-s)\,dr(s)=f(t),\quad t\ge0,\eqno (1)
\end{equation*}
где $a\in\mathbb{R}$, $h>0$, $r\colon[0,h]\to\mathbb{R}$~--- функция ограниченной вариации, $r(0)=0$, интеграл понимается в смысле Римана--Стилтьеса, а $f$~--- локально интегрируемая функция.
Назовем {\it фундаментальным решением} уравнения~(1) функцию $x_0$, являющуюся решением уравнения~(1) при $f(t)\equiv0$ и $x_0(0)=1$.
Как известно~[1], любое решение уравнения~(1) выражается через фундаментальное решение по формуле

$$
 x(t)=x_0(t)x(0) +\int\limits_{0}^{t}x_0(t-s)f(s)\,ds,\quad t\ge0.
$$

Для экспоненциально устойчивых уравнений вида~(1)  предлагается эффективный метод получения двусторонних оценок фундаментального решения.

Метод позволяет с произвольной точностью найти как показатель, так и коэффициент экспоненциальной оценки решения.
Основу метода составляет априорное предположение о положительности фундаментального решения с последующим полным  описанием его свойств на полуоси.



Обозначим $F(\lambda)=\lambda+a+\int\limits_{0}^{h}e^{-\lambda s}\,dr(s)$, $\lambda\in\mathbb{R}$ .

{\bf Теорема 1 [2].}
{\it Пусть функция $r$ не убывает на $[0,h]$.
Тогда, если для некоторого вещественного $\omega>0$ выполняются условия $F(-\omega)=0$, $F'(-\omega)>0$, то фундаментальное решение уравнения~{\rm (1)} имеет двустороннюю оценку
$$e^{-\omega t}\le x_0(t)\le \frac{1}{F'(-\omega)}e^{-\omega t}.$$}

{\bf Замечание.} Постоянные 1 и $\frac{1}{F'(-\omega)}$ точные, т.к. $x_0(0)=1$, a $\lim\limits_{t \to \infty}x_0(t)e^{\omega t}=\frac{1}{F'(-\omega)}$.

{\bf Теорема 2 [2].}
{\it Пусть функция $r$ не возрастает на $[0,h]$.
Тогда, если для некоторого вещественного $\omega>0$ выполнено условие $F(-\omega)=0$, то фундаментальное решение уравнения~{\rm (1)} имеет двустороннюю оценку
$$e^{(\omega-a)h}e^{-\omega t}\le x_0(t)\le e^{-\omega t},$$
и, кроме того,  $\lim\limits_{t \to \infty}x_0(t)e^{\omega t}=\frac{1}{F'(-\omega)}$.}


% В конце текста можно выразить благодарности, если этого не было
% сделано в ссылке с заголовка статьи, например,

% Список литературы оформляется подобно ГОСТ-2008.
% Примеры оформления находятся по этому адресу -
%     https://narfu.ru/agtu/www.agtu.ru/fad08f5ab5ca9486942a52596ba6582elit.html

\begin{thebibliography}{9} % или {99}, если ссылок больше десяти.
\bibitem{bib_amr_sab} Азбелев~Н.В., Максимов~В.П., Рахматуллина~Л.Ф. Введение в теорию функ\-ци\-о\-наль\-но-дифференциальных уравнений. М.:~Наука,~1991.

\bibitem{bib_MCh_sab} Малыгина~В.В., Чудинов~К.М. О точных двусторонних оценках устойчивых решений автономных функционально-дифференциальных уравнений~// Сиб. матем. журн. 2022. Т.~63. \textnumero~2. С.~360--378.

\end{thebibliography}

% После библиографического списка в русскоязычных статьях необходимо оформить
% англоязычный заголовок.



%\end{document}