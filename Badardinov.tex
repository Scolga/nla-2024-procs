
\iffalse
\documentclass[12pt]{llncs}

\usepackage{todonotes}

\usepackage{nla} 

\begin{document}
\fi

\title{Existence and uniqueness of solutions to polyharmonic equations in weighted Sobolev spaces}

\author{Dzhamil Badardinov}
\institute{Novosibirsk State University, Novosibirsk, Russia\\
  \email{d.badardinov@g.nsu.ru}}
  
\maketitle

\begin{abstract}

The existence and uniqueness of solutions of an equation 
$$
\Delta^m\, u = f(x), \; x\in \mathbb{R}^n,
\eqno (1)
$$
are considered in the class of weighted Sobolev spaces \(W^{2m}_{p,\, \sigma}(\mathbb{R}^n)\).

\keywords{polyharmonic operator, weighted Sobolev spaces}
\end{abstract}

\section*{The main results} %optional section

\textsc{Definition.} A function \(u(x)\) belongs to weighted Sobolev space \(W^{2m}_{p,\, \sigma}(\mathbb{R}^n)\) if there exist generalized derivatives \(D^{\beta}_x u(x), \, |\beta| \leq 2m\) of \(u(x)\) in \(\mathbb{R}^n\), and
\[\left\|(1 + | x |)^{-\sigma(2m - |\beta|)} D_x^{\beta} u(x),\, L_p(\mathbb{R}^n)\right\| < \infty.\]

The norm in the space \(W^{2m}_{p,\, \sigma}(\mathbb{R}^n)\) is defined as
\[\left\|u(x),\,W^{2m}_{p,\sigma}(\mathbb{R}^n)\right\| = \sum\limits_{0 \leq |\beta| \leq 2m}
\left\|(1 + | x |)^{-\sigma(2m - |\beta|)} D_x^{\beta} u(x),\, L_p(\mathbb{R}^n)\right\|.\]

The following results are established.

\textbf{Theorem 1.}
\textsl{If \(\sigma > \frac{n}{2mp}\), \(j \leq 2m - 1 ,\; \frac{n}{2mp} + \frac{j}{2m} < \sigma \leq 1, \;\)
or
\(\;\sigma \geq 1,\, \sigma > \frac{n}{(2m-j)p},\) then the solution is defined up to a polynomial \(P_j(x)\) of degree not greater than j. 
}

\textbf{Theorem 2.} 
\textsl{If \(n>2m,\, \sigma = \frac{n}{2mp}\), then \(\forall\, f(x) \in L_p(\mathbb{R}^n),\) supp f is compact, \(\exists!\, u(x) \in W^{2m}_{p,\,\sigma}(\mathbb{R}^n)\) -- a solution of the equation (1), and the estimate takes place:
\[
\left\|u(x),\, W^{2m}_{p, \,\sigma}(\mathbb{R}^n)\right\|
\leq c\left\|f(x), L_p(\mathbb{R}^n)\right\|,
\]
where \(c = c(supp\, f)\).}

\textbf{Theorem 3.}
\textsl{If \(n > 2m - (N+1),\; \sigma \in [0, 1-\frac{n}{2mp'}],\, N \geq 0\) -- a natural number, such that \(1-\frac{n}{2mp'} - \frac{N+1}{2m} < \sigma \leq 1- \frac{n}{2mp'} - \frac{N}{2m}\), then the conditions 
\[\int\limits_{\mathbb{R}^n}\, x^{\beta} f(x) \,dx = 0\; \scriptstyle \forall \beta:\; |\beta| \,\leq\, N,\]
 are necessary and sufficient for the existence of solutions \(u(x) \in W^{2m}_{p, \,\sigma}(\mathbb{R}^n)\) of the equation (1) \(\forall\, f(x) \in L_p(\mathbb{R}^n),\) supp f is compact; if the conditions are fulfilled, then there exist a unique solution of the equation (1) and the estimate takes place: 
 \[\left\| u(x),\, W^{2m}_{p, \,\sigma}(\mathbb{R}^n)\right\| \leq c \left\|f(x),\, L_p(\mathbb{R}^n) \right\|,\]
 where \(c = c\,(\,supp\; f).\)}

The research method is based on using the fundamental solution of the
quasielliptic operator \(L(D_x)\), which was derived from the construction of approximate solutions of the equation (1) based on the method of integral representation of functions $f(x) \in L_p(R^n)$ suggested by S.V.~ Uspenskii~[1].

Similar questions for quasielliptic operators have been considered
in the works~[2]--[5].

% At the end of the text, acknowledgments are expressed, if you haven't
% made a footnote from the title. For example, we can write

\begin{thebibliography}{9} % or {99}, if there is more than ten references.

\bibitem{Uspenskii1972} Uspenskii S.\,V. On representation of functions determined by a certain class of hypoelliptic operators. Trudy Mat. Inst. Steklov Akad. Nauk SSSR.~1972. Vol.~117. Pp.~292--299.

\bibitem{Demidenko1993} Demidenko G.\,V. Integral operators determined by quasielliptic equations. I.
Sibirsk. Mat. Zh.~1993. Vol.~34, no.~6, Pp.~52--67.

\bibitem{Demidenko1998} Demidenko G.\,V. On quasielliptic operators in $R^n$.
Sibirsk. Mat. Zh.~1998. Vol.~39, no.~5, Pp.~884--893.

\bibitem{McOwen1979} McOwen R.\,C. The behavior of the Laplacian on weighted Sobolev spaces.
Comm. Pure Appl. Math.~1979. Vol.~32, no~6 Pp.~783--795.

\bibitem{Hile G.} Hile G. N. Fundamental Solutions and Mapping Properties of Semielliptic Operators. 
Math. Nachrichten.~2006. Vol.~279, no~13-14. Pp.~1538--1572.

\end{thebibliography}

%\end{document}