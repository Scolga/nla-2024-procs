\begin{englishtitle} % Настраивает LaTeX на использование английского языка
% Этот титульный лист верстается аналогично.
\title{Practical optimization in non-convex optimal\\ control problems}
% First author
\author{Alexander Gornov, Tatiana Zarodnyuk, Anton Anikin,\\ Pavel Sorokovikov, Alexander Tyatyushkin}
\institute{Matrosov Institute for System Dynamics and Control Theory of SB RAS, Irkutsk, Russia\\
  \email{gornov@icc.ru, tz@icc.ru, anikin@icc.ru, pavel@sorokovikov.ru, tjat@icc.ru}
}
% etc

\maketitle

\begin{abstract}
A brief overview of the developed software for studying optimal control problems is presented. The authors' accumulated experience of numerical solution of applied extreme problems is described. The classes of problems that can be solved using the developed algorithms are highlighted. Using the implemented software tools, a number of applied problems from the fields of flight dynamics, space navigation, robotics, electric power, seismology, economics, ecology, chemistry, medicine and nanophysics have been successfully solved. The talk discusses the software properties and presents the results of numerical experiments.

\keywords{optimal control problems, software, numerical optimization methods} % в конце списка точка не ставится
\end{abstract}
\end{englishtitle}


\iffalse
%%%%%%%%%%%%%%%%%%%%%%%%%%%%%%%%%%%%%%%%%%%%%%%%%%%%%%%%%%%%%%%%%%%%%%%%
%
%  This is the template file for the 6th International conference
%  NONLINEAR ANALYSIS AND EXTREMAL PROBLEMS
%  June 25-30, 2018
%  Irkutsk, Russia
%
%%%%%%%%%%%%%%%%%%%%%%%%%%%%%%%%%%%%%%%%%%%%%%%%%%%%%%%%%%%%%%%%%%%%%%%%


\documentclass[12pt]{llncs}

% При использовании pdfLaTeX добавляется стандартный набор русификации babel.

\usepackage{iftex}

\ifPDFTeX
\usepackage[T2A]{fontenc}
\usepackage[utf8]{inputenc} % Кодировка utf-8, cp1251 и т.д.
\usepackage[english,russian]{babel}
\fi

\usepackage{todonotes} 

\usepackage[russian]{nla}

\begin{document}
\fi

\title{Практическая оптимизация в невыпуклых задачах оптимального управления\thanks{
	Работа выполнена за счет субсидии Минобрнауки России в рамках проекта <<Теория и методы исследования эволюционных уравнений и управляемых систем с их приложениями>> (\textnumero~гос. регистрации: 121041300060-4).}}
% Первый автор
\author{А.~Ю.~Горнов, Т.~С.~Зароднюк, А.~С.~Аникин, П.~С.~Сороковиков., А.~И.~Тятюшкин}

% Аффилиации пишутся в следующей форме, соединяя каждый институт при помощи \and.
\institute{Институт динамики систем и теории управления имени В.М. Матросова СО РАН, Иркутск, Россия\\
  \email{gornov@icc.ru, tz@icc.ru, anikin@icc.ru, pavel@sorokovikov.ru, tjat@icc.ru}
}

\maketitle

\begin{abstract}
Представлен краткий обзор разработанных программных комплексов для исследования задач оптимального управления. Описан накопленный авторами опыт численного решения прикладных экстремальных задач. Выделены классы задач, которые могут быть решены с использованием разработанных алгоритмов. С применением реализованных программных средств успешно решен ряд прикладных задач из областей динамики полета, космонавигации, робототехники, электроэнергетики, сейсмологии, экономики, экологии, химии, медицины и нанофизики. В докладе обсуждаются свойства программных комплексов и приводятся результаты численных экспериментов.

\keywords{задачи оптимального управления, программные комплексы, численные методы оптимизации} % в конце списка точка не ставится
\end{abstract}

Разработка программных средств для исследования задач оптимизации проводилась в конце ХХ века параллельно в целом ряде российских научных организаций. Среди них нельзя не упомянуть пакет прикладных программ (ППП) ДИСО (<<Диалоговая система оптимизация>>, ВЦ РАН, руководитель Ю.Г. Евтушенко), пакет прикладных программ ПАОЭМ (<<Пакет анализа экономических моделей>>, рук. В.А. Скоков), программный комплекс (ПК) CONTROL (ИПМ РАН им. М.В. Келдыша, рук. Р.П. Федоренко), программные комплексы для задач линейного программирования (ВЦ РАН, рук. А.-И.А. Станевичус, ЦЭМИ РАН, рук. У.Х. Малков), ППП МАПР (<<Математическое программирование в многомерных задачах>>, Иркутский ВЦ ВСФ СО РАН, рук. А.И. Тятюшкин), ППП КОНУС (<<Комплексная оптимизация управляемых систем», Иркутский ВЦ ВСФ СО РАН, рук. А.И. Жолудев), ППП ПСИ (<<Программная система идентификации>>, рук. В.И. Гурман) и ряд других. Большинство указанных программных разработок были реализованы для ЭВМ БЭСМ-6 или ЕС ЭВМ. К сожалению, в последние годы большинство из этих работ, возможно, по объективным причинам, были свернуты, и версии для современных компьютерных систем так и не были реализованы. Однако, за рубежом работы по созданию программных средств оптимизации ведутся все так же активно – в США, Германии, Португалии, Австралии, Франции, Китае и других странах, и реализованные программные комплексы успешно применяются при решении множества прикладных задач из различных научно-практических областей.


В Институте динамики систем и теории управления СО РАН (г. Иркутск) в настоящее время проводится активная работа по созданию, развитию и тестированию ряда программных комплексов, ориентированных на решение экстремальных задач \cite{Gor09}. В возможности программных комплексов нового поколения входят, в том числе, невыпуклые задачи оптимизации, как статические, так и динамические. Алгоритмическое наполнение новых программных комплексов, что вполне естественно, включает и лучшие алгоритмы, отобранные в прошлых периодах исследований. Для поиска локального экстремума имеется библиотека алгоритмов нулевого, первого и второго порядка, основанных как на теории оптимального управления, так и на теории конечномерной оптимизации. Для поиска глобального экстремума авторами предложено несколько новых семейств алгоритмов, как стохастических, так и детерминированных (методы случайного мультистарта; методы, основанные на аппроксимации множества достижимости; туннельные методы; методы сеток; методы криволинейного поиска; методы, основанные на нелокальном принципе максимума; методы имитации отжига и генетического поиска; методы случайных покрытий; методы криволинейного поиска и другие), лучшие из которых включены в состав программных комплексов. Сервисное наполнение программных средств, во многом зависящее от базовых программных сред, реализовано с целью расширения возможностей ПК путем использования интерактивных режимов расчетов, создания многометодных вычислительных схем и тонкой настройки алгоритмических параметров. Программные комплексы реализованы на стандарте языка С с использованием компиляторов BCC 5.5, GNU, ICC и могут работать под управлением операционных систем Windows, Linux и MacOS.

С применением реализованных программных средств возможно решать задачи оптимального управления в нелинейных системах с терминальными и интегральными функционалами и на быстродействие, с параллелепипедными, терминальными, фазовыми (смешанными, интервальными) и промежуточными ограничениями, с управлениями-функциями и управлениями-константами и другие. Применяя несложные математические редукции, удается расширить область использования ПК и решать задачи с запаздыванием, в системах дифференциальных уравнений в частных производных, с интегро-дифференциальными функционалами, в системах, не разрешенных относительно производных. 

В последние годы реализованы следующие программные средства: программный комплекс OPTCON-MD, ориентированный на аппроксимацию множества достижимости уп\-равляемой динамической системы, ПК OPTCON-A для задач параметрической идентификации, ПК OPTCON-M для оптимизации атомно-молекулярных кластеров сверхбольших размерностей, ПК OPTCON-F предназначенный для исследования систем функцио\-нально-дифференциальных уравнений точечного типа и ряд других.

Работа по созданию указанных программных средств поддержана рядом грантов: РФФИ, РГНФ и интеграционными проектами РАН и СО РАН. Свойства реализованных ПК исследованы с использованием разработанной коллекции тестовых задач. С применением реализованных программных средств успешно решен ряд невыпуклых задач из областей динамики полета и космонавигации \cite{Gor16}, робототехники \cite{Sor}, электроэнергетики, сейсмологии \cite{Ber}, экономики, экологии, химии, медицины \cite{Gor18} и нанофизики \cite{Sham}, \cite{Nen}. В докладе обсуждаются свойства программных комплексов и приводятся результаты численных экспериментов.





\begin{thebibliography}{9} % или {99}, если ссылок больше десяти.

\bibitem{Gor09}	Горнов~А.Ю. Вычислительные технологии решения задач оптимального управления. Новосибирск. Наука. 2009. 279 p.

\bibitem{Gor16}	Gornov~A.Yu., Tyatyushkin~A.I., Finkelstein~E.A. Numerical methods for solving terminal optimal control problems. Computational Mathematics and Mathematical Physics. 2016. Vol.~56, No~2. Рp. 221--234.


\bibitem{Sor} Sorokovikov~P., Gornov~A., Strelnikov~A. Algorithm for the Numerical Solution of Optimal Control Problems in Robotic Systems. Lecture Notes in Computer Science. 2021. Vol.~13078. Рp. 203--214.

\bibitem{Ber} Berzhinsky~Yu.A., Ordynskaya~A.P., Berzhinskaya~L.P., Gornov~A.Yu., Finkelstein~E.A. Analysis of the Mechanism of Transition to the Limit State of Series 111 Residential Buildings During The 1988 Spitak Earthquake. Geodynamics \& Tectonophysics. 2019. Vol.~10, No~3. Pp. 715--730.


\bibitem{Gor18}	Gornov~A.Yu., Zarodnyuk~T.S., Efimova~N.V. Air pollution and population morbidity forecasting with artificial neural networks. IOP Conference Series: Earth and Environmental Science. 2018. Vol.~211. No~1. Pp. 012053.

\bibitem{Sham} Shamirzaev~T.S., Atuchin~V.V., Zhilitskiy~V.E., Gornov~A.Yu. Dynamics of Vacancy Formation and Distribution in Semiconductor Heterostructures: Effect of Thermally Generated Intrinsic Electrons. Nanomaterials 2023. Vol.~13. No~2. Pp. 308.

\bibitem{Nen} Nenashev~A.V., Zinovieva~A.F., Dvurechenskii~A.V., Gornov~A.Yu., Zarodnyuk~T.S. Quantum logic gates from time-dependent global magnetic field in a system with constant exchange. J. of Applied Physics. 2015. Vol.~117. No~11. Рp. 113905.


\end{thebibliography}

% После библиографического списка в русскоязычных статьях необходимо оформить
% англоязычный заголовок.



%\end{document}

