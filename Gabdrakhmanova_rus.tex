\begin{englishtitle} % Настраивает LaTeX на использование английского языка
% Этот титульный лист верстается аналогично.
\title{Dynamic models in predictive analytic with examples of problem solving in different domains}
% First author
\author{N.T. Gabdrakhmanova
%\inst{1}
 % \and
 % Name FamilyName2\inst{2}
%  \and
%  Name FamilyName3\inst{1}
}
\institute{RUDN, Moscow, Russia\\
  \email{gabdrakhmanova-nt@rudn.ru}
%  \and
%Affiliation, City, Country\\
%\email{email@example.com}
}
% etc

\maketitle

\begin{abstract}
 The paper presents the results of building dynamic models using geometric methods for forecasting problems in the social and industrial sphere. Geometric methods are used to solve the problem: methods of differential geometry and algebraic topology. Algorithms for problem solving were developed, a number of computational experiments on real data were carried out, which showed high quality of problem solving. As a task in the social sphere, the problem of assessing the occurrence of conflict on the basis of social media data was solved. As a task in the industrial sphere the problem of stability of oil industry development was solved using time series data.

\keywords{dynamic system, graph curvature estimation, persistence diagram, time series
% \ldots{}
} % в конце списка точка не ставится
\end{abstract}
\end{englishtitle}


\iffalse
%%%%%%%%%%%%%%%%%%%%%%%%%%%%%%%%%%%%%%%%%%%%%%%%%%%%%%%%%%%%%%%%%%%%%%%%
%
%  This is the template file for the 6th International conference
%  NONLINEAR ANALYSIS AND EXTREMAL PROBLEMS
%  June 25-30, 2018
%  Irkutsk, Russia
%
%%%%%%%%%%%%%%%%%%%%%%%%%%%%%%%%%%%%%%%%%%%%%%%%%%%%%%%%%%%%%%%%%%%%%%%%


\documentclass[12pt]{llncs}

% При использовании pdfLaTeX добавляется стандартный набор русификации babel.

\usepackage{iftex}

\ifPDFTeX
\usepackage[T2A]{fontenc}
\usepackage[utf8]{inputenc} % Кодировка utf-8, cp1251 и т.д.
\usepackage[english,russian]{babel}
\fi

\usepackage{todonotes}

\usepackage[russian]{nla}

\begin{document}
\fi

\title{Динамические модели в предиктивной аналитике на примере решения задач в различных областях}
%\thanks{Работа выполнена при поддержке РФФИ (РНФ, другие фонды), проект \textnumero~00-00-00000.}}
% Первый автор
\author{Н.~Т.~Габдрахманова
%\inst{1}
%  \and
% Второй автор%
%  И.~О.~Фамилия\inst{2}
%  \and
% Третий автор
%  И.~О.~Фамилия\inst{1}
}

% Аффилиации пишутся в следующей форме, соединяя каждый институт при помощи \and.
\institute{РУДН им. Патриса Лумумбы, Москва, Россия \\
 \email{gabdrakhmanova-nt@rudn.ru}
%  \emaili@example.com}
%  \and   % Разделяет институты и присваивает им номера по порядку.
%Институт (название в краткой форме), Город, Страна\\
% \email{email@example.com}
% \and Другие авторы...
}

\maketitle

\begin{abstract}
В работе представлены результаты построения динамических моделей с использованием геометрических методов для задач прогнозирования в социальной и промышленной сфере. При построении моделей использованы геометрические методы: методы дифференциальной геометрии и алгебраической топологии. Разработаны алгоритмы решения задач, проведен ряд вычислительных экспериментов на реальных данных, показавших высокое качество решения задачи. В качестве задачи в социальной сфере решалась задача оценки возникновения конфликта по данным социальных медиа. В качестве задачи в промышленной сфере решалась задача устойчивости развития нефтяной отрасли по данным временных рядов.

\keywords{ динамическая система, оценка кривизны графа, персистентная диаграмма, временной ряд } % в конце списка точка не ставится
\end{abstract}

\section{Основные результаты} % не обязательное поле

Предиктивная аналитика и оценка устойчивости динамической сложной системы является важной для анализа поведения систем в самых различных областях. Такие задачи возникают в экономике, технике, медицине, биологии. Целью данной работы является исследование возможности использования аналогов кривизны Риччи для метрических пространств, порожденных графами, а также, методов алгебраической топологии для решения задач прогноза и анализа устойчивости различных динамических систем. При решении поставленных задач были построены параллельные модели с использованием статистических методов и нейросетевого моделирования с целью верификации предлагаемых методов. Основные преимущества геометрических методов 
-- это устойчивость к шумам, возможность использования машинного обучения, возможность анализировать большой объем данных.

Топологический анализ данных (TDA) - позволяет находить структуру в данных временных рядов. Персистентные диаграммы, введенные Х. Эдельсбруннером [1], являются важнейшим инструментом вычислительной топологии, позволяющим, например, получать качественную информацию о <<топологической динамике>> временного ряда.   С помощью ТДА решена одна актуальная задача в социальной сфере: изучение отношения социума к конкретным проектам по данным социальных сетей. Цель работы: выявлять конфликтные ситуации на ранних стадиях кризисных ситуаций.  На первом этапе решения задачи была проведена предобработка данных, посредством анализа текстовых данных построены временные ряды. Затем, временные ряды были сегментированы и для каждого сегмента временного ряда построены персистентные диаграммы,  барркоды, вычислены оценки характеристик Эйлера. Показано, что по динамике характеристик Эйлера можно классифицировать ситуацию по отношению возможности возникновения  конфликтной ситуации. Результаты расчетов, полученные с помощью построенных математических моделей, подтвердили отсутствие конфликта. Алгоритмы, разработанные с использованием методов TDA на данных конкретного проекта, показывают перспективность применения этих методов для решения аналогичных задач [2].

Кривизна Риччи в заданной точке характеризует среднюю кривизну секционных кривизн по всем направлениям. Понятие кривизны Риччи для метрических пространств общего вида впервые было введено в работах Бакри и Эмери [3]. В 2011 году Лин, Лу и Яу [4] модифицировали определение Олливье для кривизны Риччи цепей Маркова на метрических пространствах.
С использованием методов дифференциальной геометрии решена задача оценки устойчивости развития нефтяной отрасли по данным временных рядов. Под устойчивостью развития отрасли понимается согласованный динамический рост всех показателей отрасли, разбитых на три группы.
Для решения задачи кластеризации в работе используется формула Watts-Strogatz [5]:
curv(А)=t/(v(v-1)/2 ,

здесь v -- числа вершин и t -- число треугольников, которые образованы ребрами графа, содержащими вершину А. Данная функция -- функция двух переменных. Заметим, что величина v (v-1)/2 -- максимальное количество треугольников, которое можно составить с помощью всех вершин графа.
Корреляционная сеть строится следующим образом. Вершины сети -- экономические показатели, все вершины соединены ребрами. Вес ребра -- равен выборочному коэффициенту корреляции Пирсона. Проведена верификация полученных решений с помощью сравнения с экспертными оценками.

 Алгоритмы, разработанные с использованием методов TDA и дифференциальной геометрии, показывают перспективность применения этих методов для решения задач предиктивной аналитики в поведении динамических систем.

% Рисунки и таблицы оформляются по стандарту класса article. Например,

%\begin{figure}[htb]
  %\centering
  % Поддерживаются два формата:
  %\includegraphics[width=0.7\linewidth]{figure.pdf} % Растровый формат
  %\includegraphics[width=0.7\linewidth]{figure.png} % Векторный и растровый формат
  %

%  \begin{center}
%    \missingfigure[figwidth=0.7\linewidth]{}
%  \end{center}
%  \caption{Заголовок рисунка}\label{fig:example}
%\end{figure}


% В конце текста можно выразить благодарности, если этого не было
% сделано в ссылке с заголовка статьи, например,
%Работа выполнена при поддержке РФФИ (РНФ, другие фонды), проект \textnumero~00-00-00000.
%

\begin{thebibliography}{9} % или {99}, если ссылок больше десяти.
\bibitem{1} Edelsbunner~H., Letscher~D., Zomorodian~A. Topological persistence and simplification. Discrete Comput. Geom. 2002. Vol. 28. Pp.~ 511--533. MR 1949898 .


\bibitem{Semenov} Gabdrakhmanova~ N.  Forecasting time series using topological data analysis. ITISE 2018 International Conference on Time Series and Forecasting Proceedings of Papers 19--21 September 2018 Granada (Spain)

\bibitem{Kholl} Bakry~D. , Emery~ M.  Diffusions hypercontractives (French) [Hypercontractive diffusions], Seminaire de probabilites, XIX, 1983/84, Lecture Notes in Math. 1123, J. Azema and M. Yor (Editors), Springer, Berlin, 1985. Pp. 177--206.

\bibitem{Aleksandrov1} Lin~ Y., Lu~ L. Y., Yau~ S. T. Ricci curvature of graphs. Tohoku Mathematical Journal. 2011. Vol. 63. Pp. 605--627.   

\bibitem{Moreau1977} Watts ~D.J. and Strogatz~ S.H. Collective dynamics of small-world networks. Nature 1998, 393. Pp. 440--442.
\end{thebibliography}

% После библиографического списка в русскоязычных статьях необходимо оформить
% англоязычный заголовок.



%\end{document}
