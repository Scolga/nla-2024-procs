
\iffalse
%%%%%%%%%%%%%%%%%%%%%%%%%%%%%%%%%%%%%%%%%%%%%%%%%%%%%%%%%%%%%%%%%%%%%%%%
%
% This is the template file for the 6th International conference
% NONLINEAR ANALYSIS AND EXTREMAL PROBLEMS
% June 25-30, 2018
% Irkutsk, Russia
%
%%%%%%%%%%%%%%%%%%%%%%%%%%%%%%%%%%%%%%%%%%%%%%%%%%%%%%%%%%%%%%%%%%%%%%%%
% The preparation of the article is based on the standard llncs class
% (Lecture Notes in Computer Sciences), which is adjusted with style
% file of the conference.
%
% There are two ways of compilation of the file into PDF
% 1. Use pdfLaTeX (pdflatex), (LaTeX+DVIPS will not work);
% 2. Use LuaLaTeX (XeLaTeX will work too).
% When using LuaLaTeX You will need TTF or OTF CMU fonts
% (Computer Modern Unicode). The fonts are installed with 'cm-unicode' package in
% a distribution of LaTeX % (https://www.ctan.org/tex-archive/fonts/cm-unicode),
% either by downloading and installing these fonts system wide, the address of their page is
% http://canopus.iacp.dvo.ru/%7Epanov/cm-unicode/
% The second option won't work in XeLaTeX.
%
% For MiKTeX (LaTeX distribution for Windows),
%  1. Package 'cm-unicode' is installed manually with the MiKTeX administration Console.
%  2. For the compilation of this example, namely, the stub figure, one will also need to
% download package 'pgf' manually. This package uses in the popular
% package tikz.
%  3. Tests showed that the rest of the required packages MiKTeX loads automatically (if
%     it is allowed). The 'auto download' option is
%     configured in 'Settings' section in MiKTeX Console.
%
%
% The easiest way to compile an article is to use pdfLaTeX, but
% the final layout of the book will be compiled with LuaLaTeX,
% as a result will be of better quality thanks to the package 'microtype' and
% use vector OTF instead of standard raster fonts of pdfLaTeX.
%
% In the case of questions and problems with the article compilation,
% write letters to e-mail: eugeneai@irnok.net, Cherkashin Evgeny.
%
% New version of the correcting style file will be available at the website:
%     https://github.com/eugeneai/nla-style
%     file - nla.sty
%
% Further instructions are in the text body of the template. The template itself
% is an article example.
%
% The LaTeX2e format is used!

% 12 points font size is used.
\documentclass[12pt]{llncs}

% The correcting style file is added.
\usepackage{todonotes}

\usepackage{nla} % This package is needed for compiling
                 % this template, it should be removed
                 % from your article.

% Many popular packages (amsXXX, graphicx, etc.) are already imported in the style file.
% If there is a conflict with your packages, try disabling them and compile
% the text.
%
% It would be convenient in the layout of the proceedings if the file names
% of the figures of different authors do not clash.
% To minimize the clash, the drawings can be placed in a separate subfolder
% named after the author or the title of the paper.
%
% \graphicspath{{ivanov-petrov-pics/}} % specifies the folder with images in png, pdf formats.
% or
% \graphicspath{{great-problem-solving-paper-pics/}}.

\begin{document}
\fi
% Text should be formatted in accordance with the 'article' class, using extensions like
% AMS.
%
\title{Equilibrium Problem for a Timoshenko Plate Contacting by the Side Edge and the Bottom Boundary\thanks{The research is
supported by the Ministry of Education and Science of the Russian
Federation within the framework of the base part of the state
task, project No.~FSRG-2023-0025.}}
% First author
\author{Nyurgun Lazarev\inst{1}   \and  Djulustan Nikiforov\inst{1,2}  \and  Natalya Romanova\inst{1,3}
}
\institute{North-Eastern Federal University, Yakutsk, Russia\\
  \email{nyurgunlazarev@yandex.ru}
  \and
 \email{dju92@mail.ru}
  \and
   \email{romnatan@mail.ru}
 }
% etc

\maketitle

\begin{abstract}
Contact problems for solids with inequality type constraints have
attracted the attention of scientists since 1933s
\cite{Sign,KhludnevKovtunenko}. Problems of this kind are
associated with the use of boundary conditions that describe
non-penetration constraints on the contact surfaces or curves
\cite{KhludnevKovtunenko,RudoyESIAM}. We propose a new nonlinear
equilibrium model for a Timoshenko plate which may come into
contact either on the side edge or on the bottom surface with a
rigid obstacle of a certain given configuration. A corresponding
variational problem is formulated as a minimization problem for an
energy functional over a nonconvex set of admissible displacements
subject to a nonpenetration condition. In contrast to problems for
Timoshenko plates in the framework of nonlinear boundary
conditions \cite{laz1,laz2}, we consider two surfaces of possible
contacts. Namely, the inequality type nonpenetration condition is
given as a system of inequalities describing two cases of possible
contacts of the plate and the rigid obstacle. These two cases
correspond to different types of contacts by the plate side edge
and by the plate bottom. The solvability of the problem is
established. In particular case, when contact zones is previously
known, an equivalent differential statement is obtained under the
assumption of additional regularity for the solution to the
variational problem.

\keywords{contact problem, non-penetration condition, non-convex
set, variational problem}
\end{abstract}

% at the end of the list, there should be no final dot
%\section{The main results}


%The text of the report.


% At the end of the text, acknowledgments are expressed, if you haven't
% made a footnote from the title. For example, we can write
%The research is carried on with support of RFBR (RNF, other funds), project No.~00-00-00000.

\begin{thebibliography}{9} % or {99}, if there is more than ten references.

\bibitem{Sign} Signorini  A. Questioni di elasticità
non linearizzata e semilinearizzata. Rend. Mat. e Appl.~1959.
Vol.~18, no.~5. Pp.~95--139.


\bibitem{KhludnevKovtunenko} Khludnev A.M., Kovtunenko V.A. {Analysis of Cracks in Solids.}
WIT-Press, Southampton, 2000.

\bibitem{RudoyESIAM} Rudoy E. Domain decomposition method for crack problems with
nonpenetration condition. ESAIM: Mathematical Modelling and
Numerical Analysis~2016. Vol.~50, no.~4. Pp.~995--1009.

\bibitem{laz1} Lazarev N.P., Semenova G.M. Equilibrium problem for a Timoshenko
plate with a geometrically nonlinear condition of nonpenetration
for a vertical crack. J. Appl. Ind. Math.~2020. Vol.~14.
Pp.~532--540.

\bibitem{laz2} Lazarev N.P. Equilibrium problem for a Timoshenko plate with an
oblique crack. J. Appl. Mech. Tech. Phy.~2013. Vol.~54, no.~4. Pp.
662--671.


\end{thebibliography}
%\end{document}