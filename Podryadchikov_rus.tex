\begin{englishtitle}
\title{Optimization of the algorithm for creating periodic tasks in the electronic document management systems}
% First author
\author{Vladimir Podryadchikov
\and
% Второй автор
  Varvara Zaharchenko
}
\institute{Irkutsk State University, Irkutsk, Russia\\
\email{vpodryadchikov@bk.ru}}
% etc

\maketitle

\begin{abstract}
This paper deals with the process of developing an electronic document management system. We develop the functionality for parallelizing the workflow and conduct an evaluation of the effectiveness of the implemented modification.

\keywords{EDM, document management, modeling, optimization algorithm, modification efficiency}
\end{abstract}
\end{englishtitle}

\iffalse
\documentclass[12pt]{llncs}
\usepackage[T2A]{fontenc}
\usepackage[utf8]{inputenc}
\usepackage[english,russian]{babel}
\usepackage[russian]{nla}

%\usepackage[english,russian]{nla}

% \graphicspath{{pics/}} %Set the subfolder with figures (png, pdf).
\usepackage{multirow}

%\usepackage{showframe}
\begin{document}
%\selectlanguage{russian}
\fi

\title{Оптимизация алгоритма создания периодических поручений в системе электронного документооборота%\thanks{Работа выполнена при поддержке РФФИ (РНФ, другие фонды), проект \textnumero~00-00-00000.}
}
% Первый автор
\author{В.~В.~Подрядчиков%\inst{1}
  \and
% Второй автор
  В.~С.~Захарченко
%  \and
% Третий автор
%  И.~О.~Фамилия\inst{2}
}
\institute{Институт математики и информационных технологий\\ Иркутский государственный университет, Иркутск, Россия\\
  \email{vpodryadchikov@bk.ru}
  }
.

\maketitle

\begin{abstract}
В докладе рассматривается процесс разработки системы электронного документооборота на примере системы Тезис. Разработан функционал для распараллеливания рабочего процесса. Проведена оценка эффективности построенной модификации

\keywords{СЭД, документооборот, моделирование, оптимизация алгоритма, эффективность модификации}
\end{abstract}

\section{Основные результаты}


Существует огромное количество предприятий, на которых происходит работа с бумажными носителями информации. Система электронного документооборота позволяет упорядочить потоки документов в организации, обеспечить их хранение в едином информационном пространстве

Система документооборота Тезис \cite{Thesis1} разработана российской компанией Хоулмонт (Haulmont) и написана на CubaPlatform, высокоуровневой java-платформе для создания корпоративных информационных систем.


 В штатной версии системы Тезис существует понятие «Периодическое поручение». Такое поручение создается автоматически спустя заданный период. Но у этого функционала есть недостаток: если сотрудников много, то создавать поручение придется на каждого человека отдельно. Был оптимизирован алгоритм путём распараллеливания рабочего процесса. Благодаря доработке «Группа периодических поручений» в системе появилась возможность создавать периодические поручение для многих сотрудников на одном экране, что значительно экономит время. Ниже представлено несколько задач, выполненных в ходе доработки:

\begin{itemize}
 \item Создание необходимых сущностей с помощью инструмента Designer платформы Cuba на основании уже существующих в системе. Создание происходит на основании, так как доработка во многом представляет собой расширение штатного функционала работы, его оптимизация засчёт уменьшения количества действий, которые придется делать пользователю по ходу работы.
 \item Создание класса, отвечающего за корректное свёртывание поручений в группе. Написание сервиса, отменяющего подпоручения на основании поручений в родительской группе.
\end{itemize}

В ходе работы была проведена оценка эффективности оптимизации процесса.

\begin{table}[htbp]
    \centering
    \caption{Среднее время создания периодических поручений на 200 человек}
    \begin{tabular}{|c|c|c|}
        \hline
        \multirow{2}{*}{Операция} & \multicolumn{2}{c|}{Среднее время обработки (мин)} \\ \cline{2-3}
         & Без модификации & С модификацией \\ \hline
        Процесс заполнения полей на экране & 80 & 20 \\ \hline
        Работа с вложениями  & 4 & 3 \\ \hline
    \end{tabular}
    \label{tab:my_table}
\end{table}

Расчеты проводились с помощью формулы  \cite{Kosnikov}:
\[
E = 100 \times \left(\frac{{T1 \times N - T2 \times N}}{{T1 \times N}}\right)
\]
где \(E\) -- экономический эффект в процентах;
\(T1\) -- время на выполнение операции до модификации;
\(T2\) -- время на выполнение операции после модификации;
\(N\) -- количество деловых операций. Данные взяты из корпоративных документов  (см. таблицу 1). В результате:
\[
E =  73\%
\]

\begin{thebibliography}{9}

\bibitem{Thesis1} Автоматизация документооборота. [Электронный ресурс] -- Тезис система документооборота. URL: https://www.tezis-doc.ru/features/workflow/ (Дата обращения: 24.04.2024)

\bibitem{Kosnikov}
Косников, С. Н.  Математические методы в экономике : учебное пособие для вузов / С. Н. Косников. -- 2-е изд., испр. и доп. -- М. : Издательство Юрайт, 2024.

\bibitem{Getmanchuk}
Гетманчук А.В., Ермилов М.М. Экономико-математические методы и модели: учебное пособие для бакалавров. М. :
Издательство Дашков и К, 2017.
\end{thebibliography}

% Обязательная информация на английском языке:




%\end{document}
