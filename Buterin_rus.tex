\begin{englishtitle} % Настраивает LaTeX на использование английского языка
% Этот титульный лист верстается аналогично.
\title{On damping a control system of arbitrary order with global aftereffect on a temporal tree}
% First author
\author{Sergey Buterin}
\institute{Saratov State University, Saratov, Russia\\
  \email{buterinsa@sgu.ru}}

\maketitle

\begin{abstract}
We study the problem of damping a control system described by functional-differential equations of arbitrary order and neutral type with
non-smooth complex coefficients on an arbitrary tree with global delay. Each internal vertex of the tree gives several different scenarios of
the further course of the process in accordance with the number of edges emanating from it. We establish the existence and uniqueness of an
optimal trajectory with account of all prospectives simultaneously.

\keywords{quantum graph, functional-differential equation, global delay, optimal control problem, variational problem, nonlocal
quasi-derivative} % в конце списка точка не ставится
\end{abstract}
\end{englishtitle}

\iffalse
\documentclass[12pt]{llncs}

% При использовании pdfLaTeX добавляется стандартный набор русификации babel.

\usepackage{iftex}

\ifPDFTeX
\usepackage[T2A]{fontenc}
\usepackage[utf8]{inputenc} % Кодировка utf-8, cp1251 и т.д.
\usepackage[english,russian]{babel}
\fi

\usepackage{todonotes}

\usepackage[russian]{nla}

\begin{document}
\fi

\title{Об успокоении системы управления произвольного порядка с глобальным последействием на временном 
дереве\thanks{Работа выполнена при поддержке РНФ, проект \textnumero~22-21-00509.}}
% Первый автор
\author{С.~А.~Бутерин   % \inst ставит циферку над автором.
} % обязательное поле

% Аффилиации пишутся в следующей форме, соединяя каждый институт при помощи \and.
\institute{Саратовский университет, Саратов, Россия\\
  \email{buterinsa@sgu.ru}
}

\maketitle

\begin{abstract}
Исследуется задача об успокоении системы управления, описываемой функ\-цио\-наль\-но-диф\-ференциальными уравнениями произвольного порядка и
нейтрального типа с негладкими комплексными коэффициентами на произвольном дереве с глобальным запаздыванием. Каждая внутренняя вершина
дерева дает несколько различных сценариев дальнейшего течения процесса по числу выходящих из нее ребер. Устанавливается существование и
единственность оптимальной траектории с учетом сразу всех перспектив.

\keywords{квантовый граф, функционально-дифференциальное уравнение, глобальное запаздывание, задача оптимального управления, вариационная
задача, нелокальная квазипроизводная} % в конце списка точка не ставится
\end{abstract}

%\section{Основные результаты} % не обязательное поле

Дифференциальные операторы на геометрических графах, часто называемые квантовыми графами, активно изучаются с прошлого века в связи с
многочисленными приложениями (см., например, \cite{Pokornyi, Berkolaiko, Buterin-23}). Такие операторы возникают при исследовании процессов в
сложных системах, представимых в виде пространственных сетей, т.е. наборов одномерных континуумов, взаимодействующих только через концы
\cite{Pokornyi}. Примером служат упругие струнные сетки, в узлах которых помимо условий непрерывности характерными являются условия Кирхгофа,
выражающие баланс натяжений.

Однако в докладе предлагается иной взгляд на квантовые графы - как на временные сети, когда параметризующая ребра графа переменная
отождествляется со временем. При этом каждая внутренняя вершина понимается, как момент разветвления процесса, дающий несколько различных
сценариев дальнейшего его протекания.

Исследуется задача об успокоении системы управления, описываемой функционально-дифференциальными уравнениями произвольного натурального
порядка $n$ нейтрального типа с негладкими комплексными коэффициентами на произвольном дереве с глобальным запаздыванием. Последнее означает,
что запаздывание распространяется через внутренние вершины дерева~\cite{Buterin-23}. Примечательно, что условия типа Кирхгофа возникают и
здесь. А именно, им будет удовлетворять такая траектория течения процесса, которая является оптимальной с учетом сразу всех возможных
сценариев. Доказывается ее существование и единственность.

Минимизация функционала энергии рассматриваемой системы приводит к вариационной задаче. Установлена ее эквивалентность некоторой
самосопряженной краевой задаче на дереве для уравнений порядка $2n$ с разнонаправленными сдвигами аргумента, а также условиями типа Кирхгофа
во внутренних вершинах. Доказана однозначная разрешимость обеих задач.

Указанные результаты получены в работе~\cite{Buterin-24}. Ранее подобные задачи изучались исключительно на интервале для управляемых систем
только первого порядка с постоянными либо гладкими вещественными коэффициентами (см. \cite{Krasovskii, Skubachevskii, Adhamova} и литературу
там). Поэтому полученные результаты являются новыми даже для интервала. Кроме того, рассматриваемый случай потребовал введения семейства
специальных нелокальных квазипроизводных. В \cite{Buterin-24} также проводится их сравнение с классическими квазипроизводными для обыкновенных
дифференциальных операторов.

% кавычки << >>.

% В конце текста можно выразить благодарности, если этого не было
% сделано в ссылке с заголовка статьи, например,
%Работа выполнена при поддержке РНФ, проект \textnumero~22-21-00509.
%

\begin{thebibliography}{9} % или {99}, если ссылок больше десяти.

\bibitem{Pokornyi}
Покорный~Ю.В., Пенкин~О.М., Прядиев~В.Л., Боровских~А.В., Лазарев~К.П., Шабров~С.А. Дифференциальные уравнения на геометрических графах.
М.:~Физматлит,~2005.

\bibitem{Berkolaiko}
Berkolaiko~G., Carlson~R., Fulling~S., Kuchment~P. Quantum Graphs and Their Applications, Cont. Math. 415, AMS, Providence, RI, 2006.

\bibitem{Buterin-23}
Buterin~S. Functional-differential operators on geometrical graphs with global delay and inverse spectral problems~// Results Math. 2023.
Vol.~78. Art.~No.~79.

\bibitem{Buterin-24}
Бутерин~С.А. Об успокоении системы управления произвольного порядка с глобальным последействием на дереве~// Матем. заметки. 2024. Т.~115.
Вып.~6. С.~825--848.

\bibitem{Krasovskii}
Красовский~Н.Н. Теория управления движением. М.:~Наука,~1968.

\bibitem{Skubachevskii}
Skubachevskii~A.L. Elliptic Functional Differential Equations and Applications. Basel: Birkh\"auser,~1997.

\bibitem{Adhamova}
Адхамова~А.Ш., Скубачевский~А.Л. Об успокоении системы управления с последействием нейтрального типа~// Докл. РАН. Мат., инф., проц. упр.
2020. Т.~490. С.~81--84.

\end{thebibliography}

% После библиографического списка в русскоязычных статьях необходимо оформить
% англоязычный заголовок и аннотацию.




%\end{document}