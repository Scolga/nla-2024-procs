
\iffalse
%%%%%%%%%%%%%%%%%%%%%%%%%%%%%%%%%%%%%%%%%%%%%%%%%%%%%%%%%%%%%%%%%%%%%%%%
%
% This is the template file for the 6th International conference
% NONLINEAR ANALYSIS AND EXTREMAL PROBLEMS
% June 25-30, 2018
% Irkutsk, Russia
%
%%%%%%%%%%%%%%%%%%%%%%%%%%%%%%%%%%%%%%%%%%%%%%%%%%%%%%%%%%%%%%%%%%%%%%%%
% The preparation of the article is based on the standard llncs class
% (Lecture Notes in Computer Sciences), which is adjusted with style
% file of the conference.
%
% There are two ways of compilation of the file into PDF
% 1. Use pdfLaTeX (pdflatex), (LaTeX+DVIPS will not work);
% 2. Use LuaLaTeX (XeLaTeX will work too).
% When using LuaLaTeX You will need TTF or OTF CMU fonts
% (Computer Modern Unicode). The fonts are installed with 'cm-unicode' package in
% a distribution of LaTeX % (https://www.ctan.org/tex-archive/fonts/cm-unicode),
% either by downloading and installing these fonts system wide, the address of their page is
% http://canopus.iacp.dvo.ru/%7Epanov/cm-unicode/
% The second option won't work in XeLaTeX.
%
% For MiKTeX (LaTeX distribution for Windows),
%  1. Package 'cm-unicode' is installed manually with the MiKTeX administration Console.
%  2. For the compilation of this example, namely, the stub figure, one will also need to
% download package 'pgf' manually. This package uses in the popular
% package tikz.
%  3. Tests showed that the rest of the required packages MiKTeX loads automatically (if
%     it is allowed). The 'auto download' option is
%     configured in 'Settings' section in MiKTeX Console.
%
%
% The easiest way to compile an article is to use pdfLaTeX, but
% the final layout of the book will be compiled with LuaLaTeX,
% as a result will be of better quality thanks to the package 'microtype' and
% use vector OTF instead of standard raster fonts of pdfLaTeX.
%
% In the case of questions and problems with the article compilation,
% write letters to e-mail: eugeneai@irnok.net, Cherkashin Evgeny.
%
% New version of the correcting style file will be available at the website:
%     https://github.com/eugeneai/nla-style
%     file - nla.sty
%
% Further instructions are in the text body of the template. The template itself
% is an article example.
%
% The LaTeX2e format is used!

% 12 points font size is used.
\documentclass[12pt]{llncs}

% The correcting style file is added.
\usepackage{todonotes}

\usepackage{nla} % This package is needed for compiling
                 % this template, it should be removed
                 % from your article.

% Many popular packages (amsXXX, graphicx, etc.) are already imported in the style file.
% If there is a conflict with your packages, try disabling them and compile
% the text.
%
% It would be convenient in the layout of the proceedings if the file names
% of the figures of different authors do not clash.
% To minimize the clash, the drawings can be placed in a separate subfolder
% named after the author or the title of the paper.
%
% \graphicspath{{ivanov-petrov-pics/}} % specifies the folder with images in png, pdf formats.
% or
% \graphicspath{{great-problem-solving-paper-pics/}}.

\begin{document}
\fi
% Text should be formatted in accordance with the 'article' class, using extensions like
% AMS.
%
\title{On impulsive fractional differential inclusions with a nonconvex-valued right-hand side in Banach spaces\thanks{The research is supported by RNF, project No.~22-71-10008.}}
% First author
\author{Valeri Obukhovskii \and Maria Soroka 
}
\institute{Voronezh State Pedagogical University, Voronezh, Russia\\
  \email{valerio-ob2000@mail.ru}
  \\
\email{marya.afanasowa@yandex.ru}}
% etc

\maketitle

\begin{abstract}
We study the Cauchy problem for an impulsive semilinear fractional order differential inclusion with a nonconvex-valued almost lower semicontinuous nonlinearity in a separable Banach space.
By using the fixed point theory for condensing maps, we present a global theorem on the existence of a mild solution to this problem.

\keywords{impulsive differential inclusion, Cauchy problem, Caputo fractional derivative, almost lower semicontinuous multioperator, condensing map, fixed point, multivalued map, measure of noncompactness}
\end{abstract}

% at the end of the list, there should be no final dot
\section{The main results}

We split the segment $[0,T] $ by points $0< t_{1}<...<t_{m}<T, m\geq 1,$ and introduce the notation $I=[0,T]\setminus \lbrace t_{1}, \cdots , t_{m}\rbrace.$ For a function $c:[0,T]\rightarrow E$ we will denote
$$
c(t_{k}^{+})=\lim_{\xi\rightarrow 0+}c(t_{k}+\xi),$$
$$  c(t_{k}^{-})=\lim_{\xi\rightarrow 0-}c(t_{k}+\xi),$$
for $  k=1,...,m.$


In the present paper we consider the Cauchy problem for a semilinear fractional differential inclusion in a separable Banach space $E$ of the following form:
\begin{equation}
\label{eq:p:1}
^{C}D^{q}_0 x(t)\in Ax(t)+ F(t,x(t)),\quad t\in I,
\end{equation}
\begin{equation}
\label{eq:p:2}
x(0)=x_{0},
\end{equation}
\begin{equation}
\label{eq:p:3}
x(t_k)=x(t^+_k)-I_{k}x(t_k), \ k=1,...,m,
\end{equation}
where $^{C}D^{q}$ is the Caputo fractional derivative of an order $q,$ $0 < q < 1$ and $F: I \times  E \to K(E)$ is an almost lower semicontinuous multivalued map with compact values, $A:D(A)\subset E \rightarrow E$ is a closed linear (not necessarily bounded) operator in $E$ generating a uniformly bounded $C_{0}$--semigroup, $I_{k}: E\to E$ are the impulse functions and $x_{0}\in E.$

By using the measure of noncompactness theory and the fixed point theory for condensing maps, we present the theorem on the existence of a mild solution to problem  \eqref{eq:p:1}--\eqref{eq:p:3}.


% At the end of the text, acknowledgments are expressed, if you haven't
% made a footnote from the title. For example, we can write
The research is carried on with support of RNF project No.~22-71-10008.

\begin{thebibliography}{9} % or {99}, if there is more than ten references.

\bibitem{AOP}  Afanasova M.S.,  Obukhovskii V.V., Petrosyan G.G. On a generalized boundary value problem for a feedback control system with infinite delay. Vestnik Udmurtskogo Universiteta: Matematika, Mekhanika, Komp'yuternye Nauki.~2021. Vol.~ 31, no~2. Pp.~ 167--185.


\bibitem{BNR1}  Belmekki M.,  Nieto J.J.,  Rodriguez-Lopez R. Existence of solution to a periodic boundary value problem for a nonlinear impulsive fractional differential equation. Electronic Journal of Qualitative Theory of Differential Equations.~2014. Vol.~16. Pp.~1--27.

\bibitem{KOPY}  Kamenskii M.,  Obukhovskii V.,  Petrosyan G.,  Yao J.-C. On semilinear fractional order differential inclusions in Banach spaces, Fixed Point Theory.~2017. Vol.~ 18, no~1. Pp.~269--292.

\bibitem{KOZ} Kamenskii M., Obukhovskii V., Zecca P. Condensing Multivalued Maps and Semilinear Differential Inclusions in Banach Spaces. Walter De Gruyter, Berlin - New York, 2001.

\bibitem{KST} Kilbas A.A.,  Srivastava H.M.,  Trujillo J.J.  Theory and
Applications of Fractional Differential Equations.Elsevier Science B.V., North-Holland Mathematics Studies, Amsterdam, 2006.

\bibitem{OG}   Obukhovskii V.,  Gel'man B. Multivalued Maps and Differential Inclusions. Elements of Theory and Applications.World Scientific Publishing Co. Pte. Ltd., Hackensack, NJ, 2020.

\bibitem{OPWB}   Obukhovskii V.,  Petrosyan G.,  Wen C.-F.,  Bocharov V. On semilinear fractional differential inclusions with a nonconvex-valued right-hand side in Banach spaces. J. Nonlinear Var. Anal.~2022. Vol.~ 6. Pp.~185--197.

\bibitem{Pod}  Podlubny I. Fractional Differential Equations. Academic Press, San Diego, 1999.


\bibitem{WFZ}  Wang J.,  Fe\u{c}kan M., Zhou Y. On the new concept of solutions and existence results for impulsive fractional evolution equations. Dynamics of PDE.~2011. Vol.~8,no~4. Pp.~345--361.

\end{thebibliography}
%\end{document}

%%% Local Variables:
%%% mode: latex
%%% TeX-master: t
%%% End:
