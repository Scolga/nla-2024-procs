\begin{englishtitle} % Настраивает LaTeX на использование английского языка
% Этот титульный лист верстается аналогично.
\title{On the core of Aubin extension \\ of the elmost positive game}
% First author
\author{Valery Vasilev 
%  \and
%  Name FamilyName2\inst{2}
%  \and
%  Name FamilyName3\inst{1}
}
\institute{Sobolev Institute of Mathematics, Novosibirsk, Russia \\
\email{vasilev@math.nsc.ru}
%  \and
%Affiliation, City, Country\\
%\email{email}}
% etc
}
\maketitle
\begin{abstract}
The aim of the paper is to study the cores of the
famous Aubin extension for so-called almost positive games.
These games are characterized by nonnegativity of the
Harsanyi dividends of their non-singleton coalitions. So, being convex, these games have their
Shapley values inside the classical core. We prove that for the Aubin extension of these games
much more stronger result holds: the core of this
extension of any almost positive game consists of its Shapley value. The approach we develop
is based, mostly, on the close interconnection between the cores and super-differentials of fuzzy
cooperative games, mentioned by J.-P.Aubin.

\keywords{almost positive game, Aubin extension, Shapley value, core, super-differential of a fuzzy game}
\end{abstract}
\end{englishtitle}

\iffalse
\documentclass[12pt]{llncs}

% При использовании pdfLaTeX добавляется стандартный набор русификации babel.

\usepackage{iftex}

\ifPDFTeX
\usepackage[T2A]{fontenc}
\usepackage[utf8]{inputenc} % Кодировка utf-8, cp1251 и т.д.
\usepackage[english,russian]{babel}
\fi

\usepackage{todonotes}

\usepackage[russian]{nla}

\begin{document}
\fi
% Текст оформляется в соответствии с классом article, используя дополнения
% AMS.


\title{О ядре расширения Обэна \\ почти положительной кооперативной игры\thanks{Работа выполнена при финансовой поддержке программы фундаментальных научных исследований СО РАН (номер темы FWNF-2022-0019)}}
% Первый автор
\author{В.~А.~Васильев   % \inst ставит циферку над автором.
%  \and  % разделяет авторов, в тексте выглядит как запятая.
% Второй автор
% И.~О.~Фамилия\inst{2}
%  \and
% Третий автор
%  И.~О.~Фамилия\inst{1}
} % обязательное поле

% Аффилиации пишутся в следующей форме, соединяя каждый институт при помощи \and.
\institute{Институт математики им. С.Л.Соболева СО РАН (ИМ СО РАН), Новосибирск, Россия \\
\email{vasilev@math.nsc.ru}
%  \and   % Разделяет институты и присваивает им номера по порядку.
%Институт (название в краткой форме), Город, Страна\\
%\email{email@examle.com}
% \and Другие авторы...
}

\maketitle

\sloppy

\begin{abstract}
Ра\-бо\-та по\-свя\-ще\-на ана\-ли\-зу вза\-имо\-свя\-зи ме\-жду
зна\-че\-ния\-ми Шеп\-ли и яд\-ра\-ми рас\-ши\-ре\-ния Обэ\-на
поч\-ти по\-ло\-жи\-тель\-ных ко\-опе\-ра\-тив\-ных игр.
Пос\-лед\-ние, с точ\-нос\-тью до ад\-ди\-тив\-ной
со\-став\-ляю\-щей, ха\-рак\-те\-ри\-зу\-\-ют\-ся
не\-от\-ри\-ца\-тель\-нос\-тью своих ди\-ви\-ден\-дов
Хар\-ша\-ньи. Поэ\-то\-му, бу\-ду\-чи вы\-пук\-лы\-ми, почти
по\-ло\-жи\-тель\-ные игры содержат значения Шепли внутри своих
клас\-си\-чес\-ких ядер. В работе до\-ка\-зы\-ва\-ет\-ся, что для
рас\-ши\-ре\-ния Обэна почти по\-ло\-жи\-тель\-ных игр имеет место
еще более сильная кор\-ре\-ля\-ция: их ядра пол\-нос\-тью
ис\-чер\-пы\-ва\-ют\-ся со\-от\-вет\-ствую\-щи\-ми
зна\-че\-ния\-ми Шепли. Ис\-поль\-зуе\-мая ар\-гу\-мен\-та\-ция
опи\-ра\-ет\-ся на от\-ме\-чен\-ную Ж.-П.Обэном воз\-мож\-ность
опи\-са\-ния ядер некоторых клас\-сов не\-чет\-ких
ко\-опе\-ра\-тив\-ных игр в терминах
су\-пер\-диф\-фе\-рен\-циа\-лов их ха\-рак\-те\-рис\-ти\-чес\-ких
функ\-ций.

\keywords{почти положительная игра, значение Шепли, расширение Обэна,
ядро, супердифференциал нечеткой игры}
\end{abstract}

\section{Основные результаты} % не обязательное поле

Обозначим $V(N)$ совокупность (классических) кооперативных игр $n$ лиц, задаваемых неаддитивными функциями множества
$v: 2^N \rightarrow \mathbb{R},$ где $N =\{1, \ldots, n\}$ - множество игроков, а значения $v(S)$ трактуются как максимальные га\-ран\-ти\-ро\-ван\-ные выигрыши, достижимые ко\-опе\-ра\-ци\-ей игроков,
объе\-ди\-нив\-ших\-ся в коалиции $S$ ($v(\emptyset):= 0$). Рассматриваемые далее {\it почти положительные игры}
составляют конус  $V_{(+)}(N) := \{v \in V(N) \mid v_S \geq 0, \, |S| \geq 2\},$ где $v_S :=
\sum_{T \subseteq S}(-1)^{|S \setminus T|}v(T),$ а $|S|$ - число элементов в $S.$

Напомним определение нечеткой ко\-опе\-ра\-тив\-ной игры и ее ядра. Положим $I^N := \{(\tau_1, \ldots, \tau_n)
\in \mathbb{R}^N \mid \tau_i \in [0,1], \, i \in N\}.$ Ненулевые элементы гиперкуба $I^N$ называются {\it нечеткими коалициями} (см., например,~\cite{A93}). Компонента $\tau_i$ нечеткой коалиции $\tau$ интерпретируется как уровень участия игрока $i$ в большой коаиции $N.$  Далее, для каждой нечеткой коалиции $\tau$ положим $N(\tau) := \{i \in N \mid \tau_i > 0\}.$ Как и в случае с клас\-си\-чес\-ки\-ми играми, нечеткие ко\-опе\-ра\-тив\-ные игры отож\-дест\-вля\-ют\-ся с их ха\-рак\-те\-рис\-ти\-чес\-ки\-ми функциями $v:\sigma_F \rightarrow \mathbb{R}$~\cite{A93}, где $\sigma_F := I^N \setminus \{0\}$. К наиболее важным примерам нечетких ко\-опе\-ра\-тив\-ных игр относятся так называемые {\it расширения Обэна}  $v_{Au}$ обычных ко\-опе\-ра\-тив\-ных игр $v \in V(N):$ $v_{Au}(\tau) := \sum_{S \in \sigma_0}v_S \prod_{i \in S}\tau_i^{1/|S|}, \, \tau \in \sigma_F,$ где $\sigma_0 := 2^N \setminus \{\emptyset\}.$

Приведем еще два ключевых понятия - блокирование в нечеткой игре и определение ее ядра
(далее $e_N$ - индикаторная функция множества $N:$ $(e_N)_i = 1, \, i \in N,$ и $G_v(e_N) := \{x \in \mathbb{R}^N \mid \sum_{i \in N}x_i \leq v(e_N)\}$ ). Будем говорить, что коалиция $\tau \in \sigma_F$ {\it блокирует} распределение выигрышей (кратко: выигрыш) $x \in G_v(e_N)$ нечеткой игры $v,$ если существует вектор $y = (y_i)_{i \in N(\tau)}$ такой, что $(b1) \; y_i > x_i, \, i \in N(\tau); (b2) \; \sum_{i \in N(\tau)}\tau_i y_i \leq v(\tau).$ Совокупность $C(v)$ выигрышей $x \in G_v(e_N),$ которые не блокируются никакой нечеткой калицией $\tau,$ называется {\it ядром} нечеткой игры $v.$

Основной результат работы опирается на тесную вза\-имо\-связь ядер и су\-пер\-диф\-фе\-рен\-циа\-лов некоторых классов ко\-опе\-ра\-тив\-ных игр. Напомним сначала понятие су\-пер\-гра\-ди\-ен\-та нечеткой игры: вектор $c \in \mathbb{R}^N$ называется су\-пер\-гра\-ди\-ен\-том игры $v: \sigma_F \rightarrow \mathbb{R}$ в точке
$\tau^0 \in \sigma_F,$ если выполняется условие $v(\tau) - v(\tau^0) \leq c \cdot (\tau - \tau^0), \, \tau \in \sigma_F.$ Положим $e_N* = \frac{1}{n}e_N.$ Множество $\hat{\partial} v(e_N*)$ су\-пер\-гра\-ди\-ен\-тов игры $v$ в точке $e_N*$ будем называть {\it су\-пер\-диф\-фе\-рен\-циа\-лом игры} $v$~\cite{V21}. Условия совпадения $C(v)$ и $\hat{\partial} v(e_N^*)$ указаны в~\cite{V21}.

Наряду с результатами из~\cite{V21}, при анализе ядра расширения Обэна используется и одно из следствий известной теоремы Моро-Ро\-ка\-фел\-ла\-ра, от\-но\-ся\-щей\-ся к формуле суб\-диф\-фе\-рен\-циа\-ла суммы  
функций~\cite{IT74} (ниже $\mbox{dom}~f_i := \{x \in \mathbb{R}^N \mid f_i(x) < \infty\}$).


%\begin{corollary*}
\noindent {\bf Следствие} \cite{IT74} {\it Пусть собственные
выпуклые функции $f_1, \ldots, f_m$ таковы, что каждая из них
является непрерывной в некоторой точке $\bar{x} \in X,$ где $X =
\cap_{i=1}^m \mbox{dom}~f_i.$ Тогда в каждой точке $x \in X$
субдифферециал их суммы равен сумме (по Минковскому)
субдифференциалов этих функций.}
%\end{corollary*}

Приведем основной результат работы. Ниже $\Phi(v) = (\Phi(v)_1, \ldots, \Phi(v)_n)$ - {\it значение Шепли} игры $v \in V(N)$: $\Phi(v)_i := \sum_{S \in \sigma_i}v_S/|S|, \, i \in N,$ где $\sigma_i := \{S \subseteq N \mid i \in S\}.$

%\begin{theorem*}
\noindent {\bf Теорема} {\it Если $v \in V(N)$ является почти по\-ло\-жи\-тель\-ной игрой, то $C(v_{Au}) = \{\Phi(v)\}.$}
%\end{theorem*}


% Современные издательства требуют использовать кавычки-елочки << >>.

% В конце текста можно выразить благодарности, если этого не было
% сделано в ссылке с заголовка статьи, например,
%Работа выполнена при поддержке РФФИ (РНФ, другие фонды), проект \textnumero~00-00-00000.
%

% Список литературы оформляется подобно ГОСТ-2008.
% Примеры оформления находятся по этому адресу -
%     https://narfu.ru/agtu/www.agtu.ru/fad08f5ab5ca9486942a52596ba6582elit.html
%

\begin{thebibliography}{9} % или {99}, если ссылок больше десяти.

\bibitem{A93}
Aubin~J.-P. Optima and Equilibria. Berlin-Heidelberg:~Springer-Verlag,~1993.

\bibitem{V21}
Vasil'ev~V.A. Core and super-differential of a fuzzy TU-cooperative game~// Autom.~and~Remote~Control. 2021. Vol.~82(5). Pp. 926--934.

\bibitem{IT74}
Иоффе~А.Д., Тихомиров~В.М. Теория экстремальных проблем. М.:~Наука,~1974.

\end{thebibliography}


% После библиографического списка в русскоязычных статьях необходимо оформить
% англоязычный заголовок.



%\end{document}
