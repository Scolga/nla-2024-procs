\begin{englishtitle} % Настраивает LaTeX на использование английского языка
% Этот титульный лист верстается аналогично.
\title{Distributed software system for reservoir shoreline recognition based on a pre-trained neural network}
% First author
\author{Evgeny Cherkashin\inst{1} \and  Oksana Mazaeva\inst{2}
}
\institute{Matrosov Institute for System Dynamics and Control Theory SB RAS, Irkutsk, Russia\\
  \email{eugeneai@icc.ru}
  \and
Institute of Earth's Crust SB RAS, Irkutsk, Russia\\
\email{email@example.com}}
% etc

\maketitle

\begin{abstract}
The problem of automation of shoreline contouring on satellite images and orthophotos is considered. A number of image processing stages are carried out: recognition of separate image areas using Segment Anything technology, then algorithmic processing of the areas and assignment of additional features is performed, then, on the basis of logical inference, the areas containing the shoreline contour are identified. The technology is implemented as a distributed software system including a recognition server and a GIS. The technology is being tested on the monitoring of water reservoirs of the Irkutsk region.

\keywords{deep learning models, logical inference, image analysis, geoinformation system, engineering geology} % в конце списка точка не ставится
\end{abstract}
\end{englishtitle}

\iffalse
%%%%%%%%%%%%%%%%%%%%%%%%%%%%%%%%%%%%%%%%%%%%%%%%%%%%%%%%%%%%%%%%%%%%%%%%
%
%  This is the template file for the 6th International conference
%  NONLINEAR ANALYSIS AND EXTREMAL PROBLEMS
%  June 25-30, 2018
%  Irkutsk, Russia
%
%%%%%%%%%%%%%%%%%%%%%%%%%%%%%%%%%%%%%%%%%%%%%%%%%%%%%%%%%%%%%%%%%%%%%%%%


\documentclass[12pt]{llncs}

% При использовании pdfLaTeX добавляется стандартный набор русификации babel.

\usepackage{iftex}

\ifPDFTeX
\usepackage[T2A]{fontenc}
\usepackage[utf8]{inputenc} % Кодировка utf-8, cp1251 и т.д.
\usepackage[english,russian]{babel}
\fi

\usepackage{todonotes}

\usepackage[russian]{nla}

\begin{document}
\fi

\title{Распределенный программный комплекс распознавания береговой линии водохранилищ на основе предобученной нейронной 
сети\thanks{Работа выполнена при поддержке базовых проектов \textnumero~FWEW-2021-0005,FWEF-2021-0009.}}
% Первый автор
\author{Е.~А.~Черкашин\inst{1} \and О.~А.~Мазаева\inst{2}
}

% Аффилиации пишутся в следующей форме, соединяя каждый институт при помощи \and.
\institute{Институт динамики систем и теории управления СО РАН, Иркутск, Россия \\
  \email{eugeneai@icc.ru}
  \and   % Разделяет институты и присваивает им номера по порядку.
Институт земной коры СО РАН, Иркутск, Россия\\
  \email{moks@crust.irk.ru}
% \and Другие авторы...
}

\maketitle

\begin{abstract}
Рассматривается задача автоматизации построения контура береговой линии на спутниковых изображениях и ортофотопланах. Проводится ряд этапов обработки изображений: распознавание отдельных областей изображения при помощи технологии Segment Anything, затем производится их алгоритмическая обработка и назначение  дополнительных характеристик, далее, на основе логического вывода выделяются области, содержащие контур береговой линии. Технология реализована в виде распределенного программного комплекса, включающего сервер распознавания и ГИС. Апробация технологий осуществляется на задаче мониторинга водохранилищ Иркутской области.

\keywords{модели глубокого обучения, логический вывод, анализ изображений, геоинформационная система, инженерная геология} % в конце списка точка не ставится
\end{abstract}

% \section{Основные результаты} % не обязательное поле

Фундаментальные исследования уникальных экологических систем, какой является береговая зона крупного водного объекта, проводимые в мире и России, базируются на мониторинге, основывающимся на больших данных. Мониторинг опасных экзогенных геологических процессов входит в единую информационную систему Государственного мониторинга состояния недр, который является составной частью государственного мониторинга состояния и загрязнения окружающей среды. Результаты долговременного мониторинга Братского водохранилища показали, что после более 50-ти лет эксплуатации Ангарских водохранилищ береговая зона все еще не достигла стадии устойчивого равновесия. Разрушение берегов под воздействием природных геологических процессов (абразии, карста, эрозии, оползней) в значительной мере осложняет функционирование береговой зоны. От стабильности береговой зоны зависит возможность ее технического, рекреационного и др. видов использования, особенно в условиях, когда уровень воды регулируется технически в достаточно большом диапазоне значений сезонного (2-3~м) и многолетнего регулирования (до 10~м).

Наборы данных мониторинга характеризуются разнородностью и разноструктурированностью (электронные таблицы, тематические карты, космоснимки, 3D-модели, фото и видеоизображения), пространственно-временной привязкой, большим объемом и высокой скоростью роста объемов информации. В данном исследовании рассматривается проблема автоматизированного преобразования растрового представления изменяющегося во времени контура  береговой линии крупного равнинного водохранилища в векторный формат для дальнейшего мониторинга геологических
процессов.

Современные средства [1-3] автоматизации построения контуров береговых линий строятся с использованием нейронных сетей (НС) глубокого обучения, получаемых, в некоторых случаях, при помощи дообучения. Процесс обучения организуется на основе большого количества размеченных спутниковых изображений, разметка делается вручную, что достаточно трудоемко. Применяемый в данной работе подход основывается на использовании предобученной НС Segment Anything (SA), распознающей предметы (порядка миллиона классов) на изображении. SA не обучалась на распознавание береговых линий, но тестовые испытания показали хорошие результаты выделения контуров областей, граничащих с береговой линией. Чтобы определить контур, необходимо распознать объекты, включающие точки контура. Для этого необходимо провести процесс распознавания, задаваемый в виде правил, системы основанной на знаниях.

Процесс распознавания контура реализуется пошагово: 

1) в ГИС (геоинформационной системе) выбирается интересующая область и контур береговой линии с векторной карты OpenStreetMap, данный контур является начальным приближением результата; 

2) для выбранной области с серверов дистанционного зондирования загружаются изображения с разметкой по конкретным датам, изображения помещаются в серверное хранилище; 

3) запуск системы SegmentAnything в режиме распознавания всех возможных объектов, результат распознавания в виде набора бинарных масок сохраняется на сервере; 

4) алгоритмический анализ свойств объектов, представленных масками, и их расположение относительно друг друга и краёв изображения, результаты сохраняются в А-боксе онтологии на сервере; 

5) анализ взаимного расположения объектов, выявление множества точек, относящихся к береговой линии; 

6) построение контура по выявленным точкам (перевод в векторный формат), сохранение результата в shape-файл.

Преимущества рассмотренного подхода: 

а) отсутствие трудоемкого этапа подготовки данных для обучения НС, 

б) распознавание контура представляется (программируется) в виде правил, что дает возможность управлять процессом распознавания, в частности <<сцеплять>> объекты или контуры с разных изображений; 

в) SA, ввиду обученности на большом количестве изображений, не привязана к свойствам конкретных изображений (размеру, цветовой гамме, повороту и т.д.), что позволяет 
реализовывать (в перспективе) процедуры последовательного уточнения характеристик береговой линии, переходя к изображениям более высокого разрешения.

Разработанная платформа и модели обеспечат более эффективный способ оцифровки данных с изображений, результаты будут востребованы во многих проектных, изыскательских и научно-исследовательских организациях для проведения комплексных оценок, прогнозирования и принятия решений по управлению различными объектами береговой зоной для обеспечения их безопасного и рационального использования.

% Рисунки и таблицы оформляются по стандарту класса article. Например,

\begin{thebibliography}{9} % или {99}, если ссылок больше десяти.
\bibitem{b1} Dang K. B., Vu K. C., Nguyen H., Nguyen D. A., \emph{et al.}. Application of deep learning models to detect coastlines and shorelines.  Journal of Environmental Management. 2022. No.~320. 115732.

\bibitem{b2} Seale C., Redfern T., Chatfield P., Luo C., Dempsey K. Coastline detection in satellite imagery: A deep learning approach on new benchmark data. Remote Sensing of Environment. 2022. No.~278. 113044.

\bibitem{b3} Aghdami-Nia M., Shah-Hosseini R., Rostami A., Homayouni S. Automatic coastline extraction through enhanced sea-land segmentation by modifying Standard U-Net. International Journal of Applied Earth Observation and Geoinformation. 2022. No.~109. 102785.
\end{thebibliography}

% После библиографического списка в русскоязычных статьях необходимо оформить
% англоязычный заголовок.




%\end{document}

%%% Local Variables:
%%% mode: LaTeX
%%% TeX-master: t
%%% End:
