\begin{englishtitle} % Настраивает LaTeX на использование английского языка
% Этот титульный лист верстается аналогично.
\title{Realization of additional variables method in problems of dynamics}
% First author
\author{Mikhail Bekhovskiy}
\institute{Saint-Petersburg State University, Saint-Petersburg, Russia\\
  \email{mbexovskij@gmail.com}}
% etc

\maketitle

\begin{abstract}
We propose adapted algorithms based on additional variables method \cite{AlgoDE,AlgoFunc,AVM} and their program realization in Python. Using our program one can reduce systems of total partial differential equations to a form with polynomial right-hand sides (which in turn can be numerically solved by efficient special methods, for example \cite{TSM}) or symbolically differentiate multivariable functions. The program can transform expressions written in terms of library of functions and basic arithmetical operators. The library is organized as a separate module within the program which can be modified by a potential user. This allows our program to be used to symbolically differentiate multivariable functions written in terms of special functions not represented in popular computer algebra systems. Additionally we could not find an existing program, module or library with which one could automatically reduce systems of differential equations to polynomial form.
Our program is located in a public repository on GitHub\cite{repo} and does not have any proprietary dependencies.
The work of our program is demonstrated on some model examples.

\keywords{symbolic differentiation, polynomial system, additional variables method} % в конце списка точка не ставится
\end{abstract}
\end{englishtitle}

\iffalse
%%%%%%%%%%%%%%%%%%%%%%%%%%%%%%%%%%%%%%%%%%%%%%%%%%%%%%%%%%%%%%%%%%%%%%%%
%
%  This is the template file for the 6th International conference
%  NONLINEAR ANALYSIS AND EXTREMAL PROBLEMS
%  June 25-30, 2018
%  Irkutsk, Russia
%
%%%%%%%%%%%%%%%%%%%%%%%%%%%%%%%%%%%%%%%%%%%%%%%%%%%%%%%%%%%%%%%%%%%%%%%%


\documentclass[12pt]{llncs}  

% При использовании pdfLaTeX добавляется стандартный набор русификации babel.

\usepackage{iftex}

\ifPDFTeX
\usepackage[T2A]{fontenc}
\usepackage[utf8]{inputenc} % Кодировка utf-8, cp1251 и т.д.
\usepackage[english,russian]{babel}
\fi

\usepackage{todonotes} 

\usepackage[russian]{nla}

\begin{document}
\fi

\title{Реализация метода дополнительных переменных в задачах динамики}
% Первый автор
\author{М.~А.~Беховский}

% Аффилиации пишутся в следующей форме, соединяя каждый институт при помощи \and.
\institute{СПбГУ, Санкт-Петербург, Российская Федерация \\
  \email{mbexovskij@gmail.com}
}

\maketitle

\begin{abstract}
Предлагаются адаптированные алгоритмы и программная реализация метода дополнительных переменных \cite{AlgoDE,AlgoFunc,AVM} на языке Python. Метод позволяет сводить полные системы дифференциальных уравнений в частных производных первого порядка к эквивалентным системам с полиномиальными правыми частями. Численное решение полученных систем, в свою очередь, можно осуществлять предназначенными для полиномиальных систем методами, более эффективными, чем численные методы решения дифференциальных уравнений общего назначения\cite{TSM}. Кроме того, представленные алгоритмы позволяют осуществлять символьное дифференцирование функций многих переменных широкого класса \cite{AVM}.
Помимо самих алгоритмов \cite{AlgoDE,AlgoFunc}, адаптированных для реализации на языке Python, ключевой составляющей программы является библиотека функций.  В терминах представленной библиотеки могут быть записаны обрабатываемые выражения. Достоинством явного выделения указанной библиотеки является возможность, без особых затруднений, пополнять ее нужными пользователю функциями. Благодаря этому после незначительной модификации пользователь сможет использовать нашу программу для работы с выражениями, содержащими функции, отсутствующие в поставляемой с программой версией библиотеки. При использовании предлагаемой программы для символьного дифференцирования данное свойство, как нам кажется, положительно выделяет ее на фоне популярных пакетов символьной алгебры. Изменение ядра существующих пакетов либо невозможно (в случае коммерческого ПО с закрытым исходным кодом такого, как: Mathematica, Matlab, Maple), либо связано со значительными трудностями из-за ориентирования пакета на решение широкого круга разнородных задач (например, пакет SymPy). В то же время методы, позволяющие сводить системы уравнений к полиномиальной форме, не были нами обнаружены в указанных выше пакетах. 
Данная программа расположена в публичном репозитории\cite{repo} вместе с руководством к использованию и не имеет проприетарных зависимостей (распространяется в качестве свободного программного обеспечения с открытым исходным кодом), что позволяет всякому желающему использовать ее в своей работе без дополнительных издержек.
Корректность и актуальность нашей программы демонстрируется на модельных примерах.

\keywords{символьное дифференцирование, полиномиальная система, метод дополнительных переменных} % в конце списка точка не ставится
\end{abstract}

\begin{thebibliography}{9} % или {99}, если ссылок больше десяти.

\bibitem{AlgoDE} Бабаджанянц~Л.К., Брэгман~К.М. Алгоритм метода дополнительных переменных~// Вестник СПбГУ. Сер. 10. 2012. \textnumero~2. С.~3--12.

\bibitem{AlgoFunc} Брэгман~К.М. Алгоритм дифференцирования, основанный на методе дополнительных переменных~// Вестник СПбГУ. Сер. 10. 2013. \textnumero~2. С.~14--24.

\bibitem{AVM} Бабаджанянц~Л.К. Метод дополнительных переменных~// Вестник СПбГУ. Сер.10. 2010. \textnumero~1. С.~3--11.

\bibitem{TSM} Бабаджанянц~Л.К. Большаков А.И. Реализация метода рядов Тейлора для решения обыкновенных дифференциальных уравнений~// Вычислительные методы и программирование. Науч.-исслед. вычисл. центр Моск. ун-та им. М.В. Ломоносова. 2012. Т.~13. С.~497--510.

\bibitem{repo} AVM program GitHub repository~// URL: https://github.com/MikhailBekhovskiy/AVM
\end{thebibliography}

% После библиографического списка в русскоязычных статьях необходимо оформить
% англоязычный заголовок.




%\end{document}

