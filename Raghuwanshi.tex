
\iffalse
%%%%%%%%%%%%%%%%%%%%%%%%%%%%%%%%%%%%%%%%%%%%%%%%%%%%%%%%%%%%%%%%%%%%%%%%
%
% This is the template file for the 6th International conference
% NONLINEAR ANALYSIS AND EXTREMAL PROBLEMS
% June 25-30, 2018
% Irkutsk, Russia
%
%%%%%%%%%%%%%%%%%%%%%%%%%%%%%%%%%%%%%%%%%%%%%%%%%%%%%%%%%%%%%%%%%%%%%%%%
% The preparation of the article is based on the standard llncs class
% (Lecture Notes in Computer Sciences), which is adjusted with style
% file of the conference.
%
% There are two ways of compilation of the file into PDF
% 1. Use pdfLaTeX (pdflatex), (LaTeX+DVIPS will not work);
% 2. Use LuaLaTeX (XeLaTeX will work too).
% When using LuaLaTeX You will need TTF or OTF CMU fonts
% (Computer Modern Unicode). The fonts are installed with 'cm-unicode' package in
% a distribution of LaTeX % (https://www.ctan.org/tex-archive/fonts/cm-unicode),
% either by downloading and installing these fonts system wide, the address of their page is
% http://canopus.iacp.dvo.ru/%7Epanov/cm-unicode/
% The second option won't work in XeLaTeX.
%
% For MiKTeX (LaTeX distribution for Windows),
%  1. Package 'cm-unicode' is installed manually with the MiKTeX administration Console.
%  2. For the compilation of this example, namely, the stub figure, one will also need to
% download package 'pgf' manually. This package uses in the popular
% package tikz.
%  3. Tests showed that the rest of the required packages MiKTeX loads automatically (if
%     it is allowed). The 'auto download' option is
%     configured in 'Settings' section in MiKTeX Console.
%
%
% The easiest way to compile an article is to use pdfLaTeX, but
% the final layout of the book will be compiled with LuaLaTeX,
% as a result will be of better quality thanks to the package 'microtype' and
% use vector OTF instead of standard raster fonts of pdfLaTeX.
%
% In the case of questions and problems with the article compilation,
% write letters to e-mail: eugeneai@irnok.net, Cherkashin Evgeny.
%
% New version of the correcting style file will be available at the website:
%     https://github.com/eugeneai/nla-style
%     file - nla.sty
%
% Further instructions are in the text body of the template. The template itself
% is an article example.
%
% The LaTeX2e format is used!

% 12 points font size is used.
\documentclass[12pt]{llncs}

% The correcting style file is added.
\usepackage{todonotes}

\usepackage{nla} % This package is needed for compiling
                 % this template, it should be removed
                 % from your article.

% Many popular packages (amsXXX, graphicx, etc.) are already imported in the style file.
% If there is a conflict with your packages, try disabling them and compile
% the text.
%
% It would be convenient in the layout of the proceedings if the file names
% of the figures of different authors do not clash.
% To minimize the clash, the drawings can be placed in a separate subfolder
% named after the author or the title of the paper.
%
% \graphicspath{{ivanov-petrov-pics/}} % specifies the folder with images in png, pdf formats.
% or
% \graphicspath{{great-problem-solving-paper-pics/}}.

%\theoremstyle{plain}
%\newtheorem{thm}{Theorem}
\newtheorem{hp}{Hypothesis}


\begin{document}
\fi
% Text should be formatted in accordance with the 'article' class, using extensions like
% AMS.
%
\title{On controllability of semi-linear control delay systems
}
% First author
\author{K. Raghuwanshi}
\institute{Department of Mathematics, IGNTU Amarkantak, Madhya Pradesh, India \\
  \email{kamlesh.raghuwanshi94@gmail.com}}
% etc

\maketitle

\begin{abstract}
This work presents a study on the approximate controllability of abstract semilinear control delay system in Hilbert spaces by assuming that the semigroup generated by the linear operator is compact. Sufficient conditions have been established on the components of corresponding linear delay control system and the nonlinear term to guarantee the approximate controllability. The main result is explored by using the reachable set, the quotient space and the degree theory. The application of the obtained result is demonstrated by an example of partial differential equation.
\keywords{Delay systems, reachable sets, approximate controllability, degree theory}
\end{abstract}

% at the end of the list, there should be no final dot
\section{The main results}
Let $X$ and $U$ be Hilbert spaces of state and control variables. Consider the semi-linear control delay system.
\begin{equation}\label{eqn-main-syst}
	\left\{
	\begin{aligned}
		& \dot{x}(t)=Ax(t)+Lx(t-h)+B u(t)+B_1 u(t-h)+f(t,x(t),u(t)), \;\;t \in[0,T]  \\
		& x(\theta)=\phi(\theta), \;\; \theta \in [-h,0],  ~ h>0,
	\end{aligned}
	\right.	
\end{equation}
where $ A:D(A) \subset X \rightarrow X$ is closed linear operator; $ L:X \rightarrow X $ and $B, B_1: U \rightarrow X $ are bounded linear operators; $ \phi \in C \left( [-h,0];X \right)$ is initial condition; $f:[0,T]\times X \times U \rightarrow X  $ is a non-linear map.

\begin{hp}\label{A1}
Let us suppose that $A$ generates a $C_0$-semigroup $\{S(t)\}_{t\ge 0}$ on $X$ with $\|S(t)\| \le Me^{\omega T} = M_0$.
\end{hp}

Since $L$ is bounded linear operator on $X$, therefore $A+L$ generates a $C_0$-semigroup $\{\hat{S}(t)\}_{t\ge 0}$ on $X$, which satisfies $ \| \hat{S}(t) \| \leq M_0 e^{M_0M_LT} = M_1 $, where $ M_0=Me^{\omega T} $ and $M_L = ||L||$. If $\{S(t)\}_{t\ge 0}$ is compact, then $\{\hat{S}(t)\}_{t\ge 0}$ is compact. 

\begin{hp}\label{f1}
	The nonlinear operator $f$ is Lipschitz continuous in $x$, i.e. there exists a constant $K>0$ such that
	\begin{equation*}
		\| f(t,x_1(t),u(t))-f(t,x_2(t),u(t)) \|_X \leq K \| x_1-x_2 \|_X
	\end{equation*}
	where $x_1, x_2 \in X$.
\end{hp}
Then, the unique mild solution of \eqref{eqn-main-syst} is given by
\begin{equation}\label{eqn-main-soln}
	x(t) = \begin{cases}
		\phi(t), \mbox{ if } t \in [-h,0] \\
		S(t)\phi(0) + \int_{0}^{t} S(t-s) L x(s-h) ds \\ ~~~~~~ + \int_{0}^{t} S(t-s)[B u(s) + B_1 u(s-h) + f(s,x(s),u(s))] ds, ~ t > 0.
	\end{cases}
\end{equation}

\begin{definition}
The reachable set $\mathcal{R}_T(L,f)$ of \eqref{eqn-main-syst} is defined as the set of all $x(T,u,f,\phi) \in X$ such that $x(\cdot,u,f,\phi) \in C ( [-h, T];X )$ is the mild solution of \eqref{eqn-main-syst} under some control $u \in L^{2}([0,T]; U)$.
\end{definition}
Consider the linear delay control system associated to \eqref{eqn-main-syst} is
\begin{equation}\label{eqn-delay-linear-syst}
	\left\{
	\begin{aligned}
		& \dot{x}(t)=Ax(t)+Lx(t-h)+B u(t)+B_1 u(t-h), \;\;t \in[0,T]  \\
		& x(\theta)=\phi(\theta), \;\; \theta \in [-h,0].
	\end{aligned}
	\right.	
\end{equation}

\begin{definition}
The reachable set $\mathcal{R}_T(L)$ of \eqref{eqn-delay-linear-syst} is defined as the set of all $x(T,u, \phi) \in X$ such that $x(\cdot,u,\phi) \in C ( [-h,T];X )$ is the solution of \eqref{eqn-delay-linear-syst} for some $u \in L^{2}([0,T]; U)$.
\end{definition}

%In this work, the approximate controllability on $[0,T]$ will be presented for semi-linear control delay system \eqref{eqn-main-syst} by assuming that the linear control delay system \eqref{eqn-delay-linear-syst} is approximately controllable. The controllability concept is defined as follows:

\begin{definition}
The linear control delay system is approximately controllable on $ [0,T] $ if $ \overline{\mathcal{R}_T(L)} = X $ and exactly controllable on $ [0,T] $ if $ \mathcal{R}_T(L) = X $.
\end{definition}

\begin{definition}
	The semi-linear control delay system is approximately controllable on $ [0,T] $ if $ \overline{\mathcal{R}_T(L,f)} = X $ and exactly controllable on $ [0,T] $ if $ \mathcal{R}_T(L,f) = X $.
\end{definition}
Let us define operator $ J^T:L^{2}([0,T]; X) \rightarrow X $ by
\begin{equation*}
		(J^T)(p)=\int_{0}^{T} {S}(T-s)p(s)ds
\end{equation*}
and the controllability map $\mathcal{B} : L^{2}([0,T]; U) \rightarrow L^{2}([0,T]; X)$ by $(\mathcal{B} u)(t) = B u(t) + B_1 u(t-h)$, where $u(t-h) = 0$ for $t-h < 0$. Let $\ker(J^T)$ be the null space of $J^T$ and $ \mathcal{R}(\mathcal{B}) $ be the range space of $\mathcal{B}$. Then, we have $L^{2}([0,T]; X) = \ker(J^T) + \overline{\mathcal{R}(\mathcal{B})}$ with a projection operator $P: L^{2}([0,T]; X) \rightarrow L^{2}([0,T]; X)$ such that $\mathcal{R}(P)=\ker(J^T)$. We also consider hypotheses:
\begin{hp}\label{B1}
	For each $p \in L^{2}([0,T]; X)$, there exists $q \in \overline{\mathcal{R}(\mathcal{B})}$ such that $J^Tp=J^Tq$.
\end{hp}

\begin{hp}\label{f2}
	$f$ is uniformly bounded, i.e. there exists a constant $M_f>0$ such that
	\begin{equation*}
		\| f(t,x(t),u(t)) \| \leq M_f ~~ \forall ~~ (t,x(t),u(t)) \in [0,T] \times X \times U.
	\end{equation*}
\end{hp}

\begin{theorem}
Suppose that Hypothesis \ref{A1} - Hypothesis \ref{f2} hold. Then $\mathcal{R}_T(L) \subset \overline{\mathcal{R}_T(L,f)}$.
%Thus, as $\overline{\mathcal{R}_T(L)}=X$, the system \eqref{eqn-main-syst} is approximately controllable.
\end{theorem}

% At the end of the text, acknowledgments are expressed, if you haven't
% made a footnote from the title. For example, we can write
The authors acknowledge the IGNTU Amarkantak for providing facilities to carry out this work.

\begin{thebibliography}{9} % or {99}, if there is more than ten references.

\bibitem{Naito1987}  Naito K. Controllability of semilinear control systems dominated by the linear part. SIAM J. Control And Optimization.~1987. Vol.~25. Pp.~715--722.

\bibitem{Her2015} Henr\'{\i}quez H. R., Prokopczyk A. Controllability and stabilizability of linear time-varying distributed hereditary control systems. Math. Meth. Appl. Sci.~2015.~Vol.~38. Pp.~2250--2271.

\bibitem{SukTaf2011} Sukavanam N., Tafesse S. Approximate controllability of a delayed semilinear control system with growing nonlinear term. Nonlinear Analysis.~2011.~Vol.~74.~6868--6875.

\end{thebibliography}
%\end{document}

%%% Local Variables:
%%% mode: latex
%%% TeX-master: t
%%% End:
