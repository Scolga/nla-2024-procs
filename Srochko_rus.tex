\begin{englishtitle} % Настраивает LaTeX на использование английского языка
% Этот титульный лист верстается аналогично.
\title{Parametric regularization of the functional in a linear-quadratic optimal control problem}
% First author
\author{Vladimir Srochko
  \and
  Alexander Arguchintsev
}
\institute{Irkutsk State University, Irkutsk, Russia\\
  \email{arguch@math.isu.ru}}


\maketitle

\begin{abstract}
A linear-quadratic optimal control problem with weight coefficients and arbitrary matrices in the quadratic cost functional is considered on the set of stepwise control functions. As a result, parameter optimization problems are constructed which provide a strong convexity of the objective function on control variables together with relatively good conditionality of the corresponding quadratic programming problem.

\keywords{linear-quadratic optimal control problem, cost functional with weight coefficients, parameter optimization} % в конце списка точка не ставится
\end{abstract}
\end{englishtitle}



\iffalse
\documentclass[12pt]{llncs} 
\usepackage{iftex}

\ifPDFTeX
\usepackage[T2A]{fontenc}
\usepackage[utf8]{inputenc} % Кодировка utf-8, cp1251 и т.д.
\usepackage[english,russian]{babel}
\fi

% Для верстки в LuaLaTeX текст готовится строго в utf-8!

% В операционной системе Windows для редактирования в кодировке utf-8
% можно использовать программы notepad++ https://notepad-plus-plus.org/,
% techniccenter http://www.texniccenter.org/,
% SciTE (самая маленькая по объему программа) http://www.scintilla.org/SciTEDownload.html
% Подойдет также и встроенный в свежий дистрибутив MiKTeX редактор
% TeXworks.

% Добавляется корректирующий стилевой файл строго после babel, если он был включен.
% В параметре необходимо указать russian, что установит не совсем стандартные названия
% разделов текста, настроит переносы для русского языка как основного и т.п.

\usepackage{todonotes} % Этот пакет нужен для верстки данного шаблона, его
                       % надо убрать из вашей статьи.

\usepackage[russian]{nla}

\begin{document}
\fi

\title{Параметрическая регуляризация функционала в линейно-квадратичной задаче оптимального
 управления\thanks{Исследование А.В. Аргучинцева выполнено за счет гранта Российского научного фонда, проект \textnumero~23-21-00296, https://rscf.ru/project/23-21-00296/}}.

% Первый автор
\author{В.~А.~Срочко  % \inst ставит циферку над автором.
  \and  % разделяет авторов, в тексте выглядит как запятая.
% Второй автор
  А.~В.~Аргучинцев
  } % обязательное поле

% Аффилиации пишутся в следующей форме, соединяя каждый институт при помощи \and.
\institute{Иркутский государственный университет, Иркутск, Россия\\
  \email{arguch@math.isu.ru}
  }

\maketitle

\begin{abstract}
На множестве ступенчатых управляющих функций исследуется ли\-ней\-но-квад\-ра\-тич\-ная задача оптимального управления с весовыми коэффициентами и произвольными матрицами в квадратичном целевом функционале. Построены задачи оптимизации параметров, которые обеспечивают сильную выпуклость целевой функции по управляющим переменным вместе с относительно хорошей обусловленностью соответствующей задачи квадратичного программирования.

\keywords{линейно-квадратичная задача оптимального управления, целевой функционал с весовыми коэффициентами, оптимизация параметров} % в конце списка точка не ставится
\end{abstract}

Для линейной управляемой системы
\begin{equation*}
\dot x=A(t)x + b(t)u, \; x(t_0)=x^0
\end{equation*}
с двусторонними ограничениями на управление
\begin{equation*}
u(t) \in [u_{-}, u_{+}], \; t \in [t_0, T].
\end{equation*}
рассматривается задача на минимум квадратичного функционала
\begin{equation*}
\Phi(u, x)= 1/2 \; \alpha \langle x(T), Px(T)\rangle + 1/2  \int\limits_{t_0}^{T}
	{[\beta \langle x(t), Q(t)x(t) \rangle + \gamma u^2(t)] dt}
\end{equation*}
Здесь $t \in [t_0, T]$, $u(t) \in \Re$, $x(t) \in \Re^n$, а числа  $\alpha$, $\beta$ и $\gamma$ являются весовыми коэффициентами-параметрами. Множество допустимых управлений определяется следующим образом. Введем на отрезке $[t_0, T]$ сетку узлов $\{{t_0, t_1, \ldots , t_{m-1}, t_m = T}\}$ с условиями $t_j = t_{j-1} + h_j$, $h_j > 0$, $j = 1, 2, \ldots , m$. Выделим промежутки $T_1=[t_0, t_1]$, $T_j = (t_{j-1}, t_j]$, $j = 2, \ldots , m,$ вместе с соответствующими характеристическими функциями $\chi_1(t)$, $\chi_j(t)$, $t \in [t_0, T]$.
Для вектора $y=(y_1, y_2, \ldots ,y_m)$ введем множество допустимых значений
\begin{equation*}
Y = \{ y \in \Re^m: y_j \in [u_{-}, u_{+}], \; j = 1, 2, \ldots , m \}.
\end{equation*}
и сформируем управление
\begin{equation*}
u(t,y) =  \sum_{j=1}^{m} y_j \chi_j(t), \; t \in [t_0, T], y \in Y.
\end{equation*}
Это кусочно-постоянная функция со значениями $y_j$ на заданных промежутках $T_j$, удовлетворяющая введенным ограничениям.

Исходная задача оптимального управления в результате сводится к задаче квадратичного программирования с целевой функцией, зависящей от параметров. В качестве критерия качества допустимого набора параметров предлагается выбрать число обусловленности итоговой матрицы, которое выражается через границы ее спектра. Соответствующие спектральные подходы к  поиску параметров $\alpha$, $\beta$, $\gamma$ обеспечивают сильную выпуклость целевой функции вместе с относительно хорошей обусловленностью задачи. В этих условиях полученная задача квадратичного программирования с простейшими ограничениями допускает эффективное численное решение за конечное число итераций методами особых точек, сопряженных градиентов или опорным методом.

Предлагаемый подход продолжает направление исследований, представленное в публикациях~\cite{srochko1,srochko2}, и ликвидирует существовавшую ранее неопределенность в выборе параметров регуляризации.


\begin{thebibliography}{9} % или {99}, если ссылок больше десяти.
\bibitem{srochko1} Аргучинцев~А.В., Срочко~В.А. Процедура регуляризации билинейных задач оптимального управления на основе конечномерной модели~// Вестник С.-Петербург. ун-та. Прикл. математика. Информатика. Процессы управления. 2022. Т. 18, \textnumero~1. С.~179--187.

\bibitem{srochko2} Аргучинцев~А.В., Срочко~В.А. Решение линейно-квадратичной задачи на множестве кусочно-постоянных управлений с параметризацией функционала~// Труды ИММ УрО РАН. 2022. Т. 20, \textnumero~3. С.~5--16.

\end{thebibliography}

% После библиографического списка в русскоязычных статьях необходимо оформить
% англоязычный заголовок и аннотацию.



%\end{document} 