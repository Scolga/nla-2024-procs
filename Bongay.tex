\iffalse

%%%%%%%%%%%%%%%%%%%%%%%%%%%%%%%%%%%%%%%%%%%%%%%%%%%%%%%%%%%%%%%%%%%%%%%%
%
% This is the template file for the 6th International conference
% NONLINEAR ANALYSIS AND EXTREMAL PROBLEMS
% June 25-30, 2018
% Irkutsk, Russia
%
%%%%%%%%%%%%%%%%%%%%%%%%%%%%%%%%%%%%%%%%%%%%%%%%%%%%%%%%%%%%%%%%%%%%%%%%
% The preparation of the article is based on the standard llncs class
% (Lecture Notes in Computer Sciences), which is adjusted with style
% file of the conference.
%
% There are two ways of compilation of the file into PDF
% 1. Use pdfLaTeX (pdflatex), (LaTeX+DVIPS will not work);
% 2. Use LuaLaTeX (XeLaTeX will work too).
% When using LuaLaTeX You will need TTF or OTF CMU fonts
% (Computer Modern Unicode). The fonts are installed with 'cm-unicode' package in
% a distribution of LaTeX % (https://www.ctan.org/tex-archive/fonts/cm-unicode),
% either by downloading and installing these fonts system wide, the address of their page is
% http://canopus.iacp.dvo.ru/%7Epanov/cm-unicode/
% The second option won't work in XeLaTeX.
%
% For MiKTeX (LaTeX distribution for Windows),
%  1. Package 'cm-unicode' is installed manually with the MiKTeX administration Console.
%  2. For the compilation of this example, namely, the stub figure, one will also need to
% download package 'pgf' manually. This package uses in the popular
% package tikz.
%  3. Tests showed that the rest of the required packages MiKTeX loads automatically (if
%     it is allowed). The 'auto download' option is
%     configured in 'Settings' section in MiKTeX Console.
%
%
% The easiest way to compile an article is to use pdfLaTeX, but
% the final layout of the book will be compiled with LuaLaTeX,
% as a result will be of better quality thanks to the package 'microtype' and
% use vector OTF instead of standard raster fonts of pdfLaTeX.
%
% In the case of questions and problems with the article compilation,
% write letters to e-mail: eugeneai@irnok.net, Cherkashin Evgeny.
%
% New version of the correcting style file will be available at the website:
%     https://github.com/eugeneai/nla-style
%     file - nla.sty
%
% Further instructions are in the text body of the template. The template itself
% is an article example.
%
% The LaTeX2e format is used!

% 12 points font size is used.
\documentclass[12pt]{llncs}

% The correcting style file is added.
\usepackage{todonotes}

\usepackage{nla} % This package is needed for compiling
                 % this template, it should be removed
                 % from your article.

% Many popular packages (amsXXX, graphicx, etc.) are already imported in the style file.
% If there is a conflict with your packages, try disabling them and compile
% the text.
%
% It would be convenient in the layout of the proceedings if the file names
% of the figures of different authors do not clash.
% To minimize the clash, the drawings can be placed in a separate subfolder
% named after the author or the title of the paper.
%
% \graphicspath{{ivanov-petrov-pics/}} % specifies the folder with images in png, pdf formats.
% or
% \graphicspath{{great-problem-solving-paper-pics/}}.

\begin{document}
\fi

% Text should be formatted in accordance with the 'article' class, using extensions like
% AMS.
%
\title{On discrete equations in a multidimensional space}%\thanks{The research is supported by RFBR (RNF, other funds), project No.~00-00-00000.}}
% First author
\author{Abu Bakarr Kamanda Bongay   \and  Vladimir Vasilyev
}
\institute{Belgorod State National Research University, Belgorod, Russia\\
  \email{159720@bsu.edu.ru}\\
%  \and
%Affiliation, City, Country\\
\email{vbv57@inbox.ru}}
% etc

\maketitle

\begin{abstract}
We consider a certain discrete boundary value problem and prove its solvability in appropriate discrete spaces. A comparison between discrete and continuous solutions is given also.

\keywords{discrete space, digital pseudo-differential operator, discrete boundary value problem}
\end{abstract}

% at the end of the list, there should be no final dot
\section{Discrete spaces and digital operators}



Let $\mathbb Z^2$ be an integer lattice in a plane, $h>0, \hbar=h^{-1},$ $K_n=\{x\in\mathbb R^2: x=(x_1,x_2), x_2>a_n|x_1|a_n>0\}$, where $a_n$ can take values of the type $n, 1/n, n\in\mathbb N,$ $K_{n,d}=h\mathbb Z^2\cap K_n$. We consider functions of a discrete variable $u_d(\tilde x), \tilde x=(\tilde x_1,\tilde x_2)\in h\mathbb Z^2$, $\mathbb T^2=[-\pi,\pi]^2, \hbar=h^{-1}$. The functions defined in $\hbar\mathbb T^2$ we mean periodic functions in $\mathbb R^2$ with basic quadrate of periods $\mathbb T^2$.

We can define the discrete Fourier transform
\[
(F_du_d)(\xi)\equiv\tilde u_d(\xi)=\sum\limits_{\tilde x\in h\mathbb Z^2}e^{i\tilde x\cdot\xi}u_d(\tilde x)h^2,~~~\xi\in\hbar\mathbb T^2
\]
and discrete analogue of Schwartz space  $S(h\mathbb Z^2)$.
Let us denote $\zeta^2=h^{-2}((e^{ih\cdot\xi_1}-1)^2+(e^{ih\cdot\xi_2}-1)^2)$.
The space  $H^s(h\mathbb Z^2)$ consists of discrete functions $u_d$ and it is closure of the space  $S(h\mathbb Z^2)$ with respect to the norm
	\[
	||u_d||_s=\left(\int\limits_{\hbar\mathbb T^2}(1+|\zeta^2|)^s|\tilde u_d(\xi)|^2d\xi\right)^{1/2}.
	\]
The space $H^s(K_{n,d})$ consists of functions \cite{VV} from the space  $H^s(h\mathbb Z^2)$ for which their supports belong to the set  $\overline{K}_{n,d}$. Norm in the space  $H^s(K_{n,d})$ is induced by norm of the space $H^s)\mathbb Z^2)$.

Given periodic function $A_d(\xi)$ a digital pseudo-differential operator $A_d$ is defined as follows
\[
(A_du_d)(\tilde x)=\sum\limits_{\tilde y\in h\mathbb Z^2}h^2\int\limits_{\hbar\mathbb T^2}A_d(\xi)e^{i(\tilde x-\tilde y)\cdot\xi}\tilde u_d(\xi)d\xi,~~~\tilde x\in K_{n,d}.
\]
The periodic function $A_d(\xi)$ is called its symbol, and we assume that the inequality
\[
	c_1(1+|\zeta^2|)^{\alpha/2}\leq|A_d(\xi)|\leq c_2(1+|\zeta^2|)^{\alpha/2}
\]
holds with positive $c_1,c_2$.

\section{Main results}

We consider the following discrete boundary value problem assuming that the symbol $A_d)\xi)$ admits the periodic wave factorization with respect to $K_n$ \cite{V1,MVV,V2} with the index $\ae$ such that $1/2<\ae-s<3/2$.
\begin{equation}\label{1}
	(A_du_d)(\tilde x)=0,~~~\tilde x\in K_{n,d},
\end{equation}
\begin{equation}\label{2}
	\sum\limits_{\tilde x_2\in h\mathbb Z}u_d(\tilde x_1,\tilde x_2)h\equiv g_d(\tilde x_1),
\end{equation}
$g_d$ is given.

\begin{theorem}
 The problem $(1),(2)$ has unique solution in the space $H^s(K_{n,d})$ for arbitrary right hand side $g_d\in H^{s+1/2}(h\mathbb Z)$.
	
	A priori estimate
	\[
	||u_d||_s\leq b[g]_{s+1/2}
	\]
holds with constant $b$ non-depending on $h$.
\end{theorem}

The continuous analogue of the problem (1),(2) is the following

\begin{equation}\label{3}
(Au)(x)=0,~~~x\in K_n,
\end{equation}
\begin{equation}\label{4}
\int\limits_{-\infty}^{+\infty}u(x_1,x_2)dx_2=g(x_1).
\end{equation}

We can give a comparison between discrete and continuous solution under some additional assumptions and choice of discrete elements.

\begin{theorem}
If $A(\xi)$ admits the wave factorization with respect to $K_n$ with the index $\ae$ such that $1/2<\ae-s<3/2$, $g\in H^{s+1/2}(\mathbb R)$ then problems $(3),(4)$ and $(1),(2)$ have unique solutions $u\in H^s(K_n)$ and $u_d\in H^s(K_{n,d})$ respectively. If additionally  $\tilde g$ has a compact support then the estimate
	\begin{equation*}
		|\tilde u(\xi)-\tilde u_d(\xi)|\leq~ch,~~~\xi\in\frac{\hbar}{4b_n}\mathbb T^2,~~~b_n=\max\{1,a_n\}
	\end{equation*}
holds for small enough $h$ with constant $c$ non-depending on $h$.
\end{theorem}

% The figures and tables are drawn according to the standard class 'article'.


% At the end of the text, acknowledgments are expressed, if you haven't
% made a footnote from the title. For example, we can write
%The research is carried on with support of RFBR (RNF, other funds), project No.~00-00-00000.

\begin{thebibliography}{9} % or {99}, if there is more than ten references.

\bibitem{V1} Vasil'ev V.B., Wave Factorization of Elliptic Symbols: Theory and Applications,
Kluwer Academic Publishers, Dordrecht--Boston--London, 2000.

\bibitem{VV} Vasilyev A.V., Vasilyev V.B. Pseudo-differential operators and equations
in a discrete half-space. Math. Model. Anal.~2018. Vol.~23, no~3. Pp.~492--506.

\bibitem{MVV} A.A. Mashinets, A.V. Vasilyev, V. B. Vasilyev, On discrete Neumann problem in a quadrant. Lobachevskii J. Math.~2023. Vol.~44, no~3. Pp.~1011--1021.


\bibitem{V2} V. B. Vasilyev, Discreteness, periodicity, holomorphy, and factorization. In Integral
Methods in Science and Engineering (New York) (C. Constanda, M. Dalla
Riva, P.D. Lamberti, and P. Musolino, eds.), Theoretical Technique. Vol. 1.
Birkh\"auser, 2017. Pp. 315--324.





\end{thebibliography}
%\end{document}

