\begin{englishtitle} % Настраивает LaTeX на использование английского языка
% Этот титульный лист верстается аналогично.
\title{Iterative approach to solving polynomial Volterra integral equations of the first kind}
% First author
\author{S.~V.~Solodusha  }
\institute{ESI SB RAS, ISDCT SB RAS, Irkutsk, Russia\\
  \email{solodusha@isem.irk.ru}
  }
% etc

\maketitle

\begin{abstract}
Volterra integral equations of the first kind with polynomial nonlinearity are considered. They arise in the problem of identifying the input signal of a dynamic system.

\keywords{identification, nonstationary dynamical system, polynomial Volterra
equation of the first kind} % в конце списка точка не ставится
\end{abstract}
\end{englishtitle}

\iffalse
\documentclass[12pt]{llncs}

% При использовании pdfLaTeX добавляется стандартный набор русификации babel.

\usepackage{iftex}

\ifPDFTeX
\usepackage[T2A]{fontenc}
\usepackage[utf8]{inputenc} % Кодировка utf-8, cp1251 и т.д.
\usepackage[english,russian]{babel}
\fi

\usepackage{todonotes}

\usepackage[russian]{nla}


\begin{document}
\fi

\title{Итерационный подход к решению полиномиальных интегральных уравнений Вольтерра I рода\thanks{Работа поддержана грантом РНФ \textnumero~22-11-00173, https://rscf.ru/project/22-11-00173/.}
}
% Первый автор
\author{С.~В.~Солодуша % \inst ставит циферку над автором.
%  \and  % разделяет авторов, в тексте выглядит как запятая.
% Второй автор
%  И.~О.~Фамилия\inst{2}
%  \and
% Третий автор
%  И.~О.~Фамилия\inst{1}
} % обязательное поле

% Аффилиации пишутся в следующей форме, соединяя каждый институт при помощи \and.
\institute{ИСЭМ СО РАН, ИДСТУ СО РАН, Иркутск, Россия\\
  \email{solodusha@isem.irk.ru}
}

\maketitle

\begin{abstract}
Рассмотрены интегральные уравнения Вольтерра I рода с полиномиальной нелинейностью, возникающие в задаче идентификации входного сигнала динамической системы.

\keywords{идентификация, нестационарная динамическая система, полиномиальные интегральные уравнения Вольтерра I рода} % в конце списка точка не ставится
\end{abstract}

%\section{Основные результаты} % не обязательное поле

В докладе представлены результаты по численному решению полиномиальных интегральных уравнений Вольтерра I рода \cite{solodusha1,solodusha2}. Дано детальное описание итерационному подходу и разностным методам. Статьи \cite{solodusha3,solodusha4}  посвящены прикладным задачам.

% Рисунки и таблицы оформляются по стандарту класса article. Например,


% кавычки << >>.

% В конце текста можно выразить благодарности, если этого не было
% сделано в ссылке с заголовка статьи, например,
%Работа поддержана грантом РНФ \No~22-11-00173, https://rscf.ru/project/22-11-00173/.
%

\begin{thebibliography}{9} % или {99}, если ссылок больше десяти.
\bibitem{solodusha1}  Solodusha~S. Numerical Solution of A Class of Systems of Volterra Polynomial Equations of the First Kind. Numerical Analysis and Applications. 2018. Vol. 11. Pp.~89--97. 

\bibitem{solodusha2}  Solodusha~S. Identification of Input Signals in Integral Models of One Class of Nonlinear  Dynamic Systems. Bulletin of Irkutsk State University. Series Mathematics. 2019. Vol.~30. Pp.~73--82.

\bibitem{solodusha3} Solodusha~S. Modeling Heat Exchangers by Quadratic Volterra Polynomials./~Autom. Remote Control. 2014. Vol.~75.  \textnumero~1. Pp.~87--94.  

\bibitem{solodusha4} Solodusha~S., Bulatov~M. Integral Equations Related to Volterra Series and Inverse Problems:
Elements of Theory and Applications in Heat Power Engineering. Mathematics. 2021. Vol.~9.  \textnumero~16. P.~1905. 

\end{thebibliography}

% После библиографического списка в русскоязычных статьях необходимо оформить
% англоязычный заголовок и аннотацию.




%\end{document}
