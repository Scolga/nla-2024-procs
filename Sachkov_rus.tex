\begin{englishtitle} % Настраивает LaTeX на использование английского языка
% Этот титульный лист верстается аналогично.
\title{Flat sub-Lorentzian problems on the Martinet distribution}
% First author
\author{Yuri Sachkov
}
\institute{PSI RAS, Pereslavl-Zalessky, Russia\\
  \email{yusachkov@gmail.com}
}
\maketitle

\begin{abstract}
Two flat (nilpotent) problems of sub-Lorentzian geometry on
Martinet distribution in 3-dimensional space are considered. For the first problem, the reachable set has
non-trivial intersection with the Martinet plane, but not for the second one.
Reachable sets, optimal trajectories, sub-Lorentzian
distances and spheres are described. For the first problem, the sub-Lorentzian sphere is a topological
manifold with boundary, and for the second problem it is an analytic manifold.

\keywords{sub-Lorentzian geometry, geometric control theory, optimal control} % в конце списка точка не ставится
\end{abstract}
\end{englishtitle}


\iffalse
%%%%%%%%%%%%%%%%%%%%%%%%%%%%%%%%%%%%%%%%%%%%%%%%%%%%%%%%%%%%%%%%%%%%%%%%
%
%  This is the template file for the 6th International conference
%  NONLINEAR ANALYSIS AND EXTREMAL PROBLEMS
%  June 25-30, 2018
%  Irkutsk, Russia
%
%%%%%%%%%%%%%%%%%%%%%%%%%%%%%%%%%%%%%%%%%%%%%%%%%%%%%%%%%%%%%%%%%%%%%%%%


\documentclass[12pt]{llncs}  

% При использовании pdfLaTeX добавляется стандартный набор русификации babel.

\usepackage{iftex}

%\def\A{\mathcal{A}}
%\def\D{\Delta}
%\def\R{{\mathbb R}}
%\newcommand{\spann}{\operatorname{span}\nolimits}
%\newcommand{\pder}[2]{\frac{\partial \, #1}{\partial \, #2} }

\ifPDFTeX
\usepackage[T2A]{fontenc}
\usepackage[utf8]{inputenc} % Кодировка utf-8, cp1251 и т.д.
\usepackage[english,russian]{babel}
\fi

\usepackage{todonotes} 

\usepackage[russian]{nla}

\begin{document}
\fi

\title{Плоские сублоренцевы задачи на  распределении Мартине\thanks{Работа выполнена при поддержке  РНФ, проект \textnumero~22-11-00140.}}
% Первый автор
\author{Ю.~Л.~Сачков
}

% Аффилиации пишутся в следующей форме, соединяя каждый институт при помощи \and.
\institute{ИПС РАН, Переславль-Залесский, Россия \\
  \email{yusachkov@gmail.com}
}

\maketitle

\begin{abstract}
Исследуются две плоские (нильпотентные) задачи сублоренцевой геометрии  на
распределении Мартине в 3-мерном пространстве. Для первой задачи множество достижимости имеет
нетривиальное пересечение с плоскостью Мартине, а для второй нет.
Описаны множества достижимости, оптимальные траектории, сублоренцевы
расстояния и сферы. Для первой задачи сублоренцева сфера есть топологическое 
многообразие с краем, а для второй задачи это аналитическое многообразие.


\keywords{сублоренцева геометрия, геометрическая теория управления, оптимальное управление} 
\end{abstract}


\section{Первая задача}
Пусть $M = {\mathbb R}^3_{x, y, z}$, $X_1 = \frac{\partial \,  }{\partial \, x}$, $X_2 = \frac{\partial \,  }{\partial \, y} + \frac{x^2}{2} \frac{\partial \,  }{\partial \, z}$. Распределение $\Delta = \operatorname{span}\nolimits(X_1, X_2)\subset TM$  называется распределением Мартине \cite{mont_book,ABCK,ABB,UMN2}. Плоскость $\Pi = \{x=0\}\subset M$ называется поверхностью Мартине. 

Первая сублоренцева задача на распределении Мартине ставится как следующая задача оптимального управления \cite{notes,intro}:
\begin{align}
&\dot q = u_1 X_1 + u_2 X_2, \qquad q \in M, \label{pr11} \\
&u= (u_1, u_2) \in U_1 = \{u_2 \geq |u_1|\}, \label{pr12}\\
&q(0) = q_0 = (0, 0, 0), \qquad q(t_1) = q_1, \label{pr13}\\
&l = \int_0^{t_1} \sqrt{u_2^2-u_1^2}dt \to \max. \label{pr14}
\end{align}

\section{Вторая задача}

Вторая сублоренцева задача на распределении Мартине ставится как задача оптимального управления 
\begin{align}
&\dot q = u_1 X_1 + u_2 X_2, \qquad q \in M, \label{pr21} \\
&u= (u_1, u_2) \in U_2 = \{u_1 \geq |u_2|\}, \label{pr22}\\
&q(0) = q_0 = (0, 0, 0), \qquad q(t_1) = q_1, \label{pr23}\\
&l = \int_0^{t_1} \sqrt{u_1^2-u_2^2}dt \to \max. \label{pr24}
\end{align}

\section{Полученные результаты}

Для обеих задач получены следующие результаты:
\begin{itemize}
\item
Описаны множества достижимости,   
\item
Получена параметризация экстремальных траекторий эллиптическими функциями Якоби,
\item
Доказано существование оптимальных траекторий,
\item
Описаны оптимальные траектории,
\item
Описаны сублоренцевы
расстояния и их свойства,
\item
Описаны сублоренцевы сферы и их свойства.
\end{itemize}


\section{Заключение}
Первая задача принципиально отличается от второй следующими особенностями 
оптимального синтеза:
\begin{itemize}
\item
некоторые оптимальные траектории меняют каузальный тип,
\item
экстремальные траектории имеют точки разреза на поверхности Мартине $\Pi$,
\item
оптимальный синтез двузначен на $\Pi$,
\item
сублоренцево расстояние негладко на $\Pi$ и терпит разрыв первого рода в некоторых точках границы множества достижимости $\partial \mathcal{A}_1$,
\item
сублоренцева сфера есть многообразие с краем.
\end{itemize}
Эти особенности связаны с нетривиальным  
пересечением множества достижимости $\mathcal{A}_1$ (каузального будущего начальной точки $q_0$) и 
поверхности Мартине 
$\Pi$.   

Оптимальный синтез во второй задаче качественно не отличается от синтеза в сублоренцевой задаче на группе Гейзенберга    \cite{sl_heis}.


\begin{thebibliography}{9} % или {99}, если ссылок больше десяти.
\bibitem {mont_book}
R. Montgomery, {\em A tour of subriemannnian geometries, their geodesics and applications}, Amer. Math. Soc., 2002.
\bibitem{ABB}
A. Agrachev, D. Barilari, U. Boscain, {\em A Comprehensive Introduction to
sub-Riemannian Geometry from Hamiltonian viewpoint}, Cambridge Studies in Advanced Mathematics, Cambridge Univ. Press, 2019.
 \bibitem{ABCK}
A.Agrachev, B. Bonnard, M. Chyba, I. Kupka,
Sub-Riemannian sphere in Martinet flat case. J. ESAIM: Control, Optimisation and Calculus of Variations. 1997. Vol. 2. Pp. 377--448. 
\bibitem{UMN2}
Ю. Л. Сачков, Левоинвариантные задачи оптимального управления на группах Ли, интегрируемые в эллиптических функциях. 2023.  УМН. Vol. 78:1(469).  Pp. 67--166.
\bibitem{notes}
А.А.~Аграчев, Ю. Л. Сачков,  
{\em Геометрическая теория управления},
Физматлит, 2005.
Перевод: A.A.~Agrachev, Yu.L. Sachkov. Control Theory from the Geometric Viewpoint,
Springer, 2004.
\bibitem{intro}
Сачков Ю.Л. {\em Введение в геометрическую теорию управления}, М.: URSS, 2021.
Расширенный перевод:
Yu. Sachkov, Introduction to geometric control, Springer, 2022.
 \bibitem{sl_heis}	Yu. L. Sachkov, E.F. Sachkova, Sub-Lorentzian distance and spheres on the Heisenberg group, Journal of Dynamical and Control Systems volume 29, pages 1129–1159 (2023)
\end{thebibliography}

% После библиографического списка в русскоязычных статьях необходимо оформить
% англоязычный заголовок.



%\end{document}

