\begin{englishtitle}
\title{Stabilization of linear three-dimensional stationary discrete systems}
% First author
\author{Vladislav O. Nesterov}
\institute{Adyghe State University, Maykop, Adyghe Republic, Russia\\
\email{v.nesterov@adygnet.ru}}
% etc

\maketitle

\begin{abstract}
The paper deals with the  problem of stabilisation of a three-dimensional linear stationary discrete system by nonstationary feedback. The solution of this problem lies in the context of solving the discrete analogue of the Brockett problem on stabilisation of multidimensional linear stationary discrete controlled systems by means of linear nonstationary feedback. We consider the cases when in the controlled system the output is one of the coordinates of the state vector.The nonstationary stabilising feedback is sought in the class of periodic feedbacks of period three.

\keywords{linear discrete system, stabilisation, asymptotic stability, periodic feedback, controlled and observed system, characteristic polynomial, stabilisation region, monodromy matrix}.
\end{abstract}
\end{englishtitle}

\iffalse
\documentclass[12pt]{llncs}
\usepackage[T2A]{fontenc}
\usepackage[utf8]{inputenc}
\usepackage[english,russian]{babel}
\usepackage[russian]{nla}

%\usepackage[english,russian]{nla}

% \graphicspath{{pics/}} %Set the subfolder with figures (png, pdf).

%\usepackage{showframe}
\begin{document}
%\selectlanguage{russian}
\fi

\title{Стабилизация трехмерных линейных дискретных систем периодической обратной связью}
% Первый автор
\author{В.~О.~Нестеров%\inst{1}
%  \and
% Второй автор
%  И.~О.~Фамилия\inst{2}
%  \and
% Третий автор
%  И.~О.~Фамилия\inst{2}
} 
\institute{Факультет математики и компьютерных наук\\ Адыгейский государственный университет, Адыгея, Россия\\
  \email{v.nesterov@adygnet.ru}
  }


\maketitle

\begin{abstract}
В докладе рассматривается задача стабилизации трехмерной линейной стационарной дискретной системы нестационарной обратной связью. Решение данной задачи лежит в русле решения дискретного аналога проблемы Брокетта о стабилизации многомерных линейных стационарных дискретных управляемых систем с помощью линейной нестационарной обратной связи. Рассматриваются случаи, когда в управляемой системе выходом является одна из координат вектора состояния. Нестационарная стабилизирующая обратная связь ищется в классе периодических обратных связей периода три.

\keywords{линейная дискретная система, стабилизация, асимптотическая устойчивость, периодическая обратная связь, управляемая и наблюдаемая система, характеристический многочлен, область стабилизации, матрица монодромии.}
\end{abstract}

%\section{Основные результаты} % не обязательное поле

В докладе рассматривается задача стабилизации трехмерной линейной стационарной дискретной системы нестационарной обратной связью \cite{vor1,vor2,vor3}. 
Построены нестационарные обратные связи  -- периодические с периодом 3, стабилизирующие рассматриваемую систему. Получены коэффициентные достаточные условия стабилизируемости рассматриваемой системы. В случае, когда выходом является первая или вторая координата вектора состояния, построенная нестационарная обратная связь расширяет в целом область стационарной стабилизации. В случае, когда выходом является третья координата вектора состояния, не происходит расширения области стацио-нарной стабилизации.


% Список литературы.
\begin{thebibliography}{99}
\bibitem{vor1}
Леонов Г. А. Проблема Брокетта для линейных дискретных систем управления~// Автоматика и телемеханика 2002. \textnumero~5. С.~92–96
\bibitem{vor2}
% Format for Journal Reference:
Леонов Г. А., Шумафов. М. М, Методы стабилизации линейных управляемых систем. СПб.: Изд-во С.-Петерб. ун-та, 2005. С.~420 с.
% Format for books:
\bibitem{vor3}
Шумафов М. М. К задаче стабилизации двумерной линейной дискретной системы~// Известия вузов. Северо-Кавказский регион. 2009.  \textnumero~5. С.~71–74.
% Format for Russian Journal Reference:.
% etc
\end{thebibliography}


% Обязательная информация на английском языке:




%\end{document}
