\begin{englishtitle} % Настраивает LaTeX на использование английского языка
% Этот титульный лист верстается аналогично.
\title{About one method of decision-making with inaccurate information about the importance of criteria}
% First author
\author{Pavel Simakov  \and  Konstantin Kudryavtsev
}
\institute{SUSU (National Research University), Chelyabinsk, Russia\\
  \email{pavelsimakov35707@gmail.com }
\email{kudrkn@gmail.com}}
% etc

\maketitle

\begin{abstract}
The paper considers a multi-criteria decision-making problem in which the importance of criteria is given in the form of intervals and the set of alternatives is a convex compact. 
A modification of the EDAS method is proposed. A modelling example is considered. 

\keywords{multicriteria problem, uncertainty, decision-making} % в конце списка точка не ставится
\end{abstract}
\end{englishtitle}

\iffalse
%%%%%%%%%%%%%%%%%%%%%%%%%%%%%%%%%%%%%%%%%%%%%%%%%%%%%%%%%%%%%%%%%%%%%%%%
%
%  This is the template file for the 6th International conference
%  NONLINEAR ANALYSIS AND EXTREMAL PROBLEMS
%  June 25-30, 2018
%  Irkutsk, Russia
%
%%%%%%%%%%%%%%%%%%%%%%%%%%%%%%%%%%%%%%%%%%%%%%%%%%%%%%%%%%%%%%%%%%%%%%%%


\documentclass[12pt]{llncs}  

% При использовании pdfLaTeX добавляется стандартный набор русификации babel.

\usepackage{iftex}

\ifPDFTeX
\usepackage[T2A]{fontenc}
\usepackage[utf8]{inputenc} % Кодировка utf-8, cp1251 и т.д.
\usepackage[english,russian]{babel}
\fi

\usepackage{todonotes} 

\usepackage[russian]{nla}

\begin{document}
\fi

\title{Об одном методе принятия решений с использованием неточной информации о важности критериев\thanks{Исследование выполнено за счет гранта Российского научного фонда, проект \textnumero~23-21-00539.}}
% Первый автор
\author{П.~К.~Симаков \and К.~Н.~Кудрявцев
}

% Аффилиации пишутся в следующей форме, соединяя каждый институт при помощи \and.
\institute{ЮУрГУ (НИУ), Челябинск, Россия \\
  \email{pavelsimakov35707@gmail.com }
  \email{kudrkn@gmail.com}
% \and Другие авторы...
}

\maketitle

\begin{abstract}
В докладе рассматривается многокритериальная задача принятия решений, в которой важность критериев задана в виде интервалов, а множество альтернатив -- выпуклый компакт. Предлагается модификация метода EDAS. Рассмотрен модельный пример. 


\keywords{многокритериальная задача, неопределенность, принятие решений} % в конце списка точка не ставится
\end{abstract}


Рассматривается $N$–критериальная задача принятия решений, которая задается кортежем
\begin{equation*}
	\varGamma=\langle X, \{f_{i}(x)\}_{i=1,...,N} \rangle .
\end{equation*}
В задаче $\varGamma$ лицо принимающее решение (ЛПР) выбирает альтернативу $x \in X \subset \mathbb{R}^{n}$, которая одновременно максимизирует (или минимизирует) каждый из $N$ критериев вектор-функции
$$
f(x)=(f_{1}(x), ..., f_{N}(x)) :X \rightarrow \mathbb{R}^{N}.
$$
При этом, не имея точной информации о важности критериев, ЛПР оценивает важность $\omega_{j}$ критерия $f_{j}(x)$ через принадлежность к некоторому промежутку $[\alpha_{j};\beta_{j}]$, $(j=1, ..., N)$. А именно, считается, что $\omega_{j}$ представляет собой интервальную неопределенность, о которой ЛПР не имеет никакой стохастической информации, а знает только множество ее возможных значений.


Для задачи $\varGamma$ предлагается модификация метода EDAS \cite{Ghorabaee} с интервальными весами критериев. На основе такой модификации, и следуя подходу из \cite{Uh}, 
определяется понятие нечеткого EDAS-решения задачи $\varGamma$ с интервальными весами критериев. 

Для варианта $N$–критериальной задачи $\varGamma$ с линейными критериями
\begin{equation}
	f_{j}(x)=c_{j1}x_{1}+c_{j2}x_{2}+ ... + c_{jn}x_{n}, \;\;\; (j = 1, \ldots , N),
\end{equation} 
и множеством альтернатив $X \subset \mathbb{R}^{n}$, заданным линейными связями типа равенств
\begin{equation}
	q_{i}(x)=\alpha_{i1}x_{1} + \alpha_{i2}x_{2} + ... + \alpha_{in}x_{n}-b=0 \;\;\; (i=1, ..., k)
\end{equation} 
и условиями неотрицательности
$$
	x_{l} \geq 0 \;\;\; (l = 1, ..., n),
$$
предлагается конструктивный алгоритм построения нечеткого EDAS-решения задачи $\varGamma$ с
интервальными весами критериев. 

Заметим, что с учетом (1), (2) и вида весов критериев $\omega_{j} \in [\alpha_{j};\beta_{j}]$, $(j=1, ..., N)$, задача $\varGamma$ может трактоваться как 
многокритериальная задача линейного программирования с неточно заданной целевой функцией.

В качестве модельного примера, строится нечеткое EDAS-решение для задачи определения оптимальной по микроэлементам и витаминам рецептуры начинки фруктового пирога.


\begin{thebibliography}{9} % или {99}, если ссылок больше десяти.
\bibitem{Ghorabaee} Ghorabaee. et al. Multi-Criteria Inventory Classification Using a New Method of Evaluation Based on Distance from Average Solution (EDAS)~// Ghorabaee, Mehdi Keshavarz // Informatica 26 — (2015) — P. 435–451.
\bibitem{Uh} Ukhobotov V.I., Stabulit I.S., Kudryavtsev K.N. On Decision Making under Fuzzy Information about an Uncontrolled Factor~// Procedia Computer Science. 150 — (2019),  — P. 524-531.

\end{thebibliography}

% После библиографического списка в русскоязычных статьях необходимо оформить
% англоязычный заголовок.




%\end{document}

