\begin{englishtitle} % Настраивает LaTeX на использование английского языка
% Этот титульный лист верстается аналогично.
\title{Method for modeling infectious processes with chronization using equations randomized delay}
% First author
\author{Andrey Perevaryukha 
  }
\institute{St. Petersburg Federal Research Center RAS, St. Petersburg, Russia\\
  \email{madelf@rambler.ru} }
 
\maketitle

\begin{abstract}
We are considering the problem of modeling aggressive invasion processes into a competitive environment that offers resistance, but the development of a reaction is indefinitely delayed and depends on the level of the initial impact --- the initial dose of infection. In this way, an infectious process develops in the body. The development of a response is very individual, depends on the algorithm of the complex interaction of many immune cells and has several variable possibilities. The immune response is not a stochastic process, but with a random component of uncertainty. We call such stage processes “Aleatoric” when, at some of their transition stages, a non-deterministic choice from possible options is realized. The paper proposes equations for pulsating invasions with an aleatoric component when generating a delayed effect.

\keywords{aleatoric component when generating a delayed effect} % в конце списка точка не ставится
\end{abstract}
\end{englishtitle}

\iffalse
%%%%%%%%%%%%%%%%%%%%%%%%%%%%%%%%%%%%%%%%%%%%%%%%%%%%%%%%%%%%%%%%%%%%%%%%
%
%  This is the template file for the 6th International conference
%  NONLINEAR ANALYSIS AND EXTREMAL PROBLEMS
%  June 25-30, 2018
%  Irkutsk, Russia
%
%%%%%%%%%%%%%%%%%%%%%%%%%%%%%%%%%%%%%%%%%%%%%%%%%%%%%%%%%%%%%%%%%%%%%%%%


%  В случае возникновения вопросов и проблем с версткой статьи,
%  пишите письма на электронную почту: eugeneai@irnok.net, Черкашин Евгений.



\documentclass[12pt]{llncs}  

% При использовании pdfLaTeX добавляется стандартный набор русификации babel.
% Если верстка производится в LuaLaTeX, то следующие три строки надо
% закомментировать, русификация будет произведена в корректирующем стиле автоматом.
\usepackage{iftex}

\ifPDFTeX
\usepackage[T2A]{fontenc}
\usepackage[utf8]{inputenc} % Кодировка utf-8, cp1251 и т.д.
\usepackage[english,russian]{babel}
\fi

% Для верстки в LuaLaTeX текст готовится строго в utf-8!

% В операционной системе Windows для редактирования в кодировке utf-8
% можно использовать программы notepad++ https://notepad-plus-plus.org/,
% techniccenter http://www.texniccenter.org/,
% SciTE (самая маленькая по объему программа) http://www.scintilla.org/SciTEDownload.html
% Подойдет также и встроенный в свежий дистрибутив MiKTeX редактор
% TeXworks.

% Добавляется корректирующий стилевой файл строго после babel, если он был включен.
% В параметре необходимо указать russian, что установит не совсем стандартные названия
% разделов текста, настроит переносы для русского языка как основного и т.п.

\usepackage{todonotes} % Этот пакет нужен для верстки данного шаблона, его
                       % надо убрать из вашей статьи.

\usepackage[russian]{nla}

% Многие популярные пакеты (amsXXX, graphicx и т.д.) уже импортированы в корректирующий стиль.
% Если возникнут конфликты с вашими пакетами, попробуйте их отключить и сверстать
% текст как есть.
%
%


% Было б удобно при верстке сборника, чтобы названия рисунков разных авторов не пересекались.
% Чтоб минимизировать такое пересечение, рисунки нужно поместить в отдельную подпапку с
% названием - фамилией автора или названием статьи.
%
% \graphicspath{{ivanov-petrov-pics/}} % Указание папки с изображениями в форматах png, pdf.
% или
% \graphicspath{{great-problem-solving-paper-pics/}}.


\begin{document}
\fi

% Текст оформляется в соответствии с классом article, используя дополнения
% AMS.
%

\title{Метод моделирования инфекционных процессов с хронизацией уравнениями с рандомизированным 
запаздыванием\thanks{Работа выполнена при поддержке РНФ, проект \textnumero~23-21-00339.}}
% Первый автор
\author{А.~Ю.~Переварюха
} % обязательное поле

% Аффилиации пишутся в следующей форме, соединяя каждый институт при помощи \and.
\institute{Санкт-Петербургский Федеральный исследовательский центр РАН, Санкт-Петербург, Россия \\
  \email{madelf@rambler.ru}
  % \and Другие авторы...
}

\maketitle

\begin{abstract}
Рассматривается проблема моделирования агрессивных инвазионных процессов в конкурентную среду, оказывающую сопротивление, но выработка реакции неопределенным образом запаздывает и зависит от уровня исходного воздействия --- начальной дозы заражения. Таким образом развивается инфекционный процесс в организме. Выработка ответа очень индивидуальна, зависит от алгоритма работы сложного взаимодействия многих иммунных клеток и имеет несколько вариативных возможностей. Иммунный ответ это процесс не стохастический, но со случайной компонентой неопределенности. Мы называем такие стадийные процессы «Алеаторический», когда на их некоторых переходных стадиях реализуется недетерминированный выбор из возможных вариантов. В работе предложены уравнения для пульсирующих инвазий с алеаторической компонентой при выработке запаздывающего воздействия. 

\keywords{модели инвазии, уравнения противодействия, запаздывание с алеаторической компонентой } % в конце списка точка не ставится
\end{abstract}

\section{Основные результаты} % не обязательное поле

Цель работы --- модель стадий инвазионного процесса на основе уравнений, где стохастические факторы учтены с методом возмущения запаздывания. Проблема моделируемой ситуации инвазии в том, что регулируемое противодействие агрессивно размножающемуся виду в биологическом сообществе вырабатывается с запаздыванием и приводит к резкому переходу в фазу депрессии численности вселенца, но это запаздывание не константа. 
Направление продолжаем развивать в модификации с $\dot N=rF(N(t))^\Theta$: 
$$
\frac{dN}{dt}=rN(t)\frac{(1- N(t)/(K+\vartheta N)^\Theta}{(1-N(t))/K(1-\gamma)}.
$$
Решения подобных моделей описывают уравновешивающиеся процессы $\forall N(0)>0$. Не все уравнения имеет смысл дополнять включением $t-\tau$. Отличие моделей ограниченного роста --- положение точки перегиба $N_p\neq0$ на графике $N(t)$. Для модели ордината точки перегиба $N_p=K/2$, абсцисса  $t_p=r^{-1}\ln(K-N(0))/N(0)$. Положение ординаты точки перегиба $N_p$ установим для оптимальной эксплуатации c изъятием $\dot N=rf(N(t))-Q$.

Сравним динамику модели инвазионного процесса для агрессивного вселенца с $N(t-\tau)$ и модель инвазии в форме уравнения с отклоняющимся аргументом, где величина запаздывания $\tau$ возмущена равномерно распределенной случайной величной $\gamma\in[-0.5,0.5]$, что отражает влияение случайных факторов на небольшую исходную группу особей-вселенцев. Для включения стохастической компоненты лучше возмущать именно величину запаздывания $\gamma\tau$, что качественно отразится на сценариях завершен6ия инвазионного процесса \cite{Aleksandrov1999}.
Эффекты запаздывания разделены на три типа по биологическому генезису и роли в развитии процессов. Инвазионные процессы проходят этап кризисной динамики $N(t)\to 0+\epsilon$ и сопровождаются длительными осцилляциями. B результате биосистема получит несколько сценариев динамики кризиса, включая гибель $N(t_\infty)=0$. 

Зададим пороговое развитие инвазионного популяционного процесса в уравнении c функцией сопротивления среды $\dot N=F(N(t-\tau))-\Psi(N(t-\nu))$. Пороговый эффект реакции агрессивному росту численности вселенца выразим $\ln_K$-регуляцией в функции противодействия $\Psi(N(t-\nu))$ и при $\mathcal{Q}>q, m\geq2, N(0)<J<\mathcal{K}$. Запаздывание $\nu$ в модели возмущенно равномерно распределенной случайной величиной $\nu\times\gamma$ на отрезке $[0,0.5\nu]$:
$$
\frac{dN}{dt}= rN(t) \ln\left(\frac{\mathcal{K}}{N(t-\tau)}\right) - \mathcal{Q}\frac{N^m(t-\nu\times\gamma)}{\left(J-N(t)\right)^2} - qN(t). \eqno(1)
$$
   
Вместо стабилизации $N(t)\to K, N(t_S)<K$ и превышения равновесия $K$ стадия кризиса с возрастанием $F(N^2;J^{-1})$ при $N\to J$ и потенциал роста не нивелирован $\ln_K$-регуляцией.
Время активации вариативно, но не менее $\tau_1$. Пусть $\tau_1$ варьируется случайной величиной $\gamma$ в ограниченном диапазоне. Предложим модель с возмущенным равномерной случайной величиной запаздыванием $(t-\tau_1\gamma)$:  
$$
\frac{dN}{dt}= rN(t) \ln\left(\frac{\mathcal{K}}{N(t-\tau\gamma)}\right) - \frac{\delta N^2(t-\tau_1\gamma)}{\left(J-N(t)\right)^2} - qN(t), \delta>q, \gamma(\omega)\in[1,2]. \eqno(2)
$$
При приближении $N(t)$ к пороговому значению $J,N(0)<J<\mathcal{K}$ резкий переход в глубокий популяционный кризис $N(t)\to0+\epsilon$. Сценарий преодоления кризиса c образованием колебаний $N(t)\to N_*(t),\max N_*(t)<J$ зависит от стохастических временных факторов. Популяция погибает при увеличении репродуктивного потенциала $r$. Можно показать, что существует $r=\bar r$, что для события $\lim_{t\to\bar t} N(t;\bar r\tau)=0$ вероятность $P>0$ и $\exists \hat r>\bar r$,$t<\infty$ реализуется для данного события $P=1$. $\hat r$ критический порог репродуктивной активности. 

% Рисунки и таблицы оформляются по стандарту класса article. Например,

% В конце текста можно выразить благодарности, если этого не было
% сделано в ссылке с заголовка статьи, например,
Работа выполнена при поддержке РНФ, проект \textnumero~23-21-00339.
%

% Список литературы оформляется подобно ГОСТ-2008.
% Примеры оформления находятся по этому адресу -
%     https://narfu.ru/agtu/www.agtu.ru/fad08f5ab5ca9486942a52596ba6582elit.html
%

\begin{thebibliography}{9} % или {99}, если ссылок больше десяти.

\bibitem{Aleksandrov1999} Переварюха~А.Ю. Хаотические режимы в моделях теории формирования пополнения популяций~// Нелинейный мир. 2009. \textnumero~12. С.~925--932.

\end{thebibliography}

% После библиографического списка в русскоязычных статьях необходимо оформить
% англоязычный заголовок.




%\end{document}

%%% Local Variables:
%%% mode: latex
%%% TeX-master: t
%%% End:
