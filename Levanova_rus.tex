\begin{englishtitle} % Настраивает LaTeX на использование английского языка
% Этот титульный лист верстается аналогично.
\title{Applying a tabu search to solving the robust $p$-median problem}
% First author
\author{Tatiana Levanova  \and   Ivan Khmara
 }
\institute{Sobolev Institute of Mathematics, Omsk Division,  Omsk, Russia\\
  \email{levanovat@gmail.com}\\
 % \and
%Sobolev Institute of Mathematics, Omsk Division,  Omsk, Russia\\
\email{khmaraivan97@gmail.com}}
% etc


\maketitle

\begin{abstract}
The one-criterion  $p$-median problem in a robust formulation using threshold robustness is considered. Unlike the classical problem, it allows to take into account changes in parameters. In this work, it is assumed that demand is unstable. The possibility of applying a Tabu Search  to solving of the robust $p$-median problems being studied. Variant of the Tabu Search algorithm  are proposed. Research are conducted on specially created test instances, the quality of the results and ways of developing the approach are discussed.

\keywords{ $p$-median problem, threshold robustness, tabu search} % в конце списка точка не ставится
\end{abstract}
\end{englishtitle}

\iffalse
%%%%%%%%%%%%%%%%%%%%%%%%%%%%%%%%%%%%%%%%%%%%%%%%%%%%%%%%%%%%%%%%%%%%%%%%
%
%  This is the template file for the 6th International conference
%  NONLINEAR ANALYSIS AND EXTREMAL PROBLEMS
%  June 25-30, 2018
%  Irkutsk, Russia
%
%%%%%%%%%%%%%%%%%%%%%%%%%%%%%%%%%%%%%%%%%%%%%%%%%%%%%%%%%%%%%%%%%%%%%%%%


\documentclass[12pt]{llncs}  

% При использовании pdfLaTeX добавляется стандартный набор русификации babel.

\usepackage{iftex}

\ifPDFTeX
\usepackage[T2A]{fontenc}
\usepackage[utf8]{inputenc} % Кодировка utf-8, cp1251 и т.д.
\usepackage[english,russian]{babel}
\fi

\usepackage{todonotes} 

\usepackage[russian]{nla}

\begin{document}
\fi
%\title{Применение поиска с запретами к решению робастной задачи о $p$-медиане}

\title{Применение поиска с запретами к решению робастной задачи о $p$-медиане\thanks{Работа выполнена в рамках государственного задания ИМ СО РАН, проект \textnumero~ FWNF-2022-0020.}}
% Первый автор
\author{Т.~В.~Леванова  \and    И.~С.~Хмара
}

% Аффилиации пишутся в следующей форме, соединяя каждый институт при помощи \and.
\institute{Институт математики им. С. Л. Соболева, Омск, Россия \\
  \email{levanovat@gmail.com}\\
%  \and   % Разделяет институты и присваивает им номера по порядку.
%Институт математики им. С. Л. Соболева, Омск, Россия\\
  \email{khmaraivan97@gmail.com}
% \and Другие авторы...
}

\maketitle

\begin{abstract}
Рассматривается однокритериальная задача о $p$-медиане в робастной постановке. В отличие от классической задачи она позволяет учесть изменения параметров. В данной работе  принимается  во внимение нестабильность  спроса. Изучается возможность применения к  решению робастной задачи о $p$-медиане поиска с запретами. Предлагаются варианты алгоритма для указанной задачи, проводятся исследования на специально созданных тестовых примерах, обсуждается качество результатов и пути развития подхода.

\keywords{задача о $p$-медиане, робастность, поиск с запретами} % в конце списка точка не ставится
\end{abstract}

%\section{Основные результаты} % не обязательное поле

Развивается подход к робастности решения, начатый в работах [1,2].  
 В современном мире процесс принятия решений зачастую неустойчив, поскольку меняются условия. Поэтому строятся такие математические модели, которые позволят учесть изменения. Их цель
заключается в том, чтобы определить, насколько могут измениться параметры задачи, чтобы решение оставалось актуальным. Классическая задача о $p$-медиане хорошо известна: необходимо открыть $p$ предприятий так, чтобы все клиенты были обслужены, а затраты оказались минимальными. В однокритериальной робастной версии задачи $p$-медиане
в качестве неопределенных  параметров рассматривается спрос клиентов. Оптимизируется робастность, а затраты выступают в качестве ограничения. Рассматривается так называемая пороговая робастность.  Использование известного программного обеспечения для решения этой задачи требует большего количества времени и памяти компьютера. поэтому мы развиваем метод приближенного решения.  В данной работе предложен новый вариант алгоритма поиска с запретами [3], учитывающий особенности рассматриваемой задачи. Для его исследования созданы серии тестовых примеров, в частности, с использованием идей библиотеки <<Дискретные задачи размещения>> \cite{math}.

%Работа выполнена в рамках госзадания ИМ СО РАН, проект \textnumero~ FWNF-2022-0020.
%Исследования Еремеева выполнялись в рамках государственного задания ИМ СО РАН, проект FWNF-2022-0020. 




\begin{thebibliography}{9} % или {99}, если ссылок больше десяти.
\bibitem{Vasilev} Васильев~И.Л., Ушаков~А.В. Об одном подходе к робастности решения в задаче о $p$-медиане~//Изв.~Иркутского~гос.~ун-та. Сер.~Математика. 2012. Т.~5,  \textnumero~5. С.~2--15.

 \bibitem{Carrizosa}
Carrizosa~E., Nickel~S. Robust facility location~// Math.~Methods~Oper.~Res. 2003. Vol.~58,
№~2. Pp.~331–349.

\bibitem{Glover} Glover F., Laguna M., Marti R. Tabu Search. Springer New York, NY. 2008. DOI: 10.1007/978-1-4615-6089-0

\bibitem{math} Дискретные задачи размещения. Библиотека тестовых примеров. Точка доступа
http://old.math.nsc.ru/AP/benchmarks/. Дата обращения 01.04.2024.



\end{thebibliography}

% После библиографического списка в русскоязычных статьях необходимо оформить
% англоязычный заголовок.


%%%%%%%%



%\end{document}

