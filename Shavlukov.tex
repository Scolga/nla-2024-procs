
\iffalse
\documentclass[12pt]{llncs}

\usepackage{todonotes}

\usepackage{nla} 

\begin{document}
\fi

\title{On the elliptic umbilical singularity of solutions to nonlinear geometrical optics system\thanks{The research is supported by RNF, project No.~23-11-00009.}}

\author{Azamat Shavlukov  
}
\institute{Institute of Mathematics with Computing Centre, Ufa, Russia\\
  \email{aza3727@yandex.ru}}

\maketitle

\begin{abstract}

The local behaviour of solutions to the nonlinear geometrical optics system
$$\left\{\begin{matrix}
	u_t+u u_x-\alpha(\rho) \rho_x=0,\\
	\rho_t+(\rho u)_x=0,
\end{matrix}\right.$$ where $\alpha(\rho)>0$ is the analytical function is studied. It is shown that in the neigbourhood of the gradient catastrophe point (the finite point at which the first derivatives of solutions tend to infinity), the solutions are determined by the critical points of the function
$$G=y_1^2 y_2-\frac{y_2^3}{3}-k_3 y_2^2-k_2 y_1-k_1 y_2+\gamma$$ which is the section of canonical form of the elliptic umbilical singularity also known as $D_{4-}$ catastrophe according to Arnold's $ADE-$classification theorem.


\keywords{nonlinear geometrical optics, singularity theory, elliptic umbilic}
\end{abstract}



\begin{thebibliography}{9} % or {99}, if there is more than ten references.
\bibitem{1} Dubrovin B., Grava T., Klein C. On universality of critical behaviour in the critical behaviour in the focusing nonlinear Schr\"odinger equation, elliptic umbilic catstrophe and the tritonque to the Painlev\'e-I equation. J. Nonlinear Sc. 2009. Vol.~19, no~1. Pp.~57--94.

\bibitem{2}  Shavlukov A.M. Suleimanov B.I. Inheritance of Generic Singularities of Solutions of a Linear Wave Equation by Solutions of Isoentropic Gas Motion Equations. Math. Notes. 2022. Vol.~112, no~4. Pp.~608--620.

\bibitem{3} Picard E. Sur la détermination des intégrales de certaines équations aux dérivées partielles du second ordre par leurs valeurs le long d'un contour fermé. Journal de l'École polytechnique.~1890. Vol.~60, Pp.~89--106.

\bibitem{4}  Sedykh V.D. Mathematical methods of the catastrophe theory. MCCME, Moscow, 2021. [In Russian]

\end{thebibliography}
%\end{document}
