
\iffalse
\documentclass[12pt]{llncs}

\usepackage{todonotes}

\usepackage{nla} 

\begin{document}
\fi

\title{Comparison optimization models for an identical
parallel machine scheduling problem with\\ uncertain job
durations
}

\author{Sergei Gladyshev
%  \and
%  Name FamilyName2\inst{2}
%  \and
%  Name FamilyName3\inst{1}
}
\institute{Moscow Institute of Physics and Technology, Moscow, Russia\\
  \email{gladyshev.si@phystech.edu
}
%  \and
%Affiliation, City, Country\\
%\email{email@example.com}
}

\maketitle

\begin{abstract}
This study addresses an identical-parallel machine scheduling problem without workers. The expected duration of each operation must be considered while
scheduling to minimize the makespan. However, due to unexpected events occurring during processing, the duration of the job may change. The new processing
time is determined by a known probability distribution, set in advance for each
job. Suppose the original schedule becomes infeasible due to the changes in the
job duration. In that case, it can be made feasible only by moving jobs forward
in time without assigning them to other machines or changing the execution sequence. Overlaps are determined based on the start times from the initial schedule
and start times from the resulting schedule for each job. In our work, we mathematically formulated this problem and conducted computational experiments to
compare the performance of various constraint, linear, and quadratic programming
models that aim to find an initial schedule with a minimum makespan and try to
minimize the expected maximum overlap in the resulting schedule.

\keywords{identical parallel-machine scheduling problem, uncertain processing
times}
\end{abstract}


%\end{document}
