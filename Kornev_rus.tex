\begin{englishtitle} % Настраивает LaTeX на использование английского языка
% Этот титульный лист верстается аналогично.
\title{On one development of the method of guiding functions}
% First author
\author{Sergey Kornev \and Polina Korneva  
 }
\institute{Voronezh State Pedagogical University, Voronezh, Russia\\
\email{kornev\_vrn@rambler.ru}
% etc
}

\maketitle

\begin{abstract}
The method of guiding functions, whose base was laid in the middle of the 20th century by M.A. Krasnoselskii and A.I. Perov, is one of the most effective and geometrically clear ways to solve the periodic problem for differential equations (see, for example, \cite{k_kr}).

In this note, to effectively solve the periodic problem for random differential equations, one of the modifications of the classical notion of a guiding function is used -- a random multivalent guiding function (see, for example, \cite{k_k_o_z_2}, \cite{o_k_k}).

\keywords{differential equation, periodic solution, multivalent guiding function} % в конце списка точка не ставится
\end{abstract}
\end{englishtitle}

\iffalse
\documentclass[12pt]{llncs}

% При использовании pdfLaTeX добавляется стандартный набор русификации babel.

\usepackage{iftex}

\ifPDFTeX
\usepackage[T2A]{fontenc}
\usepackage[utf8]{inputenc} % Кодировка utf-8, cp1251 и т.д.
\usepackage[english,russian]{babel}
\fi

\usepackage{todonotes}

\usepackage[russian]{nla}

\begin{document}
\fi

% Текст оформляется в соответствии с классом article, используя дополнения
% AMS.
%

\title{Об одном развитии метода направляющих функций\thanks{Работа выполнена при поддержке Минпросвещения России в рамках государственного задания (OTGE-2024-0002), РНФ (проект \textnumero~22-71-10008).}}
% Первый автор
\author{С.~В.~Корнев  \and П.~С.~Корнева 
 } % обязательное поле

% Аффилиации пишутся в следующей форме, соединяя каждый институт при помощи \and.
\institute{Воронежский государственный педагогический университет, Воронеж, Россия\\
  \email{kornev\_vrn@rambler.ru}
%  \and   % Разделяет институты и присваивает им номера по порядку.
%Институт (название в краткой форме), Город, Страна\\
%\email{email@examle.com}
% \and Другие авторы...
}

\maketitle

\begin{abstract}
Метод направляющих функций, основы которого заложили еще в середине XX века М.А. Красносельский и А.И. Перов, является одним из наиболее эффективных и геометрически наглядных способов решения периодической задачи для дифференциальных уравнений (см., например, \cite{k_kr}).

В настоящей заметке для эффективного решения периодической задачи для случайных дифференциальных уравнений применяется одна из модификаций классического понятия направляющей функции -- случайная многолистная направляющая функция (см., например, \cite{k_k_o_z_2}, \cite{o_k_k}).

\keywords{дифференциальное уравнение, периодическое решение, многолистная направляющая функция} % в конце списка точка не ставится
\end{abstract}

\section{Основные результаты} % не обязательное поле

В конце ХХ - начале ХХI веков в связи с открывшимися новыми возможностями приложений к актуальным задачам математики, механики, теории управления, физики и других наук, возникла необходимость в существенном расширении классов рассматриваемых направляющих функций. В частности, для дифференциальных уравнений был введен класс многолистных направляющих функций (см., например, \cite{k_r}).

Пусть $(\Omega,\Sigma,\mu)$ -- полное вероятностное пространство. В настоящей работе рассматривается периодическая задача для случайного дифференциального уравнения следующего вида:
\begin{equation}\label{eq1.1}
 z'(\omega, t) = f(\omega,t,z(\omega,t)) \quad \mbox{п.в.}\,\,\,\, t\in \mathbb{R},
\end{equation}
для всех $\omega \in \Omega,$ где $f\colon \Omega \times \mathbb{R} \times \mathbb{R}^n \to \mathbb{R}^n -$ случайный $c$-оператор, удовлетворяющий для каждого $\omega \in \Omega$ условию $T$-периодичности по второму аргументу.

Для исследования рассматриваемой задачи используется модификация классического понятия направляющей функции -- случайная негладкая многолистная направляющая функция. Существенным преимуществом по сравнению с классическим подходом является возможность ``локализовать'', проверку основного условия ``направляемости'', на области, зависящей от самой направляющей функции, причем на области не всего пространства, а его подпространства меньшей размерности. В классических работах по методу направляющих функций, как правило, предполагается, что эти функции являются гладкими на всем фазовом пространстве. Это условие может представиться ограничительным, например, в таких ситуациях, когда направляющие потенциалы различны в различных областях пространства. Для снятия указанного ограничения в работе рассматриваются негладкие направляющие потенциалы и их обобщенные градиенты.

Рассматриваемый метод оказывается эффективен и при исследовании задачи о существовании периодических колебаний в нелинейных объектах, описываемых случайными дифференциальными включениями, правая часть которых является $u$-случайным мультиотображением с выпуклыми компактными значениями, удовлетворяющим условию типа подлинейного роста (см., например, \cite{k_a}).


% Список литературы оформляется подобно ГОСТ-2008.
% Примеры оформления находятся по этому адресу -
%     https://narfu.ru/agtu/www.agtu.ru/fad08f5ab5ca9486942a52596ba6582elit.html
%

\begin{thebibliography}{9} % или {99}, если ссылок больше десяти.

\bibitem{k_kr} Красносельский~М.А. Оператор сдвига по траекториям дифференциальных уравнений. M.:~Наука,~1966.

\bibitem{k_k_o_z_2} Kornev~S., Obukhovskii~V., Zecca~P. On multivalent guiding functions method in the periodic problem for random differential equations~// Journal of Dynamics and Differential Equations. 2019. Vol.~31, Issue~2. Pp.~1017--1028.

\bibitem{o_k_k} Корнев~С.В., Корнева~П.С., Якушева~Н.Э. Об одном подходе в исследовании периодической задачи для случайных дифференциальных уравнений~// Известия вузов. Математика. 2023. \textnumero~5. С.~82--88.

\bibitem{k_r} Рачинский~Д.И. Вынужденные колебания в системах управления в условиях, близких к резонансу~// Автоматика и телемеханика. 1995. \textnumero~11. С.~87--98.

\bibitem{k_a} Andres~J., G\'orniewicz~L. Random topological degree and random differential inclusions~// Topological Methods in Nonlinear Analysis. 2012. \textnumero~40. Pp.~337--358.

\end{thebibliography}

% После библиографического списка в русскоязычных статьях необходимо оформить
% англоязычный заголовок.




%\end{document}
