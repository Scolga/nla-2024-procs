\begin{englishtitle} % Настраивает LaTeX на использование английского языка
% Этот титульный лист верстается аналогично.
\title{Construction of solutions of analogues of the Schrodinger non-stationary equations corresponding to some Hamiltonian systems of the Kimura-Kawamuko list}
% First author
\author{Viktor Pavlenko}
\institute{Institute of Mathematics with Computing Centre - Subdivision of the Ufa Federal Research Centre of Russian Academy of Science, 
Ufa University of Science and Technology, Bashkir State Agrarian University Ufa, Russia\\
  \email{mail@pavlenko.school}}
% etc

\maketitle

\begin{abstract}
A pair of joint solutions of analogs of non-stationary Schrodinger equations are being constructed, 
defined by the Hamiltonians $H_{s_k}(s_1,s_2,q_1,q_2,p_1,p_2) (k=1,2)$ of the Hamiltonian system from the Kimura-Kawamuko list.
These analogues of the Schrodinger equations are linear evolutionary equations with times $s_1$ and $s_2$,
each of which depends on two spatial variables.

\keywords{Hamiltonian systems, Schrodinger equations, Painlevet equations, isomonodromic deformation method} % в конце списка точка не ставится
\end{abstract}
\end{englishtitle}

\iffalse
\documentclass[12pt]{llncs}

% При использовании pdfLaTeX добавляется стандартный набор русификации babel.

\usepackage{iftex}

\ifPDFTeX
\usepackage[T2A]{fontenc}
\usepackage[utf8]{inputenc} % Кодировка utf-8, cp1251 и т.д.
\usepackage[english,russian]{babel}
\fi

\usepackage{todonotes}

\usepackage[russian]{nla}


\begin{document}
\fi

\title{Построение решений аналогов временных уравнений Шредингера, соответствующих некоторым гамильтоновым системам списка Кимуры-Кавамуко
%\thanks{Работа выполнена при поддержке РФФИ (РНФ, другие фонды), проект \textnumero~00-00-00000.}
}
% Первый автор
\author{В.~А.~Павленко   % \inst ставит циферку над автором.
} % обязательное поле

% Аффилиации пишутся в следующей форме, соединяя каждый институт при помощи \and.
\institute{Институт математики с ВЦ УФИЦ РАН, Уфимский университет науки и технологий, Башкирский государственный агарарный университет, Уфа, Россия\\
  \email{mail@pavlenko.school}
}

\maketitle

\begin{abstract}
Строятся пара совместных решений аналогов временных уравнений
Шредингера, определяемых гамильтонианами  $H_{s_k}(s_1,s_2,q_1,q_2,p_1,p_2) (k=1,2)$ гамильтоновой системы из списка Кимуры-Кавамуко. 
Данные аналоги уравнений Шредингера представляют собой линейные эволюционные уравнения с временами $s_1$ и $s_2$, 
каждое из которых зависит от двух пространственных переменных.

\keywords{гамильтоновы системы, уравнения Шредингера, уравнения Пенлеве, метод изомонодроных деформаций} % в конце списка точка не ставится
\end{abstract}

\section{Основные результаты} % не обязательное поле

Наряду  с шестью классическими ОДУ Пенлеве в последние десятилетия все больший интерес вызывают 
нелинейные ОДУ более высокого порядка, которые также интегрируются методом ИДМ. На сегодня, 
в частности, известен конечный список совместных пар гамильтоновых систем ОДУ 
\[
(q_j)^\prime_{s_k}=(H_{s_k})^\prime_{p_j}, \quad 
(p_j)^\prime_{s_k}=-(H_{s_k})^\prime_{p_j}\quad (k=1,2)\quad (j=1,2)
\]
с гамильтонианами $H_{s_k}(s_1,s_2,q_1,q_2,p_1,p_2)$, каждое из которых есть условие совместности
двух линейных систем ОДУ вида 
\[
V^{\prime}_{s_k}=L_{s_k}V,\quad 
V^\prime_{\eta}=AV,
\]
где квадратные матрицы $L_{s_k}$ и $A$ одинаковой размерности и рациональны по переменной $\eta$. 
Этот список выписан в статье Кимуры, см. \cite{Kim}. Позднее Кавамуко дополнил этот список, см. \cite{Kaw}. 

В настоящей работе будут построены решения уравнений вида
\[
k\Psi'_{\tau_1}=H_1(\tau_1,\tau_2,x_1,x_2,-k\frac{\partial}{\partial x_1}, -k\frac{\partial}{\partial x_2})\Psi,
\]
\[
k\Psi'_{\tau_2}=H_2(\tau_1,\tau_2,x_1,x_2,-k\frac{\partial}{\partial x_1}, -k\frac{\partial}{\partial x_2})\Psi,
\]
где $H_1$, $H_2$ --- гамильтонианы из списка Кимуры-Кавамуко. 
Эти уравнения и являются аналогами временных уравнейний Шредингера. Автору удалось построить только для случая $k=1$. 
Для случая $k=i\hbar$ не удалось. Если бы удалось, то это уже были не аналоги, а настоящие уравнения Шредингера. 


% Рисунки и таблицы оформляются по стандарту класса article. Например,
% кавычки << >>.

% В конце текста можно выразить благодарности, если этого не было
% сделано в ссылке с заголовка статьи, например,
%Работа выполнена при поддержке РФФИ (РНФ, другие фонды), проект \textnumero~00-00-00000.
%

\begin{thebibliography}{9} % или {99}, если ссылок больше десяти.
\bibitem{Kim} H.~Kimura. The degeneration of the two dimensional Garnier system and the polynomial Hamiltonian structure.~/ Annali 
di Matematica pura et applicata IV. Vol. 155. No. 1. Pp. 25 -- 74. 

\bibitem{Kaw} H.~Kawamuko. On qualitative properties and asymptotic behavior of solutions to higher-order nonlinear differential equations. 
WSEAS Transact. on Math. 2017. Vol.~16. №~5. Pp.~39 -- 47. 

\end{thebibliography}

% После библиографического списка в русскоязычных статьях необходимо оформить
% англоязычный заголовок и аннотацию.




%\end{document}
