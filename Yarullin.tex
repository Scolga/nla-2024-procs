
\iffalse
\documentclass[12pt]{llncs}

\usepackage{todonotes}

\usepackage{nla} 

\begin{document}
\fi

\title{A minimization method with sequential use of constraint functions when constructing embedding sets\thanks{This paper has been supported by the Kazan Federal University Strategic Academic Leadership Program (``PRIORITY-2030'').}}

\author{Rashid Yarullin\inst{1} \and Igor Zabotin\inst{2}
}
\institute{Kazan (Volga region) Federal University, Kazan, Russia\\
  \email{yarullinrs@gmail.com}
  \and
Kazan (Volga region) Federal University, Kazan, Russia\\
\email{iyazabotin@mail.ru}}

\maketitle

\begin{abstract}
We propose a cutting method with approximation of the constraint region to solve a constrained minimization problem.
The developed method is characterized by the fact that the constraint functions are used sequentially when constructing approximating sets.
This approach of taking into account constraints is implemented as follows. 
At the initial step only one constraint is used and the number of involved constraints increases as the iteration points reach the admissible set.
As a result, the volume of computational operations performed when constructing approximating sets is reduced, which makes the proposed cutting method convenient from a practical point of view for problems with a large number of constraints.
The convergence of the developed cutting method is proven, and the estimate of the accuracy of the solution is obtained.


\keywords{constrained minimization, convex function, cutting methods, cutting plane}
\end{abstract}

Let 
$f(x)$, $g_j(x)$, $j\in J=\{1,\dots,m\}$, $m\ge 1$, be convex functions defined  in the $n$-dimensional Euclidean space ${\rm R_n}$,
$G_j = \{ x \in {\rm R_n } : g_j(x) \le 0 \}$,
$G^\prime= \underset{j\in J}{\cap} G_j,$
$G^{\prime\prime}\subset {\rm R_n}$ be a convex closed set,
$G=G^\prime \cap G^{\prime\prime}$.
Let us immediately note that the interior of the set $G$ can be empty.
We solve the problem
$\min \{ f(x) : x\in G\}$.

Suppose
$f^*= \min \{ f(x) : x\in G\}$,  %used
$X^* = \{ x\in G : f(x) = f^*\}$, %used
$x^*\in X^*$,% used
$g(x) = \max \limits_{j\in J} g_j(x)$,
$\partial f(x)$, $\partial g_j(x)$, $j\in J$, are subdifferentials of the functions
$f(x)$, $g_j(x)$, $j\in J$, at the point 
$x\in {\rm R_n}$, respectively,
$K=\{ 0, 1, \dots\}$,
$I(x, \tilde{J}) = \{ j\in \tilde{J} : x \notin G_j\}$ %used
, where
$\tilde{J}\subset J$.

The proposed method constructs points $x_k$, $k\in K$, as follows.

0. Choose a convex closed set $M_0 \subseteq {\rm R_n}$ such that $x^* \in M_0$.
Put $k=0$,
$J_0^1=\{1\}$, $J_0^2 = J \setminus \{1\}$.

1. Find the point
$x_k \in M_k \cap G^{\prime\prime} \cap \{ x \in {\rm R_n} : f(x) \le f^*\}$.

2. Form subsets 
$J_k^\prime$, $J_k^{\prime\prime} \subset J$ as follows.
Put
$J_k^{\prime} = I(x_k, J_k^1)$.
If
$I(x_k, J_k^2) = \emptyset$, then
$J_k^{\prime\prime} = \emptyset$.
Otherwise, $J_k^{\prime\prime}$ is chosen as any non-empty subset of the set $I(x_k, J_k^2)$.
Put
$J_k = J_k^{\prime} \bigcup J_k^{\prime\prime}$.
If $J_k = \emptyset$,
then the iterative process is stopped, and $x_k$ is a solution of the optimization problem.

3. Choose finite sets $A_k^j \subset \partial g_j(x_k)$ for all $j\in J_k$, and put
\begin{equation}\label{step1.1:5.1}
M_{k+1} = M_k \underset{j\in J_k}{\cap}
\{ x\in {\rm R_n} :  g_j(x_k) + \langle a, x - x_k \rangle \le 0 \ \forall a \in A_k^j\}.
\end{equation}

4. Put 
$J_{k+1}^1=J_k^1 \cup J_k^{\prime\prime}$,
$J_{k+1}^2 = J_k^2 \setminus J_k^{\prime\prime}$,
$k:=k+1$, and go to Step 1.

In cutting methods with approximation of the constraint region (see, for example, \cite{bulatov_1977,topkis_1970,polyak_1966,yarullin_2022}), to construct the next approximating set, a certain subset of constraint indices is formed at each step.
The process of forming such a subset of indices can be very labor-intensive, since all the constraint functions that define the admissible set are involved.
Therefore, in practice, there is a need to consistently take into account the restrictions involved in the construction of approximating sets, and this method of using restrictions in the proposed method is implemented as follows.

The transition from the current approximating set $M_k$ to $M_{k+1}$ occurs by separating the point $x_k$ from the set $G^\prime$ by cutting planes according to (\ref{step1.1:5.1}).
In order not to use all the restrictions that define $G^\prime$ in this transition, the set $J$ is divided into two parts.
The first part $J_k^1$ includes numbers of restrictions that are allowed to participate in the construction of cuts at the $k$th iteration, and in the second part $J_k^2$ there remain unused numbers of restrictions, and in a certain way can be transferred from $J_k^2 $ into the first part of $J_{k+1}^1$ indices to carry out sequential accounting of all functions $g_j(x)$, $j\in J$.
In particular, for $I(x_k, J_k^2) \ne \emptyset$ it is permissible to put the number $j_k \in I(x_k, J_k^2)$ of some constraint $g_{ j_k}(x)$ or a small group of numbers $\tilde{J}_k \subset I(x_k, J_k^2)$ into $J_k^{\prime\prime}$.
Then the set of constraint numbers $J_k^{\prime\prime}$ that were not taken into account at previous iterations according to Step 4 will go into the set $J^1_{k+1}$ and the numbers from $J_k^{\prime\prime}$ from the next iterations will be taken into account when cutting iteration points from the set $G^\prime$.
Due to this feature of the constructing the set $J_k^{\prime\prime}$, consistent consideration of constraint functions is implemented when constructing approximating sets.

The convergence of the method is substantiated and the estimate of the accuracy of the solution is obtained.

\begin{theorem}\label{the1}
Let the proposed method constructs the sequence
$\{x_k\}$, $k\in K$.
Then any limit point of this sequence  belongs to the set
$X^*$.
\end{theorem}

\begin{theorem}\label{theorem1.1:3}
Let $g_{\bar{j}}(x)$, $\bar{j}\in J$, be a strongly convex function in ${\rm R_n}$ with strong convexity constant $\mu_{\bar{ j}} > 0$,
$\bar{x} \in G$, and at some step
$k\in K$ the inclusion $\bar{j} \in J_k$ is performed, the points $x_k$ are constructed, $y_k = \bar{x} + \bar{\alpha} (x_k - \bar{x})$, where $\bar{\alpha}\in (0, 1)$ is a fixed number.
If for some $\xi > 0$ the inequality $g_{\bar{j}}(x_k) - g_{\bar{j}}(y_k) \le \xi$ holds, then the estimate $f^* - f(x_k) \le \|\bar{s}\| \sqrt{ 2\xi / (\mu_{\bar{j}} \bar{\alpha}(1-\bar{\alpha}) }$  is valid, where $\bar{s} \in \partial f( \bar{x})$.
\end{theorem}




\begin{thebibliography}{9} % or {99}, if there is more than ten references.
\bibitem{bulatov_1977}
 Bulatov V.P.  Embedding Methods in Optimization Problems. Nauka, Novosibirsk. 1977  [in Russian]. 
 
 \bibitem{polyak_1966}
Levitin E.S., Polyak B.T.  Constrained minimization methods. USSR Computational Mathematics and Mathematical Physics. 1966. Vol. 6, No. 5. Pp. 1--50.

\bibitem{topkis_1970}
 Topkis D. M.: Cutting plane methods without nested constraint sets. Operations Research. 1970. Vol. {18}. Pp. 404--413.

\bibitem{yarullin_2022}
Zabotin I., Shulgina O., Yarullin R. Procedures for Updating Immersion Sets in the Cutting Method and Estimating the Solution Accuracy. CCIS. 2022. Vol. {1661}. Pp. 218--229.

\end{thebibliography}
%\end{document}
