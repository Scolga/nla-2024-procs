

\iffalse
\documentclass[12pt]{llncs}

\usepackage{todonotes}

\usepackage{nla} 

\begin{document}
\fi

\title{On one generalized Lyapunov matrix equation\thanks{The work is supported by the Mathematical Center in Akademgorodok under
agreement No. 075-15-2022-282 with the Ministry of Science and Higher
Education of the Russian Federation.}}

\author{Vladislav S. Prokhorov
}
\institute{Novosibirsk State University, Novosibisk, Russia\\
  \email{v.prokhorov@g.nsu.ru}}
\maketitle

\begin{abstract}
We discuss the problem on belonging of the matrix spectrum to a region bounded by a parabola in terms of solving a generalized Lyapunov matrix equation. Using a solution to the matrix equation, an estimate for the norm of the matrix exponential is obtained.


\keywords{generalized Lyapunov matrix equation, Krein's theorem, matrix exponential}
\end{abstract}




It is well known that the spectrum of the matrix $A$ belongs to the domain
$${\cal P}_i = \left\{\lambda : \frac{1}{2p}({\rm Im}\,\lambda)^2 < {\rm Re}\,\lambda\right\}, \quad p>0,$$  if and only if there exists a solution $H = H^*>0$ to the following generalized Lyapunov matrix equation (see, for example, [1],[2])
$$HA+A^*H-\frac{1}{2p}\left[A^*HA - \frac{1}{2}(HA^2+(A^*)^2H)\right] = I. \eqno(1)$$
    
   We obtain a representation for a solution to this equation that differs from one given by M.G.Krein [2]:

\begin{theorem}
	Let A be a matrix $n\times n$ whose eigenvalues are located in ${\cal P}_i$, then the solution $H$ to (1) can be represented as follows
	$$
    H = \sum\limits_{k=0}^\infty
	\bigg(\frac{-1}{4p}\bigg)^k \Bigg( {\int\limits_{0}^{+\infty} \dots \int\limits_{0}^{+\infty}}  e^{-(t_0 + \dots + t_k)A^*}
	\bigg(\sum\limits_{r=0}^{2k}C_{2k}^r(-A^*)^{2k-r}A^r\bigg)
	e^{-(t_0 + \dots + t_k)A} dt_0 \dots dt_k \Bigg).
	$$	
		
\end{theorem}
\begin{theorem}
	Let the spectrum of the matrix A belong to ${\cal P}_i$. If $H = H^*>0$ is a solution to (1), then the following estimate holds:
	$$\| e^{-tA} \| \leq \sqrt{\mu(H)}e^{\bigg(p-\sqrt{p^2+\frac{1}{\|(A^*A)^{-1}\|}+\frac{p}{\|H\|}}\bigg)t}, \quad \mu(H) = \|H\|\|H^{-1}\|, \quad t>0.$$
\end{theorem}
 

% At the end of the text, acknowledgments are expressed, if you haven't

\begin{thebibliography}{9} % or {99}, if there is more than ten references.
\bibitem{Mazko1999} Mazko A. G. Localization of the Spectrum and Stability of Dynamical Systems. Inst. Mat., Nats. Akad. Nauk Ukrain., Kiev, 1999. [In Russian]
\bibitem{Demidenko2023} Demidenko G. V., Prokhorov V. S. On location of the matrix spectrum with respect to a parabola. Sib. Adv. Math. 2023. Vol.~33. Pp.~190–199.
\end{thebibliography}
%\end{document}
