\begin{englishtitle} % Настраивает LaTeX на использование английского языка
% Этот титульный лист верстается аналогично.
\title{On the dynamical reconstruction problem  of a continuous disturbance in a fractional order system}
% First author
\author{Platon G. Surkov\inst{1,2}
}
\institute{IMM UB RAS, Ekaterinburg, Russia
  \and
INSM UrFU, Ekaterinburg, Russia\\
  \email{spg@imm.uran.ru}}
% etc

\maketitle

\begin{abstract}
The dynamical reconstruction problem of  an unknown disturbance in a fractional order system is considered.
To solve it, the dynamical inversion method was used.
As a result of the application of which, a finite-step online algorithm for constructing approximations of an unknown external disturbance was proposed, and its convergence was established.

\keywords{online identification, external disturbance, Caputo fractional derivative} % в конце списка точка не ставится
\end{abstract}
\end{englishtitle}

\iffalse
\documentclass[12pt]{llncs}  

% При использовании pdfLaTeX добавляется стандартный набор русификации babel.

\usepackage{iftex}

\ifPDFTeX
\usepackage[T2A]{fontenc}
\usepackage[utf8]{inputenc} % Кодировка utf-8, cp1251 и т.д.
\usepackage[english,russian]{babel}
\fi

\usepackage{todonotes} 

\usepackage[russian]{nla}

\begin{document}
\fi

\title{О задаче динамического восстановления непрерывного возмущения в системе дробного порядка\thanks{Работа выполнена в рамках исследований, проводимых в Уральском математическом центре при финансовой поддержке Министерства науки и высшего образования Российской Федерации (номер соглашения 075-02-2024-1377).}}
% Первый автор
\author{П.~Г.~Сурков\inst{1,2}  % \inst ставит циферку над автором.
%  \and  % разделяет авторов, в тексте выглядит как запятая.
% Второй автор
%  И.~О.~Фамилия\inst{2}
%  \and
% Третий автор
%  И.~О.~Фамилия\inst{1}
} % обязательное поле

% Аффилиации пишутся в следующей форме, соединяя каждый институт при помощи \and.
\institute{ИММ УрО РАН, Екатеринбург, Россия
  \and   % Разделяет институты и присваивает им номера по порядку.
ИЕНиМ УрФУ, Екатеринбург, Россия\\
  \email{spg@imm.uran.ru}
% \and Другие авторы...
}

\maketitle

\begin{abstract}
Рассмотрена задача динамического восстановления неизвестного возмущения в системе дробного порядка.
Для ее решения использовался метод динамического обращения.
В результате применения которого был предложен конечно шаговый алгоритм построения аппроксимаций неизвестного внешнего воздействия в режиме онлайн, и установлена его сходимость.

\keywords{онлайн идентификация, внешнее возмущение, дробная производная Капуто} % в конце списка точка не ставится
\end{abstract}

%\section{Основные результаты} % не обязательное поле

В настоящее время активно проводимые исследования, посвященные системам дробного порядка, все более переходят от теоретической направленности к прикладной~\cite{Uchaikin}.
Также все большее внимание уделяется задачам управления в различных постановках и смежным им задачам идентификации.
Мы будем рассматривать задачу динамического восстановления неизвестной характеристики системы  дробного порядка, присутствующее в ней возмущение, если априори известно, что возмущение носит непрерывный характер.
Рассматривается управляемая система дробного порядка вида
\[
	[D^{\gamma}_* x](t)=f(t,x(t))+Bu(t),\quad t\in T:=[\sigma,\theta],\quad \gamma\in(0,1),\eqno{(1)}
\]
где $x(t)\in\mathbb R^d$ --- фазовый вектор, $t\in T$ --- конечный отрезок, $u(\cdot)$ --- внешнее воздействие, $u(t)\in P\subset\mathbb R^v$, $P$ --- выпуклый компакт, $f(\cdot,\cdot)$ --- заданная непрерывная функция, удовлетворяющая условию Липшица по второму аргументу, также использовано обозначение для дробной производной Капуто~\cite[с.~91]{Kilbas}.

Задано начальное положение системы $x(\sigma)=x_\sigma$.
Внешнее  воздействие $u(\cdot)$ заранее неизвестно, и траектория системы также неизвестна, но доступна для дискретных измерений с некоторой погрешностью в режиме онлайн, т.е. в достаточно частые моменты времени $\tau_i\in T$, $\tau_i<\tau_{i+1}$ имеются вектор $\xi^h_i\in \mathbb R^d$, удовлетворяющие неравенству
\[
	\| x(\tau_i)-\xi^h_i\|_{\mathbb R^d}\leq h,\eqno{(2)}
\]
где $h\in(0,1)$ --- допустимая погрешность.
Решается следующая задача.
В предположении, что траектория $x(\cdot)$ и внешнее воздействие $u(\cdot)$ заранее неизвестны, а информация о позиции системы поступает одновременно с ее функционированием в виде векторов $\xi^h_i$, удовлетворяющих~(2), требуется построить адаптивный работающий в режиме онлайн  устойчивый к информационным помехам и погрешностям вычислений алгоритм  нахождения аппроксимации $v^h(\cdot)$ воздействия $u(\cdot)$ близкой к нему в равномерной метрике.
Для решения рассматриваемой задачи  используется метод динамического обращения~\cite{Osipov}.
Данный метод основывается на использовании принципа экстремального прицеливания Н.Н.~Красовского в сочетании с методом регуляризации А.Н.~Тихонова.
В качестве вспомогательной системы (модели) выбирается следующая система с переключением
\[
	[D^{\gamma}_* y](t)=g(t,\xi^h(t),s(t))+Bv^h(t),\quad y(\sigma)=x_\sigma,\quad t\in T,
\]
где используются обозначения:
\[ 
	\xi^h(t):=\xi^h_i,\quad s(t):=i,\quad g(t,\xi^h(t),i):=f(\tau_i,\xi^h_i),\quad  t\in[\tau_i,\tau_{i+1}).
\]
Управление в модели выбирается кусочно-постоянным минимизирующим сглаживающий функционал
\[
	v^h(t):=v^h_i\in\mathrm{argmin}\{2(y(\tau_i)-\xi^h_i,Bv)+\alpha(h)\|v\|^2_{\mathbb R^d}\},\quad  t\in[\tau_i,\tau_{i+1}),
\]
где $(\cdot,\cdot)$ --- скалярное произведение в евклидовом пространстве.

Справедлива следующая теорема.
\begin{theorem}
	Пусть матрица $B$ несингулярная, $u(\cdot)\in W^1_\infty(T;\mathbb R^d):=\{u(\cdot)\in C(T;\mathbb R^d),\dot u(\cdot)\in L_\infty(T;\mathbb R^d)\}$, и выбрано разбиение $\Delta^h:=\{\tau^h_i\}_{i=0}^{\kappa_h}$, $\tau^h_{i+1}=\tau^h_i+\delta$, $\tau^h_{0}=\sigma$, $\tau^h_{\kappa_h}=\theta$.
	Тогда, если выполнены соотношения
\[
	\alpha(h)\to 0,\quad \delta(h)\to 0,\quad \frac{h}{\alpha(h)}\to0,\quad 
	\frac{\delta^\gamma(h)}{\alpha(h)}\to0\quad \text{при}\quad h\to0,
\]
и $u(\sigma)=0$, то имеет место сходимость $v^h(\cdot)\to u(\cdot)$ в $C(T;\mathbb R^d)$ при $h\to0$.
\end{theorem}

% Рисунки и таблицы оформляются по стандарту класса article. Например,

%\begin{figure}[htb]
%  \centering
  % Поддерживаются два формата:
  %\includegraphics[width=0.7\linewidth]{figure.pdf} % Растровый формат
  %\includegraphics[width=0.7\linewidth]{figure.png} % Векторный и растровый формат
  %
  % Векторные рисунки можно рисовать в редакторе Inkscape
  % https://inkscape.org/ru/download/
  % Основной формат этого редактора - SVG, поэтому рисунки необходимо экспортировать в
  % PDF или PNG (с разрешением - минимум 150 dpi, максимум - 300dpi).
%  \begin{center}
%    \missingfigure[figwidth=0.7\linewidth]{Уберите меня из статьи!}
%  \end{center}
%  \caption{Заголовок рисунка}\label{fig:example}
%\end{figure}

% кавычки << >>.

% В конце текста можно выразить благодарности, если этого не было
% сделано в ссылке с заголовка статьи, например,
%Работа выполнена при поддержке РФФИ (РНФ, другие фонды), проект \textnumero~00-00-00000.
%

\begin{thebibliography}{9} % или {99}, если ссылок больше десяти.
\bibitem{Uchaikin}
Uchaikin~V.V.
Fractional Derivatives for Physicists and Engineers. 
Heidelberg:~Springer~Berlin,~2013.

\bibitem{Kilbas}
Kilbas~A.A., Srivastava~H.M., Trujillo~J.J.
Theory and Applications of Fractional Differential Equations.
Amsterdam:~Elsevier Science,~2006.

\bibitem{Osipov}
Осипов~Ю.С., Кряжимский~А.В., Максимов~В.И.
Методы динамического восстановления входов управляемых систем. 
Екатеринбург:~Изд-во УрО РАН,~2011.

%\bibitem{Kholl} Современные численные методы решения обыкновенных дифференциальных уравнений~/ Под~ред.~Дж.~Холл, Дж.~Уатт. М.:~Мир,~1979.

%\bibitem{Aleksandrov1} Александров~А.Ю. Об устойчивости сложных систем в критических случаях~// Автоматика и телемеханика. 2001. \textnumero~9. С.~3--13.

%\bibitem{Moreau1977} Moreau~J.-J. Evolution problem associated with a moving convex set in a Hilbert space~// J.~Differential~Eq. 1977. Vol.~26. Pp.~347--374.

%\bibitem{Semenov} Семенов~А.А. Замечание о вычислительной сложности известных предположительно односторонних функций~// Тр.~XII Байкальской междунар. конф. <<Методы оптимизации и их приложения>>. Иркутск, 2001. С.~142--146.

\end{thebibliography}

% После библиографического списка в русскоязычных статьях необходимо оформить
% англоязычный заголовок и аннотацию.




%\end{document}