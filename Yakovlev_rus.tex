\begin{englishtitle}
\title{Algorithms for rasterizing two-dimensional objects and gradients}
\author{Vladislav Yakovlev}
\institute{Institute of Mathematics and Information Technologies\\Irkutsk State University, Irkutsk, Russia\\
  \email{zedark890@gmail.com}
  }
% etc

\maketitle

\begin{abstract}
The aim is to study the rasterization algorithms and create  tools for rasterization of closed surfaces. The implemented tools are used for rasterization of three-dimensional objects.

\keywords{rasterization algorithms, line, polygon, color, gradient}
\end{abstract}
\end{englishtitle}


\iffalse
\documentclass[12pt]{llncs}
\usepackage[T2A]{fontenc}
\usepackage[utf8]{inputenc}
\usepackage[english,russian]{babel}
\usepackage[russian]{nla}

%\usepackage[english,russian]{nla}

% \graphicspath{{pics/}} %Set the subfolder with figures (png, pdf).

%\usepackage{showframe}
\begin{document}
%\selectlanguage{russian}
\fi

\title{Алгоритмы растеризации двумерных объектов и градиента%\thanks{Работа выполнена при поддержке РФФИ (РНФ, другие фонды), проект \textnumero~00-00-00000.}
}
% Первый автор
\author{В.~А.~Яковлев%\inst{1}
%  \and
% Второй автор
%  И.~О.~Фамилия\inst{2}
%  \and
% Третий автор
%  И.~О.~Фамилия\inst{2}
} 
\institute{Институт математики и информационных технологий\\ Иркутский государственный университет, Иркутск, Россия\\
  \email{zedark890@gmail.com}
  }
.

\maketitle

\begin{abstract}
Цель работы заключается в изучении алгоритмов растеризации и создании инструментария для растеризации замкнутых поверхностей с целью дальнейшего использования реализованного инструментария для растеризации трёхмерных объектов. %Представлены алгоритмы растеризации двумерных объектов и градиента    и их реализация  на языке программирования python с использованием библиотеки pygame.   

\keywords{алгоритмы растеризации, прямая, многоугольник, цвет, градиент}
\end{abstract}

%\section{Основные результаты} % не обязательное поле

%{\color{red}

%Текст доклада на русском языке.

В докладе рассмотрены алгоритмы растеризации двумерных и трехмерных объектов и градиента %\cite{Yakovlev1,Yakovlev2} 
и их реализация на языке программирования python с использованием библиотеки pygame. 
%На текущем этапе реализации программа  будет способна строить многоугольники; на начальном этапе -- это треугольники четырех типов: каркасные, с однотонным цветом, с параметром тени и с градиентом.

В работе рассмотренно важное понятие растеризации: 4- и 8-связные развертки. Реализованы алгоритмы отрисовки прямых: интерполяция прямой по формуле y = ax + b; интерполяция по формуле y[i+1] = y[i] + a; алгоритм Брезенхейма. На основе одного из алгоритмов отрисовки прямых разобран способы растеризации многоугольников  и растеризации однотонного цвета при помощи алгоритма отрисовки множества горизонтальных прямых внутри всей поверхности многоугольника. %После этого введение параметра тени и в конце растеризация градиента. 
Разработан программный инструмент, позволяющий отображать поверхность, залитую градиентом.
Программа, содержит следующий инструментарий: функции различных алгоритмов растеризации прямой;  растеризация каркасного треугольника;   растеризация треугольника с однотонным цветом;  растеризация треугольника с однотонным цветом и параметром тени;   растеризация треугольника с градиентом.

%}



% Список литературы.
%\begin{thebibliography}{99}
%\bibitem{Yakovlev1}
%Боресков А.В. Программирование компьютерной графики. Современный OpenGL.  Москва: ДМК Пресс, 2019.  372 с.
%\bibitem{Yakovlev2}
%Гамбетта Г. Компьютерная графика. Рейтрейсинг и растеризация.  Санкт-Петербург: Питер, 2022.  224 с.
%\bibitem{1}
% Format for Journal Reference:
%Author1 N., Author2 N.  Article title. Journal. Year. Vol.~Volume, No~Number. Pp.~Page %numbers.
% Format for books:
%\bibitem{2}
%Author N. Book title. Place: Publisher, year.
% Format for Russian Journal Reference:
%\bibitem{3} Фамилия И.О. Название статьи~// Журнал. Год. Т.~том,  \textnumero~номер. %С.~страницы.
% etc
%\end{thebibliography}


% Обязательная информация на английском языке:



%\end{document}
