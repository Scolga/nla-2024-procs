
\iffalse

%%%%%%%%%%%%%%%%%%%%%%%%%%%%%%%%%%%%%%%%%%%%%%%%%%%%%%%%%%%%%%%%%%%%%%%%
%
% This is the template file for the 6th International conference
% NONLINEAR ANALYSIS AND EXTREMAL PROBLEMS
% June 25-30, 2018
% Irkutsk, Russia
%
%%%%%%%%%%%%%%%%%%%%%%%%%%%%%%%%%%%%%%%%%%%%%%%%%%%%%%%%%%%%%%%%%%%%%%%%
% The preparation of the article is based on the standard llncs class
% (Lecture Notes in Computer Sciences), which is adjusted with style
% file of the conference.
%
% There are two ways of compilation of the file into PDF
% 1. Use pdfLaTeX (pdflatex), (LaTeX+DVIPS will not work);
% 2. Use LuaLaTeX (XeLaTeX will work too).
% When using LuaLaTeX You will need TTF or OTF CMU fonts
% (Computer Modern Unicode). The fonts are installed with 'cm-unicode' package in
% a distribution of LaTeX % (https://www.ctan.org/tex-archive/fonts/cm-unicode),
% either by downloading and installing these fonts system wide, the address of their page is
% http://canopus.iacp.dvo.ru/%7Epanov/cm-unicode/
% The second option won't work in XeLaTeX.
%
% For MiKTeX (LaTeX distribution for Windows),
%  1. Package 'cm-unicode' is installed manually with the MiKTeX administration Console.
%  2. For the compilation of this example, namely, the stub figure, one will also need to
% download package 'pgf' manually. This package uses in the popular
% package tikz.
%  3. Tests showed that the rest of the required packages MiKTeX loads automatically (if
%     it is allowed). The 'auto download' option is
%     configured in 'Settings' section in MiKTeX Console.
%
%
% The easiest way to compile an article is to use pdfLaTeX, but
% the final layout of the book will be compiled with LuaLaTeX,
% as a result will be of better quality thanks to the package 'microtype' and
% use vector OTF instead of standard raster fonts of pdfLaTeX.
%
% In the case of questions and problems with the article compilation,
% write letters to e-mail: eugeneai@irnok.net, Cherkashin Evgeny.
%
% New version of the correcting style file will be available at the website:
%     https://github.com/eugeneai/nla-style
%     file - nla.sty
%
% Further instructions are in the text body of the template. The template itself
% is an article example.
%
% The LaTeX2e format is used!

% 12 points font size is used.
\documentclass[12pt]{llncs}

% The correcting style file is added.
\usepackage{todonotes}

\usepackage{nla} % This package is needed for compiling
                 % this template, it should be removed
                 % from your article.

% Many popular packages (amsXXX, graphicx, etc.) are already imported in the style file.
% If there is a conflict with your packages, try disabling them and compile
% the text.
%
% It would be convenient in the layout of the proceedings if the file names
% of the figures of different authors do not clash.
% To minimize the clash, the drawings can be placed in a separate subfolder
% named after the author or the title of the paper.
%
% \graphicspath{{ivanov-petrov-pics/}} % specifies the folder with images in png, pdf formats.
% or
% \graphicspath{{great-problem-solving-paper-pics/}}.

\begin{document}
\fi
% Text should be formatted in accordance with the 'article' class, using extensions like
% AMS.
%
\title{Solving boundary value problems for wave PDE \\with quasi-classical variational formulation and \\ neural network\thanks{The research is supported by the RUDN University named after Patrice Lumumba, project \textnumero 002092-0-000.}}
% First author
\author{Sergey Shorokhov %\inst{1}
%  \and
%  Name FamilyName2\inst{2}
%  \and
%  Name FamilyName3\inst{1}
}
\institute{Peoples' Friendship University of Russia named after Patrice Lumumba, Moscow, Russia\\
  \email{shorokhov-sg@rudn.ru}}
%  \and
%Affiliation, City, Country\\
%\email{email@example.com}}
% etc

\maketitle

\begin{abstract}
We obtain a quasi-classical variational formulation of boundary value problem for wave PDE and 
demonstrate that obtained variational functional may be used as a loss functional to train a neural network, approximating the solution to the boundary value problem.  

\keywords{boundary value problem, variational formulation, neural network} %\emph{etc.}}
\end{abstract}

% at the end of the list, there should be no final dot
\section{Variational formulation of boundary value problem}

It is well known that neural networks can successfully approximate the solutions of boundary value problems for partial differential equations (PDE) by forcing the neural network output to satisfy the PDE, boundary conditions and known physical laws, if any, using the correspondent residual functional when training the neural network \cite{Raissi2019}. 
 
We study the problem of constructing a loss functional based on a quasi-classical variational formulation \cite{Filippov1994} to train a neural network for the following boundary value problem for wave PDE

\begin{equation}
u_{tt}-u_{xx}=0,\,\left(x,\,t\right)\in\Omega\subset\mathbb{R}^{2}
\label{eq:wavePDE}
\end{equation}
in the rectangular area $\Omega=\left[0,\,l\right]\times\left[0,\,T\right]$ with boundary conditions
\begin{equation}
\left\{ \begin{array}{lll}
u\left|_{t=0}\right.=\, & \chi_{1}\left(x\right),\, & 0\leq x\leq l, \\
u_{x}\left|_{x=l}\right.=\, & \varphi_{2}\left(t\right),\, & 0\leq t\leq T,\\
u_{t}\left|_{x=l}\right.=\, & \psi_{2}\left(t\right),\, & 0\leq t\leq T,\\
u_{x}\left|_{t=T}\right.=\, & \varphi_{3}\left(x\right),\, & 0\leq x\leq l, \\
u_{t}\left|_{t=T}\right.=\, & \psi_{3}\left(x\right),\, & 0\leq x\leq l, \\
u_{x}\left|_{x=0}\right.=\, & \varphi_{4}\left(t\right),\, & 0\leq t\leq T,\\
u_{t}\left|_{x=0}\right.=\, & \psi_{4}\left(t\right),\, & 0\leq t\leq T.
\end{array}
\right.
\label{eq:wavePDE_bc}
\end{equation}
We assume that functions on the right sides of \eqref{eq:wavePDE_bc} are smooth $L_{2}$-functions and the boundary conditions \eqref{eq:wavePDE_bc} are consistent.

We prove the following statement  for the boundary value problem \eqref{eq:wavePDE}--\eqref{eq:wavePDE_bc} using the symmetrizing operator approach by V.M. Shalov \cite{Shalov1965}.

\begin{theorem}
The variational functional for the boundary value problem \eqref{eq:wavePDE}-\eqref{eq:wavePDE_bc}  can be written in the form
\begin{equation}
D\left[u\right]=\int_{\Omega}\left(u_{x}^{2}+u_{t}^{2}-u_{x}\left(\nu+\mu\right)-u_{t}\left(\nu-\mu\right)\right)\,dxdt+\int_{0}^{l}\left(u^{2}-2\chi_{1}u\right)\left|_{t=0}\right. dx,
\label{eq:wavePDE_func}
\end{equation}
where the functions $\mu,\,\nu$ are defined in area $\Omega$ according to the following formulas:
\begin{equation}
\begin{array}{ll}
\mu\left(x,\,t\right) & =\left\{\begin{array}{ll}
0.5\left(\varphi_{2}\left(-x+t+l\right)-\psi_{2}\left(-x+t+l\right)\right), & x-t>l-T,\\
0.5\left(\varphi_{3}\left(x-t+l\right)-\psi_{3}\left(x-t+l\right)\right), & x-t\leq l-T,
\end{array}\right. \\[1em]
\nu\left(x,\,t\right) & =\left\{\begin{array}{ll}
0.5\left(\varphi_{3}\left(x+t-T\right)+\psi_{3}\left(x+t-T\right)\right), & x+t>T,\\
0.5\left(\varphi_{4}\left(x+t\right)+\psi_{4}\left(x+t\right)\right), & x+t\leq T.
\end{array}\right.
\end{array}
\end{equation}
\end{theorem}

Functional \eqref{eq:wavePDE_func} contains first-order derivatives of the function $u$ and can be estimated using the Monte Carlo integration method.

\section{Training neural network with variational functional}

When training a deep neural network for a boundary value problem, the residual functional is generally used as a loss functional \cite{Raissi2019}. 
When using the residual functional for the problem  \eqref{eq:wavePDE}--\eqref{eq:wavePDE_bc},
it is necessary to evaluate five integrals (over area $\Omega$ and over four parts of its boundary $\partial\Omega$)  and calculate partial derivatives of $u$ up to the second order, inclusive.
When using the quasi-classical functional \eqref{eq:wavePDE_func},  one has to evaluate two integrals (over $\Omega$ and a part of $\partial\Omega$) and calculate first order derivatives of $u$.

Computational experiments on training neural networks for the boundary value problem \eqref{eq:wavePDE}--\eqref{eq:wavePDE_bc} using the residual functional and the constructed quasi-classical functional \eqref{eq:wavePDE_func} as a loss functional demonstrate that when using the quasi-classical functional, the neural network training occurs approximately three times faster with the achievement of significantly higher quality metrics.

Thus, the quasi-classical variational functional can have a number of advantages over the residual functional when training a neural network for a boundary value problem.

% The figures and tables are drawn according to the standard class 'article'.
%\begin{figure}[htb]
%  \centering

% Two picture formats are supported:
%\includegraphics[width=0.7\linewidth]{figure.pdf} % Raster format
%\includegraphics[width=0.7\linewidth]{figure.png} % Vector and raster format
%
% Vector drawings can be drawn in Inkscape editor
% https://inkscape.org/ru/download/
% The usual format of the editor is SVG, so the drawings must be exported in
% PDF or PNG (with a resolution of minimum 150 dpi, and maximum of 300 dpi).
%  \begin{center}
%    \missingfigure[figwidth=0.7\linewidth]{Remove me from the article!} \end{center}
%  \caption{Caption of the figure}\label{fig:example}
%\end{figure}

% At the end of the text, acknowledgments are expressed, if you haven't
% made a footnote from the title. For example, we can write
% The research is carried on with support of RFBR (RNF, other funds), project No.~00-00-00000.

\begin{thebibliography}{9} % or {99}, if there is more than ten references.
\bibitem{Raissi2019} Raissi~M., Perdikaris~P., Karniadakis~G.E. Physics-Informed Neural Networks: A Deep Learning Framework for Solving Forward and Inverse Problems Involving Nonlinear Partial Differential Equations. J. of Comput. Phys. 2019. Vol. 378. Pp. 686--707.
\bibitem{Filippov1994} Filippov~V.M., Savchin~V.M., Shorokhov~S.G. Variational principles for nonpotential operators. J. Math. Sci. 1994. Vol. 68, no 3. Pp. 275--398.
\bibitem{Shalov1965} Shalov V.M.  Minimum principle of a quadratic functional for a hyperbolic equation. Differ. Uravn. 1965. Vol. 1, no. 10.  Pp. 1338--1365.
\end{thebibliography}
%\end{document}

%%% Local Variables:
%%% mode: latex
%%% TeX-master: t
%%% End:
