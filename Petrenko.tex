
\iffalse
\documentclass[12pt]{llncs}

\usepackage{todonotes}

\usepackage{nla}

\begin{document}
\fi

\title{About solvability and controllability of differential-algebraic equations with hysteresis phenomena}%\thanks{The research is supported by RFBR (RNF, other funds), project No.~00-00-00000.}}

\author{Pavel Petrenko\inst{1, 2}
  %\and
  %Name FamilyName2\inst{2}
  %\and
  %Name FamilyName3\inst{1}
}
\institute{Matrosov Institute for System Dynamics and Control Theory of the Siberian Branch of the RAS, Irkusk, Russia\\
  %\email{petrenko\_p@mail.ru}
  \and
Irkutsk State University, Irkusk, Russia\\
\email{petrenko\_p@mail.ru}}

\maketitle

\begin{abstract}
In this note, we single out some promising classes of differential-algebraic equations (DAE) with non-linearity of hysteresis type modeled by a sweeping process. The systems under investigation arise in modeling various physical processes, in particular, in electrical circuits with hysteresis phenomena. For such a DAE, we design an equivalent structural form (in a sense of solutions). Necessary and sufficient conditions for the existence and uniqueness of a solution to an initial value problem and controllability are proved.

\keywords{differential-algebraic equations, descriptor systems, hysteresis, sweeping process, solvability, controllability}
\end{abstract}


\section{The main results} %optional section


Consider the following system of ordinary differential equations (ODE) paired with a sweeping process \cite{KM2000,Moreau1977} of a rate independent hysteresis type (modeled by the play operator \cite{BS1996,Kr1991,PSS2020})
\begin{equation} \label{pss1}
	A(t)\dot{x}(t)=B(t)x(t)+C(t)y(t), \ \ \ x(t_0)=x_0,
\end{equation}
\begin{equation}\label{pss2}
  \begin{aligned}
	-\dot{z}(t)&\in \mathcal{N}_{Q(t)}\big(z(t)\big),  \\ z(t) &= \dot{y}(t), \\
	{y}(t_0)&=y_0, \\ z(t_0)&=z_0\in Q(t_0).
  \end{aligned}
\end{equation}
Here $A(t), B(t), C(t) \in \mathbb R^{(n\times n)}$ are given matrices, wherein $\det A(t) \equiv 0$, $t \in T=[t_0,t_1]$ is a given time interval; $x(t), y(t) \in {\mathbb R}^n$ are unknown functions; $Q(t)=x(t)-Z$ is a ``moving set'', where $Z$ is a given closed convex subset of ${\mathbb R}^n$, and $\mathcal{N}_{Q}\big(q\big)$ denotes the normal cone to a closed convex set $Q$ at a point $q$.

By a solution of (\ref{pss1}), (\ref{pss2}) we mean a triple of absolutely continuous ($W^{1,1}$) functions $(x(t), y(t), z(t))$ satisfying (\ref{pss1}), (\ref{pss2}) for almost all (a.e.) $t \in T$ with respect to (w.r.t.) the usual Lebesgue measure.


The analysis is carried out under the assumptions of the existence of a structural form with separated ``differential'' and ``algebraic'' subsystems. This structural form is equivalent to the initial system in a sense of solutions, and the operator which transformes the DAE into the structural form possesses the left inverse operator. The finding of the structural form is constructive and do not use a change of variables. In addition the problem of consistency of the initial data is solved automatically.


Let's consider time-varying ODE system not resolved with respect to the derivative
\begin{equation}\label{dau}
A(t)x'(t)=B(t)x(t) + U(t)f(t), \;\;\; t \in T \subseteq {\mathbb R},
\end{equation}
where $A(t), B(t), U(t) \in \mathbb R^{(n\times n)}$ are given matrices, wherein $\det A(t) \equiv 0$, $f(t) \in {\mathbb R}^n$ is some continuous on $T$ function; $x(t) \in {\mathbb R}^n$ is unknown function of a state. Such systems are called differential-algebraic equations (DAE).

%Under some assumptions \cite{SCH2008} there exists a linear differential operator
\begin{lemma}{\rm \cite{SCH2008}}
Let $A(t), B(t) \in \mathbb C^{r+1}(T), U(t) \in \mathbb C^r(T),$ there is a resolving minor in matrix $\mathcal{D}_r(t)$ and ${\rm rank} \;\mathcal{D}_r(t) = {\rm rank} \; \mathcal{D}_{r-1}(t) + n \;\; \forall t \in T$. Then there exists linear differential operator
$$
{\mathcal R}=R_0(t) +R_1(t){{d}\over{dt}}+\ldots +R_r(t)\left( {{d}\over{dt}}\right) ^r
$$
with continuous on $T$ coefficients
$R_j(t)\; (j=\overline{0,r})$, which reduces system (\ref{dau}) into the form
$$
\dot{x}_1(t) = J_1(t)x_1(t)+{\mathcal H}(t) \overline{f}(t),
$$
$$
x_2(t) = J_2(t)x_1(t)+{\mathcal G}(t) \overline{f}(t),
$$
where $\left( x_1(t), x_2(t) \right)=S^{-1}x(t)$, $\overline{f}(t) = (f(t), \dot{f}(t),\ldots, f^{(r)}(t))$, r is unresolvability index for DAE (\ref{dau}). Matrices $\mathcal{D}_r(t), J_1(t), J_2(t)$, ${\mathcal H}(t), {\mathcal G}(t)$ are determined by formulas cited in \cite{SCH2008}.
\end{lemma}




The following statements were obtained as the main results.
\begin{theorem}
Let $A(t),B(t) \in \mathbb C^{2}(T); C(t), y(t) \in \mathbb C^{1}(T)$, there is a resolving minor in matrix ${\mathcal D}_1(t)$ and ${\rm rank} \; {\mathcal D}_{1}(t)={\rm rank} \; {\mathcal D}_{0}(t)+n \;\;  \forall t \in T.$

Then problem (\ref{pss1}), (\ref{pss2}) solvable on $T$ iff
	\begin{equation}\label{pss55}
		x_{2,0} = J_2(t_0)x_{1,0} + G_0(t_0)y_0 + G_1(t_0)z_0.
	\end{equation}
Here $(x_{1,0},x_{2,0}) = S^{-1}x_0$. Moreover, if a solution to problem (\ref{pss1}), (\ref{pss2}) exists, then it is unique.
\end{theorem}

\begin{theorem}
Let all the assumptions of Theorem 1 are satisfied.

	Denote as:\\
	1) ${\rm rank} \; (G_0(t) \; G_1(t)) = d \;\; \forall t \in T$;\\
	2) $\exists{\sigma} \in T: {\rm rank} \; \mathcal{S}(\sigma)= n-d$;\\
	3) ${\rm rank}\; \mathcal{S}(t)=n-d$ for almost all $t \in T$.

	System (\ref{pss1}), (\ref{pss2}) is $R$-controllable on $T$ if condition 2) is satisfied.

    System (\ref{pss1}), (\ref{pss2}) is completely controllable on $T$ if conditions 1), 2) are satisfied.

	System (\ref{pss1}), (\ref{pss2}) is differentialy controllable on $T$ if conditions 1), 3) are satisfied.


	Here
	$$\mathcal{S}(t) = (S_0(t) \;\; S_1(t) \;\; \ldots \; S_{n-d-1}(t)),$$
	$$S_0(t)= - (H_0(t) \; H_1(t)), \;\; S_i(t) = - J_1(t)S_{i-1}(t) +  S'_{i-1}(t), \;\; i=\overline{1,n-d-1}.
	$$
\end{theorem}



\begin{thebibliography}{9} % or {99}, if there is more than ten references.

\bibitem{KM2000} Kunze~M., Marques~M.~D.~M. An Introduction to Moreau's Sweeping Process. Impacts in Mechanical Systems. Lecture Notes in Physics. Springer, Berlin, Heidelberg, 2000. Vol.~551. https://doi.org/10.1007/3-540-45501-9\_1

\bibitem{Moreau1977} Moreau J.-J. Evolution problem associated with a moving convex set in a Hilbert space. J. Differential Eq. 1977. Vol.~26. P.~347--374. https://doi.org/10.1016/0022-0396(77)90085-7

\bibitem{BS1996} Brokate M., Sprekels J. Hysteresis and Phase Transitions. Ser. Appl. Math. Sci. Springer-Verlag, New York. 1996. Vol.~121. https://doi.org/10.1007/978-1-4612-4048-8

\bibitem{Kr1991} Krej\u{c}\'{\i} P. Vector hysteresis models // European J. Appl. Math. 1996. Vol.~2. P.~281--292. https://doi.org/10.1017/S0956792500000541

\bibitem{PSS2020} Petrenko P., Samsonyuk O., Staritsyn M. A note on Differential-Algebraic Systems with Impulsive and Hysteresis Phenomena // Cybernetics  and Physics. 2020. Vol.~9, №~1. P.~51--56. https://doi.org/10.35470/2226-4116-2020-9-1-51-56

\bibitem{SCH2008} Scheglova A.A. Controllability of nonlinear algebraic differential systems. Automat. Telemekh. 2008. Vol.~10. Pp.~57--80. [In Russian]



\end{thebibliography}
%\end{document}
