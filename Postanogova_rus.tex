\begin{englishtitle}
\title{On invertibility of the operator for derivative in a~neutral type differential equation}
% First author
\author{Irina Postanogova}
\institute{Perm National Research Polytechnic University, Perm, Russia\\
  \and
Perm State University, Perm, Russia\\
\email{ipostanogova@psu.ru}}
% etc

\maketitle

\begin{abstract}
We study the spectrum of the operator for derivative in a neutral type equation and the location of roots of the characteristic equation on the complex plane. The general form of the spectrum is found.
It is established that there exists an exact right boundary of the characteristic equation roots.
 The conditions under which all the roots lie to the left of the imaginary axis and are separated from it are identified.

\keywords{functional equations, difference equations, operator spectrum, incommensurable quantities}
\end{abstract}
\end{englishtitle}

\iffalse
\documentclass[12pt]{llncs}
\usepackage[T2A]{fontenc}
\usepackage[utf8]{inputenc}
\usepackage[english,russian]{babel}
\usepackage[russian]{nla}

%\usepackage[english,russian]{nla}

% \graphicspath{{pics/}} %Set the subfolder with figures (png, pdf).

%\usepackage{showframe}
\begin{document}
%\selectlanguage{russian}
\fi

\title{Об обратимости оператора при производной для дифференциального уравнения нейтрального типа}
% Первый автор
\author{И.~Ю.~Постаногова
 } % обязательное поле
\institute{Пермский национальный исследовательский политехнический университет (ПНИПУ), Пермь, Россия\\
  \and
    Пермский государственный национальный исследовательский университет (ПГНИУ), Пермь, Россия\\
  \email{ipostangova@psu.ru}
}


\maketitle

\begin{abstract}
Исследуется спектр оператора при производной в уравнении нейтрального типа, а также расположние корней его характеристического уравнения на комплексной плоскости. Найден общий вид спектра, установлено, что существует точная правая граница корней характеристического уравнения и найдены условия, при которых все корни лежат слева от мнимой оси и отделены от неё.

\keywords{функциональные уравнения, разностные уравнения, спектр оператора, несоизмеримые величины}
\end{abstract}

%\section{Основные результаты} % не обязательное поле

Рассмотрим автономное функционально-дифференциальное уравнение нейтрального типа, записанное в операторном виде [1]:
\begin{equation} \label{eq_MainFDE}
    (I-S)\dot{x}(t)=Tx(t)+g(t),\;t\ge0,
\end{equation}
где  \(S=aS_1+bS_{\alpha}\), \(S_1, S_{\alpha}\) -- операторы сдвига на соответствующие числа, \(T\) -- линейный ограниченный оператор, заданный интегралом Римана~-- Стилтьеса, функция \(g: \mathbb{R}_+\to\mathbb{R}\) суммируема на каждом конечном отрезке. Как известно [1], в этих предположениях уравнение (\ref{eq_MainFDE}) с заданными начальными условиями однозначно разрешимо в классе абсолютно непрерывных на каждом конечном отрезке функций. Будем предполагать, что \(\alpha>1\) и \(\alpha\notin\mathbb{Q}\); заметим, что по сравнению с хорошо изученным случаем соизмеримых запаздываний иррациональность \(\alpha\) существенно усложняет исследование.\par
При изучении асимптотических свойств решений уравнения (\ref{eq_MainFDE}) важную роль играет обратимость оператора \(I-S\) в лебеговых пространствах заданных на полуоси функций [2]. Рассмотрим функциональное уравнение
\begin{equation*}
    (I-S)y(t)=f(t),\; t\ge0,
\end{equation*}
и поставим ему в соответствие \emph{характеристическое уравнение}
\begin{equation} \label{eq_CharEq}
    1-ae^{-p}-be^{-\alpha p}=0,\;p\in\mathbb{C}.
\end{equation}\par
В случае иррационального \(\alpha\) на комплексной плоскости существует прямая \(\mathrm{Re}\,{z}=\xi_0\), такая, что все корни характеристического уравнения (\ref{eq_CharEq}) лежат слева от неё и, помимо этого, являющаяся точной границей корней. Это значит, что существует последовательность \(\left\{p_k=\xi_k+i\eta_k\right\}\) корней уравнения (\ref{eq_CharEq}), для которой \(\displaystyle\lim_{k\to\infty}{\xi_k}=\xi_0\), а \(\displaystyle\lim_{k\to\infty}{\left|\eta_k\right|}=\infty\). Показано, что на этой прямой может лежать не более одного корня характеристического уравнения и найдены условия, при которых корней на этой прямой нет.

\textbf{Теорема 1.} Спектр оператора \(S\) представляет собой круг
\begin{equation*}
\sigma\left(S\right)=\left\{\lambda\in\mathbb{C}\mid \left|\lambda\right|\le\left|a\right|+\left|b\right|\right\}.
\end{equation*}

Приведённые ниже теоремы обобщают и уточняют результаты работ [2, 3].

\textbf{Теорема 2.} Следующие утверждения эквивалентны:
\begin{enumerate}
    \item \(|a|+|b|<1;\)
    \item оператор \(I-S\) обратим в \(L_p\) при любом \(p\ge1\);
    \item все корни уравнения (\ref{eq_CharEq}) лежат слева от мнимой оси и отделены от неё (\(\xi_0<0\))
    \item уравнение (\ref{eq_MainFDE}) экспоненциально устойчиво.
\end{enumerate}

\textbf{Теорема 3.} Если \(|a|+|b|>1\), то
\begin{enumerate} 
    \item оператор \(I-S\) не обратим в \(L_p\) при любом \(p\ge1\);
    \item справа от мнимой оси существует бесконечно много корней уравнения (\ref{eq_CharEq}) (\(\xi_0>0\));
    \item уравнение (\ref{eq_MainFDE}) неустойчиво.
\end{enumerate}

\textbf{Теорема 4.} Если \(\left|a\right|+\left|b\right|=1\), то
\begin{enumerate} 
    \item оператор \(I-S\) не обратим в \(L_p\) при любом \(p\ge1\);
    \item все корни уравнения (\ref{eq_CharEq}) лежат слева от мнимой оси либо на ней, и существует вертикальная цепь корней, бесконечно приближающаяся к мнимой оси (\(\xi_0=0\)).
\end{enumerate}

% Список литературы.
\begin{thebibliography}{99}
\bibitem{1}
Азбелев~Н.~В., Максимов~В.~П., Рахматуллина~Л.~Ф. Введение в теорию функ\-ци\-о\-наль\-но-дифференциальных уравнений. М.: Наука, 1991. %- 280~с.
\bibitem{2} Чистяков~А.~В. К вопросу о множестве корней квазиполинома с несоизмеримыми показателями~// Краевые задачи: межвуз. сб. науч. тр. Перм. политехн. ин-т. 1986. С.~32--36.
\bibitem{3} Баландин~А.~С. Экспоненциальная устойчивость автономных дифференциальных уравнений нейтрального типа.~ I~// Изв. вузов. Матем. 2023. \textnumero~3. C.~3--14.
\end{thebibliography}

% Обязательная информация на английском языке:




%\end{document}


