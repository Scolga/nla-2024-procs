
\iffalse
%%%%%%%%%%%%%%%%%%%%%%%%%%%%%%%%%%%%%%%%%%%%%%%%%%%%%%%%%%%%%%%%%%%%%%%%
%
% This is the template file for the 6th International conference
% NONLINEAR ANALYSIS AND EXTREMAL PROBLEMS
% June 25-30, 2018
% Irkutsk, Russia
%
%%%%%%%%%%%%%%%%%%%%%%%%%%%%%%%%%%%%%%%%%%%%%%%%%%%%%%%%%%%%%%%%%%%%%%%%
% The preparation of the article is based on the standard llncs class
% (Lecture Notes in Computer Sciences), which is adjusted with style
% file of the conference.
%
% There are two ways of compilation of the file into PDF
% 1. Use pdfLaTeX (pdflatex), (LaTeX+DVIPS will not work);
% 2. Use LuaLaTeX (XeLaTeX will work too).
% When using LuaLaTeX You will need TTF or OTF CMU fonts
% (Computer Modern Unicode). The fonts are installed with 'cm-unicode' package in
% a distribution of LaTeX % (https://www.ctan.org/tex-archive/fonts/cm-unicode),
% either by downloading and installing these fonts system wide, the address of their page is
% http://canopus.iacp.dvo.ru/%7Epanov/cm-unicode/
% The second option won't work in XeLaTeX.
%
% For MiKTeX (LaTeX distribution for Windows),
%  1. Package 'cm-unicode' is installed manually with the MiKTeX administration Console.
%  2. For the compilation of this example, namely, the stub figure, one will also need to
% download package 'pgf' manually. This package uses in the popular
% package tikz.
%  3. Tests showed that the rest of the required packages MiKTeX loads automatically (if
%     it is allowed). The 'auto download' option is
%     configured in 'Settings' section in MiKTeX Console.
%
%
% The easiest way to compile an article is to use pdfLaTeX, but
% the final layout of the book will be compiled with LuaLaTeX,
% as a result will be of better quality thanks to the package 'microtype' and
% use vector OTF instead of standard raster fonts of pdfLaTeX.
%
% In the case of questions and problems with the article compilation,
% write letters to e-mail: eugeneai@irnok.net, Cherkashin Evgeny.
%
% New version of the correcting style file will be available at the website:
%     https://github.com/eugeneai/nla-style
%     file - nla.sty
%
% Further instructions are in the text body of the template. The template itself
% is an article example.
%
% The LaTeX2e format is used!

% 12 points font size is used.
\documentclass[12pt]{llncs}

% The correcting style file is added.
\usepackage{todonotes}

\usepackage{nla} % This package is needed for compiling
                 % this template, it should be removed
                 % from your article.

% Many popular packages (amsXXX, graphicx, etc.) are already imported in the style file.
% If there is a conflict with your packages, try disabling them and compile
% the text.
%
% It would be convenient in the layout of the proceedings if the file names
% of the figures of different authors do not clash.
% To minimize the clash, the drawings can be placed in a separate subfolder
% named after the author or the title of the paper.
%
% \graphicspath{{ivanov-petrov-pics/}} % specifies the folder with images in png, pdf formats.
% or
% \graphicspath{{great-problem-solving-paper-pics/}}.

\begin{document}
\fi
% Text should be formatted in accordance with the 'article' class, using extensions like
% AMS.
%
\title{Sparse optimal control problem governed by cyber-physical systems using PQA method}
% First author
\author{Dongyao Yang  \and  Jinlong Yuan
}

  \institute{Department of Applied Mathematics, School of Science, Dalian Maritime University, Dalian 116026, PR China\\
  \email{yuanjinlong@dlmu.edu.cn}
}
% etc

\maketitle

\begin{abstract}
We examine a linear time-invariant system influenced by a static state feedback control mechanism. Its form can be written as $Kx(t-\tau(||K||_0))$, where $K$ represents its gain matrix and $||K||_0$ means the number of nonzero entries of the matrix $K$. And the varying delay $\tau(||K||_0)$ is arises due to the time required for the transmission of the state information and the computation of the control input. Governed by this linear time-invariant system, we minimize the conventional cost functions, which are written as $J^{0}(K)$ in this study, to obtain the optimal feedback matrices $K_1^{*}$. Nevertheless, the resulting $K_1^{*}$ matrix obtained by conventional computational methods is always dense.
The objective of this paper is to minimize $||K||_0$ while satisfying the constraint $|J^{0}(K)-J^{0}(K_1^{*})|\leq \varepsilon$.
The sparsity of the gain matrix is approximated by a piecewise quadratic approximation (PQA) model.
Then, we formulate an iterative algorithm to solve the transformed problem, and undertake a numerical experiment to illustrate its practical utility and efficacy.
\keywords{sparse optimal control, sparse feedback matrix, iterative thresholding algorithm}
\end{abstract}

% at the end of the list, there should be no final dot
%\section{The main results}


%The text of the report.

% The figures and tables are drawn according to the standard class 'article'.
%\begin{figure}[htb]
%  \centering

% Two picture formats are supported:
%\includegraphics[width=0.7\linewidth]{figure.pdf} % Raster format
%\includegraphics[width=0.7\linewidth]{figure.png} % Vector and raster format
%
% Vector drawings can be drawn in Inkscape editor
% https://inkscape.org/ru/download/
% The usual format of the editor is SVG, so the drawings must be exported in
% PDF or PNG (with a resolution of minimum 150 dpi, and maximum of 300 dpi).
%  \begin{center}
%    \missingfigure[figwidth=0.7\linewidth]{Remove me from the article!} \end{center}
%  \caption{Caption of the figure}\label{fig:example}
%\end{figure}

% At the end of the text, acknowledgments are expressed, if you haven't
% made a footnote from the title. For example, we can write
%The research is carried on with support of RFBR (RNF, other funds), project No.~00-00-00000.


\begin{thebibliography}{9} % or {99}, if there is more than ten references.

\bibitem{YuanZhaoYang2024} Jinlong Yuan, Shuang Zhao, Dongyao Yang, Chongyang Liu, Changzhi Wu, Tao Zhou, Sida Lin, Yuduo Zhang, Wanli Cheng, Koopman modeling and optimal control for microbial fed-batch fermentation with switching operators. Nonlinear Analysis: Hybrid Systems.  2024. Vol. 52. 101461.

\bibitem{YuanWuTeo2023} Jinlong Yuan, Changzhi Wu, Kok Lay Teo, Jun Xie Song Wang, Computational method for feedback perimeter control of multiregion urban traffic networks with state-dependent delays. Transportation Research Part C. 2023. Vol. 153. 104231.

\bibitem{YuanZhaoliu2023} Jinlong Yuan, Changzhi Wu, Chongyang liu,Kok Lay Teo, Jun Xie, Robust suboptimal feedback control for a fed-batch nonlinear time-delayed switched system. Journal of the Franklin Institute. 2023. Vol. 360. Pp. 1835--1869.

\bibitem{YuanWuTeo2022} Jinlong Yuan, Changzhi Wu*, Kok Lay Teo, Shuang Zhao and Lixia Meng, Perimeter Control With State-Dependent Delays: Optimal Control Model and Computational Method, IEEE Transactions on Intelligent Transportation Systems. 2022. Vol. 23, no. 11. 20614--20627, 2022.

\bibitem{YuanWangZhai2021} Jinlong Yuan, Lei Wang*, Jingang Zhai, Kok Lay Teo, Changjun Yu, Ming Huang, Jun Xie, Robust optimal control for a batch nonlinear enzyme-catalytic switched time-delayed process with noisy output measurements. Nonlinear Analysis: Hybrid Systems. 2021. Vol. 41. 101059.

\end{thebibliography}
%\end{document}
