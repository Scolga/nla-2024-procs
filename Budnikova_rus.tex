\begin{englishtitle} % Настраивает LaTeX на использование английского языка
% Этот титульный лист верстается аналогично.
\title{On the construction of two-stage multistep methods for the numerical solution of integral algebraic equations}
% First author
\author{Olga Budnikova\inst{1}   \and   Mikhail Bulatov\inst{2}
 % \and
%  Name FamilyName3\inst{1}
}
\institute{ISU, ISDCT SB RAS, Irkutsk, Russia\\
  \email{osbud@mail.ru}
  \and
ISDCT SB RAS, Irkutsk, Russia\\
\email{mvbul@icc.ru}}
% etc

\maketitle

\begin{abstract}
Systems of Volterra linear integral equations with identically singular matrices in the principal part (called integral-algebraic equations) are examined. Two-stage multistep methods for the numerical solution of a selected class of such systems are proposed.

\keywords{Volterra integral equations, block methods, multistep algorithms, integral algebraic equations} % в конце списка точка не ставится
\end{abstract}
\end{englishtitle}


\iffalse
\documentclass[12pt]{llncs}

% При использовании pdfLaTeX добавляется стандартный набор русификации babel.

\usepackage{iftex}

\ifPDFTeX
\usepackage[T2A]{fontenc}
\usepackage[utf8]{inputenc} % Кодировка utf-8, cp1251 и т.д.
\usepackage[english,russian]{babel}
\fi

\usepackage{todonotes}

\usepackage[russian]{nla}



\begin{document}
\fi

\title{О построении двухстадийных многошаговых методов для численного решения интегро алгебраических уравнений\thanks{Исследование выполнено за счет гранта Российского научного фонда \textnumero~22-11-00173, https://rscf.ru/project/22-11-00173/}}
% Первый автор
\author{О.~С.~Будникова\inst{2} \and   М.~В.~Булатов\inst{2}
%  \and
% Третий автор
%  И.~О.~Фамилия\inst{1}
} % обязательное поле

% Аффилиации пишутся в следующей форме, соединяя каждый институт при помощи \and.
\institute{ИГУ, ИДСТУ СО РАН, Иркутк, Россия\\
  \email{osbud@mail.ru}
  \and   % Разделяет институты и присваивает им номера по порядку.
ИДСТУ СО РАН, Иркутк, Россия\\
\email{mvbul@icc.ru}
% \and Другие авторы...
}

\maketitle

\begin{abstract}
Доклад посвящен разработке класса численных методов для решения вырожденных систем взаимосвязанных интегральных уравнений типа Вольтеррра.

\keywords{интегральные уравнения Вольтерра, блочные методы, многошаговые алгоритмы, интегро алгебраические уравнения} % в конце списка точка не ставится
\end{abstract}

%\section{Основные результаты} % не обязательное поле

В работе рассматриваются вырожденные системы взаимосвязанных интегральных уравнений Вольтерра первого и второго рода, для которых зарубежом устоялся термин <<интегро алгебраическое уравнение>>:
\begin{equation}
\label{eq_BudnikovaOS_1}
A(t)x(t)+\int\limits^{t}_{0}{K(t,s)x(s)ds}=f(t),
\end{equation}
здесь
\begin{equation*}
\det A(t) \equiv 0,
\end{equation*}
где $A(t),$ $K(t,s)$ -- матрицы размерности $(n \times n),$ $f(t)$ и $x(t)$ -- $n$-мерные известная и искомая вектор-функции.

Для численного решения задачи (\ref{eq_BudnikovaOS_1}) предложено строить двухстадийные многошаговые методы.  В докладе будут представлены условия на весовые коэффициенты, при которых разработанные алгоритмы являются устойчивыми. Результаты численных расчетов тестовых примеров приведены для иллюстрации теоретических положений и выявления достигнутого порядка точности.

Представляемые результаты являются естественным продолжением исследований отраженных в \cite{ct_BudnikovaOS_1}.



\begin{thebibliography}{9} % или {99}, если ссылок больше десяти.
%Статья на английском:
\bibitem{ct_BudnikovaOS_1} Будникова~О.С., Булатов~М.В. Численное решение интегро-алгебраических уравнений многошаговыми методами~// Журнал вычислительной математики и математической физики. 2012. Т.~52. \textnumero~5. С.~829--839.


\end{thebibliography}

% После библиографического списка в русскоязычных статьях необходимо оформить
% англоязычный заголовок и аннотацию.



%\end{document}