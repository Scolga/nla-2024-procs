
\iffalse
\documentclass[12pt]{llncs}

\usepackage{todonotes}

\usepackage{nla} 

\begin{document}
\fi

\title{Spectral theory and asymptotic behavior of solutions for differential-algebraic equations\thanks{The research is supported by NAFOSTED, project No.~101.02-2021.43.}}

\author{Vu Hoang Linh
  }
\institute{VNU University of Science, Hanoi, Vietnam\\
  \email{linhvh@vnu.edu.vn}
  }

\maketitle

\begin{abstract}
The spectral theory (Lyapunov and Bohl exponents) for linear time-varying
ordinary differential equations (ODEs) and differential-algebraic equations (DAEs)
can be considered as generalizations of eigenvalue problems for
constant matrices and matrix pencils. The spectral theory can be used for characterizing
asymptotic behavior and stability of solutions. One of the most interesting problems 
is the investigation of Lyapunov/Bohl exponents of solutions when the system is
under small linear/nonlinear perturbations. These questions led to the stability of 
exponents and Perron theorems. In the last decade, some results were
extended from ODEs to DAEs and from the time-invariant case to the time-varying
one. First, we give a brief overview on the existing results. Next, we discuss some open problems 
in the operator setting that includes infinite dimensional systems modeled by partial differential-
algebraic (PDAEs) or delay-differential-algebraic equations (DDAEs).
\keywords{Differential-algebraic equations, Lyapunov exponents, Bohl exponents, stability of exponents}
\end{abstract}


%\end{document}
