
\iffalse
\documentclass[12pt]{llncs}

\usepackage{todonotes}

\usepackage{nla}

\begin{document}
\fi

\title{Global Limit Cycle Bifurcations in Predator--Prey\\ Models of Ecological and Biomedical Systems}

\author{Valery Gaiko
}
\institute{United~Institute~of~Informatics~Problems, National~Academy~of~Sciences~of~Belarus,
Minsk, Belarus\\
\email{valery.gaiko@gmail.com}}

\maketitle

\begin{abstract}
We carry out a global bifurcation analysis in predator--prey models of ecological and biomedical systems. In particular, applying methods of Catastrophe Theory, we prove that the Leslie--Gower
dynamical system with the Allee effect can have at most two limit cycles surrounding one singular
point.

\keywords{Leslie--Gower dynamical system, Allee effect, Catastrophe Theory, Wintner--Perko termination principle, bifurcation, limit cycle}
\end{abstract}

\section{Introduction}

We carry out a global bifurcation analysis in predator--prey models of ecological and biomedical
systems [1-5]. In particular, we complete the global qualitative
analysis of a predator-prey system derived from the Leslie--Gower type model, where the most
common mathematical form to express the Allee effect in the prey growth function is considered;
see \cite{gaiko5}. The basis for analyzing the dynamics of such complex ecological or biomedical
systems is the interactions between two species, particularly the dynamical relationship between
predators and their prey. 

To~control all limit cycle bifurcations in a dynamical system,
especially, bifurcations of multiple limit cycles, it~is necessary to know the properties and
combine the effects of all its rotation parameters. It can be done by means of the development
of new bifurcation geometric methods based on Perko's planar termination principle~\cite{gaiko1}.
This principle is a consequence of the principle of natural termination which was applied
by A.\,Wintner for studying one-parameter families of periodic orbits of the restricted
three-body problem to show that in the analytic case any one-parameter family of periodic
orbits can be uniquely continued through any bifurcation except a period-doubling bifurcation.
Such a bifurcation can happen, e.\,g., in a three-dimensional Lorenz system. But this cannot
happen for planar systems. That is why the Wintner--Perko termination principle is applied
for studying multiple limit cycle bifurcations of planar polynomial dynamical
systems~\cite{gaiko1}. If we do not know the cyclicity of the termination points,
then, applying canonical systems with field rotation parameters, we use geometric
properties of the spirals filling the interior and exterior domains of limit cycles.

\section{The main result}

The Leslie-Gower predator-prey model incorporating the Allee effect phenomenon on prey
is described by the Kolmogorov-type rational dynamical system~[5]:
\begin{equation}
    \begin{array}{l}
\displaystyle\dot{x}=x\left(r\left(1-\frac{x}{K}\right)(x-m)-qy\right)
\qquad \qquad \mbox{(prey)},
    \\[4mm]
\displaystyle\dot{y}=sy\left(1-\frac{y}{nx}\right)
\qquad\qquad\qquad\qquad\qquad \mbox{(predator)},
    \end{array}
\end{equation}
where the parameters have the following biological meanings: $r$ and $s$ represent the intrinsic
prey and predator growth rates, respectively; $K$ is the prey environment carrying capacity;
$m$ is the Allee threshold or minimum of viable population; $q$ is the maximal per capita
consumption rate, i.\,e., the maximum number of prey that can be eaten by a predator in each
time unit; $n$ is a measure of food quality that the prey provides for conversion into
predator births.

System (1) can be written in the form of a quartic dynamical system~\cite{gaiko5}:
\begin{equation}
    \begin{array}{l}
\dot{x}=x^{2}((1-x)(x-m)-\alpha\,y)\equiv P,
    \\[2mm]
\dot{y}=y\,(\beta\,x-\gamma\,y)\equiv Q.
    \end{array}
\end{equation}
Together with (2), we will also consider an auxiliary system
(see \cite{gaiko1})
    \begin{equation}
\dot{x}=P-\delta Q, \qquad \dot{y}=Q+\delta P,
    \end{equation}
applying to these systems new bifurcation methods and geometric approaches developed
in \cite{gaiko1}\,--\!\cite{gaiko5}, and completing the qualitative analysis of (1).

It is valid the following theorem.
	\par
  \medskip
\noindent \textbf{Theorem.}
\emph{The Leslie--Gower system with the Allee effect~$(2)$ can have at most
two limit cycles surrounding one singular point.}

\section{Conclusion}
In this work, we have completed the global bifurcation analysis of the Leslie--Gower system
with the Allee effect which models the dynamics of the populations of predators and their prey
in a given ecological or biomedical system. Studying global bifurcations of limit cycles, we have
proved that such a~system can have at most two limit cycles surrounding one singular point.
    \par
The mathematical tools used in this work may also be helpful in the qualitative analysis
of any two-dimensional model of species interaction in a~biological system, in particular,
in the contexts of conservation and biological control. Another line of research could be
directed, for instance, towards studying the interaction of the Allee effect with random environmental conditions such as alien species invasions or other catastrophic events,
which may increase the amplitude of population fluctuations and even drive a~population
to extinction; see \cite{gaiko5} and the references therein.

\begin{thebibliography}{9}
\bibitem{gaiko1}
Gaiko~V.\,A. Global Bifurcation Theory and Hilbert's Sixteenth Problem. Boston, Kluwer
Academic Publishers, 2003.

\bibitem{gaiko2}
Broer~H.W. Gaiko~V.\,A. Global qualitative analysis of a quartic ecological model.
Nonlinear Anal. 2010. Vol.~72. Pp.~628--634.

\bibitem{gaiko3}
Gaiko~V.\,A. Global qualitative analysis of a~Holling-type system. Int. J.~Dyn. Syst.
Differ. Equ. 2016. Vol.~6. Pp.~161--172.

\bibitem{gaiko4}
Gaiko~V.\,A. Global bifurcation analysis of a rational Holling system. Comp. Res. Model.
2017. Vol.~9, Pp.~537--545 (in Russian).

\bibitem{gaiko5}
Gaiko~V.\,A., Vuik~C. Global dynamics in the Leslie--Gower model with the Allee effect.
Int. J.~Bifurcation Chaos. 2018. Vol.~28. Pp.~1850151.
\end{thebibliography}
%\end{document}
