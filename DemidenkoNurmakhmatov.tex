

\iffalse
\documentclass[12pt]{llncs}

\usepackage{todonotes}

\usepackage{nla} 

\begin{document}
\fi

\title{An energy estimate for one pseudohyperbolic equation\thanks{The research is supported
by the Russian Science Foundation (grant no.~24-21-00370),
https://rscf.ru/project/24-21-00370/.}}

\author{Gennadii V. Demidenko   \and  Vakhobbiddin S. Nurmakhmatov
  }
\institute{Novosibirsk State University, Novosibirsk, Russia\\
  \email{g.demidenko@g.nsu.ru},
  \email{v.s.nurmuhammad@gmail.com}}

\maketitle

\begin{abstract}
We establish an energy estimate for one pseudohyperbolic operator
of the fourth order with variable coefficients.

\keywords{pseudohyperbolic operator, weighted Sobolev spaces, energy estimates}
\end{abstract}

We consider the differential operator of the fourth order
of the following form
$$
{\cal L}(x;D_t,D_x) = (\sum\limits_{|\beta|=2} a^0_{\beta}(x) D^{\beta}_x - a I) D^2_t 
+ \sum\limits_{|\beta|=3} a^1_{\beta}(x) D^{\beta}_x D_t 
+ \sum\limits_{|\beta|\le 4} a^2_{\beta}(x) D^{\beta}_x, \quad a > 0,  
$$
where the coefficients are real-valued sufficiently smooth functions,
which are constant outside of some ball.
We suppose that the operator
${\cal L}(x_0;D_t,D_x)$
at any fixed point
$x_0$  
is strictly pseudohyperbolic (see~[1]). 

{\bf Theorem.}
{\it Let
$$
\sum\limits_{|\beta|=2} a^0_{\beta}(x) \xi^{\beta}  
\ge q |\xi|^2, \quad \xi \in {\Bbb R}^n, \quad q > 0.
$$
Then there exists
$\gamma_0 > 0$
such that for any function
$u(t,x) \in W^{2,4}_{2,\gamma}({\Bbb R}^{n+1})$,
$\gamma > \gamma_0$,
such that
$$
D^2_tD^\beta_x u(t,x) \in L_{2,\gamma}({\Bbb R}^{n+1}), \quad |\beta| = 2, 
$$
the estimate holds
$$
\gamma\|(|\xi|^2 + a)\left(|\eta| + \gamma + |\xi|\right) 
\widehat u_\gamma(\eta,\xi), L_2({\Bbb R}^{n+1})\| 
\leq c \|{\cal L}(x;D_t,D_x)u(t,x), L_{2,\gamma}({\Bbb R}^{n+1})\| 
\eqno(1)
$$
with a constant
$c > 0$
independent of
$u(t,x)$.
}

Estimate~(1) is an analogue of the energy estimate for strictly
hyperbolic operators~[2, 3]. 

The obtained energy estimate can be used for studying
the correctness of the Cauchy problem for strictly
pseudohyperbolic equations
$$
\begin{array}{ll}
{\cal L}(x;D_t,D_x) u = f(t,x), \quad t>0, \ x\in {\Bbb R}^n,\\
\\
u|_{t=0} = \varphi_1(x), \quad D_t u|_{t=0} = \varphi_2(x),  
\end{array}
$$
in weighted Sobolev space
$W^{2,4}_{2,\gamma}({\Bbb R}^{n+1}_+)$
(see~[1, 4]). 

Note that pseudohyperbolic equations arise when studying some problems
of hydrodynamics, elasticity theory, when constructing waveguides, etc.
(see, for example,~[5--7]).

\begin{thebibliography}{9} % or {99}, if there is more than ten references.

\bibitem{DeUs_2003}
Demidenko G. V., Uspenskii S. V. 
Partial Differential Equations and Systems Not Solvable with Respect to the Highest-Order Derivative. 
Nauchnaya Kniga, Novosibirsk, 1998 [in Russian]; English transl. Marcel Dekker, New York and Basel. 2003.

\bibitem{Pe_1996}
Petrowsky I. G.
Selected Works. Part~1: Systems of Partial Differential Equations
and Algebraic Geometry.
Gordon and Breach, Amsterdam, 1996.

\bibitem{Le_1953}
Leray J.
Hyperbolic Differential Equations.
Institute for Advanced Study, Princeton, 1953.

\bibitem{De_2015}
Demidenko G. V.
Solvability conditions of the Cauchy problem
for pseudohyperbolic equations. Sib. Math. J.
2015. Vol.~56, no.~6. Pp.~1028--1041.

\bibitem{Vl_1961}
Vlasov V. Z.
Thin-Walled Elastic Beams. 
National Science Foundation, Washington, D.C., 1961.

\bibitem{GeEr_2014}
Gerasimov S. I., Erofeev V. I.
Problems of Wave Dynamics for Structural Elements. 
Ross. Fed. Yad. Tsentr-Vseross. Nauchn.-Issled. Inst. Eksp. Fiz.,
Sarov, 2014.~[In Russian]

\bibitem{Bi_1952} 
Bishop R. E. D.
Longitudinal waves in beams.
Aeronautical Quarterly.
1952. Vol.~3, no.~4. Pp.~280--293. 

\end{thebibliography}

%\end{document} 