\begin{englishtitle} % Настраивает LaTeX на использование английского языка
% Этот титульный лист верстается аналогично.
\title{Dualism of theories of solitonic solutions of infinite-dimensional dynamic systems and functional- differential equations of pointwise type}
% First author
\author{Levon Beklaryan\inst{1} \and  Armen Beklaryan\inst{2}
}
\institute{Central Economics and Mathematics Institute of Russian Academy of Sciences, Moscow, Russia\\
  \email{lbeklaryan@outlook.com}
  \and
HSE University, Moscow, Russia\\
\email{abeklaryan@hse.ru}}
% etc

\maketitle

\begin{abstract}
The formalism which central element is existence of one-to-one correcpondence  between solitonic solutions of an infinite-dimensional dynamic system and solutions of the induced functional-differential equation of pointwise type is presented. Within such approach the task about longitudinal fluctuations of the infinite absolutely elastic rod described  by finite  difference analog of the wave equation is studied. For such system the family of bounded solitonic solutions is described.

\keywords{solitonic solutions, functional-differential equations, wave equation} % в конце списка точка не ставится
\end{abstract}
\end{englishtitle}


\title{Дуализм теорий  солитонных решений бесконечномерных динамических систем и функционально-дифференциальных уравнений точечного типа\thanks{Исследование выполнено за счет гранта Российского научного фонда (проект \textnumero~23-11-00080).}}
% Первый автор
\author{Л.~А.~Бекларян\inst{1} \and А.~Л.~Бекларян\inst{2}
}

% Аффилиации пишутся в следующей форме, соединяя каждый институт при помощи \and.
\institute{Центральный Экономико-Математический Институт РАН, Москва, Россия \\
  \email{lbeklaryan@outlook.com}
  \and   % Разделяет институты и присваивает им номера по порядку.
Национальный исследовательский университет Высшая школа экономики, Москва, Россия\\
  \email{abeklaryan@hse.ru}
% \and Другие авторы...
}

\maketitle

\begin{abstract}
 Представлен  формализм, центральным элементом которого является существование взаимно однозначного соответствия между солитонными решениями бесконечномерной динамической системы  и решениями индуцированного функцио\-наль\-но-диф\-фе\-рен\-циаль\-но\-го уравнения точечного типа. В рамках такого подхода изучена задача о продольных колебаниях бесконечного абсолютно упругого стержня, описываемого конечно разностным аналогом волнового уравнения. Для такой системы описано семейство ограниченных солитонных решений.

\keywords{солитонные решения, функционально-дифференциальные уравнения, волновое уравнение} % в конце списка точка не ставится
\end{abstract}

\section{Основные результаты} % не обязательное поле

Совокупность конструкций $\big(\Upsilon,d,s,\eta,G_\Gamma|Q,g\big)$ называется {\it солитонным букетом} и однозначно определяется набором $\Gamma=(\Upsilon,d,s,\eta,Q,g)$, где:
\begin{itemize}
\item[(1)] конечно порожденная группа $\Upsilon$ с образующими $\{\check{\gamma_1},\ldots,\check{\gamma_d}\}$ и выделенными элементами $\{\gamma_1,\ldots,\gamma_s\}$, а также соответствующее пространство $\mathcal K^n_\Upsilon=\overline{\prod}_{\gamma\in \Gamma} R^n_\gamma$, $R^n_\gamma=\mathbb R^n$ бесконечных последовательностей $\varkappa=\{x_\gamma\}_{\gamma\in \Gamma}$, $x_\gamma\in \mathbb R^n$, $x_\gamma=(x_\gamma^1,\ldots, x_\gamma^n)^{'}$ со стандартной топологией полного прямого произведения;
\item[(2)] конечно порожденная группа сдвигов $\mathbb T_\Upsilon=\{T_\gamma: \gamma \in \Upsilon\}$, действующая в пространстве $\mathcal K^n_\Upsilon$ по следующему правилу
    $$
    T_{\bar{\gamma}}\{x_\gamma\}_{\gamma \in \Upsilon }=\{x_{\gamma\bar{\gamma}}\}_{\gamma \in \Upsilon},\quad \bar{\gamma}\in \Upsilon,\quad \{x_\gamma\}_{\gamma \in \Upsilon }\in \mathcal K^n_\Upsilon,\quad T_{\bar{\gamma}} \in \mathbb T_\Upsilon;
    $$
\item[(3)] эпиморфизм $\eta: \Upsilon \rightarrow Q$, где $Q$ группа диффеоморфизмов прямой, сохраняющих ориентацию, и, соответственно, $Q$ конечно порожденная группа с образующими $\check{q_j}=\eta(\check{\gamma_j})$, $j=1,\ldots,d$ и выделенными элементами $q_j=\eta({\gamma_j})$, $j=1,\ldots,s$;
\item[(4)] функция $g:\mathbb R\times \mathbb R^{ns} \rightarrow \mathbb R^{n}$ измеримая по $t\in \mathbb R$ при каждом $x_1,\ldots,x_s\in \mathbb R$ и при почти каждом $t\in \mathbb R$ непрерывная по $x_1,\ldots,x_s\in \mathbb R^n$ (условия Каратеодори);
\item[(5)] оператор
$$
G_{\Gamma}: \mathbb R\times \mathcal K^n_\Upsilon \rightarrow \mathcal K^n_\Upsilon,\quad \Gamma=(\Upsilon,d,s,\eta,Q,g)
$$
такой, что координата $\big(G_{\Gamma}(t,\varkappa)\big)_e$ бесконечномерной вектор-функции $G_{\Gamma}(t,\varkappa)$, соответствующая единичному элементу $e$ группы $\Upsilon$, зависит только лишь от конечного числа координат и равна
$$
\big(G_{\Gamma}(t,\varkappa)\big)_e=g\big(t,x_{{\gamma}_1},\ldots,x_{{\gamma}_s}\big);
$$
\item[(6)] при почти каждом $t \in \mathbb R $ выполняются ``почти перестановочные'' соотношения
$$
T_{\bar\gamma} G_{\Gamma}(t,\varkappa)=\frac{d}{dt}\eta(\bar\gamma)(t)\cdot G_{\Gamma}(\eta(\bar\gamma)(t),T_{\bar\gamma}\varkappa), \forall \varkappa \in \mathcal  K^n_\Upsilon, \forall \bar\gamma\in \Upsilon.
$$
\end{itemize}

Для солитонного букета $\big(\Upsilon,d,s,\eta,G_\Gamma| Q,g\big)$ с $\Gamma=(\Upsilon,d,s,\eta,Q,g)$ в фазовом пространстве ${\mathcal K}^n_\Upsilon$ с фазовой переменной $\varkappa\in{\mathcal K}^n_\Upsilon$ определим систему
$$
\dot{\varkappa}(t)=G_\Gamma(t,\varkappa),\quad  \text{для п.в. } t\in \mathbb R,\eqno{(1)}
$$
$$
\varkappa(\eta(\bar{\gamma})(t))=T_{\bar{\gamma}}\varkappa(t),\quad \forall t\in \mathbb R, \quad \forall\bar{\gamma}\in \Upsilon,\eqno{(2)}
$$
где производная в бесконечномерном ОДУ (1) понимается  как {\it производная по Гато}, а нелокальные ограничения (2) означают, что для решений системы {\it сдвиг по пространству равен сдвигу по времени}. Решения такой системы называются {\it решениями типа бегущей волны (солитонные решения)}, а группа $Q$ называется {\it характеристикой бегущей волны}.

В паре с системой (1)--(2) рассматривается функ\-ци\-о\-наль\-но-диф\-фе\-рен\-ци\-аль\-ное уравнение
$$
\dot{x}(t)=g(t,x(q_1(t),\ldots,q_s(t)),\quad t\in \mathbb R. \eqno{(3)}
$$

Каждый солитонный букет $\big(\Upsilon,d,s,\eta,G_\Gamma| Q,g\big)$ с $\Gamma=(\Upsilon,d,s, \eta,Q,g)$ определяет дуальную пару $\big(G_\Gamma| Q,g\big)$ {\it функция-оператор}. Для каждого солитонного букета (дуальной пары) существует {\it канонический солитонный букет (каноническая дуальная пара)} вида $\big(Q,d,s,\mathcal I,G_\Gamma| Q,g\big)$ с $\Gamma=(Q,d,s,\mathcal I,Q,g)$ ($\big(G_\Gamma| Q,g\big)$), где $\mathcal I$ тождественный автоморфизм группы $Q$. Для заданных $Q,g$ канонический букет выделяется наиболее простой структурой набора $\Gamma=(Q,d,s,\mathcal I,Q,g) $ и, соответственно, оператора $G_\Gamma$.

Представленное исследование демонстрирует  фрагмент некоторого общего подхода. В рамках такого подхода разработан формализм \cite{Beklaryan1}, центральным элементом которого является существование взаимно однозначного соответствия между солитонными решениями бесконечномерной динамической системы (1) (решениями системы (1)--(2)) и решениями  функцио\-наль\-но-диф\-фе\-рен\-циаль\-но\-го уравнения точечного типа (3).

В теории пластической деформации изучается бесконечномерная динамическая система
$$
m \ddot{y}_i = y_{i-1} - 2y_i + y_{i+1} +\phi(y_i), \quad i \in \mathbb Z, \quad y_i\in \mathbb R,\quad t\in \mathbb R, \eqno{(4)}
$$
где потенциал $\phi(\cdot)$, в частности, задается гладкой периодической функцией. Уравнение (4) является системой с потенциалом Френ\-ке\-ля-Кон\-то\-ровой \cite{Frenkel}. Такая система является конечно разностным аналогом нелинейного волнового уравнения, моделирует поведение счетного числа шаров массы $m$, помещенных в целочисленных точках числовой прямой, где каждая пара соседних шаров соединена между собой упругой пружиной, и описывает распространение продольных волн в бесконечном однородном абсолютно упругом стержне. Наиболее важный класс волн описывается решениями типа бегущих волн (солитонные решения).
Для представленного конечно разностного аналога волнового уравнения с нелинейным потенциалом общего вида (4) ключевым является также и наличие ряда дополнительных симметрий.
Для такой системы установлено существование семейства ограниченных солитонных решений \cite{Beklaryan2,Beklaryan3}.


\begin{thebibliography}{9} % или {99}, если ссылок больше десяти.
\bibitem{Frenkel} Френкель Я.И., Конторова Я.И. О теории пластической деформации и двойственности~// ЖЭТФ.  1938.  Т.~8.  С.~89--97.
\bibitem{Beklaryan1} Бекларян Л.А. Введение в теорию функ\-ци\-о\-наль\-но-диф\-фе\-рен\-ци\-о\-наль\-ных уравнений. Групповой подход.  М.~: Факториал Пресс, 2007.  286~с.
\bibitem{Beklaryan2} Бекларян Л.А., Бекларян А.Л.  Вопрос существования ограниченных солитонных решений в задаче о продольных колебаниях упругого бесконечного стержня в поле с сильно нелинейным потенциалом~// Ж. вычисл. матем. и матем. физ. 2021. Т.~61, \textnumero~12. С.~2024--2039.
\bibitem{Beklaryan3} Бекларян Л.А., Бекларян А.Л. Вопрос существования ограниченных солитонных решений в задаче о продольных колебаниях упругого бесконечного стержня в поле с нелинейным потенциалом общего вида~// Ж. вычисл. матем. и матем. физ.  2022.  Т.~62, \textnumero~6. С.~933--950.
\end{thebibliography}

% После библиографического списка в русскоязычных статьях необходимо оформить
% англоязычный заголовок.




%\end{document}
