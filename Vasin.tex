


\iffalse
%%%%%%%%%%%%%%%%%%%%%%%%%%%%%%%%%%%%%%%%%%%%%%%%%%%%%%%%%%%%%%%%%%%%%%%%
%
% This is the template file for the 6th International conference
% NONLINEAR ANALYSIS AND EXTREMAL PROBLEMS
% June 25-30, 2018
% Irkutsk, Russia
%
%%%%%%%%%%%%%%%%%%%%%%%%%%%%%%%%%%%%%%%%%%%%%%%%%%%%%%%%%%%%%%%%%%%%%%%%
% The preparation of the article is based on the standard llncs class
% (Lecture Notes in Computer Sciences), which is adjusted with style
% file of the conference.
%
% There are two ways of compilation of the file into PDF
% 1. Use pdfLaTeX (pdflatex), (LaTeX+DVIPS will not work);
% 2. Use LuaLaTeX (XeLaTeX will work too).
% When using LuaLaTeX You will need TTF or OTF CMU fonts
% (Computer Modern Unicode). The fonts are installed with 'cm-unicode' package in
% a distribution of LaTeX % (https://www.ctan.org/tex-archive/fonts/cm-unicode),
% either by downloading and installing these fonts system wide, the address of their page is
% http://canopus.iacp.dvo.ru/%7Epanov/cm-unicode/
% The second option won't work in XeLaTeX.
%
% For MiKTeX (LaTeX distribution for Windows),
%  1. Package 'cm-unicode' is installed manually with the MiKTeX administration Console.
%  2. For the compilation of this example, namely, the stub figure, one will also need to
% download package 'pgf' manually. This package uses in the popular
% package tikz.
%  3. Tests showed that the rest of the required packages MiKTeX loads automatically (if
%     it is allowed). The 'auto download' option is
%     configured in 'Settings' section in MiKTeX Console.
%
%
% The easiest way to compile an article is to use pdfLaTeX, but
% the final layout of the book will be compiled with LuaLaTeX,
% as a result will be of better quality thanks to the package 'microtype' and
% use vector OTF instead of standard raster fonts of pdfLaTeX.
%
% In the case of questions and problems with the article compilation,
% write letters to e-mail: eugeneai@irnok.net, Cherkashin Evgeny.
%
% New version of the correcting style file will be available at the website:
%     https://github.com/eugeneai/nla-style
%     file - nla.sty
%
% Further instructions are in the text body of the template. The template itself
% is an article example.
%
% The LaTeX2e format is used!

% 12 points font size is used.
\documentclass[12pt]{llncs}

% The correcting style file is added.
\usepackage{todonotes}

\usepackage{nla} % This package is needed for compiling
                 % this template, it should be removed
                 % from your article.

% Many popular packages (amsXXX, graphicx, etc.) are already imported in the style file.
% If there is a conflict with your packages, try disabling them and compile
% the text.
%
% It would be convenient in the layout of the proceedings if the file names
% of the figures of different authors do not clash.
% To minimize the clash, the drawings can be placed in a separate subfolder
% named after the author or the title of the paper.
%
% \graphicspath{{ivanov-petrov-pics/}} % specifies the folder with images in png, pdf formats.
% or
% \graphicspath{{great-problem-solving-paper-pics/}}.

\begin{document}
\fi
% Text should be formatted in accordance with the 'article' class, using extensions like
% AMS.
%
\title{On identity for dyadic coverings of closed sets with zero measure\thanks{The research is supported by RNF, project No.~23-11-00171.}}
% First author
\author{Andrei Vasin 
}
\institute{Admiral Makarov State University of Maritime and Inland Shipping, St-Petersburg, Russia\\
  \email{vasinav@gumrf.ru}
  }


\maketitle

\begin{abstract}
We relate a dyadic covering of a closed  set  with zero measure in Euclidean space and a  dyadic covering of its complement. We apply an identity proved for certain porous and weakly porous sets

\keywords{ entropy, Dyn'kin inequality, Borichev-Nicolau-Thomas characteristic}
\end{abstract}

% at the end of the list, there should be no final dot
\section{Background}


A set $E$ of the unit circle $\mathbb{T} $ is called porous if there exists $C$ such that for any arc  $I\subset\mathbb{T}$ there
is  a subarc $M(I)\subset I$ containing no points of $E$ and satisfying
\[
|M(I)|> C | I|.
\]
  Dyn'kin proved  uniform  discrete entropy characteristization of the porous set \cite{D}:
\begin{equation}\label{e_pri}
 \sum_{J \subset I,\;J\cap E=\emptyset}|J| \log \frac{1}{|J|}\leq |I|\left( \log \frac{1}{|I|} + C\right)
\end{equation}
where $C$ is not depending on $I$. 



The porosity condition  and estimate (\ref{e_pri})  appear naturally
in the free interpolation problems for different classes of analytic functions
smooth up to the boundary of the unit disc $\mathbb{D} \subset \mathbb{R}^2$ (see \cite{D}).

Another characterization of porosity with application  for the  weak embedding property  of singular inner functions  on the unit circle $\mathbb{T}$  was found  by Borichev, Nicolau and Thomas in \cite[Lemma7]{BNT}: a set $E$ is porous if and only if there exists $C$ such that  for  any arc $I \subset \mathbb{T}$ one has
\begin{equation}\label{e_prdya}
\sum_{J \cap E \neq \emptyset,\; J \in \mathcal{D}(I)}|J|\leq C |I|,
\end{equation}
 where $\mathcal{D}(I)$ is a dyadic decomposition  of $I$.


So, in  the Dyn'kin inequality, the description of a porous set involves arcs free from points of the set, while in the Borichev-Nicolau-Thomas characteristization,  arcs intersecting this set are used.
Our main goal in this paper is to extend   the related characterization  
on the higher dimensional case. 
In fact, we will show that the  discrete  Dyn'kin inequality (\ref{e_pri}) coincides, in a sense, with   the
Borichev-Nicolau-Thomas property (\ref{e_prdya}).

\section{Main results}
We deal  with a certain generalization, and consider  general closed sets of zero Lebesgue measure  in $\mathbb{R}^d$.
The dyadic formulation is convenient from the point of view of our
proofs. 
The dyadic decomposition of a cube $R\in \mathbb{R}^d $ is
\[ \mathcal{D}(R)= \bigcup_{j\geq 0} \mathcal{D}_j(R)\]
where each $\mathcal{D}_j(R)$ consists of the $2^{jd}$  pairwise disjoint (half-open) cubes $Q$, with side length
$\ell(Q) = 2^{-j}\ell(R)$, such that
\[R= \bigcup_{Q \in \mathcal{D}_j(R)} Q\]
for every $j =0,1, \dots $.

\begin{theorem}\label{p_id}
  Let $E \in \mathbb{R}^d$ be a closed set with $|E|=0$, and let $R \in \mathbb{R}^d$ be a cube.

  Define two families of dyadic cubes.
      The first one   $\mathcal{D}(R,E)=\{Q \in \mathcal{D}(R): \:  Q \cap E \neq \emptyset\}$ is a family of all dyadic cubes intersecting $E$.
    The second family
  $\mathcal{F}(R,E)$  consists of all  pairwise disjoint dyadic cubes $Q' \in \mathcal{D}(R)$  such that
  $Q'\cap E=\emptyset$, but $\pi Q'\cap E \neq \emptyset$. Then
   $Q'\cap E=\emptyset$, but $\pi Q'\cap E \neq \emptyset$. Then
 \begin{equation}\label{e_id}
 \sum_{Q \in \mathcal{D}(E,R)}|Q|
 = \sum_ {Q'\in \mathcal{F}(R,E) }|Q'|\log \frac{|R|}{2^d|Q'|}
\end{equation}
\end{theorem}

\begin{remark}
  The sums in (\ref{e_id}) are bounded for  a set  $E$  of finite entropy over $R$:
  \[\int_{R}\log\frac{1}{dist(x, E)}dx < \infty,\]
  otherwise, both the sums diverge to  infinity.
\end{remark}
We apply  Theorem \ref{p_id}  two precise characterization of    weakly porous sets.

% At the end of the text, acknowledgments are expressed, if you haven't
% made a footnote from the title. For example, we can write


\begin{thebibliography}{9} % or {99}, if there is more than ten references.


\bibitem{BNT} 
Borichev, A., Nicolau, A.,  Thomas, P. J., Weak embedding property, inner functions and entropy,
 Math. Ann.~2017.Vol. ~368(3). Pp.~ 987--1015.
 
\bibitem{D}
Dyn'kin, E.M. Free interpolation sets for H\"{o}lder classes. Mat. Sb.  ~1979. Vol.~109. Pp.~ 107--128  [In Russian]
[English translation in Math. USSR Sb.~1980. Vol.~37. Pp.~ 97-117.




\end{thebibliography}
%\end{document}

%%% Local Variables:
%%% mode: latex
%%% TeX-master: t
%%% End:
