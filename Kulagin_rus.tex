\begin{englishtitle} % Настраивает LaTeX на использование английского языка
% Этот титульный лист верстается аналогично.
\title{Semiclassical solutions to the nonlocal nonlinear Schr\"{o}dinger equation with an anti-Hermitian term associated with the dynamics of quasiparticles}
% First author
\author{Anton Kulagin\inst{1,2}  \and  Alexander Shapovalov\inst{3}
}
\institute{Tomsk Polytechnic University, Tomsk, Russia\\
  \email{aek8@tpu.ru}
  \and
Institute of Atmospheric Optics SB RAS, Tomsk, Russia\\
\and
Tomsk State University, Tomsk, Russia\\
  \email{shpv@mail.tsu.ru}
}
% etc

\maketitle

\begin{abstract}
The method of constructing asymptotic solutions to the Schr\"{o}dinger equation with a nonlocal nonlinearity and an anti-Hermitian term within the semiclassical approximation. The solutions under consideration are localized in a neighbourhood of few trajectories associated with classical mechanics of quasiparticles. The general formalism is applied to the specific one-dimensional model equation for the Bose--Einstein condensate.


\keywords{quasiparticles, open quantum system, semiclassical approximation, Maslov complex germ method} % в конце списка точка не ставится
\end{abstract}
\end{englishtitle}


\iffalse
%%%%%%%%%%%%%%%%%%%%%%%%%%%%%%%%%%%%%%%%%%%%%%%%%%%%%%%%%%%%%%%%%%%%%%%%
%
%  This is the template file for the 6th International conference
%  NONLINEAR ANALYSIS AND EXTREMAL PROBLEMS
%  June 25-30, 2018
%  Irkutsk, Russia
%
%%%%%%%%%%%%%%%%%%%%%%%%%%%%%%%%%%%%%%%%%%%%%%%%%%%%%%%%%%%%%%%%%%%%%%%%


\documentclass[12pt]{llncs}

% При использовании pdfLaTeX добавляется стандартный набор русификации babel.

\usepackage{iftex}

\ifPDFTeX
\usepackage[T2A]{fontenc}
\usepackage[utf8]{inputenc} % Кодировка utf-8, cp1251 и т.д.
\usepackage[english,russian]{babel}
\fi

\usepackage{todonotes}

\usepackage[russian]{nla}

\begin{document}
\fi

\title{Квазиклассические решения нелокального нелинейного уравнения Шредингера с антиэрмитовой частью, ассоциированные с динамикой квазичастиц\thanks{Исследование выполнено за счет гранта Российского научного фонда \textnumero~23-71-01047, https://rscf.ru/project/23-71-01047/.}}
% Первый автор
\author{А.~Е.~Кулагин\inst{1,2}  \and  А.~В.~Шаповалов\inst{3}
}

% Аффилиации пишутся в следующей форме, соединяя каждый институт при помощи \and.
\institute{Томский политехнический университет, г. Томск, Россия \\
  \email{aek8@tpu.ru}
  \and   % Разделяет институты и присваивает им номера по порядку.
  Институт оптики атмосферы СО РАН, г. Томск, Россия \\
  \and   % Разделяет институты и присваивает им номера по порядку.
Томский государственный университет, г. Томск, Россия\\
  \email{shpv@mail.tsu.ru}
% \and Другие авторы...
}

\maketitle

\begin{abstract}
Предлагается метод построения асимптотических решений уравнения Шредингера с нелокальной нелинейностью и антиэрмитовой частью в квазиклассическом приближении. Рассматриваемые решения локализованы в окрестности нескольких траекторий, ассоциированных с классической механикой квазичастиц. Общий формализм применен к конкретному одномерному модельному уравнению для бозе-эйнштейновского конденсата.

\keywords{квазичастицы, открытая квантовая система, квазиклассическое приближение, метод комплексного ростка Маслова} % в конце списка точка не ставится
\end{abstract}

%\section{Основные результаты} % не обязательное поле

Метод квазиклассически сосредоточенных состояний (МКСС), основанный на методе комплексного ростка Маслова \cite{mas94}, позволяет строить решения нелокального нелинейного уравнения Шредингера (ННУШ), которые локализованы в окрестности траектории в фазовом пространстве \cite{bel02}. Эта работа посвящена более сложной проблеме построения решений, которые локализованы в окрестности нескольких траекторий, а не одной. Такие решения ассоциированы с "классической механикой"\, нескольких квазичастиц, каждая из которых имеет свою траекторию и взаимодействие которых друг с другом диктуется ННУШ.

Рассматривается задача Коши для следующего ННУШ с антиэрмитовой частью:
\begin{equation}
\begin{array}{l}
\bigg\{ -i\hbar\partial_t + H(\hat{z},t)[\Psi]-i\hbar \Lambda \breve{H}(\hat{z},t)[\Psi] \bigg\}\Psi(\vec{x},t)=0, \quad \Psi(\vec{x},0)=\psi(\vec{x}),\cr
H(\hat{z},t)[\Psi]=V(\hat{z},t)+\kappa\displaystyle\int\limits_{{\mathbb{R}}^n}d\vec{y}\,\Psi^{*}(\vec{y},t)W(\hat{z},\hat{w},t)\Psi(\vec{y},t), \cr
\breve{H}(\hat{z},t)[\Psi]=\breve{V}(\hat{z},t)+\kappa\displaystyle\int\limits_{{\mathbb{R}}^n}d\vec{y}\,\Psi^{*}(\vec{y},t)\breve{W}(\hat{z},\hat{w},t)\Psi(\vec{y},t),
\end{array}
\label{eq1}
\end{equation}
где $\vec{x},\vec{y}\in{\mathbb{R}}^n$, $\hat{\vec{p}}=-i\hbar\partial_x$, $\hat{\vec{p}}_y=-i\hbar\partial_y$, $\hat{z}=(\hat{\vec{p}},\vec{x})$, $\hat{w}=(\hat{\vec{p}}_y,\vec{y})$ и $\hbar$ -- формальный малый параметр квазиклассического приближения. Операторы $V(\hat{z},t)$, $\breve{V}(\hat{z},t)$, $W(\hat{z},\hat{w},t)$ и $\breve{W}(\hat{z},\hat{w},t)$ -- псевдодифференциальные операторы с гладкими символами.

Расширение МКСС на случай ННУШ с анти-эрмитовой частью было предложено в работе \cite{kul24}. Здесь мы обобщаем этот формализм на асимптотические решения уравнения \eqref{eq1}, ассоциированные с динамической системой обыкновенных дифференциальных уравнений, описывающих квазиклассическую динамику квазичастиц.

С помощью предложенного формализма построены решения задачи Коши одномерного уравнения вида
\begin{equation}
\bigg\{ \displaystyle\frac{-i\hbar\partial_t}{1-i\hbar\Lambda} + \displaystyle\frac{1}{2}\hat{p}^2+\epsilon\cos x +\kappa\displaystyle\int\limits_{-\infty}^{\infty} \displaystyle\frac{c^2}{\big((x-y)^2+c^2\big)^{3/2}}|\Psi(y,t)|^2dy \bigg\}\Psi(x,t)=0,
\label{primereq1}
\end{equation}
описывающего бозе-эйнштейновский конденсат в периодической потенциальной ловушке с диполь-дипольным взаимодействием и феноменологическим затуханием. В рамках квазиклассического приближения показано, что для одиночной квазичастицы в потенциальной яме, возникает коллапс решения $\Psi(\vec{x},t)$, в то время как при добавлении в соседнюю потенциальную яму второй такой же квазичастицы коллапс пропадает даже при слабом взаимодействии между квазичастицами, что свидетельствует о солитоноподобном поведении ансамбля взаимодействующих квазичастиц.


% В конце текста можно выразить благодарности, если этого не было
% сделано в ссылке с заголовка статьи, например,
%



\begin{thebibliography}{9} % или {99}, если ссылок больше десяти.

\bibitem{mas94} Maslov~V.P. The Complex WKB Method for Nonlinear Equations. I. Linear Theory~/ Birkhauser Verlag, Basel,~1994.

\bibitem{bel02} Belov~V.V., Trifonov~A.Yu., Shapovalov~A.V. The trajectory-coherent approximation and the system of moments for the Hartree type equation~// Int. J. of Mathematics and Mathematical Sciences. 2002. Vol.~32. Pp.~325--370.

\bibitem{kul24} Kulagin~A.E., Shapovalov~A.V. Semiclassical approach to the nonlocal nonlinear Schrodinger equation with a non-Hermitian term~// Mathematics. 2024. Vol.~12. 580.


\end{thebibliography}


%}





%\end{document}

