
\iffalse
\documentclass[12pt]{llncs}

\usepackage{todonotes}

\usepackage{nla} 

\begin{document}
\fi

\title{On boundary value problems with nonsmooth solutions and nonlinear boundary conditions\thanks{The research is supported by the Russian Science Foundation, project No.~22-71-10008.}}


\author{Margarita Zvereva\inst{1} \and  Mikhail Kamenskii\inst{2}
 }
\institute{Voronezh State University, Voronezh, Russia\\
  \email{margz@rambler.ru}
  \and
Voronezh State Pedagogical University, Voronezh, Russia\\
\email{mikhailkamenski@mail.ru}}

\maketitle

\begin{abstract}

We study boundary value problems with  nonlinear boundary conditions. Theorems for the existence and uniqueness of solutions are proved, the qualitative properties of solutions are studied and formulas for representing solutions in explicit form are presented.



\keywords{Stieltjes integral, functions of bounded variation, nonlinear boundary condition}   
\end{abstract}


\section{The main results} %optional section

We study  boundary value problems with nonsmooth solutions and nonlinear boundary conditions. Such problems model
 deformations  of elastic systems having localized interactions
with the environment. We assume that  the movement of one  of  the ends of the investigated physical system  is limited by  a bounded, closed, convex set $C$.  Depending on the applied external force,  this   end
either remains an internal point of $C$ or touches the boundary of
$C$. This leads to a nonlinear boundary condition at the
corresponding point.

The deviation of the investigated physical system  from the equilibrium position is described by an integro-differential equation with the Stieltjes integral, that makes it possible to take into account singularities localized at separate  points (concentrated forces, elastic supports) and to carry out point-by-point analysis. 
 The corresponding equation is an analogue of the Euler equation and together with boundary conditions it is obtained by the variational method from the problem of the energy functional minimizing.
 
We obtain the necessary and sufficient
conditions for the extremum of the energy functional, prove the
existence and uniqueness theorems, find  explicit
formulas for  solutions, study the qualitative properties of solutions and  dependence of solutions on the size of
the set $C$.

Moreover, models of oscillations of string systems are studied under the assumption that the limiter can move in the perpendicular direction to the plane where the investigated  physical system is located in the equilibrium position. For such problems, existence and uniqueness theorems for solutions are proved,  an analogue of d'Alembert's formula is obtained, problems of boundary control of oscillations are solved.


\begin{thebibliography}{9} % or {99}, if there is more than ten references.


\bibitem{Kamenskii1}  Kamenskii M., Wen Ch.-F., Zvereva M.  On a variational problem for a model of a Stieltjes string with a backlash at the end. Optimization. 2020. Vol.~69, no~9. Pp.~1935--1959.

\bibitem{Kamenskii2}  Kamenskii M., Liou Y.Ch., Wen Ch.-F., Zvereva M.  On a hyperbolic equation on a geometric graph with hysteresis type boundary conditions. Optimization. 2020. Vol.~69, no~2. Pp.~283--304.

\bibitem{Kamenskii3}  Kamenskii M., Voskovskaya N., Zvereva M.  On periodic oscillations of some points of a string with a nonlinear boundary condition.  Applied Set-Valued Analysis and Optimization. 2020. Vol.~2, no~1. Pp.~35--48.
 
\bibitem{Kamenskii4}  Kamenskii M., Raynaud de Fitte P., Wong N.-Ch., Zvereva M.  A model of deformations of  a discontinuous Stieltjes string with a nonlinear boundary condition. Journal of Nonlinear and Variational Analysis. 2021. Vol.~5, no~5. Pp.~737--759.

\bibitem{Zvereva1} Zvereva M., Wen Ch.-F., Kamenskii M., Raynaud de Fitte P.  A model of deformations of a beam with nonlinear boundary conditions. Journal of Nonlinear and Variational Analysis. 2022. Vol.~6, no~3. Pp.~279--298.

\bibitem{Zvereva2} Zvereva M. A two-dimensional model of string deformations with a nonlinear boundary condition.Journal of Nonlinear and Convex Analysis. 2022. Vol.~23, no~12. Pp.~2775--2793.

\bibitem{Zvereva3} Zvereva M., Kamenskii M., Raynaud de Fitte P., Wen Ch.-F.  The deformations problem for the Stieltjes strings system with a nonlinear condition. Journal of Nonlinear and Variational Analysis. 2023. Vol.~7, no~2. Pp.~291--308.

\bibitem{Zvereva4} Zvereva M. The problem of two-dimensional string vibrations with a nonlinear condition. Differential Equations. 2023. Vol.~59.  Pp.~1050--1060.

\end{thebibliography}
%\end{document}
