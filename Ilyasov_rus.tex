\begin{englishtitle} % Настраивает LaTeX на использование английского языка
% Этот титульный лист верстается аналогично.
\title{Finding bifurcations by extended nonlinear Rayleigh quotient method}
% First author
\author{Y.~Sh.~Il'yasov }
\institute{Institute of Mathematics with Computer Center of UFRC RAS, Russia\\
  \email{ilyasov02@gmail.com}}
% etc

\maketitle

\begin{abstract}
We discuss finding bifurcations of solutions to nonlinear equation systems. We present a method based on the use of new type variational functionals, whose critical points correspond to solutions and bifurcation points of equations. Through this method, saddle-node bifurcations for systems of equations can be found as well as variational formulas for their direct determination. 
The method will be demonstrated for solving nonlinear partial differential equations, systems of equations with finite-difference approximations, and algebraic-trigonometric equations that arise during the calculation of the maximum voltage power in electrical circuits (blackout problem).).
\keywords{nonlinear problems, variational method, bifurcation, Rayleigh quotient} % в конце списка точка не ставится
\end{abstract}
\end{englishtitle}


\iffalse
%%%%%%%%%%%%%%%%%%%%%%%%%%%%%%%%%%%%%%%%%%%%%%%%%%%%%%%%%%%%%%%%%%%%%%%%
%
%  This is the template file for the 6th International conference
%  NONLINEAR ANALYSIS AND EXTREMAL PROBLEMS
%  June 25-30, 2018
%  Irkutsk, Russia
%
%%%%%%%%%%%%%%%%%%%%%%%%%%%%%%%%%%%%%%%%%%%%%%%%%%%%%%%%%%%%%%%%%%%%%%%%


\documentclass[12pt]{llncs}  

% При использовании pdfLaTeX добавляется стандартный набор русификации babel.

\usepackage{iftex}

\ifPDFTeX
\usepackage[T2A]{fontenc}
\usepackage[utf8]{inputenc} % Кодировка utf-8, cp1251 и т.д.
\usepackage[english,russian]{babel}
\fi

\usepackage{todonotes} 

\usepackage[russian]{nla}

\begin{document}
\fi

\title{Нахождение бифуркаций по методу продолженных нелинейных отношения Рэлея}
% Первый автор
\author{Я.~Ш.~Ильясов  }

% Аффилиации пишутся в следующей форме, соединяя каждый институт при помощи \and.
\institute{Институт математики с ВЦ УФИЦ\ РАН, г. Уфа \\
  \email{ilyasov02@gmail.com}
}

\maketitle

\begin{abstract}

В докладе обсуждается нахождение бифуркаций ветвей решений систем нелинейных уравнений. Будет предложен метод, основанный на использовании нового типа вариационных функционалов, критические точки которых соответствуют решениям и бифуркационным точкам уравнений. Метод позволяет найти как достаточные условия существования седло-узловых бифуркаций для систем уравнений, так и вывести вариационную формулу для прямого их нахождения.  
Применение метода будет проиллюстрировано на нелинейных уравнениях в частных производных, системах уравнений конечно-разностных аппроксимаций, а также конечномерных  алгебра-тригонометрических систем уравнений, возникающих при расчете максимальных допустимых мощностей напряжений в электрических сетях  (blackout problem).

\keywords{нелинейные уравнения, вариационные методы, бифуркация, отношения Рэлея} % в конце списка точка не ставится
\end{abstract}


\begin{thebibliography}{9} % или {99}, 
	

\bibitem{3} Y. Il'yasov. \newblock Finding Saddle-Node Bifurcations via a Nonlinear Generalized Collatz–Wielandt Formula. \newblock {\em International Journal of Bifurcation and Chaos}. 31(01):2150008, 2021.
	
\bibitem{2} Y. Il'yasov, A finding of the maximal saddle-node bifurcation for systems of differential equations. Journal of Differential Equations.   2024. Vol. 378. Pp.   610--625.
	
\bibitem{Salazar} P. D. P. Salazar, Y. Il'yasov, L. F. C. Alberto,  E. C. M. Costa \& M. B. Salles.  \newblock Saddle-node bifurcations of power systems in the context of variational theory and nonsmooth optimization, \newblock {  IEEE Access} 8:110986--110993, 2020.
	
\bibitem{1} Y.S. Il'yasov,   Ivanov, A.A. 2016, Computation of maximal turning points to nonlinear  equations by nonsmooth optimization,''  {Optim. Meth. and   Softw.}. Vol.  {31}, No (1). Pp. 1--23.

\end{thebibliography}

% После библиографического списка в русскоязычных статьях необходимо оформить
% англоязычный заголовок.




%\end{document}

