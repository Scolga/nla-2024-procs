


\iffalse
\documentclass[12pt]{llncs}

\usepackage{todonotes}

\usepackage{nla}

\begin{document}
\fi

\title{Operator continued $J$-fractions: basic facts and possible applications\thanks{This work is done at SRISA according to the project FNEF-2024-0001, reg. No  1023032100070-3-1.2.1 }}

\author{Andrey Osipov 
}
\institute{Federal State Institution \\
``Scientific-Research Institute for System Analysis\\
of the Russian Academy of Sciences''\\
Nakhimovskii pr. 36-1, Moscow, 117218, \\ Russia\\
  \email{osipa68@yahoo.com}
}
\maketitle

\begin{abstract}
We introduce an operator analog of the continued $J-$ fractions. The links between this object and the spectral theory for one class of band operators are provided.

\keywords{continued fractions, band operators, operator equations}
\end{abstract}

Since the works of Chebyshev, Stieltjes, Markov and Stone, the continued $J-$ fractions have found a wide application in the study of orthogonal polynomials, rational approximations, moment problems and Jacobi operators, including the use of the latter for integration of Toda and Volterra (Langmuir) nonlinear lattices, see \cite{NS}. Here we consider
the $J-$ fractions with operator coefficients 
\begin{eqnarray*}
J_{\mathcal A} := 
(zE-B_0-(zE-B_1-(\dots)^{-1}A_1)^{-1}A_0)^{-1}:= \\
:= \frac{E}{\displaystyle zE - B_0 -
\frac{A_0}{\displaystyle
zE-B_1-\frac{A_1}{\displaystyle\frac{\qquad \ddots}{ \displaystyle zE - B_n -\frac{\displaystyle \; A_n}{\displaystyle \quad \qquad \ddots}}}}}
\end{eqnarray*}
where $A_i, B_i$ are bounded operators in a Hilbert space $H,$ $A_i$ are invertible, and $E$ is the identity operator. As in the case of scalar continued $J-$ fractions and Jacobi operators, these continued fractions are related to the band operators ${\cal A}$ generated by the infinite tridiagonal matrices

\begin{equation*}
A=\begin{pmatrix} B_{0}&E&O&O&\dots\\
A_{0}&B_{1}&E&O&\dots\\
O&A_{1}&B_{2}&E&\dots\\
\vdots&\vdots&\vdots&\vdots&\ddots\\
\end{pmatrix} \quad,
\end{equation*}
and the relation is as follows
: $J_{\mathcal A}$ is an expansion at infinity of the Weyl (operator) function of the operator $\cal A$ denoted as $\mathfrak{M}(z,{\cal A})$ \cite{osfr}. We establish a criterion under which a formal power series with operator coefficients can be expanded into $J_{\mathcal A},$ as well as obtain an expansion algorithm and the uniqueness theorem for this continued fraction. Then we study the convergence of 
$J_{\mathcal A}$ to $\mathfrak{M}(z,{\cal A})$; we find that for $|z|>r({\cal A})$ $J_{\mathcal A}$ converges to $\mathfrak{M}(z,{\cal A})$ in the operator norm, where $r({\cal A})$ is the spectral radius of the operator $\cal A$. Also we show how these results can be applied for finding the solutions of quadratic operator equations, which may  be of interest in the application context.

% At the end of the text, acknowledgments are expressed, if you haven't
% made a footnote from the title. For example, we can write

\begin{thebibliography}{9} % or {99}, if there is more than ten references.

\bibitem{NS} E. M. Nikishin, V. N. Sorokin. Rational
Approximations and Orthogonality, Translations of Mathematical
Monographs, vol. 92. AMES, Providence, R.I., 1991.
\bibitem{osfr} A.S. Osipov. On one class of continued fractions
with operator elements. Doklady Mathematics. 2001. Vol.~63, no. ~3. Pp.~383--386.
\end{thebibliography}
%\end{document}
