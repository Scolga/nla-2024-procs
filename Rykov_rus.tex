\begin{englishtitle} % Настраивает LaTeX на использование английского языка
% Этот титульный лист верстается аналогично.
\title{Variational formulation for one-dimensional systems of conservation laws}
% First author
\author{Yuri Rykov
%  \and
%  Name FamilyName2\inst{2}
%  \and
%  Name FamilyName3\inst{1}
} 
\institute{Keldysh Institute of Applied Mathematics RAS, Moscow,
Russia\\
  \email{yu-rykov@yandex.ru}
%  \and
%Affiliation, City, Country\\
%\email{email}
}
% etc
\maketitle

\begin{abstract}
The report provides alternative formulations of the concept of a
generalized solution for systems of quasi-linear conservation laws.
The system of equations is associated with some nonlinear
functional on trajectories in the space of independent variables
and the concept of a generalized solution is defined through the
extreme properties of this functional. It is also possible to give
a formulation with functional minimization in a certain function
space. The latter formulation makes it possible to interpret
solutions for Keyfitz-Kranzer type systems with delta singularity.

\keywords{conservation laws, generalized solutions, quasi-linear systems,
shock waves, delta singularities, variational formulation} % в конце списка точка не ставится
\end{abstract}
\end{englishtitle}


\iffalse
\documentclass[12pt]{llncs}

% При использовании pdfLaTeX добавляется стандартный набор русификации babel.

\usepackage{iftex}

\ifPDFTeX
\usepackage[T2A]{fontenc}
\usepackage[utf8]{inputenc} % Кодировка utf-8, cp1251 и т.д.
\usepackage[english,russian]{babel}
\fi

\usepackage{todonotes}

\usepackage[russian]{nla}

\begin{document}
\fi
\title{Вариационная формулировка для одномерных систем законов сохранения
%\thanks{Работа выполнена при поддержке РФФИ (РНФ, другие фонды), проект \textnumero~00-00-00000.}
}
% Первый автор
\author{Ю.~Г.~Рыков  % \inst ставит циферку над автором.
%  \and  % разделяет авторов, в тексте выглядит как запятая.
% Второй автор
%  И.~О.~Фамилия\inst{2}
%  \and
% Третий автор
%  И.~О.~Фамилия\inst{1}
} % обязательное поле

% Аффилиации пишутся в следующей форме, соединяя каждый институт при помощи \and.
\institute{ИПМ им. М.В. Келдыша РАН, Москва, Россия\\
  \email{yu-rykov@yandex.ru}
%  \and   % Разделяет институты и присваивает им номера по порядку.
%Институт (название в краткой форме), Город, Страна\\
%\email{email@examle.com}
% \and Другие авторы...
}

\maketitle

\begin{abstract}
В докладе приведены альтернативные формулировки понятия обобщенного
решения для систем квазилинейных законов сохранения. Система
уравнений ассоциируется с некоторым нелинейным функционалом на
траекториях в пространстве независимых переменных и понятие
обобщенного решения определяется через экстремальные свойства этого
функционала. Также возможно привести формулировку с минимизацией
функционала в некотором пространстве функций. Последняя
формулировка позволяет интерпретировать решения для систем типа
Кейфитц-Кранзера с дельта особенностью.


\keywords{законы сохранения, обобщенные решения, квазилинейные системы, ударные волны, дельта особенности,
вариационная формулировка} % в конце списка точка не ставится
\end{abstract}


Будем рассматривать задачу Коши для следующей системы квазилинейных
уравнений, которая предполагается строго гиперболической,
\begin{equation}\label{1}
\mathbf{U}_{t}+\mathbf{F}(\mathbf{U})_{x}=0\,,\quad\mathbf{U}(0,x)=\mathbf{U}_{0}(x)\,,
\end{equation}
здесь
$(t,x)\in\prod_{T}\equiv\{(t,x):(t,x)\in[0,T]\times\mathbb{R}\}$,
$\mathbf{U}(t,x)=(u_{1}(t,x),\ldots,u_{n}(t,x))$, а
$\mathbf{F}=(f_{1}\ldots,f_{n})$ -- достаточно гладкая (по крайней
мере, $\mathbf{F}\in C^{1}(\mathbb{R}^{n})$) вектор-функция
переменных $(u_{1},\ldots,u_{n})$. Здесь и далее жирным шрифтом в
формулах будут обозначаться векторные величины. Решения
системы~\ref{1}, принимающие заданные начальные значения, как
правило, понимаются в обобщенном смысле, т.е. в смысле выполнения
соответствующего интегрального тождества.

Системы уравнений типа~(\ref{1}) изучались очень интенсивно,
включая и многомерный случай, общее представление о предмете
приведено, например, в \cite{LTP}. Однако многие вопросы, связанные
с существованием обобщенного решения (кратко: о.р.), все же
остаются открытыми. Ниже приведен альтернативный взгляд на понятие
о.р. для систем типа~(\ref{1}), основанный на вариационном подходе.
Отметим, что понятие энтропийного о.р. и вопрос единственности
здесь не рассматриваются.

Рассмотрим функционал $\mathbf{J}$ такой, что
$\mathbf{J}:\chi(\tau)\in C^{1}\rightarrow\mathbb{R}^{n}$,
\begin{equation}\label{2}
\mathbf{J}\equiv\int\limits_{0}^{T}\mathbf{L}(\dot{\chi},\mathbf{U})d\tau;\quad\mathbf{L}(\dot{\chi},\mathbf{U})\equiv\mathbf{U}(\tau,\chi(\tau))
\,\dot{\chi}(\tau)-\mathbf{F}\circ\mathbf{U}(\tau,\chi(\tau))\,.
\end{equation}
ПРЕДЛОЖЕНИЕ 1. Пусть $\mathbf{U}(t,x)$ является кусочно
непрерывно-дифференцируемой, и пусть существует траектория
$\hat{\chi}$ такая, что для этой траектории $\delta\mathbf{J}=0$.
Тогда в тех точках $\hat{\chi}$, где $\mathbf{U}(t,x)$ является
гладкой, выполняется~(\ref{1}) в классическом смысле, а в точках
пересечения $\hat{\chi}$ и линий разрыва функции $\mathbf{U}(t,x)$
выполняются соотношения Ренкина-Гюгонио для~(\ref{1}).

На основе данного факта можно предложить следующее определение, см.
также \cite{R1}.

ОПРЕДЕЛЕНИЕ 1. Пусть $\mathbf{J}$ определено соотношением
~\eqref{2}. Назовем функцию $\mathbf{U}(t,x)$ из подходящего класса
функций о.р. системы~(\ref{1}), если для любого пути $\chi(\tau)$
вариация $\delta\mathbf{J}=0$, т.е. любая точка функционала
$\mathbf{J}$ является критической.

Из Предложения 1 возможно получить и другое определение, см. также
\cite{R2}.

ОПРЕДЕЛЕНИЕ 2. Рассмотрим функцию $\mathbf{V}(t,x)\in
W^{1,\infty}(\prod_{T})\equiv B$ и начальное условие
$\mathbf{V}_{0}(x)\in W^{1,\infty}(\mathbb{R})$,
$\mathbf{V}_{0}^{\prime}(x)=\mathbf{U}_{0}(x)$. Обозначим через
$\mathcal V$ подмножество $B$, такое что
$\mathbf{V}(+0,x)=\mathbf{V}_{0}(x)$. Также на пространстве $B$
рассмотрим функционал $${\mathcal L}(\mathbf{V})\equiv\underset
{t}{ess\,sup}\ \sum_{i=1}^{n}\underset{x}{Var}\left(\frac{\partial
v_{i}}{\partial t}+f_{i}\left(\frac{\partial\mathbf{V}}{\partial
x}\right)\right)\,,$$
%\begin{equation}\label{3}
%{\mathcal L}(\mathbf{V})\equiv\underset {t}{ess\,sup}\ \sum_{i=1}^{n}\underset{x}{Var}\left(\frac{\partial v_{i}}{\partial t}+
%f_{i}\left(\frac{\partial\mathbf{V}}{\partial x}\right)\right)\,,
%\end{equation}
где $\mathbf{V}=\left(v_{1},\ldots,v_{n}\right)$, а
$\underset{x}{Var}$ обозначает вариацию выражения по переменной
$x$. Тогда функцию $\mathbf{U}=\partial\mathbf{V}/\partial x$
назовем о.р.~(\ref{1}), если функция $\mathbf{V}$ реализует минимум
функционала $\mathcal L$ на множестве $\mathcal V$.

ПРЕДЛОЖЕНИЕ 2. Пусть минимум $m$ функционала $\mathcal L$
достигается. Тогда, если $m=0$, то
$\mathbf{U}=\partial\mathbf{V}/\partial x$ является о.р.~(\ref{1})
в смысле интегрального тождества.

Если же $m\neq 0$, то Определение 2 можно использовать для
интерпретации решений систем типа Кейфитц-Кранзера, см. \cite{Key}.

% Рисунки и таблицы оформляются по стандарту класса article. Например,

%\begin{figure}[htb]
%  \centering
  % Поддерживаются два формата:
  %\includegraphics[width=0.7\linewidth]{figure.pdf} % Растровый формат
  %\includegraphics[width=0.7\linewidth]{figure.png} % Векторный и растровый формат
  %
  % Векторные рисунки можно рисовать в редакторе Inkscape
  % https://inkscape.org/ru/download/
  % Основной формат этого редактора - SVG, поэтому рисунки необходимо экспортировать в
  % PDF или PNG (с разрешением - минимум 150 dpi, максимум - 300dpi).
%  \begin{center}
%    \missingfigure[figwidth=0.7\linewidth]{Уберите меня из статьи!}
%  \end{center}
%  \caption{Заголовок рисунка}\label{fig:example}
%\end{figure}

% кавычки << >>.

% В конце текста можно выразить благодарности, если этого не было
% сделано в ссылке с заголовка статьи, например,
%Работа выполнена при поддержке РФФИ (РНФ, другие фонды), проект \textnumero~00-00-00000.
%

\begin{thebibliography}{9} % или {99}, если ссылок больше десяти.
\bibitem{LTP} Liu~Tai-Ping. Shock waves. American Math. Soc.
    Providence. Rhode Island, 2021. 458 p.
\bibitem{R1} Рыков~Ю.Г. О вариационном подходе к системам
    квазилинейных законов сохранения~// Труды МИАН. 2018. Т.~301.
    С.~225--240.
\bibitem{R2} Рыков~Ю.Г. Вариационная постановка задачи поиска
    обобщенных решений для квазилинейных гиперболических систем законов
    сохранения~// Математические заметки. 2021. Т.~110:6. С.~944--947.
\bibitem{Key} Keyfitz~B.L. Singular shocks, retrospective and
    prospective~// Confl. Math. 2011. Vol.~3:3. Pp.~445-–470.
\end{thebibliography}


% После библиографического списка в русскоязычных статьях необходимо оформить
% англоязычный заголовок и аннотацию.




%\end{document}
