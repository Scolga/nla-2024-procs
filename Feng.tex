

\iffalse
\documentclass[12pt]{llncs}

% The correcting style file is added.
\usepackage{todonotes}

\usepackage{nla} % This package is needed for compiling
                 % this template, it should be removed
                 % from your article.

% Many popular packages (amsXXX, graphicx, etc.) are already imported in the style file.
% If there is a conflict with your packages, try disabling them and compile
% the text.
%
% It would be convenient in the layout of the proceedings if the file names
% of the figures of different authors do not clash.
% To minimize the clash, the drawings can be placed in a separate subfolder
% named after the author or the title of the paper.
%
% \graphicspath{{ivanov-petrov-pics/}} % specifies the folder with images in png, pdf formats.
% or
% \graphicspath{{great-problem-solving-paper-pics/}}.

\begin{document}
\fi
% Text should be formatted in accordance with the 'article' class, using extensions like
% AMS.
%
\title{A proximal bundle algorithm for solving the equilibrium problems with inexact
data\thanks{The research is supported by the National Key Research and Development Program of China, project No.2022YFB3304600 the International Postdoctoral Exchange Fellowship Program 2020 by the Office of China Postdoctoral Council, project No.2020003; the Project by China Postdoctoral Science Foundation, project No.2019M651091; the Fundamental Research Funds for the Central Universities of DMU, under No. 3132023205.}}
% First author
\author{Yong Xiu Feng \and Ming Huang \and Si Qi Zhang \and Xiao Dan Chao
  }
\institute{School of Science, Dalian Maritime University, Dalian 116026, China\\
  \email{fengyongxiu2001@dlmu.edu.cn}
}

\maketitle

\begin{abstract}
 In this paper, an approach bundle method for solving non-smooth equilibrium problems is proposed. We consider a general algorithm that converges so that the sequence generated by the algorithm converges to an approximate solution of the equilibrium problem, and then we explain the realizability of the algorithm. In this strategy, the inexact cutting plane linearization of the objective function is established by using the inexact value of the objective function and the approximate gradient of the adjacent points, which makes the subproblem easy to solve and maintains convergence. In addition, we choose a suitable descent criterion to measure the decline of the function and we introduce a stop criterion, which is satisfied after a finite number of iterations.

\keywords{equilibrium problems, proximal bundle method, non-smooth optimization, inexact data}
\end{abstract}

\section{The main results}
In this paper, we try to solve the equilibrium problem (for short, EP):
 \begin{equation}
              (EP)\qquad \text{Find}~ x^{*}\in C ,~\text{such that}~f(x^{*},y)\geq 0,~\text{for all}~y\in C,
             \nonumber
  \end{equation}
where $C$ is a nonempty convex compact subset of ${\mathbb R}^n$ and $f:C\times C\rightarrow \mathbb{R}$ is continuous differentiable, it satisfies $f(x,x)=0$ and $f(x,\cdot)$  is convex for all $x \in C$. \\
\indent Then, the inexact oracle is given as the following type: for given  $y_{k}\in C$ and $\varepsilon\geq 0$, there are\\$$\tilde {f_{k}}={f_{k}}(y_{k})-\varepsilon,$$
where ${f_{k}}(\cdot)$ denotes $f(x^{k},\cdot).$\\
\indent It is expensive to solve the subgradient of a non-smooth function, so we propose and analyze an algorithm that is capable to deal with approximate subgradients, then we take an already computed subgradient $g(\tilde {x})$ of $f$ at some $\tilde {x}$ near $x$. Then
$$
\begin{aligned}
f(x)+g(\tilde {x})^{T}(z-x) & =f(\tilde {x})+g(\tilde {x})^{T}(z-\tilde{x})+\alpha_0(x,\tilde{x})\\
 & \leq f(z)+\alpha_0(x,\tilde{x})\quad \text{for all}~z\in {\mathbb{R}^{n}}, \nonumber
\end{aligned}
$$
$\noindent\text {with}~0\leq \alpha_0{(x,\tilde{x})}:=f(x)-f(\tilde {x})-g(\tilde {x})^{T}(x-\tilde{x}).~  \text{Thus},~g(\tilde {x})\in \partial_{\alpha_0(x,\tilde{x})}f(x).$\\

\newpage\noindent\textbf{Proximal Bundle Algorithm For Solving Equilibrium Problem}\\
\noindent \textbf{Step 0.} Let an initial point $x^{0}$ be given, together with a tolerance $\mu\in(0,1)$ and a positive sequence $\{\epsilon_{k}\}_{k\in\mathbb{N} }$. Set $y_{0}^{0}=x^{0}, k=0,i=1$.\\
\noindent \textbf{Step 1.} Choose a piecewise linear convex function $\bar{{f}_{k}}^{i}$ and solve
$$
({{P}_{k}}^{i})\quad \begin{aligned}
\min_{y \in C}  \{\epsilon_{k}\bar{{f}_{k}}^{i}+ h(y)-h(x_{k})+\langle\nabla h(x^{k}),x-x^{k}\rangle\}
\end{aligned}
$$
to get a unique optimal solution $y^{i}\in C$, where $h:C\rightarrow\mathbb{R} $ is strongly convex differentiable function.\\
\textbf{Step 2.} If
$${\tilde{f}_{k}}({y_{k}}^{i})\leq\mu{\bar{f}_{k}}^{i}({y_{k}}^{i}),$$
set $x^{k+1}=y^{i}$, $y_{k+1}^{0}=x^{k+1}$ and increase $k$ by 1 and set $i=0$, otherwise, $ x_k$ is kept fixed for the next inner iteration and set $x_{k+1}=x_k$.\\
\textbf{Step 3.}  Increased $i$ by 1 and go to Step 1.\\
\noindent\textbf{Conclusion}\\
\indent The running time will be reduced by using inexact information and the algorithm will converge to an approximate solution of the equilibrium problem. It provides reliability and attractiveness in solving non-smooth and non-convex optimization problems. In the future, we will introduce numerical experiments and some examples to show that the algorithm is effective in solving non-smooth equilibrium problems.

\begin{thebibliography}{9} % or {99}, if there is more than ten references.
\bibitem{ref1} T. T. V. Nguyen, J. J. Strodiot, V. H. Nguyen. A bundle method for solving equilibrium problems. Mathematical Programming, Ser. B, 2009, 116: 529-552.
\bibitem{ref2} Hinterm¨¹ller M. A proximal bundle method based on approximate subgradients. Computational Optimization and Applications, 2001, 20: 245-266.
\bibitem{ref3} Salmon G, Strodiot J J, Nguyen V H. A bundle method for solving variational inequalities. SIAM Journal on Optimization, 2004, 14(3): 869-893.
\bibitem{ref4} M. Huang, J. L. Yuan, S. D. Lin, X. J. Liang and C. Y. Liu. On solving the convex semi-infinite minimax problems via superlinear VU incremental bundle technique with partial inexact oracle. Asia-Pacific Journal of Operational Research, 2021, 38(5): 2140015.
\bibitem{ref5} M. Huang, H. M. Niu, S. D. Lin, Z. R. Yin, J. L. Yuan. A redistributed proximal bundle method for nonsmooth nonconvex functions with inexact information. Journal of Industrial and Management Optimization, 2023, 19 (12): 8691-8708.

\end{thebibliography}
%\end{document}
