\begin{englishtitle} % Настраивает LaTeX на использование английского языка
% Этот титульный лист верстается аналогично.
\title{Longitudinal-transverse bending of a rod under the action of nonlinear distributed loads}
% First author
\author{M.~I.~Makarov\inst{1}  \and  I.~L.~Kuzmin\inst{2}}
\institute{UUST, Ufa, Russia\\
  \email{ufamax2@gmail.com}\\
 \and
UUST, Ufa, Russia\\
\email{igor\_l\_kuzmin@mail.ru}
 }
% etc

\maketitle

\begin{abstract}
Calculation of rod bending using differential bending equation and Euler bending theory under the influence of Gaussian shear forces.

\keywords{bend, distributed load, normal distribution} % в конце списка точка не ставится
\end{abstract}
\end{englishtitle}

\iffalse
%%%%%%%%%%%%%%%%%%%%%%%%%%%%%%%%%%%%%%%%%%%%%%%%%%%%%%%%%%%%%%%%%%%%%%%%
%
%  This is the template file for the 6th International conference
%  NONLINEAR ANALYSIS AND EXTREMAL PROBLEMS
%  June 25-30, 2018
%  Irkutsk, Russia
%
%%%%%%%%%%%%%%%%%%%%%%%%%%%%%%%%%%%%%%%%%%%%%%%%%%%%%%%%%%%%%%%%%%%%%%%%

\documentclass[12pt]{llncs}  

% При использовании pdfLaTeX добавляется стандартный набор русификации babel.

\usepackage{iftex}

\ifPDFTeX
\usepackage[T2A]{fontenc}
\usepackage[utf8]{inputenc} % Кодировка utf-8, cp1251 и т.д.
\usepackage[english,russian]{babel}
\fi

\usepackage{todonotes} 

\usepackage[russian]{nla}

\begin{document}
\fi

\title{Продольно-поперечный изгиб стержня при действии нелинейных распределенных нагрузок}
% Первый автор
\author{М.~И.~Макаров\inst{1}    \and   И.~Л.~Кузьмин\inst{2}
}

% Аффилиации пишутся в следующей форме, соединяя каждый институт при помощи \and.
\institute{УУНиТ, Уфа, Россия \\
  \email{ufamax2@gmail.com}\\
\and
УУНиТ, Уфа, Россия \\
\email{igor\_l\_kuzmin@mail.ru}
% \and Другие авторы...
}

\maketitle

\begin{abstract}
Расчёт изгиба стержня, используя дифференциальное уравнение изгиба и теории изгиба Эйлера, при воздействии Гауссовых перерезывающих сил.

\keywords{изгиб, распределенная нагрузка, нормальное распределение} % в конце списка точка не ставится
\end{abstract}

%\section{Основные результаты} % не обязательное поле

Рассмотрим дифференциальное уравнение
$$
EJ\frac{d^4w}{dz^4}+N\frac{d^2w}{dz^2}+c\cdot w=q(z),
$$
где $w$ -- величина прогиба, $z$ -- координата стержня, $E$ -- модуль Юнга, $J$ -- момент инерции, $N$ -- продольные силы, $c$ -- коэффициент жесткости основания с учетом площади соприкосновения, $q(z)$ -- распределенная нагрузка с нелинейной зависимостью.


 Изгиб стержня под действием поперечной нагрузки с учетом продольных сил называется продольно-поперечным изгибом. Продольно-поперечный изгиб возникает при потере Эйлеровой устойчивости, то есть при превышении критической силы. Максимальный изгиб при равномерно распределенной нагрузке происходит в середине стержня, а при воздействии нелинейных нагрузок может быть смещён.

 В работе создаём нелинейную нагрузку, используя величины нормального распределения, скалированные на величину заданной нагрузки.


% В конце текста можно выразить благодарности, если этого не было
% сделано в ссылке с заголовка статьи, например,
%Работа выполнена при поддержке РФФИ (РНФ, другие фонды), проект \textnumero~00-00-00000.
%



\begin{thebibliography}{9} % или {99}, если ссылок больше десяти.
\bibitem{Sv} Светлицкий~В.~А. Механика стержней. Ч.~1. Статика. М.:~Высшая школа, 1987.

\bibitem{Feod} Феодосьев~В.~И. Избранные задачи и вопросы по сопротивлению материалов.  М.:~Наука, 1996.


\bibitem{Birg} Биргер~И.А., Пановко~Я.Г. Прочность, устойчивость, колебания. Том~1. М.:~Машиностроение, 1968.

\bibitem{Darkov} Дарков~А.В., Шпиро~Г.С. Сопротивление материалов. М.:~Высшая школа, 1975.

\bibitem{Tim1}	Тимошенко~С.П., Гудьер~Дж. Теория упругости. М.:~Наука, 1979.

\bibitem{Tim2}	Тимошенко~С.П. Сопротивление материалов. Том~2. М.:~Наука, 1965.
\end{thebibliography}
% После библиографического списка в русскоязычных статьях необходимо оформить
% англоязычный заголовок.




%\end{document}