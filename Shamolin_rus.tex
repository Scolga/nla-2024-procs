\begin{englishtitle} % Настраивает LaTeX на использование английского языка
% Этот титульный лист верстается аналогично.
\title{Invariants of dynamical systems with dissipation}
% First author
\author{Maxim V. Shamolin 
%  \and
 % Name FamilyName2\inst{2}
  %\and
  %Name FamilyName3\inst{1}
}
\institute{Lomonosov Moscow State University, Moscow, Russian Federation\\
  \email{shamolin@rambler.ru,~shamolin.maxim@yandex.ru}
 % \and
%Affiliation, City, Country\\
%\email{email}
}
% etc

\maketitle

\begin{abstract}
New cases of integrable homogeneous dynamical systems of the fifth, seventh, ninth and any odd-order are presented, in which a system on a tangent bundle to a two-dimensional, three-dimensional, four-dimensional and even-dimensional manifold can be distinguished. In this case, the force field is divided into an internal (conservative) and an external one, which has a dissipation of different signs. The external field is introduced using some unimodular transformation and generalizes the previously considered fields. Complete sets of both the first integrals and invariant differential forms are given.

\keywords{dynamical system, integrability, dissipation, first integral with essentially singular points, invariant differential form} % в конце списка точка не ставится
\end{abstract}
\end{englishtitle}

\iffalse
\documentclass[12pt]{llncs}

% При использовании pdfLaTeX добавляется стандартный набор русификации babel.

\usepackage{iftex}

\ifPDFTeX
\usepackage[T2A]{fontenc}
\usepackage[utf8]{inputenc} % Кодировка utf-8, cp1251 и т.д.
\usepackage[english,russian]{babel}
\fi

\usepackage{todonotes}

\usepackage[russian]{nla}

\begin{document}
\fi
% Текст оформляется в соответствии с классом article, используя дополнения
% AMS.
%

\title{Инварианты динамических систем с диссипацией}
% Первый автор
\author{М.~В.~Шамолин   % \inst ставит циферку над автором.
  %\and  % разделяет авторов, в тексте выглядит как запятая.
% Второй автор
%  И.~О.~Фамилия\inst{2}
  %\and
% Третий автор
  %И.~О.~Фамилия\inst{1}
} % обязательное поле

% Аффилиации пишутся в следующей форме, соединяя каждый институт при помощи \and.
\institute{МГУ имени М. В. Ломоносова, Москва, Россия\\
  \email{shamolin@rambler.ru,~shamolin.maxim@yandex.ru}
%  \and   % Разделяет институты и присваивает им номера по порядку.
%Институт (название в краткой форме), Город, Страна\\
%\email{email@examle.com}
% \and Другие авторы...
}

\maketitle

\begin{abstract}
Представлены новые случаи интегрируемых однородных по части переменных динамических систем
пятого, седьмого, девятого и любого нечетного порядка, в которых может быть выделена система на касательном расслоении к двумерному, трехмерному, четырехмерному и четномерному многообразию. При этом силовое поле разделяется на внутреннее (консервативное) и внешнее, которое обладает диссипацией разного знака. Внешнее поле вводится с помощью некоторого унимодулярного преобразования и обобщает ранее рассмотренные поля. Приведены полные наборы как первых интегралов, так и инвариантных дифференциальных форм.


\keywords{динамическая система, интегрируемость, диссипация, первый интеграл с существенно особыми точками, инвариантная дифференциальная форма} % в конце списка точка не ставится
\end{abstract}

%\section{Основные результаты} % не обязательное поле

Нахождение достаточного количества тензорных инвариантов (не только автономных первых интегралов), как известно [1, 2, 3], облегчает исследование, а иногда позволяет точно проинтегрировать систему дифференциальных уравнений. Например, наличие инвариантной дифференциальной формы фазового объема позволяет уменьшить количество требуемых первых интегралов. Для консервативных (в частности, гамильтоновых) систем этот факт естествен, когда фазовый поток сохраняет объем с гладкой (или постоянной) плотностью. 

Сложнее (в смысле гладкости инвариантов) дело обстоит для систем, обладающих притягивающими или отталкивающими предельными множествами. Для таких систем коэффициенты искомых инвариантов должны, вообще говоря, включать функции, обладающие существенно особыми точками (см. также [4, 5, 6]). 

Наш подход состоит в том, что для точного интегрирования автономной системы порядка $m$ надо знать $m-1$ независимый тензорный инвариант. При этом для достижения точной интегрируемости приходится соблюдать также ряд дополнительных условий на эти инварианты.

Важные случаи интегрируемых систем с малым числом степеней свободы в неконсервативном поле сил уже рассматривались в работах автора [5, 7]. Настоящее исследование распространяет результаты этих работ на более широкий класс динамических систем. При этом в этих работах упор делался на нахождение достаточного количества именно первых интегралов. Но, как известно, иногда полного набора первых интегралов для систем может и не быть, зато достаточное количество инвариантных форм может быть обеспечено. 

Для систем классической механики понятия ``консервативность'', ``силовое поле'', ``диссипация'' и др. вполне естественны. 
Поскольку в данной работе изучаются динамические системы на касательном расслоении к гладкому многообразию
(пространству положений), уточним данные понятия для таких систем. 

Анализ ``в целом'' начинается с исследования приведенных уравнений геодезических, %на поверхности, 
левые части которых при правильной параметризации представляют собой записи координат ускорения движения материальной 
частицы, %по такой поверхности, 
а правые части приравнены к нулю. Соответственно, величины, которые ставятся в дальнейшем в правую часть,
можно рассматривать как некоторые обобщенные силы. Такой подход традиционен для классической механики, а теперь он
естественно распространяется на более общий случай касательного расслоения к гладкому многообразию. Последнее позволяет, в некотором смысле,
конструировать ``силовые поля''. Так, например, введя в 
систему коэффициенты, линейные по одной из координат (по одной из квазискоростей системы) 
касательного пространства, получим силовое поле с диссипацией разного знака
(в зависимости от знака самого коэффициента).

И хотя словосочетание ``диссипация разного знака'' несколько противоречиво, тем не менее, будем его употреблять. Учитывая при этом, что в математической физике диссипация ``со знаком ``плюс'' --- это рассеяние полной энергии в обычном смысле, а диссипация ``со знаком ``минус'' --- это своеобразная ``подкачка'' энергии (при этом в механике силы, обеспечивающие рассеяние энергии называются диссипативными, а силы, обеспечивающие подкачку энергии называются разгоняющими).

Консервативность для систем на касательных расслоениях можно понимать в традиционном смысле, но мы добавим к этому следующее.
Будем говорить, что система консервативна, если она обладает полным набором гладких первых интегралов, что говорит о том, что
она не 
обладает притягивающими или отталкивающими предельными множествами. Если же она последними обладает, то будем говорить, что
система в той или иной области фазового пространства обладает диссипацией какого-то знака. Как следствие этого --- обладание системы хотя бы одним первым интегралом 
(если они вообще есть) с существенно особыми точками.

В данной работе силовое поле разделяется на так называемые внутреннее и внешнее. Внутреннее поле характерно тем, что оно не меняет
консервативности системы. А внешнее может вносить в систему диссипацию разного 
знака. Заметим также, что вид внутренних силовых полей заимствован из классической 
динамики твердого тела (см. также [5]).



% Рисунки и таблицы оформляются по стандарту класса article. Например,


% Современные издательства требуют использовать кавычки-елочки << >>.

% В конце текста можно выразить благодарности, если этого не было
% сделано в ссылке с заголовка статьи, например,
%Работа выполнена при поддержке РФФИ (РНФ, другие фонды), проект \textnumero~00-00-00000.
%

% Список литературы оформляется подобно ГОСТ-2008.
% Примеры оформления находятся по этому адресу -
%     https://narfu.ru/agtu/www.agtu.ru/fad08f5ab5ca9486942a52596ba6582elit.html
%

\begin{thebibliography}{9} % или {99}, если ссылок больше десяти.
\bibitem{1}
{Poincar\'{e} H.}
Calcul des probabilit\'{e}s. Gauthier--Villars, Paris. 1912. %340 p. 

\bibitem{2} 
{Колмогоров А.Н.}
О динамических системах с интегральным инвариантом на торе 
// Доклады АН СССР. 1953. Т. 93. \textnumero 5. 763--766.

\bibitem{3} 
{Козлов В.В.}
Тензорные инварианты и интегрирование дифференциальных
уравнений // Успехи матем. наук. 2019. Т. 74. \textnumero 1(445). 117--148.

\bibitem{4} 
{Шамолин М.В.}
Об интегрируемости в трансцендентных функциях // Успехи матем. наук. 1998. Т. 53. \textnumero 3. 209--210.

\bibitem{5} 
{Шамолин М.В.}
Полный список первых интегралов динамических уравнений движения четырехмерного твердого тела в неконсервативном поле при наличии линейного демпфирования
// Доклады РАН. 2013. Т. 449. \textnumero 4. 416--419.

\bibitem{6} 
{Шамолин М.В.}
Инварианты однородных динамических систем пятого порядка с диссипацией
// Доклады РАН. Математика, информатика, процессы управления. 2023. Т. 514. \textnumero 1. 98--106.

\bibitem{7} 
{Шамолин М.В.}
Инвариантные формы объема систем с тремя степенями свободы с переменной диссипацией
// Доклады РАН. Математика, информатика, процессы управления. 2022. Т. 507. \textnumero 1. 86--92.
\end{thebibliography}

% После библиографического списка в русскоязычных статьях необходимо оформить
% англоязычный заголовок.




%\end{document}
