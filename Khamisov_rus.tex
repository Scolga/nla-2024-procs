\begin{englishtitle}
\title{Optimization of heat and power sources system operation taking into account losses at cross-flows}
% First author
\author{Oleg Khamisov \and Nadezhda Ulianova}
\institute{Melentiev Energy Systems Institute
Siberian Branch of the Russian Academy of Sciences, Irkutsk, Russia\\
\email{khamisov@isem.irk.ru, nadinmix@mail.ru}}
% etc

\maketitle

\begin{abstract}
The mathematical formulation of the problem of optimization of the operation of an energy system consisting of sources, consumers and transit nodes is considered. The total load generated by all producers should satisfy the needs of the consuming stations, taking into account losses due to overflows. The objective is to minimize the total costs incurred by the sources to provide the required load, as well as the costs of energy transportation.

\keywords{energy system, optimization, mixed integer programming, linear approximations, counting algorithm}
\end{abstract}
\end{englishtitle}

\iffalse
\documentclass[12pt]{llncs}
\usepackage[T2A]{fontenc}
\usepackage[utf8]{inputenc}
\usepackage[english,russian]{babel}
\usepackage[russian]{nla}

%\usepackage[english,russian]{nla}

% \graphicspath{{pics/}} %Set the subfolder with figures (png, pdf).

%\usepackage{showframe}
\begin{document}
%\selectlanguage{russian}
\fi

\title{Оптимизация работы системы теплоэнергетических источников с учетом потерь при перетоках}

% Первый автор
 \author{ О.~В.~Хамисов  %\inst{1}
  \and
 %Второй автор
Н.Ю. Ульянова %\inst{2}
%  \and
% Третий автор
%  И.~О.~Фамилия\inst{2}
} 
\institute{Институт систем энергетики им. Л.А. Мелентьева СО РАН, Иркутск, Россия\\
  \email{khamisov@isem.irk.ru, nadinmix@mail.ru}
  }

\maketitle

\begin{abstract}
Рассматривается математическая постановка задачи оптимизации 
 включения -- выключения оборудования на примере теплоэнергетической системы, состоящей из источников, потребителей и транзитных узлов. Предложенная постановка является продолжением работы \cite{Khamisov1} с учётом дискретной структуры. Общая нагрузка, создаваемая всеми производителями, должна удовлетворять потребности потребляющих станций с учетом потерь при перетоках. Цель состоит в минимизации общих издержек, которые несут источники для обеспечения требуемой нагрузки, а также расходов на транспортировку энергии. Издержки каждого источника включают переменную компоненту, зависящую от объема производимой энергии (представленной кубической функцией), и постоянную составляющую, связанную с содержанием источника в горячем резерве. Также учитываются периоды, когда источники находятся в режиме ожидания (не производят энергию), но все же несут издержки.  

Используя бинарные переменные для определения состояния источников, задача преобразуется в задачу частичного целочисленного программирования. Решение этой задачи строится на приближении исходной задачи к задачам линейного частичного целочисленного программирования путем замены кубической функции переменных издержек их кусочно-линейными аппроксимациями. Разработан алгоритм для решения задачи на основе этих аппроксимаций и выполнена численная реализация. Проведен ряд тестовых расчетов для проверки реализуемости предложенного метода.

\keywords{энергетическая система, оптимизация, смешанное целочисленное программирование, линейные аппроксимации, счетный алгоритм }
\end{abstract}

%\section{Основные результаты} % не обязательное поле

% Список литературы.
\begin{thebibliography}{99}
\bibitem{Khamisov1} Стенников В.А., Хамисов О.В., Стенников Н.В. Оптимизация совместной  работы источников тепловой энергии // Электрические станции. 2011. \textnumero 3. С.27--33.
% Format for Journal Reference:
%Стенников В.А., Хамисов О.В..  ~Оптимизация совместной работы источников тепловой энергии// НТФ <<Энергопресс>>, <<Электрические станции>>.   Vol.~2011, No~Number. P.~Page numbers 27.
% 


\end{thebibliography}


% Обязательная информация на английском языке:




%\end{document}
