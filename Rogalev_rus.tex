\begin{englishtitle} % Настраивает LaTeX на использование английского языка
% Этот титульный лист верстается аналогично.
\title{Numerical-symbolic estimates of geometric characteristics of reachable sets and functional differential equations}
% First author
\author{A.N. Rogalev 
  %\and
 % Name FamilyName2\inst{2}
  %\and
  %Name FamilyName3\inst{1}
}
\institute{Institute of computing modelling SB RAS, Krasnoyarsk, Russia\\
  \email{rogalyov@icm.krasn.ru}
  \and
660036, Krasnoyarsk, Russia\\
%\email{email}
}
% etc
\maketitle
\begin{abstract}
Controlled systems described by ordinary differential equations with initial or boundary conditions are considered.  In the report  the inclusion of the reachability domain of control systems using a guaranteed method of estimating the sets of solutions of systems of ordinary differential equations on the basis of symbolic formulas for approximating the shift operator along the trajectory are building.

\keywords{Uncertain dynamic systems, Reachability, Symbolic formulas } % в конце списка точка не ставится
\end{abstract}
\end{englishtitle}


\iffalse
\documentclass[12pt]{llncs}

% При использовании pdfLaTeX добавляется стандартный набор русификации babel.

\usepackage{iftex}

\ifPDFTeX
\usepackage[T2A]{fontenc}
\usepackage[utf8]{inputenc} % Кодировка utf-8, cp1251 и т.д.
\usepackage[english,russian]{babel}
\fi

\usepackage{todonotes}

\usepackage[russian]{nla}


\begin{document}
\fi
% Текст оформляется в соответствии с классом article, используя дополнения
% AMS.
%

\title{Численно-символьные оценки геометрических характеристик множеств достижимости и функционально-дифференциальные уравнения
%\thanks{Работа была выполнена при поддержке ФИЦ КНЦ СО РАН, проект \textnumero~0287-2021-0002.}
}
% Первый автор
\author{А.~Н.~Рогалев 
} % обязательное поле

% Аффилиации пишутся в следующей форме, соединяя каждый институт при помощи \and.
\institute{ИВМ СО РАН, Красноярск, Россия\\
  \email{rogalyov@icm.krasn.ru}
 }

\maketitle

\begin{abstract}
В докладе описывается алгоритм построения включения областей достижимости управляемой системы, использующий оценки геометрических характеристик множеств достижимости управляемых систем, а также свойства функционально-дифференциальных уравнений. Оцениваются вычислительные затраты предложенного алгоритма в сравнении с затратами некоторых известных алгоритмов оценки множеств достижимости.  Приводятся примеры оценки множеств достижимости.

\keywords{Динамические системы с возмущениями, области достижимости, символьные формулы, граничные и внутренние точки} % в конце списка точка не ставится
\end{abstract}

%\section{Основные результаты} % не обязательное поле

При решении многих задач движение  управляемой системы описывается  системой дифференциальных уравнений:
\begin{equation}\frac{dy}{dt} = f (t, y, u) ,
\label{for1}
\end{equation}
с заданным классом допустимых управлений $u \in U$ , с известными начальным и конечным состоянием управляемой системы, при выполнении ряда условий, наложенных на правую часть системы дифференциальных уравнений.
Множеством достижимости в момент времени  $t$ называется множество всех точек из фазового пространства, в которые можно перейти на отрезке времени $[t_{0},T ]$   из всех возможных точек начального множества фазовых состояний  по решениям системы (\ref{for1}) с начальным условием   и с допустимым управлением.
Понятие  области достижимости позволяет решать различные конкретные задачи теории управления.
 %Это подтверждает достаточно большое число публикаций, посвященных вопросам  оценивания множеств достижимости.
 В тезисах сложно привести полное описание работ  по этой тематике, опубликованных учеными из Екатеринбурга, Москвы, Иркутска и др. Приводятся ссылки к нескольким статьям \cite{Kurzhansky}-\cite{Gornov},
 полный обзор всех работ, посвященных оцениванию множеств достижимости -- это задача, требующая существенно больших ресурсов и выходящая за рамки данных тезисов.

 %В докладе описывается включение множеств достижимости  с помощью гарантированного метода, основанного на символьной формуле оператора сдвига вдоль  траектории и вычислении %множеств значений этой формулы
 %на области всех управляющих воздействий \cite{Rogalev1}-\cite{Rogalev6}.
%Символьная формула (аналитическое выражение) – это запись символьной  формы включающей знаки операций, имена функций и констант, вычисление по которой позволяет получить %значение решения.
%Для получения символьных формул использовалось последовательное исполнение метода хранения и переработки символьной информации при продвижении вдоль траектории решений на %основе статичного хранения этой информации, работы с адресацией памяти с помощью функций поточной обработки. %\begin{figure}[htb]

 Модели, основанные на символьных формулах решений и не использующие символьные формулы вариации постоянных, изложены в работах \cite{Rogalev1}, \cite{Rogalev5}, \cite{Rogalev6}, \cite{Rogalev7}.
 %Опубликованные результаты полезны для оценки множеств достижимости, но они не закрывают полностью этот вопрос, как и другие подходы.
  В данной статье описывается применение символьной нелинейной формулы вариации произвольных постоянных для  оценки множеств достижимости нелинейных управляемых систем.
 % Решение нелинейной системы интегральных уравнений (\ref{for10})  строится в символьном виде методом последовательных приближений.
%Сформулируем  основные понятия и определения.
Символьная формула (аналитическое выражение)–- запись имен констант, переменных и действий, которые нужно проделать в определенном порядке над значениями этих переменных.
При записи символьных формул, аппроксимирующих оператор сдвига вдоль траектории, допускается включение в них числовых констант, с отложенным выполнением действий с ними.

%Тогда результат последовательного исполнения преобразований формул
%\begin{eqnarray*}
%&&Y^1=\Lambda(y,Y^0,Y^1)=S^1(Y^0),\\
%&&Y^2=\Lambda(y,Y^0,Y^1,Y^2)=S^2\circ S^1(Y^0),\\
%&&\dots \dots \dots \dots \dots \dots\\
%&&Y^k=\Lambda(y,Y^0,Y^1),...,Y^k=S^{k}\circ S^{k-1}\dots S{^2}\circ S^1(Y^0)
%\end{eqnarray*}
%называется символьной формулой метода сдвига вдоль траектории решения системы ОДУ.

 %В качестве формулы сдвига вдоль траектории можно выбрать формулы, числовые оригиналы которых соответствуют приближенным методам решения систем ОДУ (например, линейным %одношаговым и многошаговым методам, коллокационным методам), а также символьные формулы Коши и символьные формулы вариации произвольных постоянных.

%Принципиальное отличие заключается в способе реализации этих методов, то есть получении числовых оригиналов символьных формул.
%В процессе конструирования символьной формулы целесообразно использовать экономичные символьные формулы, то есть последовательности имен переменных и действий,  позволяющие %хранить формулы в памяти машины и обрабатывать их за минимальное время и с разумными  затратами памяти.

%В общем случае для описываемого класса методов предлагается модель вычислений (преобразований и вычислений) символьных формул, основанная на поэтапном статичном хранении %информации и преобразовании ее в завершающей стадии метода.

Алгоритм нахождения символьной формулы решения задачи
 \begin{eqnarray}
                        \label{for10}
                        \frac{dy}{dt}=A(t)y+F(t,y)+u(t),
                        \end{eqnarray}
состоит из нескольких этапов.

%1) Строится символьная формула фундаментальной матрицы $\Phi(t)$ решений  \cite{Abramov}  линейной однородной системы
% \begin{eqnarray}
%                        \label{for11}
%                        \frac{dy}{dt}=A(t)y,
%                        \end{eqnarray}
%столбцы которой образуют фундаментальную систему решений  системы.

%2) Строится символьная формула обратной матрицы $\Phi(t)^{-1}$ фундаментальной матрицы решений в точке $t_0$.

%3)  Выполняется умножение символьных величин формулы фундаментальной матрицы $\Phi(t)$ решений на формулу обратной матрицу $\Phi(t)^{-1}$ в точке $t_0$ и на символьную %переменную начальных данных $y_0.$

%4) Формируется формула подинтегрального члена  выражения.
%       \begin{eqnarray}
%       \Phi(\tau)^{-1}\Big[ F(\tau,y(\tau))+u(\tau))\Big] d\tau.
%       \end{eqnarray}
Выполняется полная сборка формулы
 \begin{eqnarray}
      \label{for12}
       y(t)=\Phi(t)\Phi(t_0)^{-1}y_0 + \Phi(t) \int_{t_0}^{t}\Phi(\tau)^{-1}\Big[ F(\tau,y(\tau))+u(\tau))\Big] d\tau.
       \end{eqnarray}
 В (\ref{for12}) сформирована система  нелинейных интегральных уравнений Фредгольма второго рода. Коэффициенты уравнений заданы как символьные величины. Решение системы определяется методом последовательных приближений как функция.
%\begin{eqnarray}
%\label{for13}
%&&y^{1}(t)= \Phi(t)\Phi^{-1}(t_0)y_0, \\
%&&y^{k}(t)= \Phi(t)\Phi^{-1}(t_0)y_0 + \Phi(t) \int_{t_0}^{t}\Phi(\tau)^{-1}\times\nonumber\\ &&\times\Big[ F(\tau,y^{k-1}(\tau))+u(\tau))\Big] d\tau,
%k=2,3,...\nonumber
%\end{eqnarray}

%7) Ядро этой системы интегральных уравнений $$ K(t,\tau) = \Phi(t)\Phi(\tau)^{-1}\Big[ F(\tau,y(\tau))+u(\tau))\Big] d\tau,$$ может иметь разрывы, при условии, что двойной %интеграл
%$$ \int_{a}^{b} \int_{a}^{b} |K(t,\tau)|^2 dt d\tau = B^2<\infty. $$
%ограничен конечным значением  $B^2=const.$
Показано, при выполнении каких условий метод последовательных приближений сходится.

%8) Можно описать метод последовательных приближений в виде
 %        $$ \phi_{k}(t) =g(t) + \int_{t_0}^{t}K(t,\tau)\phi_{k-1} d\tau. $$
 %        Достаточными условиями применимости метода простой итерации являются неравенства
 %        $$q_1=max_{t_0\le t \le T}\int_{t_0}^{T}|K(t,\tau)| d\tau <1.$$

%В качестве примеров рассмотрим оценки множеств достижимости следующих управляемых систем.

% Для первого примера оценку множества достижимости  удается построить на интервале времени от 0 до 70 и далее вплоть до 100.  For the first example, the estimate of the %reachability set can be constructed on the time interval from 0 to 70 and further up to 100.

 %В опубликованных работах результаты оценивания множеств достижимости на таком интервале  времени (или большем) не удается найти.
%\begin{eqnarray}
%\label{ex1}
% &&\frac{dy_{1}}{dt} = y_{2},\nonumber \\
%&&\frac{dy_{2}}{dt} =-y_1+u,\\
%&&\left|u\right|\le 1,  y_{1}(0)=y_{2}(0)=1.\nonumber
%\end{eqnarray}

%In the published papers, the results of estimating the reachability sets on such a time interval (or longer) cannot be found.
%\begin{eqnarray}
%\label{ex1}
%   &&\frac{dy_{1}}{dt} = y_{2},\nonumber \\
%&&\frac{dy_{2}}{dt} =-y_1+u,\\
%&&\left|u\right|\le 1, y_{1}(0)=y_{2}(0)=1.\nonumber
%\end{eqnarray}

%\begin{figure}
%\centering
%\caption{Оценка области достижимости второй компоненты вектора решения  Estimation of the reachability domain of the second component of the solution vector}
%\includegraphics[width=\textwidth]{Component N2_70.jpg}
%\label{FIG:1}
%\end{figure}

Метод применяется для оценки множества достижимости управляемой системы
\begin{eqnarray}
\label{ex2}
 &&\frac{dy_{1}}{dt}=\frac{1}{2}y_{1}(1-y_{2})+u_1,\nonumber \\
&&\frac{dy_{2}}{dt}=\frac{1}{2}y_{2}(1-y_{1})+u_2,\\
&&\ 0  \le y_{1}(0) \le 10,  0 \le y_{2}(0) \le 10,\nonumber\\
&&\left|u_1\right|\le 1, \left|u_2 \right|\le 1. \nonumber
\end{eqnarray}

При $t=0.5$ получена оценка множества достижимости $-0.26445443 \le y_1 \le 9.40980746, $  $-0.01 \le y_2 \le 8.2. $

При $t=1.5$ вычислена оценка $-0.3284561 \le y_1 \le 18.9746654, $  $-0.237211 \le y_2 \le 16.4834202. $

При больших значениях $t$ у решений появляются точки сингулярности, поэтому оценки множеств  решений увеличиваются экспоненциально и не несут никакой информации. Приводить их нет смысла.
%\section{Заключение }

Для нелинейных управляемых систем качество и точность построения множеств достижимости зависит от сложного характера множества  решений ОДУ –- для решений могут появляться дихотомия спектра собственных значений дифференциального оператора, бифуркации решений, точки возврата и многие другие характеристики множеств решений, на которые влияет управление, являющееся негладкой функцией.
%Это является одной из причин, по которой границы оценок могут сильно отличаться от границ геометрических фигур, применяемых для аппроксимации множеств решений.

 %Поэтому в статье описана  модель, в которой  строится оценка множества достижимости с помощью формул, записанных  в символьном виде.  Затем вычисляются  граничные точки этих %множеств.  Граничные точки определяются  как экстремумы символьных выражений либо в направлении координатных осей, либо в иных заданных направлениях.  По заданному набору %граничных точек возможно построить множество, включающее множество достижимости.

%Предварительно полезно выполнить регуляризацию задачи оценивания границ множеств решений, переходя к линейному приближению исходной системы. Под регуляризацией понимается %нахождение информации  о множестве точных решений.  Эту регуляризацию задают величины сжатия/растяжения в заданных направлениях, смещения по оси времени, и поворота на %некоторый угол.  В некотором смысле можно говорить о деформации множества в линейном приближении  согласно линейной теории упругости.

Формула нелинейной вариации постоянных задает соотношение, которому удовлетворяет решение нелинейной системы ОДУ с управлениями. Для линейных систем эта формула вариации постоянных называется формулой Коши. Для нелинейных систем формула вариации постоянных является по сути системой интегральных уравнений, в правую часть которой входит управление. Далее предлагается решать систему интегральных уравнений так, чтобы получить зависимость для точного решения нелинейной системы с управлением и оценить максимальные и минимальные значения данной зависимости при изменении управления в заданных пределах управления. Зависимость будет получена как формула решения. В этом суть подхода. Поэтому можно оценить область возможных значений формулы, это область есть оценка множества достижимости.
Полученные результаты подтверждают эффективность модели оценки множеств достижимости с применением символьных формул вариации постоянных.

% Рисунки и таблицы оформляются по стандарту класса article. Например,

%\begin{figure}[htb]
 % \centering
  % Поддерживаются два формата:
  %\includegraphics[width=0.7\linewidth]{figure.pdf} % Растровый формат
  %\includegraphics[width=0.7\linewidth]{figure.png} % Векторный и растровый формат
  %
  % Векторные рисунки можно рисовать в редакторе Inkscape
  % https://inkscape.org/ru/download/
  % Основной формат этого редактора - SVG, поэтому рисунки необходимо экспортировать в
  % PDF или PNG (с разрешением - минимум 150 dpi, максимум - 300dpi).
  %\begin{center}
   % \missingfigure[figwidth=0.7\linewidth]{Уберите меня из статьи!}
  %\end{center}
  %\caption{Заголовок рисунка}\label{fig:example}
%\end{figure}

% Современные издательства требуют использовать кавычки-елочки << >>.

% В конце текста можно выразить благодарности, если этого не было
% сделано в ссылке с заголовка статьи, например,
%Работа выполнена при поддержке РФФИ (РНФ, другие фонды), проект \textnumero~00-00-00000.
%

% Список литературы оформляется подобно ГОСТ-2008.
% Примеры оформления находятся по этому адресу -
%     https://narfu.ru/agtu/www.agtu.ru/fad08f5ab5ca9486942a52596ba6582elit.html
%

\begin{thebibliography}{9} % или {99}, если ссылок больше десяти.
%\bibitem{Gantmakher} Гантмахер~Ф.Р. Теория матриц. М.:~Наука,~1966.

%\bibitem{Kholl} Современные численные методы решения обыкновенных дифференциальных уравнений~/ Под~ред.~Дж.~Холл, Дж.~Уатт. М.:~Мир,~1979.

%\bibitem{Aleksandrov1} Александров~А.Ю. Об устойчивости сложных систем в критических случаях~// Автоматика и телемеханика. 2001. \textnumero~9. С.~3--13.

%\bibitem{Moreau1977} Moreau~J.-J. Evolution problem associated with a moving convex set in a Hilbert space~// J.~Differential~Eq. 1977. Vol.~26. Pp.~347--374.

%\bibitem{Semenov} Семенов~А.А. Замечание о вычислительной сложности известных предположительно односторонних функций~// Тр.~XII Байкальской междунар. конф. <<Методы оптимизации и их приложения>>. Иркутск, 2001. С.~142--146.
\bibitem{Kurzhansky} Куржанский~А.Б. Управление и наблюдение в условиях неопределенности. M.:~ Наука, 1977 .

\bibitem{Chernousjko}Черноусько~Ф.Л. Оценивание фазового состояния динамических систем. М.:~ Наука, 1988.

\bibitem{Tyatushkin} Тятюшкин~А. И.,  Моржин О. В.  Численное исследование множеств достижимости нелинейных управляемых дифференциальных систем // Автоматика и телемеханика. 2011. \textnumero 6. C. 160--170.

\bibitem{Filippova}Филиппова~Т.Ф. Оценки множеств достижимости управляемых систем с нелинейностью и параметричскими возмущениями // Труды Института Математики и Механики УрО РАН. 2014. Т. 20. \textnumero 4. С. 287--296.

\bibitem{Gornov} Финкельштейн~ Е.А., Горнов~ А.Ю., Алгоритмы квазиравномерного заполнения множсевт достижимости нелинейной управляемой системы // Известия Иркутского государственного университета. Серия: Математика. 2017. Т. 19. С.217--223.

\bibitem{Rogalev1} Рогалев~ А.Н. Гарантированные методы решения систем обыкновенных дифференциальных уравнений на основе преобразования символьных формул // Вычислительные Технологии. 2003. Т. 8 \textnumero 5. С. 102--116.

%\bibitem{Rogalev2} Рогалев~А.Н. Символьные вычисления в гарантированных методах, выполненные на нескольких процессорах // Вестник НГУ. Серия: Информационные технологии.  2006.  № 1 (4).  С. 56--62

%\bibitem{Rogalev3}
% Rogalev  A.N., Rogalev A.A., Feodorova N.A.  Numerical computations of the safe boundaries of complex technical systems and practical stability. J. Phys: Conf.  Ser. 2019. Vol. 1399.  33112.

%\bibitem{Rogalev4}
% Rogalev A.N., Rogalev A.A., Feodorova N.A.  Malfunction analysis and safety of mathematical models of technical systems. J. Phys.: Conf. Ser.  2020.  Vol. 1515.  022064

\bibitem{Rogalev5}
Rogalev A.N.   Set of Solutions of Ordinary Differential Equations in Stability Problems // Continuum Mechanics, Applied Mathematics and Scientific Computing: Godunov's Legacy / Ed. by G. Demidenko, E. Romenski, E.Toro, M. Dumbser. Springer Nature Switzerland AG, 2020. Pp. 307--312. %https://doi.org/10.1007/978-3-030-38870-6
\bibitem{Rogalev6}
Rogalev A.N., Rogalev A.A., Feodorova N.A.
Mathematical modelling of risk and safe regions of technical systems and surviving trajectories. J. Phys: Conf Ser. 2021. Vol. 1889. P. 022108. %doi:10.1088/1742-6596/1889/2/022108

\bibitem{Rogalev7} Rogalev A.N. Symbolic Methods for Estimating the Sets of Solutions of Ordinary Differential Equations with Perturbations on a Finite Time Interval. Journal of Vibration Testing and System Dynamics. 2023. Vol.7. Is.1. P.31-37. doi:10.5890/JVTSD.2023.03.005

\end{thebibliography}

% После библиографического списка в русскоязычных статьях необходимо оформить
% англоязычный заголовок.



%\end{document}
