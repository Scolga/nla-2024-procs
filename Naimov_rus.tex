\begin{englishtitle} % Настраивает LaTeX на использование английского языка
% Этот титульный лист верстается аналогично.
\title{On calculating the mapping degree of the gradient of a smooth positively homogeneous function}
% First author
\author{Alizhon Naimov  \and  Alexander Korovin 
}
\institute{Vologda State University, Vologda, Russia\\
 \email{naimovan@vogu35.ru ,   korovinal@vogu35.ru}
 }  

\maketitle

\begin{abstract}
The report addresses the issue of calculating the mapping degree of the gradient of a smooth positively homogeneous function on the unit sphere of the Euclidean space $\mathrm{R}^n$, $n\geq 2$. A scheme for proving the formula for calculating rotation is discussed and some consequences are given. The formula for calculating the mapping degree is refined in cases where the set of zeros of a function has a certain structure or
  a function is represented by a product of simpler functions.

\keywords{positively homogeneous function, gradient of a function, 
vector field,  mapping degree of a vector field (topological degree), 
Euler characteristic} % в конце списка точка не ставится
\end{abstract}
\end{englishtitle}

\iffalse
\documentclass[12pt]{llncs}

% При использовании pdfLaTeX добавляется стандартный набор русификации babel.

\usepackage{iftex}

\ifPDFTeX
\usepackage[T2A]{fontenc}
\usepackage[utf8]{inputenc} % Кодировка utf-8, cp1251 и т.д.
\usepackage[english,russian]{babel}
\fi

\usepackage{todonotes}

\usepackage[russian]{nla}

\begin{document}
\fi
% Текст оформляется в соответствии с классом article, используя дополнения
% AMS.
%

\title{О вычислении вращения градиента гладкой положительно однородной функции\thanks{Исследование выполнено за счет гранта
Российского научного фонда \textnumero 23-21-00032, https://rscf.ru/project/23-21-00032/}}
% Первый автор
\author{А.~Н.~Наимов \and А.~Л.~Коровин
 } % обязательное поле

% Аффилиации пишутся в следующей форме, соединяя каждый институт при помощи \and.
\institute{Вологодский государственный университет (ВоГУ), Вологда, Россия\\
  \email{naimovan@vogu35.ru,  korovinal@vogu35.ru}
 % \and Другие авторы...
}

\maketitle

\begin{abstract}
В докладе рассматривается вопрос о вычислении вращения (степени отображения) градиента гладкой положительно однородной функции на единичной сфере евклидового  пространства $\mathrm{R}^n$, $n\geq 2$. 
Обсуждается схема доказательства формулы вычисления вращения и приводятся некоторые следствия.
Формула вычисления вращения уточняется в случаях, когда множество нулей функции имеет определенную структуру или
 функция представлена произведением более простых функций.
 
\keywords{положительно однородная функция,
 градиент функции, векторное поле, вращение векторного поля,
 эйлерова характеристика} % в конце списка точка не ставится
\end{abstract}


Вычисление вращения (степени отображения) векторных полей представляет интерес в приложении методов нелинейного функционального анализа 
в теории дифференциальных и интегральных уравнений. Можно отметить работы \cite{K66},  \cite{KZ75}, \cite{M96}, где методы вычисления 
вращения векторных полей применяются к исследованию существования 
периодических и ограниченных решений систем нелинейных 
обыкновенных дифференциальных уравнений. 

В работе \cite{M96} анонсирована формула 
\begin{equation}\label{for1}
\gamma(\nabla v, S^{n-1})=1-\chi (\overline{\Omega}_-(v)),
\end{equation}
где $\gamma(\nabla v, S^{n-1})$ -- вращение градиента $\nabla v$
гладкой положительно однородной функции 
$v\in C^1(\mathrm{R}^n\setminus\{0\};\mathrm{R}^1)$, 
$\nabla v(x)\neq 0$ $\forall x\in S^{n-1}$, 
на единичной сфере $S^{n-1}=\{x\in \mathrm{R}^n : |x|=1 \}$ 
евклидового пространства $\mathrm{R}^n$, $n\geq 2$,
 $\chi (\overline{\Omega}_-(v))$ -- эйлерова характеристика
замыкания множества
$$
\Omega_-(v)=\{x\in S^{n-1} : v(x)<0 \},
$$ 
при этом $\chi (\overline{\Omega}_-(v))=0$, если 
$\Omega_-(v)$ пусто. В работах \cite{MN21}, \cite{MN22},  используя формулу (\ref{for1}),  доказаны теоремы о существовании периодических и ограниченных  решений для систем обыкновенных дифференциальных уравнений первого порядка с главной градиентной положительно однородной нелинейностью.

В работе \cite[с. 3-4]{A78} дано пояснение, 
как формулы вида (\ref{for1}) можно получить 
непосредственным применением теории Морса. 
В настоящей работе предложена схема 
доказательства формулы (\ref{for1}). Суть предложенной схемы состоит в том, что вычисление  $\gamma(\nabla v, S^{n-1})$ 
сводится к вычислению вращений  касательной составляющей
$$
\nabla v(x)-\langle \nabla v(x) , x \rangle \frac{x}{|x|^2}, 
\quad x\in S^{n-1},
$$
 на границах связных компонент множества $\Omega_-(v)$, где 
$
\langle x , y \rangle = x_1y_1 + \ldots + x_ny_n
$
 --  скалярное произведение в $\mathrm{R}^n$.
Установлено, что вращение касательной составляющей на границе каждой 
связной компоненты равно эйлеровой характеристике замыкания
данной связной компоненты. 
 Схему доказательства  можно
применить для вычисления вращения неградиентных векторных полей (см., например, \cite{MN23}). 

Справедливы следующие уточнения формулы (\ref{for1}).

{\bf Теорема 1}. {\it Если множество
$
\Omega_0(v)=\{x\in S^{n-1} : v(x)=0 \}
$
состоит из $p_0(v)$ связных компонент, каждая из которых диффеоморфна сфере $S^{n-2}$, то верна формула
$$
\gamma(\nabla v, S^{n-1})=\left\{
\begin{array}{cl}
1-p_0(v), & \text{если n -- четно,}\\
p_+(v)-p_-(v), & \text{если n -- нечетно,}\\
\end{array}
\right.
$$
где $p_{\pm}(v)$ -- число связных компонент множества 
$\Omega_{\pm}(v)=\{x\in S^{n-1} : \pm v(x)>0 \}$.
}

Условие теоремы 1 при $n=3$  всегда выполняется. При $n=4$ 
  множество $\Omega_0(v)$ состоит из $p_0(v)$ двумерных
ориентируемых гладких многообразий без края. Каждое такое многообразие,
как известно, диффеоморфно сфере с ручками. Зная набор количества ручек
$r_1, \ldots , r_{p_0(v)}$ связных компонент, находим
$$
\chi(\Omega_0(v))=2(1-r_1) +  \ldots + 2(1- r_{p_0(v)}).
$$

{\bf Теорема 2}. {\it Если $n=4$, то верна формула
$$
\gamma(\nabla v, S^{3})=1 - p_0(v) + r_1  +  \ldots + r_{p_0(v)},
$$
где $r_1, \ldots , r_{p_0(v)}$ --  набор количества ручек связных компонент 
множества $\Omega_0(v)$.
}

{\bf Теорема 3}. {\it Пусть функции 
$v_i\in C^1(\mathrm{R}^n\setminus\{0\};\mathrm{R}^1)$, $i=\overline{1, N}$
удовлетворяют условиям

а) $\nabla v_i(x)\neq 0$ $\forall x\neq 0$, $i=\overline{1, N}$;

б) $|v_i(x)|+|v_j(x)|>0$  $\forall x\neq 0$, $i\neq j$;

в) $\forall  i=\overline{1, N-1}$ либо 
$\Omega_-(v_i)\subset \Omega_+( v_{i+1}\cdot \ldots \cdot v_N)$,
либо 
$\Omega_-(v_i)\subset \Omega_-( v_{i+1}\cdot \ldots \cdot v_N)$.

Тогда верна формула
$$
\chi(\overline{\Omega}_-(v_1\cdot \ldots \cdot v_N))=
\sum_{i=1}^{N-1}
\sigma(v_i; v_{i+1}\cdot \ldots \cdot v_N)
\chi(\overline{\Omega}_-(v_i)) +
\chi(\overline{\Omega}_-(v_N)),
$$
где
$$
\sigma(v_1; v_2)=\left\{
\begin{array}{cl}
1, & \text{если}\quad \Omega_-(v_1)\subset\Omega_+(v_2), \\
(-1)^n, & \text{если}\quad \Omega_-(v_1)\subset\Omega_-(v_2).\\
\end{array}
\right.
$$
В частности, 
$$
\chi(\overline{\Omega}_-(v_1\cdot \ldots \cdot v_N))=
\sum_{i=1}^{N}
\chi(\overline{\Omega}_-(v_i)), \qquad \text{если} \quad n - \text{четно}.
$$
}

Аналогично работе \cite[Lemma 2]{MN_IA21} можно показать, что любую положительно однородную (порядка $m>0$) функцию 
$v\in C^1(\mathrm{R}^n\setminus\{0\};\mathrm{R}^1)$,
градиент которой не обращается в ноль, можно представить в виде произведения 
$v(x)=|x|^{m-N}v_1(x)\cdot \ldots \cdot v_N(x)$, где $N=p_0(v)$,
функции $v_i$, $i=\overline{1, N}$ 
положительно однородные порядка 1 и
удовлетворяют условиям а) -- в), а соответствующие им множества $\Omega_0(v_i)$, 
$i=\overline{1, N}$ совпадают со связными компонентами $\Omega_0(v)$.

% Список литературы оформляется подобно ГОСТ-2008.
% Примеры оформления находятся по этому адресу -
%     https://narfu.ru/agtu/www.agtu.ru/fad08f5ab5ca9486942a52596ba6582elit.html
%


\begin{thebibliography}{9} % или {99}, если ссылок больше десяти.

\bibitem{K66}  Красносельский, М.А.   
{ Оператор сдвига по траекториям дифференциальных уравнений}.   М.: Наука, 1966.  330~с.

\bibitem{KZ75}  Красносельский, М.А., Забрейко, П.П.    
{ Геометрические методы нелинейного анализа}.
  М.: Наука, 1975.  512~с.

\bibitem{M96} Мухамадиев, Э. { Ограниченные решения и гомотопические
инварианты систем  нелинейных дифференциальных уравнений} 
 // Докл. РАН.  1996.  Т.~351. №~5.  С.~596-598.

\bibitem{MN21} Мухамадиев, Э., Наимов, А.Н. { Критерии существования периодических и ограниченных решений 
для трехмерных систем дифференциальных уравнений} 
 // Тр. ИММ УрО РАН.  2021.  Т.~27. №~1.
  С.~157–172.

\bibitem{MN22} Мухамадиев, Э., Наимов, А.Н. { Об априорной оценке 
и существовании периодических решений для одного класса 
систем нелинейных обыкновенных дифференциальных уравнений} 
 // Изв. вузов. Матем.  2022.  №~4.
  С.~37–48.

\bibitem{A78} Арнольд, В.И. { Индекс особой точки векторного поля, неравенства 
Петровского–Олейник и смешанные структуры Ходжа} 
 // Функц. анализ и его прил.  1978.  Т.~12. №~1.
  С.~1–14.
 
\bibitem{MN23} Мухамадиев, Э., Наимов, А.Н. { К вопросу о вычислении 
вращения конечномерного векторного поля} 
// Изв. вузов. Матем.  2023.  №~6.
 С.~67–73. 

\bibitem{MN_IA21} Mukhamadiev, E., Naimov, A.N. 
{ On the homotopy classification of positively homogeneous 
functions of three variables}. Issues Anal.   2021.   Vol.~10. №~2.
  Pp.~67–78.

\end{thebibliography}

% После библиографического списка в русскоязычных статьях необходимо оформить
% англоязычный заголовок.




%\end{document}
