\begin{englishtitle}
\title{Generalized Heilbronn problem}
% First author
\author{Vladimir Kistner, Vsevolod Voronov}
\institute{Caucasus Mathematical Center, Maikop, Russia\\
\email{vovakistner1@gmail.comm}
\and
Moscow Institute of Physics and Technology, \\
Dolgoprudniy, Russia,\\
\email{v-vor@yandex.ru}
}
% etc

\maketitle

\begin{abstract}
The paper deals with the following optimisation problem: we need to locate points inside a given subset of Euclidean space in such a way that the smallest possible area of triangles (simplex volumes) with vertices at these points is as large as possible. In other words, we need to find $k$ points of the most general position in the compact $X \subset \mathbb{R}^n$, where the degeneracy criterion is the smallest of the volumes of $n$-simplexes. The paper will present an algorithm for finding local minima and some results obtained when $X$ is a flat torus and a sphere.

\keywords{discrete geometry, stochastic gradient descent}
\end{abstract}
\end{englishtitle}


\iffalse
\documentclass[12pt]{llncs}
\usepackage[T2A]{fontenc}
\usepackage[utf8]{inputenc}
\usepackage[english,russian]{babel}
\usepackage[russian]{nla}

%\usepackage[english,russian]{nla}

% \graphicspath{{pics/}} %Set the subfolder with figures (png, pdf).

%\usepackage{showframe}
\begin{document}
%\selectlanguage{russian}
\fi

\title{Обобщенная задача Хейльбрённа%\thanks{Работа выполнена при поддержке РФФИ (РНФ, другие фонды), проект \textnumero~00-00-00000.}
}
% Первый автор
\author{В.~Д.~Кистнер\inst{1}
  \and
% Второй автор
  В.~А.~Воронов\inst{1,2}
%  \and
% Третий автор
%  И.~О.~Фамилия\inst{2}
} 
 \institute{Кавказский математический центр\\ Адыгейского государственного университета, Майкоп, Россия\\
   \email{vovakistner1@gmail.comm}
  \and
  Московский физико-технический институт\\ Долгопрудный, Россия\\
   \email{v-vor@yandex.ru}
  }


\maketitle

\begin{abstract}
В докладе рассматривается следующая оптимизационная задача: требуется расположить точки внутри заданного подмножества евклидова пространства таким образом, чтобы наименьшая из площадей треугольников (объемов симплексов) с вершинами в этих точках была как можно больше [1-3]. Иными словами, требуется найти  $k$ точек как можно более общего положения в компакте $X \subset \mathbb{R}^n$, где критерием <<почти вырожденности>> является наименьший из объемов $n$-симплексов. В докладе будет представлен алгоритм поиска локальных минимумов и некоторые результаты, полученные в случаях, когда $X$ является плоским тором и сферой.
\keywords{дискретная геометрия, стохастический градиентный спуск}
\end{abstract}


% Список литературы.
\begin{thebibliography}{99}
\bibitem{vv1}
Roth, K. F.  On a problem of Heilbronn. Journal of the London Mathematical Society. 1951. Vol.~1. No.~3. Pp.~198-204.

\bibitem{vv2}
Komlós, J., Pintz, J.,  Szemerédi, E.  On Heilbronn's triangle problem. Journal of the London Mathematical Society. 1981.  Vol.~2. No.~3. Pp.~385-396.

\bibitem{vv3}
Cohen, A., Pohoata, C.,  Zakharov, D.  A new upper bound for the Heilbronn triangle problem. 2023. arXiv preprint arXiv:2305.18253.

\end{thebibliography}


% Обязательная информация на английском языке:




%\end{document}
