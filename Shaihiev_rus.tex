\begin{englishtitle} % Настраивает LaTeX на использование английского языка
% Этот титульный лист верстается аналогично.
\title{Time series of percolation sensor currents under conditions of potentiostatic chronoamperometry of liquid media}
% First author
\author{E.~R.~Shaihiev\inst{1}  \and  M.~D.~Bitkulov\inst{1}  \and  N.~E.~Murtazin\inst{2}
}
\institute{UUST, Ufa, Russia\\
  \email{erik08082002@mail.ru}\\
\email{bitkulovmarat@gmail.com}
  \and
USPTU, Ufa, Russia\\
  \email{people.ufa@gmail.com}}
% etc

\maketitle

\begin{abstract}
Restoring the state of the system from the temporal realization of an observed value on a limited time interval using the theorems of F.~Takens for dynamic systems with one observation.

\keywords{time series, dynamic system, diffeomorphism} % в конце списка точка не ставится
\end{abstract}
\end{englishtitle}

\iffalse
%%%%%%%%%%%%%%%%%%%%%%%%%%%%%%%%%%%%%%%%%%%%%%%%%%%%%%%%%%%%%%%%%%%%%%%%
%
%  This is the template file for the 6th International conference
%  NONLINEAR ANALYSIS AND EXTREMAL PROBLEMS
%  June 25-30, 2018
%  Irkutsk, Russia
%
%%%%%%%%%%%%%%%%%%%%%%%%%%%%%%%%%%%%%%%%%%%%%%%%%%%%%%%%%%%%%%%%%%%%%%%%


%  В случае возникновения вопросов и проблем с версткой статьи,
%  пишите письма на электронную почту: eugeneai@irnok.net, Черкашин Евгений.



\documentclass[12pt]{llncs}  

% При использовании pdfLaTeX добавляется стандартный набор русификации babel.
% Если верстка производится в LuaLaTeX, то следующие три строки надо
% закомментировать, русификация будет произведена в корректирующем стиле автоматом.
\usepackage{iftex}

\ifPDFTeX
\usepackage[T2A]{fontenc}
\usepackage[utf8]{inputenc} % Кодировка utf-8, cp1251 и т.д.
\usepackage[english,russian]{babel}
\fi

% Для верстки в LuaLaTeX текст готовится строго в utf-8!

% В операционной системе Windows для редактирования в кодировке utf-8
% можно использовать программы notepad++ https://notepad-plus-plus.org/,
% techniccenter http://www.texniccenter.org/,
% SciTE (самая маленькая по объему программа) http://www.scintilla.org/SciTEDownload.html
% Подойдет также и встроенный в свежий дистрибутив MiKTeX редактор
% TeXworks.

% Добавляется корректирующий стилевой файл строго после babel, если он был включен.
% В параметре необходимо указать russian, что установит не совсем стандартные названия
% разделов текста, настроит переносы для русского языка как основного и т.п.

\usepackage{todonotes} % Этот пакет нужен для верстки данного шаблона, его
                       % надо убрать из вашей статьи.

\usepackage[russian]{nla}

% Многие популярные пакеты (amsXXX, graphicx и т.д.) уже импортированы в корректирующий стиль.
% Если возникнут конфликты с вашими пакетами, попробуйте их отключить и сверстать
% текст как есть.
%
%


% Было б удобно при верстке сборника, чтобы названия рисунков разных авторов не пересекались.
% Чтоб минимизировать такое пересечение, рисунки нужно поместить в отдельную подпапку с
% названием - фамилией автора или названием статьи.
%
% \graphicspath{{ivanov-petrov-pics/}} % Указание папки с изображениями в форматах png, pdf.
% или
% \graphicspath{{great-problem-solving-paper-pics/}}.


\begin{document}
\fi
% Текст оформляется в соответствии с классом article, используя дополнения
% AMS.
%

\title{Временные ряды токов перколяционного сенсора в условиях потенциостатической хроноамперометрии жидких сред}
% Первый автор
\author{Э.~Р.~Шайхиев\inst{1}  \and  М.~Д.~Биткулов\inst{1}  \and  Н.~Э.~Муртазин\inst{2}
} % обязательное поле

% Аффилиации пишутся в следующей форме, соединяя каждый институт при помощи \and.
\institute{УУНиТ, Уфа, Россия \\
  \email{erik08082002@mail.ru}\\
\email{bitkulovmarat@gmail.com}
   \and   % Разделяет институты и присваивает им номера по порядку.
УГНТУ, Уфа, Россия\\
  \email{people.ufa@gmail.com}
% \and Другие авторы...
}

\maketitle

\begin{abstract}
Восстановление состояния системы по временной реализации наблюдаемой величины на ограниченном интервале времени используя теоремы Ф.~Такенса для динамических систем с одним наблюдением.

\keywords{временной ряд, динамическая система, диффеоморфизм} % в конце списка точка не ставится
\end{abstract}

%\section{Основные результаты} % не обязательное поле

Рассмотрим систему
$$
\begin{array}{l}
y(t+t_0)=\phi_t(y(t_0)),\\
\eta(t)=h(y(t)),
\end{array}
$$
где $y$ -- вектор состояния, $\phi_t$ -- оператор эволюции, $h$ -- измерительная функция.


 Вектор $\eta\in\mathbb{R}$ в случае $m=1$ является рядом $\eta(t_i)$, $i=1\ldots N$.
	

Пусть движение системы происходит на компактном многообразии $M$ конечной размерности $d$. Каждой фазовой траектории $y(t)$ на многообразии $M$ соответствует временная реализация наблюдаемой величины $\eta=h(y(t))$, $0<t<\infty$. Вектор $y(t_0)$ однозначно определяет поведение системы, в частности реализацию $\eta(t)$ при $t>t_0$. 

В работе по временной реализации $\eta(t)$  в окрестности момента времени $t_0$ однозначно определено положение системы на многообразии $M$ в момент времени $t_0$. Анализируем временной ряд с постоянным шагом химической реакции используя различные алгоритмы: метод временных задержек, метод ложных соседей, метод главных компонент.


% В конце текста можно выразить благодарности, если этого не было
% сделано в ссылке с заголовка статьи, например,
%Работа выполнена при поддержке РФФИ (РНФ, другие фонды), проект \textnumero~00-00-00000.
%

% Список литературы оформляется подобно ГОСТ-2008.
% Примеры оформления находятся по этому адресу -
%     https://narfu.ru/agtu/www.agtu.ru/fad08f5ab5ca9486942a52596ba6582elit.html
%

\begin{thebibliography}{9} % или {99}, если ссылок больше десяти.
\bibitem{Takens} Такенс~Ф. Обнаружение странных аттракторов в турбулентности // Динамические системы и турбулентность. 1981. Том 898. С.~366--381.

\bibitem{Sugiharal} Сугихара~Г. Нелинейное прогнозирование для классификации естественных временных рядов.  1994. 348 (1688). C.~477-495.


\bibitem{Nizamova}	Валиев~A.А., Низамова~А.Д. Комплексное исследование вытеснения нефти водой в плоском канале // Многофазные системы, 2020. Т.~15. \textnumero~1-2. С.~25.

\bibitem{Nizamova2}	Низамова~А.Д. Влияние температурной зависимости вязкости на устойчивость плоскопараллельного течения жидкости // Труды Института механики им. Р.Р. Мавлютова Уфимского научного центра РАН, 2014. Т.~10. С.~90--94.

\bibitem{Nizamova3}	Nizamova~A.D., Kireev~V.N., Urmancheev~S.F. Influence of temperature dependence of viscosity on the stability of fluid flow in an annular channel // Журнал Лобачевского по математике, 2023. Т.~44. \textnumero~5. С.~1778--1784.
\end{thebibliography}
% После библиографического списка в русскоязычных статьях необходимо оформить
% англоязычный заголовок.




%\end{document}

%%% Local Variables:
%%% mode: latex
%%% TeX-master: t
%%% End:
