\begin{englishtitle} % Настраивает LaTeX на использование английского языка
% Этот титульный лист верстается аналогично.
\title{On the control reconstruction for trajectories of differential inclusions in problems of control theory}
% First author
\author{Nina Subbotina \and  Evgenii Krupennikov 
}
\institute{IMM UrB RAS, Yekaterinburg, Russia\\
  \email{subb@uran.ru}
%  \and
%UrFU, Yekaterinburg, Russia\\
%\email{krupennikov@imm.uran.ru
%}
}
% etc

\maketitle

\begin{abstract}
In the talk, the control reconstruction problem for
dynamic systems with dynamics described by differential inclusions, based on discrete inaccurate measurements of the observed trajectory, is studied.
We consider the case of an observed trajectory generated by a sliding
regime under non-convex geometric restrictions on controls. A correct formulation of the control reconstruction problem is introduced. An approach to constructing
approximations of the solution in the weak topology of the space L2 is suggested.

\keywords{control reconstruction, differential inclusions, sliding controls, weak convergence, non-convex constraints} % в конце списка точка не ставится
\end{abstract}
\end{englishtitle}

\iffalse
%%%%%%%%%%%%%%%%%%%%%%%%%%%%%%%%%%%%%%%%%%%%%%%%%%%%%%%%%%%%%%%%%%%%%%%%
%
%  This is the template file for the 6th International conference
%  NONLINEAR ANALYSIS AND EXTREMAL PROBLEMS
%  June 25-30, 2018
%  Irkutsk, Russia
%
%%%%%%%%%%%%%%%%%%%%%%%%%%%%%%%%%%%%%%%%%%%%%%%%%%%%%%%%%%%%%%%%%%%%%%%%


\documentclass[12pt]{llncs}

% При использовании pdfLaTeX добавляется стандартный набор русификации babel.

\usepackage{iftex}

\ifPDFTeX
\usepackage[T2A]{fontenc}
\usepackage[utf8]{inputenc} % Кодировка utf-8, cp1251 и т.д.
\usepackage[english,russian]{babel}
\fi

\usepackage{todonotes}

\usepackage[russian]{nla}

\begin{document}
\fi

\title{О реконструкции управлений для траекторий дифференциальных включений в задачах теории управления\thanks{Работа выполнена в рамках исследований, проводимых в Уральском математическом центре при финансовой поддержке Министерства науки и высшего образования Российской Федерации (номер соглашения 075-02-2024-1377).}}
% Первый автор
\author{Н.~Н.~Субботина \and  Е.~А.~Крупенников
  \and
}

% Аффилиации пишутся в следующей форме, соединяя каждый институт при помощи \and.
\institute{ИММ УрО РАН, Россия, Екатеринбург \\
  \email{subb@uran.ru}
%  \and   % Разделяет институты и присваивает им номера по порядку.
%УрФУ, Россия, Екатеринбург\\
%  \email{krupennikov@imm.uran.ru
 % }
% \and Другие авторы...
}

\maketitle

\begin{abstract}
В докладе рассматривается задача реконструкции управления по дискретным неточным замерам наблюдаемой траектории управляемой системы, описываемой дифференциальным включением.
Рассматривается случай наблюдаемой траектории, порожденной скользящим
режимом при невыпуклых геометрических ограничениях на управления. Вводится корректная постановка задачи реконструкции управления. Предлагается метод построения аппроксимаций решения этой задачи в слабой топологии пространства L2.

\keywords{реконструкция управлений, дифференциальные включения, скользящие управления, слабая сходимость, невыпуклые ограничения} % в конце списка точка не ставится
\end{abstract}


Рассматриваются обратные задачи динамики для аффинно-управляемых
детерминированных систем с динамикой, описываемой дифференциальными включениями вида
\begin{equation}\label{Krupennikov:dyn}
\begin{gathered}
\dot x(t)\in F(t,x(t))=\{G(t,x(t))u + f(t,x(t)):\ u\in \mathbf{U}\},\\
x\in R^n,\quad u\in R^m,\quad t\in[0,T],\quad x(0)=x_0.
\end{gathered}
\end{equation}
где $\mathbf{U}\subset R^m$ --- невыпуклый компакт. А именно, рассматривается задача реконструкции управления по неточным дискретным замерам порождаемой
им траектории $x^*(\cdot):[0,T]\to R^n$.

%Задачи реконструкции управлений исследовались многими авторами. Отметим подход, предложенный в работах Ю. С. Осипова и А. В.
%Кряжимского~\cite{Krupennikov:OK83}. Этот подход опирается на процедуру оптимального
%прицеливания, истоки которой лежат в работах школы Н. Н. Красовского по теории позиционного оптимального управления~\cite{Krupennikov:KNN_74}.
В системах с невыпуклыми ограничениями на управления могут возникать скользящие управления. Траектория $x^*(\cdot)$, порождаемая таким управлением, описывается дифференциальным включением
\begin{equation}\label{Krupennikov:dyn2}
\dot x^*(t)\in \textnormal{co}F(t,x^*(t))=\{G(t,x^*(t))u + f(t,x^*(t)):\ u\in \textnormal{co}\mathbf{U}\},\quad t\in[0,T],\quad x^*(0)=x_0.
\end{equation}
%Для
%таких систем возникает ряд вопросов, связанных с постановкой задачи. В частности, одна траектория может порождаться не единственным
%управлением.
%Сформулировано понятие нормального управления --- управления, порождающего наблюдаемую траекторию и определяемого единственным образом.
Как известно~\cite{Krupennikov:BF}, для траектории $x^*(\cdot)$ найдется множество измеримых селекторов $\{u(t):[0,T]\to\textnormal{co}\mathbf{U}\}$ --- управлений, порождающих эту траекторию. Вводится понятие нормального управления. Это измеримая функция со значениями из $\textnormal{co}\mathbf{U}$, порождающая $x^*(\cdot)$ и имеющая минимальную $L^2$-норму. Задача реконструкции управления для динамики~(\ref{Krupennikov:dyn}) понимается как задача реконструкции нормального управления.

По известным неточным замерам наблюдаемой траектории нужно построить кусочно-постоянные управления, аппроксимирующие нормальное при стремлении параметров точности замеров к нулю. При этом значения аппроксимаций должны содержаться в невыпуклом множестве $\mathbf{U}$, а порождаемые ими траектории должны сходиться равномерно к наблюдаемой.


Известно~\cite{Krupennikov:Tolst}, что для любой траектории $x^*(\cdot)$ включения~(\ref{Krupennikov:dyn2}) существует последовательность траекторий включения~(\ref{Krupennikov:dyn}), сходящихся к ней равномерно. Однако, встает вопрос как понимать сходимость к нормальному управлению аппроксимаций, порождающих такие траектории. В работе показано, что можно строить кусочно-постоянные аппроксимации,
удовлетворяющие заданным невыпуклым ограничениям $\mathbf{U}$, сходящиеся к нормальному управлению в смысле слабой топологии пространства $L^2$, такие, что порождаемые ими траектории сходятся равномерно к наблюдаемой. Приведен пример, в котором невозможна сходимость к нормальному управлению таких кусочно-постоянных аппроксимаций в сильной топологии пространства $L^2$.

Предложен метод построения искомых аппроксимаций. Он опирается на описанный ранее метод построения кусочно-постоянных аппроксимаций нормального управления в задачах с выпуклыми ограничениями~\cite{Krupennikov:subb4}.

% Рисунки и таблицы оформляются по стандарту класса article. Например,



% В конце текста можно выразить благодарности, если этого не было
% сделано в ссылке с заголовка статьи, например,

%



\begin{thebibliography}{9} % или {99}, если ссылок больше десяти.

\bibitem{Krupennikov:BF}
Благодатских В.И., Филиппов А.Ф., Дифференциальные включения и оптимальное управление // Тр. МИАН СССР, 1985. Т. 169. С 194-–252.

\bibitem{Krupennikov:Tolst} Толстоногов А.А. Дифференциальные включения с невыпуклой правой частью в банаховом пространстве: Диссертация на соискание ученой степени д-ра физико-математических наук: 01.01.02. Иркутск, 1982.

\bibitem{Krupennikov:subb4}	Subbotina N.N., Krupennikov E.A. Variational Approach to Construction of Piecewise-Constant Approximations of the Solution of Dynamic Reconstruction Problem  // Differential Equations, Mathematical Modeling and Computational Algorithms. 2023. Vol. 423. Springer, Cham. P. 227--242.
\end{thebibliography}

% После библиографического списка в русскоязычных статьях необходимо оформить
% англоязычный заголовок.




%\end{document}

