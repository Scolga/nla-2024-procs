\begin{englishtitle} % Настраивает LaTeX на использование английского языка
% Этот титульный лист верстается аналогично.
\title{Hidden stability boundaries of one system with monotonic nonlinearity in the Hurwitz sector}
% First author
\author{Igor Burkin\inst{1}  \and  Oksana Kuznetsova\inst{2}
%  \and
%  Name FamilyName3\inst{1}
}
\institute{TulSU, Tula, Russia\\
  \email{i-burkin@yandex.ru}
  \and
TSPU, Tula, Russia\\\
\email{oxxy4893@mail.ru}
}
% etc

\maketitle

\begin{abstract}
Estimates are found for hidden stability boundaries in the parameter space of a system with monotonic nonlinearity that satisfies the assumptions of the Kalman conjecture. 

\keywords{Kalman conjecture, global stability, hidden boundaries of stability} % в конце списка точка не ставится
\end{abstract}
\end{englishtitle}

\iffalse
%%%%%%%%%%%%%%%%%%%%%%%%%%%%%%%%%%%%%%%%%%%%%%%%%%%%%%%%%%%%%%%%%%%%%%%%
%
%  This is the template file for the 6th International conference
%  NONLINEAR ANALYSIS AND EXTREMAL PROBLEMS
%  June 25-30, 2018
%  Irkutsk, Russia
%
%%%%%%%%%%%%%%%%%%%%%%%%%%%%%%%%%%%%%%%%%%%%%%%%%%%%%%%%%%%%%%%%%%%%%%%%


\documentclass[12pt]{llncs}  

% При использовании pdfLaTeX добавляется стандартный набор русификации babel.

\usepackage{iftex}

\ifPDFTeX
\usepackage[T2A]{fontenc}
\usepackage[utf8]{inputenc} % Кодировка utf-8, cp1251 и т.д.
\usepackage[english,russian]{babel}
\fi

\usepackage{todonotes} 

\usepackage[russian]{nla}
\newcommand{\sign}{\mathop{\mathrm{sign}}}
\begin{document}
\fi

\title{Скрытые границы устойчивости одной системы с монотонной нелинейностью в гурвицевом секторе\thanks{Работа выполнена в рамках государственного задания Министерства просвещения РФ соглашение № 073-00033-24-01 от 09.02.2024 тема научного исследования «Теоретико-числовые методы в приближенном анализе и их приложения в механике и физике».}}
% Первый автор
\author{И.~М.~Буркин\inst{1} \and О.~И.~Кузнецова\inst{2}
}

% Аффилиации пишутся в следующей форме, соединяя каждый институт при помощи \and.
\institute{ТулГУ, Тула, Россия \\
  \email{i-burkin@yandex.ru}
  \and   % Разделяет институты и присваивает им номера по порядку.
ТГПУ им. Л.~Н. Толстого, Тула, Россия\\
  \email{oxxy4893@mail.ru}
% \and Другие авторы...
}

\maketitle

\begin{abstract}
Найдены оценки скрытых границ устойчивости в пространстве параметров системы с монотонной нелинейностью, удовлетворяющей предположениям гипотезы Калмана.

\keywords{гипотеза Калмана, глобальная устойчивость, скрытые границы устойчивости} % в конце списка точка не ставится
\end{abstract}

%\section{Основные результаты} % не обязательное поле

Центральными направлениями изучения динамики нелинейных систем являются задачи выявления всех нетривиальных (колебательных) аттракторов в фазовом пространстве или доказательство их отсутствия (глобальной устойчивости, когда все траектории притягиваются к стационарному множеству, состоящему из тривиальных аттракторов). Границей глобальной устойчивости в пространстве параметров является граница замыкания множества точек в пространстве параметров системы, для которых система не является глобально устойчивой, а точки границы глобальной устойчивости являются точками бифуркации рождения незатухающих колебаний. Точка границы называется скрытой, если для некоторой ее окрестности в пространстве параметров потеря глобальной устойчивости вызвана только глобальными бифуркациями рождения скрытых колебаний, область притяжения которых в фазовом пространстве не касается неустойчивых состояний равновесия. В противном случае точка называется тривиальной. В данной работе приведен пример, который показывает трудности определения точной границы глобальной устойчивости, а также получения ее внутренних и внешних оценок, связанные с нелокальным рождением скрытых аттракторов. 

В работе рассмотрена система в форме Лурье
\begin{equation}\label{1}
\begin{aligned}
\dot x=Ax+b\varphi (\sigma), \sigma=c^Tx,
\end{aligned}%\eqno(1)
\end{equation}
где $A$ – постоянная вещественная $n \times n$ - матрица,  $b$ и $c$ – постоянные $n$ - векторы,   $\varphi (\sigma)$ – скалярная непрерывная функция. Передаточная функция системы $W(p)=c^T(A-pI)^{-1}b$  имеет вид  $W(p)=(p^3+2p^2)(p^4+p^3+ap^2+p+2)^{-1}$, где  $a \geq 3$ – параметр. Функция $\varphi (\sigma)$ предполагается кусочно-дифференцируемой и в точках дифференцируемости удовлетворяющей условиям  $0<\varphi' (\sigma) \leq k$. Для такой системы сектор  $((3-a)(a-2)^{-1}, \infty)$ является сектором гурвицевости.  Условие на нелинейность означает, что рассматриваемая система удовлетворяет предположениям известной в теории автоматического управления  гипотезы Калмана [1]. Согласно этой гипотезе такая система с любой нелинейностью, удовлетворяющей указанному выше условию, является глобально асимптотически устойчивой. 

	В настоящее время хорошо известно, что гипотеза Калмана для систем (1), порядок которых превышает 3, вообще говоря, неверна. Например, в работе [2] приведен пример системы, удовлетворяющей всем предположениям этой гипотезы,  и содержащей сосуществующие хаотические и периодические аттракторы. 

	Для рассматриваемой здесь системы в пространстве параметров  $(a, k)$ в диапазоне изменения параметра  $a \in [3, 5]$  найдены оценки верхней и нижней границ её глобальной устойчивости.  Для нахождения условий абсолютной устойчивости системы применен частотный критерий В.А. Якубовича. Для поиска скрытых аттракторов системы (скрытых циклов) применен прием продолжения по параметру, использованный в работе [2]. В качестве нелинейности  $\varphi (\sigma)$  в системе (1) рассматривается функция 

\[ \varphi (k, \sigma) =
  \begin{cases}
    0.1\sigma, |\sigma| \leq 0.2, \\
   k \sigma +(0.02-0.2k)\sign \sigma, 0.2<|\sigma|<0.4, \\
	0.1\sigma+(0.2k-0.02)\sign \sigma, |\sigma| \geq0.4.
  \end{cases}
\]
Производная этой функции при любом значении $k>0$ и $a \in [3, 5]$  находится в секторе Гурвица. Для каждого значения  параметра $a$  из указанного диапазона удается найти значения $k=k_1$ и $k=k_2$ такие, что  при $k<k_1$ система абсолютно устойчива, а при $k=k_2$ она имеет цикл. Найденные значения  $k_1$ и $k_2$ дают, соответственно, оценки внутренней  и внешней границ устойчивости для фиксированного значения параметра   $a$. 

	%Результаты исследования представлены в приведенной ниже таблице. Для %каждого из указанных значений параметра  $a$  сектор $(0, k)$ является сектором %абсолютной устойчивости в классе монотонных нелинейностей, в то время как %среди монотонных кусочно-дифференцируемых нелинейностей   $\varphi (\sigma)$, %для которых в некоторых точках выполняется условие  $\varphi' (\sigma) \geq k_0$, %можно указать такую функцию, для которой  система (1) имеет нетривиальный %аттрактор (цикл). 

%\begin{table}
%\caption{}
%\begin{tabular}{ | m{1cm} | m{1cm}| m{1cm} | m{1cm} | m{1cm} | m{1cm} |m{1cm} %| m{1cm}  |m{1cm}  |  m{1cm} | m{1cm} | m{1cm} |  m{1cm} | m{1cm} | m{1cm} |} 
%  \hline
%  $a$ & 3 & 3.05 & 3.1 & 3.15 & 3.2 &3.25	&3.3	&3.35	&3.4	&3.45	&3.5& 3.55&	%3.6&	3.65\\ 
%  \hline
%  $k$ & 0.36 &	0.52 & 	0.58	& 0.64	 & 0.7& 0.75&	0.79&	0.84&	0.88&	0.92	%&0.96& 1.07&	1.35	&1.53\\ 
%  \hline
%  $k_0$ & 1.57	 & 1.80 & 2.10	& 2.28	& 2.55 &2.85	&3.15	&3.35	&3.47	&3.59	%&3.72& 3.85&	4	&4.14\\ 
%  \hline
%\end{tabular}

%\begin{tabular}{ | m{1cm} | m{1cm}| m{1cm} | m{1cm} | m{1cm} | m{1cm} |m{1cm} %| m{1cm}  |m{1cm}  |  m{1cm} | m{1cm} | m{1cm} |  m{1cm} | m{1cm} | m{1cm} |} 
%  \hline
%  $a$ & 3.7 & 3.75	&3.8	&3.85	&3.9	&3.95	&4.0&	4.05&	4.1&	4.15&	4.2	%&4.25&4.3&	4.35\\ 
%  \hline
%  $k$ & 1.59 &	1.64	&1.70&	1.75&	1.83	&1.89	&1.97	&2.03	&2.09	&2.15	&2.21	%&2.27&2.33	&2.38\\ 
%  \hline
%  $k_0$ & 4.30	 & 4.48	&4.67	&4.89&	5.20&	5.39	&5.67	&6.0	&6.28&	6.6&	6.95	%&7.30&7.63	&7.97\\ 
%  \hline
%\end{tabular}

%\begin{tabular}{ | m{1cm} | m{1cm}| m{1cm} | m{1cm} | m{1cm} | m{1cm} |m{1cm} %| m{1cm}  |m{1cm}  |  m{1cm} | m{1cm} | m{1cm} |  m{1cm} | m{1cm} | } 
%  \hline
%  $a$ & 4.4	&4.45 & 4.5	&4.55	&4.6&	4.65	&4.7&	4.75	&4.8&	4.85	&4.9&	%4.95&	5.0\\ 
%  \hline
%  $k$ & 2.45&	2.50 & 	2.56&	2.61	&2.67&	2.73&	2.78	&2.84	&2.89	&2.95	&3.01&	%3.06	&3.12\\ 
%  \hline
%  $k_0$ & 8.35	&8.70 & 9.10	&9.44&	9.9&	10.22&	10.63	&11.05&	11.45	&11.89&	%12.35	&12.81&	13.25\\ 
%  \hline
%\end{tabular}
%\end{table} 

В процессе исследования также обнаружено, что при некоторых фиксированных значениях параметра $a$  из указанного диапазона и различных значениях параметра    $k>k_2$ в системе могут наблюдаться несколько сосуществующих циклов. Эти циклы рождаются в результате нелокальных бифуркаций, поскольку единственное состояние равновесия системы всегда остается локально устойчивым.
% Рисунки и таблицы оформляются по стандарту класса article. Например,

%\begin{figure}[htb]
%  \centering
  % Поддерживаются два формата:
  %\includegraphics[width=0.7\linewidth]{figure.pdf} % Растровый формат
  %\includegraphics[width=0.7\linewidth]{figure.png} % Векторный и растровый формат
  %
  
%  \begin{center}
%    \missingfigure[figwidth=0.7\linewidth]{Уберите меня из статьи!}
%  \end{center}
%  \caption{Заголовок рисунка}\label{fig:example}
%\end{figure}


% В конце текста можно выразить благодарности, если этого не было
% сделано в ссылке с заголовка статьи, например,
%Работа выполнена в рамках государственного задания Министерства просвещения РФ %соглашение № 073-00033-24-01 от 09.02.2024 тема научного исследования «Теоретико-%числовые методы в приближенном анализе и их приложения в механике и физике».
%



\begin{thebibliography}{9} % или {99}, если ссылок больше десяти.
\bibitem{Kalman} Kalman~R.E. Physical and mathematical mechanisms of instability in nonlinear automatic control systems~// Transactions of ASME. 1957. Vol.~79(3). Pp.~553–566.
%Гантмахер~Ф.Р. Теория матриц. М.:~Наука,~1966.

\bibitem{Burkin} Burkin~I.M., Kuznetsov~N.V., Mokaev~T.N. Coexisting Chaotic and Periodic Attractors in a Counterexample to the Kalman Conjecture~// 16th Int. Conf. <<Stability and Oscillations of Nonlinear Control Systems (Pyatnitskiy's Conference)>>. Moscow, 2022. Pp.~1-4.
%Современные численные методы решения обыкновенных дифференциальных уравнений~/ %Под~ред.~Дж.~Холл, Дж.~Уатт. М.:~Мир,~1979.

%\bibitem{Якубович} Якубович~В.А. Частотные критерии абсолютной устойчивости %нелинейных систем автоматического регулирования~// Тр. Межвузовской конф. по %прикл. теории устойчивости и аналит. механике. Казань, 1964. С.~123-134.
%Александров~А.Ю. Об устойчивости сложных систем в критических случаях~// Автоматика %и телемеханика. 2001. \textnumero~9. С.~3--13.

%\bibitem{Moreau1977} Moreau~J.-J. Evolution problem associated with a moving convex set in a %Hilbert space~// J.~Differential~Eq. 1977. Vol.~26. Pp.~347--374.

%\bibitem{Semenov} Семенов~А.А. Замечание о вычислительной сложности известных %предположительно односторонних функций~// Тр.~XII Байкальской междунар. конф. %<<Методы оптимизации и их приложения>>. Иркутск, 2001. С.~142--146.

\end{thebibliography}

% После библиографического списка в русскоязычных статьях необходимо оформить
% англоязычный заголовок.

\begin{englishtitle} % Настраивает LaTeX на использование английского языка
% Этот титульный лист верстается аналогично.
\title{Hidden stability boundaries of one system with monotonic nonlinearity in the Hurwitz sector}
% First author
\author{Igor Burkin\inst{1}  \and  Oksana Kuznetsova\inst{2}
%  \and
%  Name FamilyName3\inst{1}
}
\institute{TulSU, Tula, Russia\\
  \email{i-burkin@yandex.ru}
  \and
TSPU, Tula, Russia\\\
\email{oxxy4893@mail.ru}
}
% etc

\maketitle

\begin{abstract}
Estimates are found for hidden stability boundaries in the parameter space of a system with monotonic nonlinearity that satisfies the assumptions of the Kalman conjecture. 

\keywords{Kalman conjecture, global stability, hidden boundaries of stability} % в конце списка точка не ставится
\end{abstract}
\end{englishtitle}


%\end{document}

