\begin{englishtitle} % Настраивает LaTeX на использование английского языка
% Этот титульный лист верстается аналогично.
\title{On the properties of the collocation-variational approach for solving differential algebraic equations}
% First author
\author{Mikhail Bulatov
  \and
  Liubov Solovarova
	}
  

\institute{ISDCT SB RAS, Irkutsk, Russia\\
  \email{mvbul@icc.ru, soleilu@mail.ru}}
% etc

\maketitle

\begin{abstract}
The report deals with the initial value problem for systems of ordinary differential equations with an identically singular matrix in front of the derivative. For such systems, collocation-variational difference schemes have been proposed, the construction of which is based on solving a special type of mathematical programming problem. Calculations of model examples are given.

\keywords{differential algebraic equations, numerical methods, index} % в конце списка точка не ставится
\end{abstract}
\end{englishtitle}

\iffalse

%%%%%%%%%%%%%%%%%%%%%%%%%%%%%%%%%%%%%%%%%%%%%%%%%%%%%%%%%%%%%%%%%%%%%%%%
%
%  This is the template file for the 6th International conference
%  NONLINEAR ANALYSIS AND EXTREMAL PROBLEMS
%  June 25-30, 2018
%  Irkutsk, Russia
%
%%%%%%%%%%%%%%%%%%%%%%%%%%%%%%%%%%%%%%%%%%%%%%%%%%%%%%%%%%%%%%%%%%%%%%%%


\documentclass[12pt]{llncs}  

% При использовании pdfLaTeX добавляется стандартный набор русификации babel.

\usepackage{iftex}

\ifPDFTeX
\usepackage[T2A]{fontenc}
\usepackage[utf8]{inputenc} % Кодировка utf-8, cp1251 и т.д.
\usepackage[english,russian]{babel}
\fi

\usepackage{todonotes} 

\usepackage[russian]{nla}

\begin{document}
\fi

\title{О свойствах коллокационно-вариационного подхода для решения дифференциально-алгебраических уравнений}
% Первый автор
\author{М.~В.~Булатов  
  \and  
% Второй автор
  Л.~С.~Соловарова
}

\institute{ИДСТУ СО РАН, Иркутск, Россия \\
  \email{mvbul@icc.ru,soleilu@mail.ru}
% \and Другие авторы...
}



\maketitle

\begin{abstract}
В докладе рассмотрена начальная задача для систем обыкновенных дифференциальных уравнений с тождественно вырожденной матрицей перед производной. Для таких систем предложены коллокационно-вариационные разностные схемы, построение которых основано на решении задачи математического программирования специального вида. Приведены расчеты модельных примеров.

\keywords{дифференциально-алгебраические уравнения, численные методы, индекс} % в конце списка точка не ставится
\end{abstract}

\section{Основные результаты} % не обязательное поле


В докладе рассмотрена задача
$$
A(t) x^{'}(t) + B(t)x(t) = f(t), \; x(0) = x_{0}, \; t \in [0,1], 
$$
где $A(t)$, $B(t)$ -- $(n \times n)$-матрицы, $f(t)$ и $x(t)$~-- заданная и искомая $n$-мерные вектор-функции, элементы матриц $A(t)$, $B(t)$ и $f(t)$ достаточно гладкие, и
  $$detA\equiv0.$$ Такие задачи принято называть дифференциально-алгебраическими уравнениями (ДАУ). Предполагается, что начальное условие задано корректно, а из исходной задачи и $r$ ее производных путем элементарных преобразований можно выделить классическую систему обыкновенных дифференциальных уравнений $x^{'}(t)+\bar{B}(t)x(t)=\bar{f}(t)$. Значение $r$ принято называть индексом рассматриваемой задачи. Многие известные неявные методы могут порождать неустойчивый процесс или принципиально неприменимы для ДАУ.
	
	Авторы для таких задач предлагают коллокационно-вариационные разностные схемы, которые имеют принципиальное отличие от классических алгоритмов. Описан общий подход к созданию коллокационно-вариационных разностных схем, основанный на построении задачи квадратичного программирования специального вида. Приведены конкретные алгоритмы с одной и двумя точками коллокации и результаты численных расчетов \cite{BulSol1},\cite{BulSol2} для задач индекса два, задач, содержащих жесткие компоненты.


% В конце текста можно выразить благодарности, если этого не было
% сделано в ссылке с заголовка статьи, например,

%



\begin{thebibliography}{9} % или {99}, если ссылок больше десяти.

\bibitem{BulSol1} Bulatov~M.V., Solovarova~L.S. Collocation-variation difference schemes for differential-algebraic equations~// Mathematical Methods in the Applied Sciences. 2018. Vol.~41, No. 18. P. 9048--9056.

\bibitem{BulSol2} Bulatov~M., Solovarova~L. Collocation-variation difference schemes with several collocation points for differential-algebraic equations~// Applied Numerical Mathematics. 2020. Vol.~149. P.~153--163. 

\end{thebibliography}

% После библиографического списка в русскоязычных статьях необходимо оформить
% англоязычный заголовок.




%\end{document}

