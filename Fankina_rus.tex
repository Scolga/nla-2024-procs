\begin{englishtitle} % Настраивает LaTeX на использование английского языка
% Этот титульный лист верстается аналогично.
\title{On junction problem for  elastic inclusions\\in an elastic body}
% First author
\author{Irina Fankina
%  \and
%  Name FamilyName2\inst{2}
%  \and
%  Name FamilyName3\inst{1}
}
\institute{Novosibirsk State University, Novosibirsk, Russia\\
  \email{fankina.iv@gmail.com}
%  \and
%Affiliation, City, Country\\
%\email{email@example.com}
}
% etc

\maketitle

\begin{abstract}
The equilibrium problems for a 2D elastic body with thin elastic inclusions with a junction at a point are considered. It is assumed that a crack exists between the inclusions and the body. Inequality-type boundary conditions are imposed at the crack faces to prevent mutual penetration. The problems depend on various rigidity parameters of inclusions: we are talking about two families of problems. A convergence of solutions of the families of problems in suitable function spaces is proved. By this convergence, we pass to the limit in the problems and establish the form of limit problems. Optimal control problems are considered, the control parameters are the inclusion rigidity parameters.

\keywords{crack, elastic body, thin inclusion, junction conditions, variational inequality} % в конце списка точка не ставится
\end{abstract}
\end{englishtitle}

\iffalse
%%%%%%%%%%%%%%%%%%%%%%%%%%%%%%%%%%%%%%%%%%%%%%%%%%%%%%%%%%%%%%%%%%%%%%%%
%
%  This is the template file for the 6th International conference
%  NONLINEAR ANALYSIS AND EXTREMAL PROBLEMS
%  June 25-30, 2018
%  Irkutsk, Russia
%
%%%%%%%%%%%%%%%%%%%%%%%%%%%%%%%%%%%%%%%%%%%%%%%%%%%%%%%%%%%%%%%%%%%%%%%%


\documentclass[12pt]{llncs}

% При использовании pdfLaTeX добавляется стандартный набор русификации babel.

\usepackage{iftex}

\ifPDFTeX
\usepackage[T2A]{fontenc}
\usepackage[utf8]{inputenc} % Кодировка utf-8, cp1251 и т.д.
\usepackage[english,russian]{babel}
\fi

\usepackage{todonotes}

\usepackage[russian]{nla}

\begin{document}
\fi

\title{О задаче сопряжения упругих включений \\в упругом теле\thanks{Работа выполнена при поддержке Математического Центра в Академгородке, соглашение № 075-15-2022-282 с Министерством науки и высшего образования Российской Федерации.}}
% Первый автор
\author{И.~В.~Фанкина%\inst{1}
 % \and
% Второй автор
%  И.~О.~Фамилия\inst{2}
%  \and
% Третий автор
 % И.~О.~Фамилия\inst{1}
}

% Аффилиации пишутся в следующей форме, соединяя каждый институт при помощи \and.
\institute{Новосибирский государственный университет, Новосибирск, Россия \\
  \email{fankina.iv@gmail.com}
 % \and   % Разделяет институты и присваивает им номера по порядку.
%Институт (название в краткой форме), Город, Страна\\
 % \email{email@example.com}
% \and Другие авторы...
}

\maketitle

\begin{abstract}
Рассматриваются задачи равновесия двумерного упругого тела с тонкими упругими включениями, сопрягающимися в точке. Предполагается наличие трещины отслоения с условиями непроникания между включениями и телом. Задачи зависят от различных параметров жесткости включений: речь идет о двух семействах задач. Доказывается сходимость решений указанных семейств задач в подходящих функциональных пространствах, с помощью которой осуществляются предельные переходы в задачах и устанавливается вид предельных задач. Также рассматриваются задачи оптимального управления,  в которых управляющими параметрами служат параметры жесткости включений.


\keywords{трещина, упругое тело, тонкое включение, условия сопряжения, вариационное неравенство } % в конце списка точка не ставится
\end{abstract}

%\section{Основные результаты} % не обязательное поле

В докладе будет рассмотрена задача о равновесии двумерного упругого тела с двумя тонкими упругими включениями при наличии отслоения. Наличие отслоения означает, что между включениями и упругим телом есть трещина. Для описания трещины применяются нелинейные краевые условия, исключающие взаимное проникновение ее берегов, что приводит к задаче с неизвестным множеством контакта. Поведение тонких включений описывается в рамках моделей балки Бернулли -- Эйлера и балки Тимошенко. Предполагается, что включения контактируют в точке. В работе \cite{KhP2016} получен полный набор краевых условий в точке сопряжения включений. Кроме того, установлено, что поставленная задача имеет решение, а также получены формулировки задач, отвечающих предельным значениям параметра жесткости включения Бернулли -- Эйлера в изотропном случае.

В докладе будут предложены результаты, связанные с предельным переходом в семействе задач указанного выше типа, зависящих от параметров жесткости анизотропного включения Тимошенко. Параметрами жесткости являются параметр $\alpha$, пропорциональный модулю сдвига включения Тимошенко,  и параметр $\beta$, пропорциональный модулю упругости включения Тимошенко в продольном и поперечном направлении.
По каждому из параметров в семействе задач с помощью вариационного подхода осуществлены предельные переходы при стремлении $\alpha$ и $\beta$ к бесконечности; получены вариационные и дифференциальные формулировки предельных задач. Кроме того, в обоих случаях доказаны сильные сходимости решений семейств задач равновесия, на основе которых установлено существование решений задач оптимального управления. Указанные задачи сформулированы в соответствии с критерием разрушения Гриффитса, и параметры $\alpha$ и $\beta$ в них являются управляющими.


% Рисунки и таблицы оформляются по стандарту класса article. Например,




% В конце текста можно выразить благодарности, если этого не было
% сделано в ссылке с заголовка статьи, например,
%Работа выполнена при поддержке Математического Центра в Академгородке, соглашение № 075-15-2022-282 с Министерством науки и высшего образования Российской Федерации.
%



\begin{thebibliography}{9} % или {99}, если ссылок больше десяти.
\bibitem{KhP2016} Khludnev~A.M. and Popova~T.S. Junction problem for Euler-Bernoulli and Timoshenko elastic inclusions in elastic bodies~// Journal: Quart. Appl. Math. 2016. Vol.~74. Pp.~705--718.
\end{thebibliography}

% После библиографического списка в русскоязычных статьях необходимо оформить
% англоязычный заголовок.




%\end{document}

