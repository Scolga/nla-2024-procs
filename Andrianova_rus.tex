\begin{englishtitle} % Настраивает LaTeX на использование английского языка
% Этот титульный лист верстается аналогично.
\title{Reduction of computational complexity for solving the non-guillotine placement problem on an exact model}
% First author
\author{Anastasiya Andrianova 
}
\institute{Kazan Federal University, Kazan, Russia\\
  \email{Anastasiya.Andrianova@kpfu.ru}
% etc
}
\maketitle

\begin{abstract}
The paper proposes an approach to reducing the computational complexity of solving the problem of non-guillotine placement of a set of parts on a half-strip based on an exact model in the form of a high-dimensional partially Boolean linear programming problem.

\keywords{non-guillotine placement of a set of rectangles, partial Boolean linear programming problem, branch and bound method} % в конце списка точка не ставится
\end{abstract}
\end{englishtitle}

\iffalse
%%%%%%%%%%%%%%%%%%%%%%%%%%%%%%%%%%%%%%%%%%%%%%%%%%%%%%%%%%%%%%%%%%%%%%%%
%
%  This is the template file for the 6th International conference
%  NONLINEAR ANALYSIS AND EXTREMAL PROBLEMS
%  June 25-30, 2018
%  Irkutsk, Russia
%
%%%%%%%%%%%%%%%%%%%%%%%%%%%%%%%%%%%%%%%%%%%%%%%%%%%%%%%%%%%%%%%%%%%%%%%%


\documentclass[12pt]{llncs}  

% При использовании pdfLaTeX добавляется стандартный набор русификации babel.

\usepackage{iftex}

\ifPDFTeX
\usepackage[T2A]{fontenc}
\usepackage[utf8]{inputenc} % Кодировка utf-8, cp1251 и т.д.
\usepackage[english,russian]{babel}
\fi

\usepackage{todonotes} 

\usepackage[russian]{nla}

\begin{document}
\fi

\title{Уменьшение вычислительной сложности при решении задачи негильотинного размещения с помощью точной модели\thanks{ Работа выполнена за счет средств Программы стратегического академического лидерства Казанского
(Приволжского) федерального университета (<<ПРИОРИТЕТ-2030>>)}}
% Первый автор
\author{А.~А.~Андрианова
}

% Аффилиации пишутся в следующей форме, соединяя каждый институт при помощи \and.
\institute{Казанский (Приволжский) федеральный университет, Казань, Россия \\
  \email{Anastasiya.Andrianova@kpfu.ru}
}

\maketitle

\begin{abstract}
В работе предлагается подход к уменьшению вычислительной сложности решения задачи негильотинного размещения набора деталей на полуполосе на основе точной модели в виде частично булевой задачи линейного программирования большой размерности.

\keywords{негильотинное размещение набора прямоугольников, задача частично булевого линейного программирования, метод ветвей и границ} % в конце списка точка не ставится
\end{abstract}

Задача негильотинного размещения набора из $n$ прямоугольников заданных размеров ($a_i \times b_i$, $i=1\ldots n$) на полуполосе заданной ширины $B$ решается на сегодняшний день, в основном, эвристическими алгоритмами. Тем не менее, существуют модели, которые могут позволить получить точное решение задачи, однако, в основном они имеют вид труднорешаемых задач. Например, такая модель в виде задачи частично булевого линейного программирования большой размерности была поставлена в \cite{Andr_Faz_Much}. Вид данной модели следующий:
$$A\to min$$
$$x_i +(b_i-a_i)z_i\le A-a_i \;  i=1\ldots n$$
$$y_i +(a_i-b_i)z_i\le B-b_i \; i=1\ldots n$$
$$-x_i+x_j -(b_i-a_i)z_i +\bar{A}t_{ij}+\bar{A}s_{ij}\ge a_i \; i=1\ldots n-1, j=i+1\ldots n $$
$$x_i-x_j -(b_j-a_j)z_j +\bar{A}t_{ij}-\bar{A}s_{ij}\ge a_j-\bar{A} \; i=1\ldots n-1, j=i+1\ldots n $$
$$-y_i+y_j -(a_i-b_i)z_i -\bar{B}t_{ij}+\bar{B}s_{ij}\ge b_i-\bar{B} \; i=1\ldots n-1, j=i+1\ldots n $$
$$y_i-y_j -(a_j-b_j)z_j -\bar{B}t_{ij}-\bar{B}s_{ij}\ge b_j-2\bar{B} \; i=1\ldots n-1, j=i+1\ldots n $$
$$x_i \ge 0 \; i=1\ldots n$$
$$y_i \ge 0 \; i=1\ldots n$$
$$z_i \in\{0,1\} \; i=1\ldots n$$
$$s_{ij}\in \{0,1\} \: i=1\ldots n-1, j=i+1\ldots n$$
$$t_{ij}\in \{0,1\} \: i=1\ldots n-1, j=i+1\ldots n$$
%
Здесь переменные $x_i, y_i$ означают координтаы размещения $i$-ого прямоугольника, $z_i$ - перемернная, означающая вертикальную или горизонтальную ориентацию. Набор булевых переменных $s_{ij}$ и $t_{ij}$ предназначены для определения относительного расположения каждой пары деталей друг относительно друга и неналожения их друг на друга - один набор определяет отношение предшествования, другой определяет по горизонтали или по вертикали нужно разделить местоположения пары деталей. Переменная $A$ определяет неизвестную длину полуполосы, необходимую для размещения набора прямоугольников. Константы $\bar{A}$ и $\bar{B}$ выбираются не меньшими размеров листа, достаточных для размещения всего набора прямоугольников.

Решение данной модели представляет достаточно большие вычислительные сложности. Так, для $n=10$ в задаче будет присутствовать 20 обычных переменных и 100 булевых переменных. Для решения задачи можно воспользоваться, например, методом Лэнд и Дойг, который является универсальным алгоритмом на основе метода ветвей и границ. Эксперименты показывают, что приемлемое время решения задачи показывают только задачи при $n\le 5$ (среднее время решения не превышает 10 минут). Для задач с $n=6$ уже встречаются задачи, которые решались за время более 3 часов. Поэтому решение задачи на основе точной модели не представляется возможным для реальных задач, возникающих, например, в промышленности.

Предлагается многоэтапный эвристический подход сведения задачи размерности $n\ge 6$ к серии задач размера $n\le 5$, который требует значительно меньших вычислений, тем не менее, использует точную модель для решения небольших задач.

Идея подхода основывается на разбиении набора прямоугольников на поднаборы размера $n\le 5$. Пусть выделен поднабор из 5 прямоугольников $N=\{i_1, i_2\ldots i_5\}$. Для каждого такого поднабора ставится задача оптимизации длины полуполосы как точная модель частично булевого линейного программирования и решается для получения оптимального размещения на полуполосе ширины $B$. Для каждого поднабора высчитываются отходы материала $P_N$ применения такого размещения как разница между плошадью листа, который получается в результате решения задачи и суммарной площади всех размещенных  деталей. 

Для получения решения исходной задачи необходимо найти разбиение полного набора прямоугольников на поднаборы с минимальными суммарными отходами. Эта задача также является вычислительно сложной. Тем не менее, чтобы избежать полного перебора разбиений набора прямоугольников на поднаборы можно использовать различные эвристические процедуры, включая алгоритмы локального поиска и генетические алгоритмы. 

Еще большего эффекта можно добиться, если поднабор размещается не на полуполосе такого же размера, как в исходной задаче, а на полуполосе меньшего размера, например, можно устанавливать ширину полуполосы немного большей максимального размера детали из поднабора. Таким образом, каждый поднабор умещается на лист, размеры которого определяют <<деталь-контейнер>>. Определив для всех поднаборов размеры <<деталей-контейнеров>> можно перейти к решению задачи размещения <<деталей-контейнеров>> на полуполосе, которую также можно решать с помощью точной модели частично булевой задачи линейного программирования. Таким образом, получается иерархическая группировка исходных деталей в <<детали-контейнеры>> и применение точной модели только в случае, когда количество деталей в задаче не будет превышать 5 штук.

\begin{thebibliography}{9} % или {99}, если ссылок больше десяти.
\bibitem{Andr_Faz_Much} Андрианова А.А., Мухтарова Т.М., Фазылов В.Р. Модели негильотинного размещения
набора прямоугольных деталей на листе и полуполосе~// Учен. зап. Казан. ун-та. Сер. Физ.матем. науки. 2013. Т. 155, кн. 2. С.~5–-18.

\end{thebibliography}

% После библиографического списка в русскоязычных статьях необходимо оформить
% англоязычный заголовок.



%\end{document}

