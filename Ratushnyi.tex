
\iffalse
\documentclass[12pt]{llncs}  

\usepackage{iftex}

\ifPDFTeX
\usepackage[T2A]{fontenc}
\usepackage[utf8]{inputenc}
\usepackage[english,russian]{babel}
\fi


% Добавляется корректирующий стилевой файл строго после babel, если он был включен.
% В параметре необходимо указать russian, что установит не совсем стандартные названия
% разделов текста, настроит переносы для русского языка как основного и т.п.

\usepackage{todonotes} % Этот пакет нужен для верстки данного шаблона, его
                       % надо убрать из вашей статьи.

\usepackage[russian]{nla}


% Было б удобно при верстке сборника, чтобы названия рисунков разных авторов не пересекались.
% Чтоб минимизировать такое пересечение, рисунки нужно поместить в отдельную подпапку с
% названием - фамилией автора или названием статьи.
%
% \graphicspath{{ivanov-petrov-pics/}} % Указание папки с изображениями в форматах png, pdf.
% или
% \graphicspath{{great-problem-solving-paper-pics/}}.


\begin{document}
\fi
% Текст оформляется в соответствии с классом article, используя дополнения AMS.

\title{Шаблонизация для динамической задачи упаковки в контейнеры с группами размещения\thanks{Исследование выполнено в рамках государственного задания ИМ СО РАН (проект FWNF–2022–0019).}}

\author{
А.~В.~Ратушный\inst{1} \and A.~A.~Панин\inst{2}  \and Е.~А.~Бражников\inst{3}
}


\institute{
ИМ СО РАН, Новосибирск, Россия\\
\email{alexeyratushny@gmail.com}
  \and
ИМ СО РАН, Новосибирск, Россия\\
\email{aapanin1988@gmail.com}
  \and
Новосибирский Государственный Университет, Новосибирск, Россия\\
\email{brazhnikov.eugene@gmail.com}
}

\maketitle

\begin{abstract}

Мы рассматриваем NP-трудную задачу динамической упаковки в контейнеры, которая возникла в облачных вычислениях.
Каждый предмет представляет собой виртуальную машину, которая определяется временными метками её создания и удаления, а также вектором запросов на выделение ресурсов.
Виртуальные машины делятся на две категории: большие и маленькие.
Каждый контейнер представляет собой сервер, состоящий из нескольких NUMA-узлов.
Маленькая виртуальная машина может быть полностью размещена на одном узле, в то время как большая виртуальная машина разделяется на две идентичные половины, которые размещаются на разных узлах.
Серверы располагаются в стойках фиксированного размера.
Некоторые виртуальные машины объединены в группы, каждая из которых далее разделяется на подгруппы, называемые партициями.
Виртуальные машины из разных партиций в рамках одной группы конфликтуют друг с другом, т.е. они не могут совместно размещаться на серверах из одной стойки в один и тот же момент времени.
Это ограничение имеет решающее значение для обеспечения отказоустойчивости системы.
Наша цель состоит в том, чтобы эффективно разместить все виртуальные машины в минимальное количество стоек.
Для решения задачи мы используем метод генерации столбцов для создания шаблонов упаковки виртуальных машин, которые затем распределяются по серверам.
Шаблоны адаптированы таким образом, чтобы учитывать конфликты, связанные с существованием партиций.
Эффективность подхода протестирована на обширном открытом наборе примеров \cite{benchmark}, которые содержат до 70 тысяч виртуальных машин.

\keywords{упаковка в контейнеры, конфликты, генерация столбцов, распределение ресурсов, жадный алгоритм} % в конце списка точка не ставится
\end{abstract}


% Список литературы оформляется подобно ГОСТ-2008.
% Примеры оформления находятся по этому адресу -
%     https://narfu.ru/agtu/www.agtu.ru/fad08f5ab5ca9486942a52596ba6582elit.html
%

\begin{thebibliography}{9} % или {99}, если ссылок больше десяти.

\bibitem{benchmark}
Temporal Bin Packing Problem with Placement Groups. \url{http://old.math.nsc.ru/AP/benchmarks/Temporal\%20Bin\%20Packing/binpack.html}

% \bibitem{Gantmakher} Гантмахер~Ф.Р. Теория матриц. М.:~Наука,~1966.

\end{thebibliography}
\hfill \break

% После библиографического списка в русскоязычных статьях необходимо оформить
% англоязычный заголовок.

\begin{englishtitle} % Настраивает LaTeX на использование английского языка
% Этот титульный лист верстается аналогично.
\title{
Template-based Approach to Dynamic Bin Packing with Placement Groups
}


\author{
A. V. Ratushnyi\inst{1}  \and A. A. Panin\inst{2}  \and E. A. Brazhnikov\inst{3}
}
\institute{
Sobolev Institute of Mathematics of SB RAS, Novosibirsk, Russia\\
\email{alexeyratushny@gmail.com}
    \and
Sobolev Institute of Mathematics of SB RAS, Novosibirsk, Russia\\
\email{aapanin1988@gmail.com}
    \and
Novosibirsk State University, Novosibirsk, Russia\\
\email{brazhnikov.eugene@gmail.com}
}
\maketitle


\begin{abstract}

We consider an NP-hard temporal bin packing problem that arises in cloud computing.
Each item represents a virtual machine, defined by its creation and deletion timestamps, as well as a vector of resource allocation requests.
Virtual machines are divided into two categories: large and small.
Each container represents a server consisting of multiple NUMA nodes.
A small virtual machine can be fully placed on one node, while a large virtual machine is split into two identical halves that are placed on different nodes.
Servers are located in fixed-size racks.
Some virtual machines are grouped together, and each group is further divided into subgroups called partitions.
Virtual machines from different partitions within the same group conflict with each other, meaning they cannot be co-located on servers from the same rack at the same time.
This constraint is crucial for ensuring system fault tolerance.
Our goal is to efficiently pack all virtual machines into a minimum number of racks.
To solve the problem, we use a column generation method to create packing templates for virtual machines, which are then distributed across servers.
The templates are adapted to take into account conflicts related to partition existence.
The effectiveness of the approach has been tested on an extensive open benchmark \cite{benchmark}, which contains up to 70,000 virtual machines.

\keywords{bin packing, conflicts, column generation, resource allocation, greedy algorithm} % в конце списка точка не ставится
\end{abstract}
\end{englishtitle}


%\end{document}
