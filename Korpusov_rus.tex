\begin{englishtitle} % Настраивает LaTeX на использование английского языка
% Этот титульный лист верстается аналогично.
\title{On critical exponents for weak solutions of the Cauchy problem for one $2+1$-dimensional nonlinear equation of wave theory in semiconductors}
% First author
\author{Maxim Korpusov\inst{1,2}  \and  Alexandra Matveeva\inst{1}
  %\and
  %Name FamilyName3\inst{1}
}
\institute{{Lomonosov Moscow State University, Moscow, Russia\\
  \email{matveeva2778@yandex.ru}
  \and
Peoples’ Friendship University of Russia named after Patrice Lumumba, Moscow, Russia\\
\email{korpusov@gmail.com}}
% etc
}
\maketitle

\begin{abstract}

The Cauchy problem for a model nonlinear equation with gradient nonlinearity is considered.
We prove the existence of two critical exponents  $q_1=2$ and $q_2=3,$  such that this problem has no local-in-time weak (in some sense) solution  t for $1<q\leqslant q_1$,  while such a solution exists for  $q>q_1$ , but , for $q_1<q\leqslant q_2$ , there is no global-in-time weak solution.

\keywords{nonlinear Sobolev-type equations, blow-up, local solvability, nonlinear capacity,  blow-up time estimates} % в конце списка точка не ставится
\end{abstract}
\end{englishtitle}


\iffalse
%%%%%%%%%%%%%%%%%%%%%%%%%%%%%%%%%%%%%%%%%%%%%%%%%%%%%%%%%%%%%%%%%%%%%%%%
%
%  This is the template file for the 6th International conference
%  NONLINEAR ANALYSIS AND EXTREMAL PROBLEMS
%  June 25-30, 2018
%  Irkutsk, Russia
%
%%%%%%%%%%%%%%%%%%%%%%%%%%%%%%%%%%%%%%%%%%%%%%%%%%%%%%%%%%%%%%%%%%%%%%%%


\documentclass[12pt]{llncs}

% При использовании pdfLaTeX добавляется стандартный набор русификации babel.

\usepackage{iftex}

\ifPDFTeX
\usepackage[T2A]{fontenc}
\usepackage[utf8]{inputenc} % Кодировка utf-8, cp1251 и т.д.
\usepackage[english,russian]{babel}
\usepackage[tbtags]{amsmath}
\usepackage{amsfonts,amssymb,mathrsfs,amscd}
\fi

\usepackage{todonotes}

\usepackage[russian]{nla}

\begin{document}
\fi

\title{О критических показателях для слабых решений задачи Коши для одного $2+1$--мерного нелинейного уравнения теории волн в полупроводниках\thanks{Работа выполнена при финансовой поддержке РНФ (проект \textnumero~23-11-00056) и при поддержке  Фонда развития теоретической физики и математики БАЗИС  (проект \textnumero~22-2-2-17-1)}}
% Первый автор
\author{M.~О.~Корпусов\inst{1,2} \and А.~К.~Матвеева\inst{1}
  %\and
% Третий автор
 % И.~О.~Фамилия\inst{1}
}

% Аффилиации пишутся в следующей форме, соединяя каждый институт при помощи \and.
\institute{МГУ им. Ломоносова, Москва, Россия \\
  \email{matveeva2778@yandex.ru}
  \and   % Разделяет институты и присваивает им номера по порядку.
РУДН, Москва, Россия\\
  \email{korpusov@gmail.com}
  %  \and
 % НИЯУ МИФИ, Москва, Россия\\
 % \email{email@example.com}
% \and Другие авторы...
}

\maketitle

\begin{abstract}
В работе рассматривается задача Коши для одного модельного нелинейного уравнения с градиентной нелинейностью.
Для этой задачи Коши в работе доказано существование двух критических показателей $q_1=2$ и $q_2=3,$ таких что при $1<q\leqslant q_1$ отсутствует локальное во времени в некотором смысле слабое решение, при $q>q_1$ локальное во времени слабое решение появляется, однако при $q_1<q\leqslant q_2$ отсутствует глобальное во времени слабое решение.

\keywords{нелинейные уравнения соболевского типа, разрушение, blow-up, локальная разрешимость, нелинейная емкость, оценки времени разрушения} % в конце списка точка не ставится
\end{abstract}

%\section{Основные результаты} % не обязательное поле
В работе рассматривается следующая задача Коши:
\begin{equation}\label{eq_intro-1}
\dfrac{\partial}{\partial t}\Delta u(x,t)+\dfrac{\partial}{\partial x_1}\Delta u(x,t)=|\nabla u(x,t)|^q,\quad (x,t)\in\mathbb{R}^2\otimes(0,T],
\end{equation}
\begin{equation}\label{eq_intro-2}
u(x,0)=u_0(x),\quad x\in\mathbb{R}^2,\quad\Delta=\dfrac{\partial^2}{\partial x^2_1}+\dfrac{\partial^2}{\partial x^2_2}.
\end{equation}
Вывод уравнения (\ref{eq_intro-1}) имеется в работе \cite{bagdoev}. Уравнение (\ref{eq_intro-1}) относятся к классу нелинейных уравнений типа С. Л. Соболева. Отметим, что исследованию линейных и нелинейных уравнений соболевского типа посвящено много работ. Так, в работах Г. А. Свиридюка, С. А. Загребиной, А. А. Замышляевой \cite{svirid1}--\cite{svirid3} были рассмотрены в общем виде и в виде примеров начально--краевые задачи для большого многообразия классов линейных и нелинейных уравнений соболевского типа.

Отметим, что впервые теория потенциала для неклассических уравнений типа С. Л. Соболева была рассмотрена в работе Б. В. Капитонова \cite{Kapitonov}. В дальнейшем теория потенциала изучалась в работах С. А. Габова и А.~Г.~Свешникова \cite{sveshnikov}, \cite{Gabov2}, а также в работах их учеников (см., например, работу Ю. Д. Плетнера \cite{pletner}).

В классической работе \cite{pokhozaev1} С. И. Похожаева и Э. Митидиери были достаточно простым методом нелинейной емкости были получены глубокие результаты
о роли так называемых критических показателей. Отметим также работы Е. И. Галахова и О. А. Салиевой \cite{gala1} и \cite{gala2}. В настоящей работе мы получили результат о существовании двух критических показателей $q_1=2$ и $q_2=3$ таких, что
в широких классах начальных функций $u_0(x)$ при $1<q\leqslant q_1$ отсутствует даже локальные во времени слабые решения задачи Коши (\ref{eq_intro-1}), (\ref{eq_intro-2}), а при $q_1<q$ локальные во времени слабые решения уже существуют, однако, при $q_1<q\leqslant q_2$ глобальных во времени слабых решений нет --- все слабые решения разрушаются за конечное время.

Данная работа продолжает исследования, начатые нами в работах \cite{korpusov1}--\cite{korpusov5}. Причем в работе \cite{korpusov5}
была рассмотрена следующая задача Коши:
$$
\dfrac{\partial}{\partial t}\Delta_x u+\dfrac{\partial}{\partial x_1}\Delta_xu=|u|^q,\quad x=(x_1,x_2,x_3)\in\mathbb{R}^3,\quad t>0,\quad q>1,
$$
$$
u(x,0)=u_0(x),\quad x\in\mathbb{R}^3,\quad\Delta_x=\dfrac{\partial^2}{\partial x^2_1}+\dfrac{\partial^2}{\partial x^2_2}+\dfrac{\partial^2}{\partial x_3^2}.
$$
Для которой были тоже получены два критических показателя $q_1=3$ и $q_2=4$ для аналогичных утверждений, что и в этой работе.
%Текст доклада на русском языке.

% Рисунки и таблицы оформляются по стандарту класса article. Например,

%\begin{figure}[htb]
 % \centering
  % Поддерживаются два формата:
  %\includegraphics[width=0.7\linewidth]{figure.pdf} % Растровый формат
  %\includegraphics[width=0.7\linewidth]{figure.png} % Векторный и растровый формат
  %

  %\begin{center}
   % \missingfigure[figwidth=0.7\linewidth]{Уберите меня из статьи!}
 % \end{center}
  %\caption{Заголовок рисунка}\label{fig:example}
%\end{figure}


% В конце текста можно выразить благодарности, если этого не было
% сделано в ссылке с заголовка статьи, например,
Работа выполнена при финансовой поддержке РНФ (проект \textnumero~23-11-00056) и при поддержке  Фонда развития теоретической физики и математики БАЗИС  (проект \textnumero~22-2-2-17-1).

%Работа выполнена при поддержке РФФИ (РНФ, другие фонды), проект \textnumero~00-00-00000.
%



\begin{thebibliography}{9} % или {99}, если ссылок больше десяти.
%\bibitem{Gantmakher} Гантмахер~Ф.Р. Теория матриц. М.:~Наука,~1966.

%\bibitem{Kholl} Современные численные методы решения обыкновенных дифференциальных уравнений~/ Под~ред.~Дж.~Холл, Дж.~Уатт. М.:~Мир,~1979.

%\bibitem{Aleksandrov1} Александров~А.Ю. Об устойчивости сложных систем в критических случаях~// Автоматика и телемеханика. 2001. \textnumero~9. С.~3--13.

%\bibitem{Moreau1977} Moreau~J.-J. Evolution problem associated with a moving convex set in a Hilbert space~// J.~Differential~Eq. 1977. Vol.~26. Pp.~347--374.

%\bibitem{Semenov} Семенов~А.А. Замечание о вычислительной сложности известных предположительно односторонних функций~// Тр.~XII Байкальской междунар. конф. <<Методы оптимизации и их приложения>>. Иркутск, 2001. С.~142--146.



\bibitem{bagdoev} Линейные и нелинейные волны в диспергирующих средах ~/ Под~ред.Багдоев~ А. Г., Ерофеев~В. И. ,  Шекоян~А. В. М.: ФИЗМАТЛИТ  ~2009.

\bibitem{svirid1}  Свиридюк~Г. А. К общей теории полугрупп операторов~// УМН. 1994\textnumero~49(4) С.~47--74.

\bibitem{svirid2}  Загребина~С. А. Начально-конечная задача для уравнений соболевского типа с сильно (L,p)--радиальным оператором~// Мат. заметки ЯГУ. 2012 \textnumero~ 19(2). С.39--48.


\bibitem{svirid3}  Zamyshlyaeva ~A. A., Sviridyuk ~G. A.  Nonclassical equations of mathematical physics. Linear Sobolev type equations of higher order~// Вестн. Южно-Ур. ун-та. Сер. Матем. Мех. Физ. 2016. \textnumero~ 8. Pp.5--16.

\bibitem{Kapitonov}   Капитонов~Б. В. Теория потенциала для уравнения малых колебаний вращающейся жидкости~// Матем. сб.1979. \textnumero~109(151). C.607--628.

\bibitem{sveshnikov}   Габов~С. А.,  Свешников~А. Г.  Линейные задачи теории нестационарных внутренних волн. М.:~Наука,~1990.

\bibitem{Gabov2} Габов~С. А.   Новые задачи математической теории волн. М.:~Физматлит, ~1998.

\bibitem{pletner}   Плетнер~Ю. Д. Фундаментальные решения операторов типа Соболева и некоторые начально-краевые задачи~// Ж. вычисл. матем. и матем. физ. 1992. \textnumero~32. C.1885--1899.


\bibitem{pokhozaev1}   Похожаев~С. И. ,  Митидиери~Э.  Априорные оценки и отсутствие решений нелинейных уравнений и неравенств в частных производных~// Тр. МИАН.2001  \textnumero 234  C.3--383.

\bibitem{gala1}  Galakhov ~E. I.  Some nonexistence results for quasilinear elliptic problems~// J. Math. Anal. Appl.Vol.2000. Vol.~252. Pp.256--277.

\bibitem{gala2} Галахов ~Е. И. , Салиева ~О. А. Об отсутствии неотрицательных монотонных решений для некоторых коэрцитивных неравенств в полупространстве ~// СФМ. 2017.  \textnumero~63. C. 573--585.

\bibitem{korpusov1}  Корпусов ~М. О.  Критические показатели мгновенного разрушения или локальной разрешимости нелинейных уравнений соболевского типа~// Изв. РАН. Сер. матем. 2015.\textnumero~79. C.103--162.

\bibitem{korpusov2}   Корпусов~М. О.  О разрушении решений нелинейных уравнений типа уравнения Хохлова–Заболотской~// ТМФ.2018. \textnumero~194. C.403--417.

\bibitem{korpusov3}  Korpusov~M. O., Ovchinnikov~A. V., Panin~A. A.  Instantaneous blow-up versus local solvability of solutions to the Cauchy problem for the equation of a semiconductor in a magnetic field ~// Math. Methods Appl. Sci.2018. Vol.~41. Pp.8070--8099.

\bibitem{korpusov4}  Корпусов  ~M.O., Панин ~А.А.  Мгновенное разрушение versus локальная разрешимость задачи Коши для двумерного уравнения полупроводника с тепловым разогревом ~// Изв. РАН. Сер. матем. 2019 Vol.~83. C. 1174--1200.

\bibitem{korpusov5}   Korpusov ~M.O., Matveeva ~A.K.  О критических показателях для слабых решений задачи Коши для одного нелинейного уравнения составного типа ~// Изв. РАН. Сер. матем.2021 Vol.~ 85 C.96--136.

\bibitem{korpusov51}   Korpusov ~M.O., Matveeva ~A.K.  On critical exponents for weak solutions to the Cauchy problem for one nonlinear equation with gradient nonlinearity~// MMAS.2023








\bibitem{Vlad}  Владимиров~В. С.   Уравнения математической физики~//М.: Наука. 1988.

\bibitem{Gilbarg}  Гилбарг~Д., Трудингер~М. Эллиптические дифференциальные уравнения с частными производными второго порядка ~// М.: Наука. 1989.

\bibitem{Fridman}   Фридман~А. Уравнения с частными производными параболического типа~// М.: Мир.1968.

\bibitem{panin5}  Панин ~А. А. О локальной разрешимости и разрушении решения абстрактного нелинейного интегрального уравнения Вольтерра~// Математические заметки. 2015 \textnumero~97. C.884--903.




\end{thebibliography}

% После библиографического списка в русскоязычных статьях необходимо оформить
% англоязычный заголовок.




%\end{document}

