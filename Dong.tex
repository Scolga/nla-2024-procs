
\iffalse
\documentclass[12pt]{llncs}

\usepackage{todonotes}

\usepackage{nla}

\begin{document}
\fi

\title{Stabilization of discrete-time generalized quasi-one-sided Lipschitz nonlinear systems with multiple delays}

\author{Wenqiang Dong
}
\institute{Shanghai Customs College, Shanghai, China\\
  \and
Sobolev Institute of Mathematics, Novosibirsk, Russia\\
\email{dongwenqiangshu@163.com}}


\maketitle

\begin{abstract}
In this article, a generalized quasi-one-sided Lipschitz condition is introduced to investigate the stabilization problem for a kind of discrete-time nonlinear systems with multiple delays. The feedback stabilization and observer-based stabilization for discrete-time nonlinear multiple delays systems are studied, respectively. It is shown that the nonlinear terms of systems will yield a beneficial impact on the whole systems for stabilization under generalized quasi-one-sided Lipschitz condition. A simulation example is given to illustrate the feasibility of the obtained results.
\keywords{discrete-time nonlinear systems with multiple delays, generalized quasi-one-sided Lipschitz condition, feedback stabilization, observer-based feedback stabilization}
\end{abstract}

\section{The main results} %optional section


Consider discrete-time nonlinear systems with multiple delays as follows:
\begin{equation}\label{eq:sys1}
  \left\{
    \begin{array}{ll}
      x(k+1)=A_{0}x(k)+\sum_{j=1}^{m}A_{j}x(k-d_{j})+Bu(k)+F(x(k),x(k-d_{1}),\cdots,x(k-d_{m})),  \\
      y(k)=Cx(k),
    \end{array}
  \right.
\end{equation}
where constant matrices $A_{0},A_{j}\in \mathbb R^{n\times n}$, $B\in\mathbb R^{n\times p}$, $C\in\mathbb R^{q\times n}$, state vector $x(k)\in \mathbb R^{n}$, delayed state vector $x(k-d_{j})\in \mathbb R^{n}$, constant delays $d_{j}>0$ for $j=1,2,\cdots,m$, output vector $y(k)\in \mathbb R^{q}$, control vector $u(k)\in \mathbb R^{p}$, $F(x(k),x(k-d_{1}),\cdots,x(k-d_{m}))$ is a nonlinear function with respect to $x(k),x(k-d_{1}),\cdots, x(k-d_{m})$, assume that $F(0,0,\cdots,0)=0$.
\par At first, a state feedback controller is designed to achieve feedback stabilization for discrete-time nonlinear systems with multiple delays (\ref{eq:sys1}). Let the structure of the feedback controller of systems (\ref{eq:sys1}) be
\begin{equation}\label{eq:controller}
  u(k)=-K_{0}x(k)-\sum_{j=1}^{m}K_{j}x(k-d_{j}),
\end{equation}
where $K_{0},K_{1},\cdots,K_{m}\in \mathbb R^{p\times n}$ are the controller gain matrices. By (\ref{eq:sys1}) and (\ref{eq:controller}), the nonlinear closed-loop systems with multiple delays can be obtained as:
\begin{equation}\label{eq:clo1}
  \left\{
  \begin{array}{ll}
    x(k+1)=(A_{0}-BK_{0})x(k)+\sum_{j=1}^{m}(A_{j}-BK_{j})x(k-d_{j})+F(x(k),x(k-d_{1}),\cdots,x(k-d_{m})), \\
    y(k)=Cx(k).
  \end{array}
\right.
\end{equation}
\par The sufficient condition for asymptotic stability of zero solutions of multiple delays closed-loop systems is derived under the generalized weak quasi-one-sided Lipschitz condition.
\par Subsequently, we will achieve the observer-based stabilization for systems (\ref{eq:sys1}). Firstly, a state observer is designed for estimating the state of systems (\ref{eq:sys1}). then, a observer-based controller is designed for achieving the stabilization of the systems.
\par Consider the following structure of state observer of systems (\ref{eq:sys1}):
\begin{equation}\label{eq:obs1}
  \left\{
  \begin{array}{ll}
  \begin{split}
    \hat{x}(k+1)&=A_{0}\hat{x}(k)+\sum_{j=1}^{m}A_{j}\hat{x}(k-d_{j})+Bu(k)+F(\hat{x}(k),\hat{x}(k-d_{1}),\cdots,\hat{x}(k-d_{m}))\\
      &\quad+L_{0}(y(k)-C\hat{x}(k))+\sum_{j=1}^{m}L_{j}(y(k-d_{j})-C\hat{x}(k-d_{j})), \\
  \end{split}
\\
    \hat{y}(k)=C\hat{x}(k),
  \end{array}
\right.
\end{equation}
where $L_{0},L_{1},\cdots,L_{m}\in \mathbb R^{n\times q}$ are observer gain matrices.

\par By Lyapunov-Krasovskii theory, the sufficient condition that observer (\ref{eq:obs1}) is a asymptotically stable observer for systems (\ref{eq:sys1}) is presented.
\par Next, combining the controller
\begin{equation}\label{eq:controller2}
  u(k)=-K_{0}\hat{x}(k)-\sum_{j=1}^{m}K_{j}\hat{x}(k-d_{j})
\end{equation}
with the observer (\ref{eq:obs1}), by construct the series Lyapunov-Krasovskii functional
\begin{equation}
\begin{split}
V(x(k),e(k))=&a[x^{T}(k)Q_{0}x(k)+\sum_{j=1}^{m}\sum_{i=1}^{d_{j}}x(k-d_{j})Q_{j}x(k-d_{j})]+e^{T}(k)P_{0}e(k)\\
&+\sum_{j=1}^{m}\sum_{i=1}^{d_{j}}e^{T}(k-d_{j})P_{j}e(k-d_{j}), \\
\end{split}
\end{equation}
We will proof that the feedback controller and state observer can be designed separately when discrete-time systems with multiple delays and nonlinearities carried out the control based on observer under the generalized quasi-one-sided condition. Finally, an illustrative numerical example is given to show the feasibility and superiority of the proposed methods.\\



The research is carried on with support of the Mathematical Center in Akademgorodok under agreement No. 075-15-2022-281 with the Ministry of Science and Higher Education of the Russian Federation, the National Natural Science Foundation of China (Grant No.12371399), Shanghai Science and Technology Talents Sailing Plan (22YF1415300).


\begin{thebibliography}{9} % or {99}, if there is more than ten references.


\bibitem{Hale1993} Hale J.K., Lunel S. M. V. Introduction to Functional Differential Equations. New York: Springer, 1993.

\bibitem{Hu2008} Hu G.D. A note on observer for one-sided Lipschitz non-linear systems. IMA J. Math. Control Inf.~2008. Vol.~25, no~2, Pp.~297--303.

\bibitem{Demidenko2017} Demidenko G.V., Baldanov D. Sh. Asymptotic stability of solutions to delay difference equations. J. Math. Sci.~2017. Vol.~221, no~6, Pp.~815-825.

\bibitem{Hu2020} Hu G.D., Dong W.Q., Cong Y.H. Separation principle for quasi-one-sided Lipschitz nonlinear systems with time delay. Int. J. Robust Nonlinear Control 2020, Vol.~30, Pp.~2430-2442.






\end{thebibliography}
%\end{document}
